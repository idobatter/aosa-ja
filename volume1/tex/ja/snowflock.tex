\begin{aosachapter}{SnowFlock}{s:snowflock}{Roy Bryant and Andr\'e{s} Lagar-Cavilla}
%% Based on EN-Revision r272

%% Cloud computing provides an attractively affordable computing
%% platform.  Instead of buying and configuring a physical server, with
%% all the associated time, effort and up front costs, users can rent
%% ``servers'' in the cloud with a few mouse clicks for
%% less than 10 cents per hour. Cloud providers keep their costs low by
%% providing virtual machines (VMs) instead of physical computers. The
%% key enabler is the virtualization software, called a virtual machine
%% monitor (VMM), that emulates a physical machine. Users are safely
%% isolated in their ``guest'' VMs, and are blissfully unaware that they
%% typically share the physical machine (``host'') with many others.
クラウドコンピューティングは、お手頃な価格で魅力的な計算プラットフォームを提供してくれる。
物理的なサーバーを購入して設定するのには時間や手間もかかるし、初期費用も必要だ。
クラウドコンピューティングなら、クラウドにある「サーバー」を簡単にレンタルできる。
マウスを何度かクリックするだけだし、一時間あたりのコストは10セントに満たない。
クラウド事業者がこんなに安価でサーバーを提供できている理由は、
物理的なコンピューターではなく仮想マシン(VM)を使っているからだ。
それを実現する肝となるのが仮想化ソフトウェアで、これは仮想マシンモニター(VMM)
と呼ばれている。物理的なマシンをエミュレートする仕組みだ。
各ユーザーはそれぞれ安全に隔離された「ゲスト」VMを利用する。
ふつうは複数のゲストが一台の物理マシン(「ホスト」)を共有しているのだが、
ありがたいことにユーザーはそれを気にせずに済む。

%% \begin{aosasect1}{Introducing SnowFlock}
\begin{aosasect1}{SnowFlockのご紹介}

Clouds are a boon to agile organizations. With physical servers, users
are relegated to waiting impatiently while others (slowly) approve the
server purchase, place the order, ship the server, and install and
configure the Operating System (OS) and application stacks. Instead of
waiting weeks for others to deliver, the cloud user retains control of
the process and can create a new, standalone server in minutes.

Unfortunately, few cloud servers stand alone.  Driven by the quick
instantiation and pay-per-use model, cloud servers are typically
members of a variable pool of similarly configured servers performing
dynamic and scalable tasks related to parallel computing, data mining,
or serving web pages.  Because they repeatedly boot new instances from
the same, static template, commercial clouds fail to fully deliver on
the promise of true on-demand computation. After instantiating the
server, the cloud user must still manage cluster membership and broker
the addition of new servers.

SnowFlock addresses these issues with VM Cloning, our proposed cloud
API call. In the same way that application code routinely invokes OS
services through a syscall interface, it could now also invoke cloud
services through a similar interface. With SnowFlock's VM Cloning,
resource allocation, cluster management, and application logic can be
interwoven programmatically and dealt with as a single logical
operation.

The VM Cloning call instantiates multiple cloud servers that are
identical copies of the originating parent VM up to the point of
cloning. Logically, clones inherit all the state of their parent,
including OS- and application-level caches. Further, clones are
automatically added to an internal private network, thus effectively
joining a dynamically scalable cluster. New computation resources,
encapsulated as identical VMs, can be created on-the-fly and can be
dynamically leveraged as needed.

To be of practical use, VM cloning has to be applicable, efficient,
and fast. In this chapter we will describe how SnowFlock's
implementation of VM Cloning can be effectively interwoven in several
different programming models and frameworks, how it can be implemented
to keep application runtime and provider overhead to a minimum, and
how it can be used to create dozens of new VMs in five seconds or
less.

With an API for the programmatic control of VM Cloning with bindings
in C, C++, Python and Java, SnowFlock is extremely flexible and
versatile.  We've successfully used SnowFlock in prototype
implementations of several, quite different, systems.  In parallel
computation scenarios, we've achieved excellent results by explicitly
cloning worker VMs that cooperatively distribute the load across many
physical hosts. For parallel applications that use the Message Passing
Interface (MPI) and typically run on a cluster of dedicated servers,
we modified the MPI startup manager to provide unmodified applications
with good performance and much less overhead by provisioning a fresh
cluster of clones on demand for each run. Finally, in a quite
different use case, we used SnowFlock to improve the efficiency and
performance of elastic servers.  Today's cloud-based elastic servers
boot new, cold workers as needed to service spikes in demand. By
cloning a running VM instead, SnowFlock brings new workers on line 20
times faster, and because clones inherit the warm buffers of their
parent, they reach their peak performance sooner.

\end{aosasect1}

%% \begin{aosasect1}{VM Cloning}
\begin{aosasect1}{VMのクローニング}

As the name suggests, VM clones are (nearly) identical to their parent
VM\@. There are actually some minor but necessary differences to avoid
issues such as MAC address collisions, but we'll come back to that
later. To create a clone, the entire local disk and memory state must
be made available, which brings us to the first major design tradeoff:
should we copy that state up-front or on demand?

The simplest way to achieve VM cloning is to adapt the standard VM
``migration'' capability.  Typically, migration is used when a running
VM needs to be moved to a different host, such as when the host
becomes overloaded or must be brought down for maintenance. Because
the VM is purely software, it can be encapsulated in a data file that
can then be copied to a new, more appropriate host, where it picks up
execution after a brief interruption. To accomplish this,
off-the-shelf VMMs create a file containing a ``checkpoint'' of the VM,
including its local filesystem, memory image, virtual CPU (VCPU)
registers, etc. In migration, the newly booted copy replaces the
original, but the process can be altered to produce a clone while
leaving the original running. In this ``eager'' process, the entire VM
state is transferred up front, which provides the best initial
performance, because the entire state of the VM is in place when
execution begins.  The disadvantage of eager replication is that the
laborious process of copying the entire VM must happen before
execution can begin, which significantly slows instantiation.

The other extreme, adopted by SnowFlock, is ``lazy'' state
replication.  Instead of copying everything the VM might ever need,
SnowFlock transfers only the vital bits needed to begin execution, and
transfers state later, only when the clone needs it. This has two
advantages.  First, it minimizes the instantiation latency by doing as
little work as possible up front. Second, it increases the efficiency
by copying only the state that is actually used by the clone. The
yield of this benefit, of course, depends on the clone's behavior, but
few applications access every page of memory and every file in the
local filesystem.

However, the benefits of lazy replication aren't free. Because the
state transfer is postponed until the last moment, the clone is left
waiting for state to arrive before it can continue execution. This
This situation parallels swapping of memory to disk in time-shared workstation:
applications are blocked waiting for state to be fetched from a high latency source.
In the case of SnowFlock, the
blocking somewhat degrades the clone's performance; the severity of
the slowdown depends on the application.  For high performance
computing applications we've found this degradation has little impact,
but a cloned database server may perform poorly at first. It should be
noted that this is a transient effect: within a few minutes, most of
the necessary state has been transferred and the clone's performance
matches that of the parent.

As an aside, if you're well versed in VMs, you're likely wondering if
the optimizations used by ``live'' migration are useful here. Live
migration is optimized to shorten the interval between the original
VM's suspension and the resumption of execution by the new copy.  To
accomplish this, the Virtual Machine Monitor (VMM) pre-copies the VM's state while the original
is still running, so that after suspending it, only the recently
changed pages need to be transferred. This technique does not affect
the interval between the migration request and the time the copy
begins execution, and so would not reduce the instantiation latency of
eager VM cloning.

\end{aosasect1}

%% \begin{aosasect1}{SnowFlock's Approach}
\begin{aosasect1}{SnowFlockの手法}

SnowFlock implements VM cloning with a primitive called ``VM Fork'',
which is like a standard Unix \code{fork}, but with a few
important differences.  First, rather than duplicating a single
process, VM Fork duplicates an entire VM, including all of memory, all
processes and virtual devices, and the local filesystem.  Second,
instead of producing a single copy running on the same physical host,
VM Fork can simultaneously spawn many copies in parallel. Finally, VMs
can be forked to distinct physical servers, letting you quickly
increase your cloud footprint as needed.

The following concepts are key to SnowFlock:

\begin{aosaitemize}

  \item Virtualization: The VM encapsulates the computation
  environment, making clouds and machine cloning possible.

  \item Lazy Propagation: The VM state isn't copied until it's needed,
  so clones come alive in a few seconds.

  \item Multicast: Clone siblings have similar needs in terms of VM
  state. With multicast, dozens of clones start running as quickly
  as one.

  \item Page Faults: When a clone tries to use missing memory, it
  faults and triggers a request to the parent. The clone's execution
  is blocked until the needed page arrives.

  \item Copy on Write (CoW): By taking a copy of its memory and disk
  pages before overwriting them, the parent VM can continue to run
  while preserving a frozen copy of its state for use by the clones.

\end{aosaitemize}

We've implemented SnowFlock using the Xen virtualization system, so
it's useful to introduce some Xen-specific terminology for clarity. In
a Xen environment, the VMM is called the hypervisor, and VMs are
called domains. On each physical machine (host), there is a privileged
domain, called ``domain 0'' (dom0), that has full access to the host
and its physical devices, and can be used to control additional guest, or ``user'',
VMs that are called ``domain U'' (domU).

In broad strokes, SnowFlock consists of a set of modifications to the
Xen hypervisor that enable it to smoothly recover when missing
resources are accessed, and a set of supporting processes and systems
that run in dom0 and cooperatively transfer the missing VM state, and
some optional modifications to the OS executing inside clone
VMs. There are six main components.

\begin{aosaitemize}

  \item VM Descriptor: This small object is used to seed the clone,
  and holds the bare-bones skeleton of the VM as needed to begin
  execution.  It lacks the guts and muscle needed to perform any
  useful work.

  \item Multicast Distribution System (\code{mcdist}): This
  parent-side system efficiently distributes the VM state
  information simultaneously to all clones.

  \item Memory Server Process: This parent-side process maintains a
  frozen copy of the parent's state, and makes it available to all
  clones on demand through \code{mcdist}.

  \item Memtap Process: This clone-side process acts on the clone's
  behalf, and communicates with the memory server to request pages
  that are needed but missing.

  \item Clone Enlightenment: The guest kernel running inside the
  clones can alleviate the on-demand transfer of VM state by
  providing hints to the VMM\@. This is optional but highly desirable
  for efficiency.

  \item Control Stack: Daemons run on each physical host to
  orchestrate the other components and manage the SnowFlock parent and clone VMs.

\end{aosaitemize}

%% \aosafigure{../images/snowflock/drawing.pdf}{SnowFlock VM Replication Architecture}{fig.snowflock.arch}
\aosafigure{../images/snowflock/drawing.pdf}{SnowFlock VMレプリケーションのアーキテクチャ}{fig.snowflock.arch}

Pictorially speaking, \aosafigref{fig.snowflock.arch} depicts the
process of cloning a VM, showing the the four main steps: (1)
suspending the parent VM to produce an architectural descriptor; (2)
distributing this descriptor to all target hosts; (3) initiating
clones that are mostly empty state-wise; and (4) propagating state
on-demand. The figure also depicts the use of multicast distribution
with \code{mcdist}, and fetch avoidance via guest enlightenment.

If you're interested in trying SnowFlock, it's available in two
flavors.  The documentation and open source code for the original
University of Toronto SnowFlock research project are
available\footnote{\url{http://sysweb.cs.toronto.edu/projects/1}}
If you'd prefer to take
the industrial-strength version for a spin, a free, non-commercial
license is available from GridCentric 
Inc.\footnote{\url{http://www.gridcentriclabs.com/architecture-of-open-source-applications}}
Because SnowFlock includes changes to the hypervisor and requires
access to dom0, installing SnowFlock requires privileged access on the
host machines. For that reason, you'll need to use your own hardware,
and won't be able to try it out as a user in a commercial cloud
environment such as Amazon's EC2.

Throughout the next few sections we'll describe the different pieces
that cooperate to achieve instantaneous and efficient cloning. All
the pieces we will describe fit together as shown in
\aosafigref{fig.snowflock.components}.

%% \aosafigure{../images/snowflock/drawing-arch.eps}{Software Components of SnowFlock}{fig.snowflock.components}
\aosafigure{../images/snowflock/drawing-arch.eps}{SnowFlockのソフトウェアコンポーネント}{fig.snowflock.components}

\end{aosasect1}

%% \begin{aosasect1}{Architectural VM Descriptor}
\begin{aosasect1}{Architectural VM Descriptor}

The key design decision for SnowFlock is to postpone the replication
of VM state to a lazy runtime operation. In other words, copying the
memory of a VM is a late binding operation, allowing for many
opportunities for optimization.

The first step to carry out this design decision is the generation of
an architectural descriptor of the VM state. This is the seed that
will be used to create clone VMs. It contains the bare minimum
necessary to create a VM and make it schedulable. As the name implies,
this bare minimum consists of data structures needed by the underlying
architectural specification.  In the case of SnowFlock, the
architecture is a combination of Intel x86 processor requirements and
Xen requirements. The architectural descriptor thus contains data
structures such as page tables, virtual registers, device metadata,
wallclock timestamps, etc. We refer the interested reader to~\cite{bib:snowflocktocs}
for an in-depth description of the contents of the
architectural descriptor.

An architectural descriptor has three important properties: First, it
can be created in little time; 200 milliseconds is not
uncommon. Second, it is small, typically three orders of magnitude
smaller than the memory allocation of the originating VM (1 MB for a 1
GB VM). And third, a clone VM can be created from a descriptor in less
than a second (typically 800 milliseconds).

The catch, of course, is that the cloned VMs are missing most of their
memory state by the time they are created from the descriptor. The
following sections explain how we solve this problem---and how we take
advantage of the opportunities for optimization it presents.

\end{aosasect1}

%% \begin{aosasect1}{Parent-Side Components}
\begin{aosasect1}{親側のコンポーネント}

Once a VM is cloned it becomes a parent for its children or
clones.  Like all responsible parents, it needs to look out for the
well-being of its descendants. It does so by setting up a set of
services that provision memory and disk state to cloned VMs on demand.

%% \begin{aosasect2}{Memserver Process}
\begin{aosasect2}{Memserverプロセス}

When the architectural descriptor is created, the VM is stopped in its
tracks throughout the process. This is so the VM memory state settles;
before actually pausing a VM and descheduling from execution,
internal OS drivers quiesce into a state from which clones can
reconnect to the external world in their new enclosing VMs. We take
advantage of this quiescent state to create a ``memory server'',
or \code{memserver}.

The memory server will provide all clones with the bits of memory they
need from the parent. Memory is propagated at the granularity of an
x86 memory page (4 kbytes). In its simplest form, the memory server sits
waiting for page requests from clones, and serves one page at a time,
one clone at a time.

However, this is the very same memory the parent VM needs to use to
keep executing. If we would allow the parent to just go ahead and
modify this memory, we would serve corrupted memory contents to clone
VMs: the memory served would be different from that at the point of
cloning, and clones would be mightily confused. In kernel hacking
terms, this is a sure recipe for stack traces.

To circumvent this problem, a classical OS notion comes to the rescue:
Copy-on-Write, or CoW memory. By enlisting the aid of the Xen
hypervisor, we can remove writing privileges from all pages of memory
in the parent VM\@. When the parent actually tries to modify one page, a
hardware page fault is triggered.  Xen knows why this happened, and
makes a copy of the page. The parent VM is allowed to write to the original
page and continue execution, while the memory server is told to use
the copy, which is kept read-only. In this manner, the memory state at the point of cloning
remains frozen so that clones are not confused, while the parent is
able to proceed with execution. The overhead of CoW is minimal:
similar mechanisms are used by Linux, for example, when creating new
processes.

\end{aosasect2}

%% \begin{aosasect2}{Multicasting with Mcdist}
\begin{aosasect2}{Mcdistによるマルチキャスト}

Clones are typically afflicted with an existential syndrome known as
``fate determinism.'' We expect clones to be created for a single
purpose: for example, to align X chains of DNA against a segment Y of
a database. Further, we expect a set of clones to be created so that
all siblings do the same, perhaps aligning the same X chains against
different segments of the database, or aligning different chains
against the same segment Y\@. Clones will thus clearly exhibit a great
amount of temporal locality in their memory accesses: they will use
the same code and large portions of common data.

We exploit the opportunities for temporal locality through \code{mcdist}, our
own multicast distribution system tailored to SnowFlock. Mcdist uses
IP multicast to simultaneously distribute the same packet to a set of
receivers. It takes advantage of network hardware parallelism to
decrease the load on the memory server. By sending a reply to all
clones on the first request for a page, each clone's requests act as a
prefetch for its siblings, because of their similar memory access
patterns.

Unlike other multicast systems, \code{mcdist} does not have to be reliable,
does not have to deliver packets in an ordered manner, and does not
have to atomically deliver a reply to all intended
receivers. Multicast is strictly an optimization, and delivery need
only be ensured to the clone explicitly requesting a page. The design
is thus elegantly simple: the server simply multicasts responses,
while clients time-out if they have not received a reply for a
request, and retry the request.

Three optimizations specific to SnowFlock are included in \code{mcdist}:

\begin{aosaitemize}

  \item Lockstep Detection: When temporal locality does happen,
  multiple clones request the same page in very close
  succession. The \code{mcdist} server ignores all but the first of such
  requests.

  \item Flow Control: Receivers piggyback their receive rate on
  requests.  The server throttles its sending rate to a weighted
  average of the clients' receive rate. Otherwise, receivers will be
  drowned by too many pages sent by an eager server.

  \item End Game: When the server has sent most pages, it falls back
    to unicast responses. Most requests at this point are retries, and
    thus blasting a page through the wire to all clones is
    unnecessary.

\end{aosaitemize}

\end{aosasect2}

%% \begin{aosasect2}{Virtual Disk}
\begin{aosasect2}{仮想ディスク}

SnowFlock clones, due to their short life span and fate determinism,
rarely use their disk. The virtual disk for a SnowFlock VM houses the
root partition with binaries, libraries and configuration files. Heavy
data processing is done through suitable filesystems such as HDFS
(\aosachapref{s:hdfs}) or PVFS\@. Thus, when SnowFlock clones decide
to read from their root disk, they typically have their requests
satisfied by the kernel filesystem page cache.

Having said that, we still need to provide access to the virtual disk
for clones, in the rare instance that such access is needed. We
adopted the path of least resistance here, and implemented the disk by
closely following the memory replication design. First, the state of
the disk is frozen at the time of cloning. The parent VM keeps using
its disk in a CoW manner: writes are sent to a separate location in
backing storage, while the view of the disk clones expect remains
immutable.  Second, disk state is multicast to all clones, using
\code{mcdist}, with the same 4 KB page granularity, and under the same
expectations of temporal locality. Third, replicated disk state for a
clone VM is strictly transient: it is stored in a sparse flat file
which is deleted once the clone is destroyed.

\end{aosasect2}

\end{aosasect1}

%% \begin{aosasect1}{Clone-Side Components}
\begin{aosasect1}{クローン側のコンポーネント}

Clones are hollow shells when they are created from an architectural
descriptor, so like everybody else, they need a lot of help from their
parents to grow up: the children VMs move out and immediately call
home whenever they notice something they need is missing, asking their
parent to send it over right away.

%% \begin{aosasect2}{Memtap Process}
\begin{aosasect2}{Memtapプロセス}

Attached to each clone after creation, the \code{memtap} process is the
lifeline of a clone. It maps all of the memory of the clone and fills
it on demand as needed.  It enlists some crucial bits of help from the
Xen hypervisor: access permission to the memory pages of the clones is
turned off, and hardware faults caused by first access to a page are
routed by the hypervisor into the \code{memtap} process.

In its simplest incarnation, the \code{memtap} process simply asks the
memory server for the faulting page, but there are more complicated
scenarios as well.  First, \code{memtap} helpers use \code{mcdist}. This means that at any
point in time, any page could arrive by virtue of having been
requested by another clone---the beauty of asynchronous
prefetching. Second, we allow SnowFlock VMs to be multi-processor VMs.
There wouldn't be much fun otherwise. This means that multiple faults
need to be handled in parallel, perhaps even for the same page.
Third, in later versions \code{memtap} helpers can explicitly prefetch a
batch of pages, which can arrive in any order given the lack of
guarantees from the \code{mcdist} server.  Any of these factors could have led
to a concurrency nightmare, and we have all of them.

The entire \code{memtap} design centers on a page presence bitmap. The bitmap
is created and initialized when the architectural descriptor is
processed to create the clone VM\@.
The bitmap is a flat bit array sized by the number of pages the VM's memory can hold. 
Intel processors have handy atomic
bit mutation instructions: setting a bit, or doing a test and set, can
happen with the guarantee of atomicity with respect to other
processors in the same box.  This allows us to avoid
locks in most cases, and thus to provide access to the bitmap by
different entities in different protection domains: the Xen
hypervisor, the \code{memtap} process, and the cloned guest kernel itself.

When Xen handles a hardware page fault due to a first access to a
page, it uses the bitmap to decide whether it needs to alert
\code{memtap}. It also uses the bitmap to enqueue multiple faulting virtual
processors as dependent on the same absent page. Memtap buffers pages
as they arrive.  When its buffer is full or an explicitly requested
page arrives, the VM is paused, and the bitmap is used to discard any
duplicate pages that have arrived but are already present. Any
remaining pages that are needed are then copied into the VM memory,
and the appropriate bitmap bits are set.

\end{aosasect2}

%% \begin{aosasect2}{Clever Clones Avoid Unnecessary Fetches}
\begin{aosasect2}{Clever Clones Avoid Unnecessary Fetches}

We just mentioned that the page presence bitmap is visible to the
kernel running inside the clone, and that no locks are needed to
modify it. This gives clones a powerful ``enlightenment'' tool: they can prevent the
fetching of pages by modifying the bitmap and pretending they are
present. This is extremely useful performance-wise, and safe to do
when pages will be completely overwritten before they are used.

There happens to be a very common situation when this happens and
fetches can be avoided. All memory allocations in the kernel (using
\code{vmalloc}, \code{kzalloc}, \code{get\_free\_page},
user-space \code{brk}, and the like) are ultimately
handled by the kernel page allocator. Typically pages are requested by
intermediate allocators which manage finer-grained chunks: the slab
allocator, the glibc malloc allocator for a user-space process,
etc. However, whether allocation is explicit or implicit,
one key semantic implication
always holds true: no one cares about what the page contained,
because its contents will be arbitrarily overwritten. Why
fetch such a page, then? There is no reason to do so, and empirical
experience shows that avoiding such fetches is tremendously
advantageous.

\end{aosasect2}

\end{aosasect1}

%% \begin{aosasect1}{VM Cloning Application Interface}
\begin{aosasect1}{VMクローニングアプリケーションのインターフェイス}

So far we have focused on the internals of cloning a VM
efficiently. As much fun as solipsistic systems can be, we need to
turn our attention to those who will use the system: applications.

%% \begin{aosasect2}{API Implementation}
\begin{aosasect2}{APIの実装}

VM Cloning is offered to the application via the simple SnowFlock API,
depicted in \aosafigref{fig.snowflock.cloningapi}. Cloning is
basically a two-stage process. You first request an allocation for the
clone instances, although due to the system policies that are in effect, that
allocation may be smaller than requested.  Second, you use the
allocation to clone your VM\@.  A key assumption is that your VM focuses
on a single operation.  VM Cloning is appropriate for
single-application VMs such as a web server or a render farm
component.  If you have a hundred-process desktop environment in which
multiple applications concurrently call VM cloning, you're headed for
chaos.

\begin{table}\centering
  \begin{tabular}{ |l p{7cm}| }
    \hline
    \code{sf\_request\_ticket(n)}
    &
    Requests an allocation for \code{n} clones.  Returns a
    \code{ticket} describing an allocation for \code{m$\leq$n}
    clones.
    \\

    \code{sf\_clone(ticket)}
    &
    Clones, using the \code{ticket} allocation. Returns the clone
    \code{ID}, \code{0$\leq$ID{\textless}m}.
    \\

    \code{sf\_checkpoint\_parent()}
    &
    Prepares an immutable checkpoint \code{C} of the parent VM to be
    used for creating clones at an arbitrarily later time.
    \\

    \code{sf\_create\_clones(C, ticket)}
    &
    Same as \code{sf\_clone}, uses the checkpoint \code{C}.
    Clones will begin execution at the point at which
    the corresponding \code{sf\_checkpoint\_parent()} was invoked.
    \\

    \code{sf\_exit()}
    &
    For children \code{(1$\leq$ID{\textless}m)}, terminates
    the child.
    \\

    \code{sf\_join(ticket)}
    &
    For the parent \code{(ID = 0)}, blocks until all children in the
    \code{ticket} reach their \code{sf\_exit} call.  At that point all
    children are terminated and the \code{ticket} is discarded.
    \\

    \code{sf\_kill(ticket)}
    &
    Parent only, discards \code{ticket} and immediately kills all
    associated children.
    \\
    \hline
  \end{tabular}
  \caption{The SnowFlock VM Cloning API}
  \label{fig.snowflock.cloningapi}
\end{table}

The API simply marshals messages and posts them to the XenStore, a
shared-memory low-throughput interface used by Xen for control plane
transactions. A SnowFlock Local Daemon (SFLD) executes on the
hypervisor and listens for such requests. Messages are unmarshalled,
executed, and requests posted back.

Programs can control VM Cloning directly through the API, which is
available for C, C++, Python and Java. Shell scripts that harness the
execution of a program can use the provided command-line scripts
instead.  Parallel frameworks such as MPI can embed the API: MPI
programs can then use SnowFlock without even knowing, and with no
modification to their source.
Load balancers sitting in front of web or application
servers can use the API to clone the servers they manage.

SFLDs orchestrate the execution of VM Cloning requests. They create
and transmit architectural descriptors, create cloned VMs, launch disk
and memory servers, and launch \code{memtap} helper processes.  They are a
miniature distributed system in charge of managing the VMs in a
physical cluster.

SFLDs defer allocation decisions to a central SnowFlock Master Daemon
(SFMD).  SFMD simply interfaces with appropriate cluster management
software. We did not see any need to reinvent the wheel here, and
deferred decisions on resource allocation, quotas, policies, etc.\  to
suitable software such as Sun Grid Engine or Platform EGO.

\end{aosasect2}

%% \begin{aosasect2}{Necessary Mutations}
\begin{aosasect2}{欠かせない変化}

After cloning, most of the cloned VM's processes have no idea that
they are no longer the parent, and that they are now running in a
copy. In most aspects, this just works fine and causes no
issues. After all, the primary task of the OS is to isolate
applications from low-level details, such as the network identity.
Nonetheless, making the transition smooth requires a set of mechanisms
to be put in place. The meat of the problem is in managing the clone's
network identity; to avoid conflicts and confusion, we must introduce
slight mutations during the cloning process.  Also, because these
tweaks may necessitate higher-level accommodations, a hook is inserted
to allow the user to configure any necessary tasks, such as
(re)mounting network filesystems that rely on the clone's identity.

Clones are born to a world that is mostly not expecting them. The
parent VM is part of a network managed most likely by a DHCP server,
or by any other of the myriad ways sysadmins find to do their
job. Rather than assume a necessarily inflexible scenario, we place
the parent and all clones in their own private virtual network. Clones
from the same parent are all assigned a unique ID, and their IP
address in this private network is automatically set up upon cloning
as a function of the ID\@. This guarantees that no intervention from a
sysadmin is necessary, and that no IP address collisions will ever
happen.

IP reconfiguration is performed directly by a hook we place on the
virtual network driver. However, we also rig the driver to
automatically generate synthetic DHCP responses. Thus, regardless of
your choice of distribution, your virtual network interface will
ensure that the proper IP coordinates are propagated to the guest OS,
even when you are restarting from scratch.

To prevent clones from different parents colliding with each others'
virtual private networks---and to prevent mutual DDoS attacks---clone
virtual network are isolated at the Ethernet (or layer 2) level. We
hijack a range of Ethernet MAC OUIs~\footnote{OUI, or Organizational Unique ID,
is a range of MAC addresses assigned to a vendor.} and dedicate them to clones. The
OUI will be a function of the parent VM\@. Much like the ID of a VM
determines its IP address, it also determines its non-OUI Ethernet MAC
address.  The virtual network driver translates the MAC address the VM
believes it has to the one assigned as a function of its ID, and
filters out all traffic to and from virtual private network with
different OUIs. This isolation is equivalent to that achievable via
\code{ebtables}, although much simpler to implement.

Having clones talk only to each other may be fun, but not fun enough.
Sometimes we will want our clones to reply to HTTP requests from the
Internet, or mount public data repositories. We equip any set of
parent and clones with a dedicated router VM\@. This tiny VM performs
firewalling, throttling and NATing of traffic from the clones to the
Internet. It also limits inbound connections to the parent VM and
well-known ports.  The router VM is lightweight but represents a
single point of centralization for network traffic, which can
seriously limit scalability. The same network rules could be applied
in a distributed fashion to each host upon which a clone VM runs. We
have not released that experimental patch.

SFLDs assign IDs, and teach the virtual network drivers how they
should configure themselves: internal MAC and IP addresses, DHCP
directives, router VM coordinates, filtering rules, etc.

\end{aosasect2}

\end{aosasect1}

%% \begin{aosasect1}{Conclusion}
\begin{aosasect1}{結論}

By tweaking the Xen hypervisor and lazily transferring the VM's state,
SnowFlock can produce dozens of clones of a running VM in a few
seconds. Cloning VMs with SnowFlock is thus instantaneous and
live---it improves cloud usability by automating cluster management
and giving applications greater programmatic control over the cloud
resources. SnowFlock also improves cloud agility by speeding up VM
instantiation by a factor of 20, and by improving the performance of
most newly created VMs by leveraging their parent's warm, in-memory OS
and application caches. The keys to SnowFlock's efficient performance
are heuristics that avoid unnecessary page fetches, and the multicast
system that lets clone siblings cooperatively prefetch their
state. All it took was the clever application of a few tried-and-true
techniques, some sleight of hand, and a generous helping of
industrial-strength debugging.

We learned two important lessons throughout the SnowFlock experience.
The first is the often-underestimated value of the KISS theorem. We
were expecting to implement complicated prefetching techniques to
alleviate the spate of requests for memory pages a clone would issue
upon startup.  This was, perhaps surprisingly, not necessary. The
system performs very well for many workloads based on one single
principle: bring the memory over as needed.  Another example of the
value of simplicity is the page presence bitmap. A simple data
structure with clear atomic access semantics greatly simplifies what
could have been a gruesome concurrency problem, with multiple virtual
CPUs competing for page updates with the asynchronous arrival of pages
via multicast.

The second lesson is that scale does not lie. In other words, be
prepared to have your system shattered and new bottlenecks uncovered
every time you bump your scale by a power of two. This is intimately
tied with the previous lesson: simple and elegant solutions scale well
and do not hide unwelcome surprises as load increases. A prime example
of this principle is our \code{mcdist} system. In large-scale tests, a
TCP/IP-based page distribution mechanism fails miserably for hundreds
of clones. Mcdist succeeds by virtue of its extremely constrained and
well-defined roles: clients only care about their own pages; the
server only cares about maintaining a global flow control. By keeping
\code{mcdist} humble and simple, SnowFlock is able to scale extremely well.

If you are interested in knowing more, you can visit the University of
Toronto site\footnote{\url{http://sysweb.cs.toronto.edu/projects/1}} for the
academic papers and open-source code licensed under GPLv2, and
GridCentric\footnote{\url{http://www.gridcentriclabs.com/}} for an industrial strength
implementation.

\end{aosasect1}

\end{aosachapter}

