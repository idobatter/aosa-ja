\addtocontents{toc}{\vspace{\baselineskip}}
\begin{aosachapter}{Mercurial}{s:mercurial}{Dirkjan Ochtman}
%% Based on EN-Revision r229

Mercurial is a modern distributed version control system (VCS), written
mostly in Python with bits and pieces in C for performance. In this
chapter, I will discuss some of the decisions involved in designing
Mercurial's algorithms and data structures. First, allow me to go into
a short history of version control systems, to add necessary context.

%% \begin{aosasect1}{ A Short History of Version Control}
\begin{aosasect1}{バージョン管理システムの簡単な歴史}

While this chapter is primarily about Mercurial's software
architecture, many of the concepts are shared with other version
control systems. In order to fruitfully discuss Mercurial, I'd like to
start off by naming some of the concepts and actions in different
version control systems. To put all of this in perspective, I will
also provide a short history of the field.

Version control systems were invented to help developers work on
software systems simultaneously, without passing around full copies
and keeping track of file changes themselves. Let's generalize from
software source code to any tree of files. One of the primary
functions of version control is to pass around changes to the
tree. The basic cycle is something like this:

\begin{aosaenumerate}

  \item Get the latest tree of files from someone else

  \item Work on a set of changes to this version of the tree

  \item Publish the changes so that others can retrieve them

\end{aosaenumerate}

\noindent
The first action, to get a local tree of files, is called a
\emph{checkout}. The store where we retrieve and publish our changes
is called a \emph{repository}, while the result of the checkout is
called a \emph{working directory}, \emph{working tree}, or
\emph{working copy}. Updating a working copy with the latest files
from the repository is simply called \emph{update}; sometimes this
requires \emph{merging}, i.e., combining changes from different users
in a single file. A diff command allows us to review changes between
two revisions of a tree or file, where the most common mode is to
check the local (unpublished) changes in your working copy. Changes
are published by issuing a \emph{commit} command, which will save the
changes from the working directory to the repository.

%% \begin{aosasect2}{Centralized Version Control}
\begin{aosasect2}{中央集中型バージョン管理}

The first version control system was the Source Code Control System,
SCCS, first described in 1975. It was mostly a way of saving deltas to
single files that was more efficient than just keeping around copies,
and didn't help with publishing these changes to others. It was
followed in 1982 by the Revision Control System, RCS, which was a more
evolved and free alternative to SCCS (and which is still being
maintained by the GNU project).

After RCS came CVS, the Concurrent Versioning System, first released
in 1986 as a set of scripts to manipulate RCS revision files in
groups. The big innovation in CVS is the notion that multiple users can
edit simultaneously, with merges being done after the fact
(concurrent edits). This also required the notion of edit conflicts.
Developers may only commit a new version of some file if it's based
on the latest version available in the repository. If there are changes
in the repository and in my working directory, I have to resolve any
conflicts resulting from those changes (edits changing the same lines).

CVS also pioneered the notions of \emph{branches}, which allow
developers to work on different things in parallel, and \emph{tags},
which enable naming a consistent snapshot for easy reference. While
CVS deltas were initially communicated via the repository on a shared
filesystem, at some point CVS also implemented a client-server
architecture for use across large networks (such as the Internet).

In 2000, three developers got together to build a new VCS, christened
Subversion, with the intention of fixing some of the larger
warts in CVS\@. Most importantly, Subversion works on whole trees at a
time, meaning changes in a revisions should be atomic, consistent,
isolated, and durable. Subversion working copies also retain a
pristine version of the checked out revision in the working directory,
so that the common diff operation (comparing the local tree against
a checked-out changeset) is local and thus fast.

One of the interesting concepts in Subversion is that tags and
branches are part of a project tree. A Subversion project is usually
divided into three areas: \code{tags}, \code{branches}, and
\code{trunk}. This design has proved very intuitive to users who were
unfamiliar with version control systems, although the flexibility
inherent in this design has caused
numerous problems for conversion tools, mostly because \code{tags} and
\code{branches} have more structural representation in other systems.

All of the aforementioned systems are said to be \emph{centralized};
to the extent that they even know how to exchange changes (starting
with CVS), they rely on some other computer to keep track of the
history of the repository. \emph{Distributed} version control systems
instead keep a copy of all or most of the repository history on each
computer that has a working directory of that repository.

\end{aosasect2}

%% \begin{aosasect2}{Distributed Version Control}
\begin{aosasect2}{分散型バージョン管理}

While Subversion was a clear improvement over CVS, there are still a
number of shortcomings. For one thing, in all centralized systems,
committing a changeset and publishing it are effectively the same
thing, since repository history is centralized in one place. This
means that committing changes without network access is
impossible. Secondly, repository access in centralized systems always
needs one or more network round trips, making it relatively slow compared
to the local accesses needed with distributed systems. Third, the systems
discussed above were not very good at tracking merges (some have since
grown better at it). In large groups working concurrently, it's
important that the version control system records what changes have
been included in some new revision, so that nothing gets lost and
subsequent merges can make use of this information. Fourth, the
centralization required by traditional VCSes sometimes seems
artificial, and promotes a single place for integration. Advocates of
distributed VCSes argue that a more distributed system allows for a
more organic organization, where developers can push around and
integrate changes as the project requires at each point in time.

A number of new tools have been developed to address these needs. From
where I sit (the open source world), the most notable three of these
in 2011 are Git, Mercurial and Bazaar. Both Git and Mercurial were
started in 2005 when the Linux kernel developers decided to no longer
use the proprietary BitKeeper system. Both were started by Linux
kernel developers (Linus Torvalds and Matt Mackall, respectively) to
address the need for a version control system that could handle
hundreds of thousands of changesets in tens of thousands of files (for
example, the kernel). Both Matt and Linus were also heavily influenced
by the Monotone VCS\@. Bazaar was developed separately but gained
widespread usage around the same time, when it was adopted by
Canonical for use with all of their projects.

Building a distributed version control system obviously comes with
some challenges, many of which are inherent in any distributed
system. For one thing, while the source control server in centralized
systems always provided a canonical view of history, there is no such
thing in a distributed VCS\@. Changesets can be committed in parallel,
making it impossible to temporally order revisions in any given
repository.

The solution that has been almost universally adopted is to use a
directed acyclic graph (DAG) of changesets instead of a linear
ordering (\aosafigref{fig.hg.dag}). That is, a newly committed
changeset is the child revision of the revision it was based on, and
no revision can depend on itself or its descendant revisions. In this
scheme, we have three special types of revisions: \emph{root
revisions} which have no parents (a repository can have multiple
roots), \emph{merge revisions} which have more than one parent, and
\emph{head revisions} which have no children. Each repository starts
from an empty root revision and proceeds from there along a line of
changesets, ending up in one or more heads. When two users have
committed independently and one of them wants to pull in the changes
from the other, he or she will have to explicitly merge the other's
changes into a new revision, which he subsequently commits as a merge
revision.

\aosafigure[275pt]{../images/mercurial/dag.eps}{Directed Acyclic Graph of Revisions}{fig.hg.dag}

Note that the DAG model helps solve some of the problems that are hard
to solve in centralized version control systems: merge revisions are used to
record information about newly merged branches of the DAG. The
resulting graph can also usefully represent a large group of parallel
branches, merging into smaller groups, finally merging into one
special branch that's considered canonical.

This approach requires that the system keep track of the ancestry
relations between changesets; to facilitate exchange of changeset data,
this is usually done by having changesets keep track of their
parents. To do this, changesets obviously also need some kind of
identifier. While some systems use a UUID or a similar kind of scheme,
both Git and Mercurial have opted to use SHA1 hashes of the contents
of the changesets. This has the additional useful property that the
changeset ID can be used to verify the changeset contents. In fact,
because the parents are included in the hashed data, all history
leading up to any revision can be verified using its hash. Author
names, commit messages, timestamps and other changeset metadata is
hashed just like the actual file contents of a new revision, so that
they can also be verified. And since timestamps are recorded at commit
time, they too do not necessarily progress linearly in any given
repository.

All of this can be hard for people who have previously only used
centralized VCSes to get used to: there is no nice integer to globally
name a revision, just a 40-character hexadecimal string. Moreover,
there's no longer any global ordering, just a local ordering; the only
global ``ordering'' is a DAG instead of a line. Accidentally starting
a new head of development by committing against a parent revision that
already had another child changeset can be confusing when you're used
to a warning from the VCS when this kind of thing happens.

Luckily, there are tools to help visualize the tree ordering, and
Mercurial provides an unambiguous short version of the changeset hash
\emph{and} a local-only linear number to aid identification. The
latter is a monotonically climbing integer that indicates the order in
which changesets have entered the clone. Since this order can be
different from clone to clone, it cannot be relied on for non-local
operations.

\end{aosasect2}

\end{aosasect1}

%% \begin{aosasect1}{Data Structures}
\begin{aosasect1}{データ構造}

Now that the concept of a DAG should be somewhat clear, let's try
and see how DAGs are stored in Mercurial. The DAG model is central to
the inner workings of Mercurial, and we actually use several different
DAGs in the repository storage on disk (as well as the in-memory
structure of the code). This section explains what they are and how
they fit together.

%% \begin{aosasect2}{Challenges}
\begin{aosasect2}{課題}

Before we dive into actual data structures, I'd like to provide some
context about the environment in which Mercurial evolved. The first
notion of Mercurial can be found in an email Matt Mackall sent to the
Linux Kernel Mailing List on April 20, 2005. This happened shortly
after it was decided that BitKeeper could no longer be used for the
development of the kernel. Matt started his mail by outlining some
goals: to be simple, scalable, and efficient.

In~\cite{bib:mackall:better}, Matt claimed that a modern VCS must deal
with trees containing millions of files, handle millions of
changesets, and scale across many thousands of users creating new
revisions in parallel over a span of decades. Having set the goals,
he reviewed the limiting technology factors:

\begin{aosaitemize}

  \item speed: CPU

  \item capacity: disk and memory

  \item bandwidth: memory, LAN, disk, and WAN

  \item disk seek rate

\end{aosaitemize}

Disk seek rate and WAN bandwidth are the limiting factors today, and
should thus be optimized for. The paper goes on to review common
scenarios or criteria for evaluating the performance of such a system
at the file level:

\begin{aosaitemize}

  \item Storage compression: what kind of compression is best suited
  to save the file history on disk? Effectively, what algorithm makes
  the most out of the I/O performance while preventing CPU time
  from becoming a bottleneck?

  \item Retrieving arbitrary file revisions: a number of version control
  systems will store a given revision in such a way that a large
  number of older revisions must be read to reconstruct the newer one
  (using deltas). We want to control this to make sure that retrieving
  old revisions is still fast.

  \item Adding file revisions: we regularly add new revisions. We don't
  want to rewrite old revisions every time we add a new one, because
  that would become too slow when there are many revisions.

  \item Showing file history: we want to be able to review a history of
  all changesets that touched a certain file. This also allows us to
  do annotations (which used to be called \code{blame} in CVS but was
  renamed to \code{annotate} in some later systems to remove the
  negative connotation): reviewing the originating changeset for each
  line currently in a file.

\end{aosaitemize}

The paper goes on to review similar scenarios at the project level.
Basic operations at this level are checking out a revision, committing
a new revision, and finding differences in the working directory. The
latter, in particular, can be slow for large trees (like those of the
Mozilla or NetBeans projects, both of which use Mercurial for their
version control needs).

\end{aosasect2}

%% \begin{aosasect2}{Fast Revision Storage: Revlogs}
\begin{aosasect2}{高速リビジョンストレージ: Revlog}

The solution Matt came up with for Mercurial is called the \emph{revlog}
(short for revision log). The revlog is a way of efficiently storing
revisions of file contents (each with some amount of changes compared
to the previous version). It needs to be efficient in both access time
(thus optimizing for disk seeks) and storage space, guided by the common
scenarios outlined in the previous section. To do this, a revlog is
really two files on disk: an index and the data file.

\begin{table}[h!]
  \centering
  \begin{tabular}{|rl|}
    \hline
    6 bytes & hunk offset \\
    2 bytes & flags \\
    4 bytes & hunk length \\
    4 bytes & uncompressed length \\
    4 bytes & base revision \\
    4 bytes & link revision \\
    4 bytes & parent 1 revision \\
    4 bytes & parent 2 revision \\
    32 bytes & hash \\
    \hline
  \end{tabular}
  \caption{Mercurial Record Format}
  \label{tbl.hg.records}
\end{table}

The index consists of fixed-length records, whose contents are
detailed in \aosatblref{tbl.hg.records}. Having fixed-length records
is nice, because it means that having the local revision number allows
direct (i.e., constant-time) access to the revision: we can simply
read to the position (index-length~$\times$~revision) in the index
file, to locate the data. Separating the index from the data also
means we can quickly read the index data without having to seek the
disk through all the file data.

The hunk offset and hunk length specify a chunk of the data file to
read in order to get the compressed data for that revision. To get the
original data we have to start by reading the base revision, and apply
deltas through to this revision. The trick here is the decision on
when to store a new base revision. This decision is based on the
cumulative size of the deltas compared to the uncompressed length of
the revision (data is compressed using zlib to use even less space on
disk). By limiting the length of the delta chain in this way, we make
sure that reconstruction of the data in a given revision does not
require reading and applying lots of deltas.

Link revisions are used to have dependent revlogs point back to the
highest-level revlog (we'll talk more about this in a little bit), and
the parent revisions are stored using the local integer revision
number. Again, this makes it easy to look up their data in the
relevant revlog. The hash is used to save the unique identifier for
this changeset. We have 32 bytes instead of the 20 bytes required for
SHA1 in order to allow future expansion.

\end{aosasect2}

%% \begin{aosasect2}{The Three Revlogs}
\begin{aosasect2}{三つのRevlog}

With the revlog providing a generic structure for historic data, we
can layer the data model for our file tree on top of that. It consists
of three types of revlogs: the \emph{changelog}, \emph{manifests}, and
\emph{filelogs}. The changelog contains metadata for each revision,
with a pointer into the manifest revlog (that is, a node id for one
revision in the manifest revlog). In turn, the manifest is a file that
has a list of filenames plus the node id for each file, pointing to a
revision in that file's filelog. In the code, we have classes for
changelog, manifest, and filelog that are subclasses of the generic
revlog class, providing a clean layering of both concepts.

\aosafigure[275pt]{../images/mercurial/revlogs.eps}{Log Structure}{fig.hg.revlog}

A changelog revision looks like this:

\begin{verbatim}
0a773e3480fe58d62dcc67bd9f7380d6403e26fa
Dirkjan Ochtman <dirkjan@ochtman.nl>
1276097267 -7200
mercurial/discovery.py
discovery: fix description line
\end{verbatim}

\noindent This is the value you get from the revlog layer; the changelog layer
turns it into a simple list of values. The initial line provides the
manifest hash, then we get author name, date and time (in the form of
a Unix timestamp and a timezone offset), a list of affected files, and
the description message. One thing is hidden here: we allow arbitrary
metadata in the changelog, and to stay backwards compatible we added
those bits to go after the timestamp.

Next comes the manifest:

\begin{verbatim}
.hgignore\x006d2dc16e96ab48b2fcca44f7e9f4b8c3289cb701
.hgsigs\x00de81f258b33189c609d299fd605e6c72182d7359
.hgtags\x00b174a4a4813ddd89c1d2f88878e05acc58263efa
CONTRIBUTORS\x007c8afb9501740a450c549b4b1f002c803c45193a
COPYING\x005ac863e17c7035f1d11828d848fb2ca450d89794
...
\end{verbatim}

\noindent This is the manifest revision that changeset 0a773e points to
(Mercurial's UI allows us to shorten the identifier to any unambiguous
prefix). It is a simple list of all files in the tree, one per line,
where the filename is followed by a NULL byte, followed by the
hex-encoded node id that points into the file's filelog. Directories
in the tree are not represented separately, but simply inferred from
including slashes in the file paths. Remember that the manifest is
diffed in storage just like every revlog, so this structure should
make it easy for the revlog layer to store only changed files and
their new hashes in any given revision. The manifest is usually
represented as a hashtable-like structure in Mercurial's Python code,
with filenames as keys and nodes as values.

The third type of revlog is the filelog. Filelogs are stored in Mercurial's internal \code{store} directory,
where they're named almost exactly like the file they're tracking. The
names are encoded a little bit to make sure things work across all
major operating systems. For example, we have to deal with casefolding
filesystems on Windows and Mac~OS~X, specific disallowed filenames on
Windows, and different character encodings as used by the various
filesystems. As you can imagine, doing this reliably across operating
systems can be fairly painful. The contents of a filelog revision, on
the other hand, aren't nearly as interesting: just the file contents,
except with some optional metadata prefix (which we use for tracking
file copies and renames, among other minor things).

This data model gives us complete access to the data store in a
Mercurial repository, but it's not always very convenient. While the
actual underlying model is vertically oriented (one filelog per file),
Mercurial developers often found themselves wanting to deal with all
details from a single revision, where they start from a changeset from
the changelog and want easy access to the manifest and filelogs from
that revision.  They later invented another set of classes, layered
cleanly on top of the revlogs, which do exactly that. These are
called \code{contexts}.

One nice thing about the way the separate revlogs are set up is the
ordering. By ordering appends so that filelogs get appended to first,
then the manifest, and finally the changelog, the repository is always
in a consistent state. Any process that starts reading the changelog
can be sure all pointers into the other revlogs are valid, which takes
care of a number of issues in this department. Nevertheless, Mercurial
also has some explicit locks to make sure there are no two processes
appending to the revlogs in parallel.

\end{aosasect2}

%% \begin{aosasect2}{The Working Directory}
\begin{aosasect2}{作業ディレクトリ}

A final important data structure is what we call the
\emph{dirstate}. The dirstate is a representation of what's in the
working directory at any given point. Most importantly, it keeps track
of what revision has been checked out: this is the baseline for all
comparisons from the \code{status} or \code{diff} commands, and also
determines the parent(s) for the next changeset to be committed. The
dirstate will have two parents set whenever the \code{merge} command
has been issued, trying to merge one set of changes into the other.

Because \code{status} and \code{diff} are very common operations (they
help you check the progress of what you've currently got against the
last changeset), the dirstate also contains a cache of the state of
the working directory the last time it was traversed by Mercurial.
Keeping track of last modified timestamps and file sizes makes it
possible to speed up tree traversal. We also need to keep track of the
state of the file: whether it's been added, removed, or merged in the
working directory. This will again help speed up traversing the
working directory, and makes it easy to get this information at commit
time.

\end{aosasect2}

\end{aosasect1}

%% \begin{aosasect1}{Versioning Mechanics}
\begin{aosasect1}{バージョン管理の仕組み}

Now that you are familiar with the underlying data model and the
structure of the code at the lower levels of Mercurial, let's move up
a little bit and consider how Mercurial implements version control
concepts on top of the foundation described in the previous section.

%% \begin{aosasect2}{Branches}
\begin{aosasect2}{ブランチ}

Branches are commonly used to separate different lines of development
that will be integrated later. This might be because someone is
experimenting with a new approach, just to be able to always keep the
main line of development in a shippable state (feature branches), or
to be able to quickly release fixes for an old release (maintenance
branches). Both approaches are commonly used, and are supported by all
modern version control systems. While implicit branches are common in
DAG-based version control named branches (where the branch name is
saved in the changeset metadata) are not as common.

Originally, Mercurial had no way to explicitly name branches. Branches
were instead handled by making different clones and publishing them
separately. This is effective, easy to understand, and especially
useful for feature branches, because there is little
overhead. However, in large projects, clones can still be quite
expensive: while the repository store will be hardlinked on most
filesystems, creating a separate working tree is slow and may require
a lot of disk space.

Because of these downsides, Mercurial added a second way to do
branches: including a branch name in the changeset metadata. A
\code{branch} command was added that can set the branch name for the
current working directory, such that that branch name will be used for
the next commit. The normal \code{update} command can be used to
update to a branch name, and a changeset committed on a branch will
always be related to that branch. This approach is called \emph{named
branches}. However, it took a few more Mercurial releases before
Mercurial started including a way to close these branches up
again (closing a branch will hide the branch from view in a list of
branches). Branch closing is implemented by adding an extra field in the changeset
metadata, stating that this changeset closes the branch. If the branch
has more than one head, all of them have to be closed before the
branch disappears from the list of branches in the repository.

Of course, there's more than one way to do it.
Git has a different way of naming branches, using
references. References are names pointing to another object in the
Git history, usually a changeset. This means that Git's branches are
ephemeral: once you remove the reference, there is no trace of the
branch ever having existed, similar to what you would get when using a
separate Mercurial clone and merging it back into another clone. This
makes it very easy and lightweight to manipulate branches locally, and
prevents cluttering of the list of branches.

This way of branching turned out to be very popular, much more popular
than either named branches or branch clones in Mercurial. This has
resulted in the \code{bookmarks}q extension, which will probably be
folded into Mercurial in the future. It uses a simple unversioned file
to keep track of references. The wire protocol used to exchange
Mercurial data has been extended to enable communicating about
bookmarks, making it possible to push them around.

\end{aosasect2}

%% \begin{aosasect2}{Tags}
\begin{aosasect2}{タグ}

At first sight, the way Mercurial implements tags can be a bit
confusing. The first time you add a tag (using the \code{tag}
command), a file called \code{.hgtags} gets added to the repository and
committed. Each line in that file will contain a changeset node id and
the tag name for that changeset node. Thus, the tags file is treated
the same way as any other file in the repository.

There are three important reasons for this. First, it must be possible
to change tags; mistakes do happen, and it should be possible to fix
them or delete the mistake. Second, tags should be part of changeset
history: it's valuable to see when a tag was made, by whom, and for
what reason, or even if a tag was changed. Third, it should be
possible to tag a changeset retroactively. For example, some projects
extensively test drive a release artifact exported from the version
control system before releasing it.

These properties all fall easily out of the \code{.hgtags} design. While some
users are confused by the presence of the \code{.hgtags} file in their
working directories, it makes integration of the tagging mechanism
with other parts of Mercurial (for example, synchronization with other
repository clones) very simple. If tags existed outside the source
tree (as they do in Git, for example), separate mechanisms would have
to exist to audit the origin of tags and to deal with conflicts from
(parallel) duplicate tags. Even if the latter is rare, it's nice to
have a design where these things are not even an issue.

To get all of this right, Mercurial only ever appends new lines to the
\code{.hgtags} file. This also facilitates merging the file if the
tags were created in parallel in different clones. The newest node id
for any given tag always takes precedence, and adding the null node id
(representing the empty root revision all repositories have in common)
will have the effect of deleting the tag. Mercurial will also consider
tags from all branches in the repository, using recency calculations
to determine precedence among them.

\end{aosasect2}

\end{aosasect1}

%% \begin{aosasect1}{General Structure}
\begin{aosasect1}{構造の概要}

Mercurial is almost completely written in Python, with only a few bits
and pieces in C because they are critical to the performance of the
whole application. Python was deemed a more suitable choice for most
of the code because it is much easier to express high-level concepts
in a dynamic language like Python. Since much of the code is not
really critical to performance, we don't mind taking the hit in
exchange for making the coding easier for ourselves in most parts.

A Python module corresponds to a single file of code. Modules can
contain as much code as needed, and are thus an important way to
organize code. Modules may use types or call functions from other
modules by explicitly importing the other modules. A directory
containing an \code{\_\_init\_\_.py} module is said to be a package,
and will expose all contained modules and packages to the Python
importer.

Mercurial by default installs two packages into the Python path:
\code{mercurial} and \code{hgext}. The \code{mercurial} package
contains the core code required to run Mercurial, while \code{hgext}
contains a number of extensions that were deemed useful enough to be
delivered alongside the core. However, they must still be enabled by
hand in a configuration file if desired (which we will discuss later.)

To be clear, Mercurial is a command-line application. This means that
we have a simple interface: the user calls the \code{hg} script with a
command. This command (like \code{log}, \code{diff} or \code{commit})
may take a number of options and arguments; there are also some
options that are valid for all commands. Next, there are three
different things that can happen to the interface.

\begin{aosaitemize}

  \item \code{hg} will often output something the user asked for or show
  status messages

  \item \code{hg} can ask for further input through command-line prompts

  \item \code{hg} may launch an external program (such as an editor for the
  commit message or a program to help merging code conflicts)

\end{aosaitemize}

\aosafigure{../images/mercurial/modules.eps}{Import Graph}{fig.hg.importgraph}

The start of this process can neatly be observed from the import graph
in \aosafigref{fig.hg.importgraph}. All command-line arguments are
passed to a function in the dispatch module. The first thing that
happens is that a \code{ui} object is instantiated. The \code{ui}
class will first try to find configuration files in a number of
well-known places (such as your home directory), and save the
configuration options in the \code{ui} object. The configuration files
may also contain paths to extensions, which must also be loaded at
this point. Any global options passed on the command-line are also
saved to the \code{ui} object at this point.

After this is done, we have to decide whether to create a repository
object. While most commands require a local repository (represented
by the \code{localrepo} class from the \code{localrepo} module), some
commands may work on remote repositories (either HTTP, SSH, or some
other registered form), while some commands can do their work without
referring to any repository. The latter category includes the
\code{init} command, for example, which is used to initialize a new
repository.

All core commands are represented by a single function in the
\code{commands} module; this makes it really easy to find the code for
any given command. The commands module also contains a hashtable that
maps the command name to the function and describes the options that
it takes. The way this is done also allows for sharing common sets of
options (for example, many commands have options that look like the
ones the \code{log} command uses). The options description allows the
dispatch module to check the given options for any command, and to
convert any values passed in to the type expected by the command
function. Almost every function also gets the \code{ui} object and the
\code{repository} object to work with.

\end{aosasect1}

%% \begin{aosasect1}{Extensibility}
\begin{aosasect1}{拡張性}

One of the things that makes Mercurial powerful is the ability to
write extensions for it. Since Python is a relatively easy language to
get started with, and Mercurial's API is mostly quite well-designed
(although certainly under-documented in places), a number of people
actually first learned Python because they wanted to extend Mercurial.

%% \begin{aosasect2}{Writing Extensions}
\begin{aosasect2}{拡張機能の作成}

Extensions must be enabled by adding a line to one of the
configuration files read by Mercurial on startup; a key is provided
along with the path to any Python module. There are several ways to
add functionality:

\begin{aosaitemize}

  \item adding new commands;

  \item wrapping existing commands;

  \item wrapping the repository used;

  \item wrap any function in Mercurial; and

  \item add new repository types.

\end{aosaitemize}

Adding new commands can be done simply by adding a hashtable called
\code{cmdtable} to the extension module. This will get picked up by
the extension loader, which will add it to the commands table
considered when a command is dispatched. Similarly, extensions can
define functions called \code{uisetup} and \code{reposetup} which are
called by the dispatching code after the UI and repository have been
instantiated. One common behavior is to use a \code{reposetup}
function to wrap the repository in a repository subclass provided by
the extension. This allows the extension to modify all kinds of basic
behavior. For example, one extension I have written hooks into the
uisetup and sets the \code{ui.username} configuration property based on the
SSH authentication details available from the environment.

More extreme extensions can be written to add repository types. For
example, the \code{hgsubversion} project (not included as part of
Mercurial) registers a repository type for Subversion
repositories. This makes it possible to clone from a Subversion
repository almost as if it were a Mercurial repository. It's even
possible to push back to a Subversion repository, although there are a
number of edge cases because of the impedance mismatch between the two
systems. The user interface, on the other hand, is completely
transparent.

For those who want to fundamentally change Mercurial, there is
something commonly called ``monkeypatching'' in the world of dynamic
languages. Because extension code runs in the same address space as
Mercurial, and Python is a fairly flexible language with extensive
reflection capabilities, it's possible (and even quite easy) to modify
any function or class defined in Mercurial. While this can result in
kind of ugly hacks, it's also a very powerful mechanism. For example,
the \code{highlight} extension that lives in \code{hgext} modifies the
built-in webserver to add syntax highlighting to pages in the
repository browser that allow you to inspect file contents.

There's one more way to extend Mercurial, which is much simpler:
\emph{aliases}. Any configuration file can define an alias as a new
name for an existing command with a specific group of options already
set. This also makes it possible to give shorter names to any
commands. Recent versions of Mercurial also include the ability to
call a shell command as an alias, so that you can design complicated
commands using nothing but shell scripting.

\end{aosasect2}

%% \begin{aosasect2}{Hooks}
\begin{aosasect2}{フック}

Version control systems have long provided hooks as a way for VCS
events to interact with the outside world. Common usage includes
sending off a notification to a continuous integration
system\footnote{See \aosachapref{s:integration}.} or updating the
working directory on a web server so that changes become
world-visible. Of course, Mercurial also includes a subsystem to
invoke hooks like this.

In fact, it again contains two variants. One is more like traditional
hooks in other version control systems, in that it invokes scripts in
the shell. The other is more interesting, because it allows users to
invoke Python hooks by specifying a Python module and a function name
to call from that module. Not only is this faster because it runs in
the same process, but it also hands off \code{repo} and \code{ui} objects, meaning
you can easily initiate more complex interactions inside the VCS.

Hooks in Mercurial can be divided in to pre-command, post-command,
controlling, and miscellaneous hooks. The first two are trivially
defined for any command by specifying a pre-\emph{command} or
post-\emph{command} key in the hooks section of a configuration
file. For the other two types, there's a predefined set of events. The
difference in controlling hooks is that they are run right before
something happens, and may not allow that event to progress further.
This is commonly used to validate changesets in some way on a central
server; because of Mercurial's distributed nature, no such checks can
be enforced at commit time. For example, the Python project uses a
hook to make sure some aspects of coding style are enforced throughout
the code base---if a changeset adds code in a style that is not
allowed, it will be rejected by the central repository.

Another interesting use of hooks is a pushlog, which is used by
Mozilla and a number of corporate organizations. A pushlog
records each push (since a push may contain any number of
changesets) and records who initiated that push and when, providing
a kind of audit trail for the repository.

\end{aosasect2}

\end{aosasect1}

%% \begin{aosasect1}{Lessons Learned}
\begin{aosasect1}{学んだこと}

One of the first decisions Matt made when he started to develop
Mercurial was to develop it in Python. Python has been great for the
extensibility (through extensions and hooks) and is very easy to code
in. It also takes a lot of the work out of being compatible across
different platforms, making it relatively easy for Mercurial to work
well across the three major OSes. On the other hand, Python is slow
compared to many other (compiled) languages; in particular,
interpreter startup is relatively slow, which is particularly bad for
tools that have many shorter invocations (such as a VCS) rather than
longer running processes.

An early choice was made to make it hard to modify changesets after
committing. Because it's impossible to change a revision without
modifying its identity hash, ``recalling'' changesets after having
published them on the public Internet is a pain, and Mercurial makes
it hard to do so. However, changing unpublished revisions should
usually be fine, and the community has been trying to make this easier
since soon after the release. There are extensions that try to solve
the problem, but they require learning steps that are not very
intuitive to users who have previously used basic Mercurial.

Revlogs are good at reducing disk seeks, and the layered architecture
of changelog, manifest and filelogs has worked very well. Committing
is fast and relatively little disk space is used for revisions.
However, some cases like file renames aren't very efficient due to
the separate storage of revisions for each file; this will eventually
be fixed, but it will require a somewhat hacky layering violation.
Similarly, the per-file DAG used to help guide filelog storage isn't
used a lot in practice, such that some code used to administrate
that data could be considered to be overhead.

Another core focus of Mercurial has been to make it easy to learn.  We
try to provide most of the required functionality in a small set of
core commands, with options consistent across commands. The intention
is that Mercurial can mostly be learned progressively, especially for
those users who have used another VCS before; this philosophy extends
to the idea that extensions can be used to customize Mercurial even
more for a particular use case. For this reason, the developers also
tried to keep the UI in line with other VCSs, Subversion in
particular.  Similarly, the team has tried to provide good
documentation, available from the application itself, with
cross-references to other help topics and commands. We try hard to
provide useful error messages, including hints of what to try instead
of the operation that failed.

Some smaller choices made can be surprising to new users. For example,
handling tags (as discussed in a previous section) by putting them in
a separate file inside the working directory is something many users
dislike at first, but the mechanism has some very desirable properties
(though it certainly has its shortcomings as well). Similarly, other
VCSs have opted to send only the checked out changeset and any
ancestors to a remote host by default, whereas Mercurial sends every
committed changeset the remote doesn't have. Both approaches make some
amount of sense, and it depends on the style of development which one
is the best for you.

As in any software project, there are a lot of trade-offs to be made.
I think Mercurial made good choices, though of course with the benefit
of 20/20 hindsight some other choices might have been more appropriate.
Historically, Mercurial seems to be part of a first generation of
distributed version control systems mature enough to be ready for
general use. I, for one, am looking forward to seeing what the next
generation will look like.

\end{aosasect1}

\end{aosachapter}
