%% \begin{aosachapter}{Continuous Integration}{s:integration}{C.\ Titus Brown and Rosangela Canino-Koning}
\begin{aosachapter}{継続的インテグレーション}{s:integration}{C.\ Titus Brown and Rosangela Canino-Koning}
%% Based on EN-Revision r229

%% Continuous Integration (CI) systems are systems that build and test
%% software automatically and regularly.  Though their primary benefit
%% lies in avoiding long periods between build and test runs, CI systems
%% can also simplify and automate the execution of many otherwise tedious
%% tasks. These include cross-platform testing, the regular running of slow,
%% data-intensive, or difficult-to-configure tests, verification of
%% proper performance on legacy platforms, detection of infrequently
%% failing tests, and the regular production of up-to-date release
%% products. And, because build and test automation is necessary for
%% implementing continuous integration, CI is often a first step towards
%% a \emph{continuous deployment} framework wherein software updates can be
%% deployed quickly to live systems after testing.
継続的インテグレーション(Continuous Integration: CI)システムとは、ソフトウェアのビルドやテストを自動的かつ定期的に行うシステムのことである。CIシステムを使う最大のメリットは、ビルドやテストの実行間隔が長くなるのを避けられるということだ。それ以外にも、その他の退屈な作業を単純化して自動化させることもできる。たとえば、クロスプラットフォームでのテスト実行、遅かったりデータを扱ったり設定の難しかったりするテストの定期実行、レガシーな環境での適切なパフォーマンスの確保、ごくまれに失敗するテストの検出、そしてリリースする製品の定期的な作成などといった作業がそれにあたる。また、ビルドやテストの自動化は継続的インテグレーションのために必須となるので、CIは\emph{継続的デプロイ}用フレームワークに向けた第一歩にもなる。これは、ソフトウェアを更新したら、テストをしてすぐに稼働中のシステムへ展開できるようにする仕組みである。

%% Continuous integration is a timely subject, not least because of its
%% prominence in the Agile software methodology.  There has been an
%% explosion of open source CI tools in recent years, in and for a
%% variety of languages, implementing a huge range of features in the
%% context of a diverse set of architectural models.  The purpose of this
%% chapter is to describe common sets of features implemented in
%% continuous integration systems, discuss the architectural options
%% available, and examine which features may or may not be easy to
%% implement given the choice of architecture.
昨今アジャイルソフトウェア方法論が広まってきたこともあり、継続的インテグレーションはタイムリーな話題である。オープンソースのCIツールもここ数年急増し、さまざまな言語向けにさまざまな言語で書かれている。そして、さまざまなアーキテクチャモデルに対応した幅広い機能が実装されている。本章では継続的インテグレーションシステムが実装する一般的な機能群について説明し、アーキテクチャに関する選択肢について議論する。そして、選んだアーキテクチャごとにどの機能が実装しやすくてどの機能が実装しづらいのかを検討する。
 
%% Below, we will briefly describe a set of systems that exemplify the
%% extremes of architectural choices available when designing a CI
%% system. The first, Buildbot, is a master/slave system; the second,
%% CDash is a reporting server model; the third Jenkins, uses a hybrid model; and the fourth, Pony-Build, 
%% is a Python-based decentralized reporting server that we will use
%% as a foil for further discussion.
これ以降では、CIシステムを設計する際に選択可能なアーキテクチャのよい例となるシステム群について簡単に説明する。最初に取り上げるBuildbotはマスター/スレーブ型のシステムだ。それに続くCDashはレポートサーバー型、Jenkinsはハイブリッド型、そして最後のPony-BuildはPythonベースの分散型レポートサーバーで、これを使ってさらに議論を深めていく。

%% \begin{aosasect1}{The Landscape}
\begin{aosasect1}{概観}

%% The space of architectures for continuous integration systems seems to
%% be dominated by two extremes: master/slave architectures, in which a
%% central server directs and controls remote builds; and reporting
%% architectures, in which a central server aggregates build reports
%% contributed by clients. All of the continuous integration systems of
%% which we are aware have chosen some combination of features from these
%% two architectures.
継続的インテグレーションシステムのアーキテクチャの世界は二大勢力に支配されているようだ。一方はマスター/スレーブ型のアーキテクチャで、中央のサーバーがリモートのビルドを指揮して制御する。その対極にあるのがレポーティングアーキテクチャで、各クライアントからのレポートを中央のサーバーが集約する。我々の知る範囲では、すべての継続的インテグレーションシステムはこの二種類のアーキテクチャの機能を組み合わせて使っている。

%% Our example of a centralized architecture, Buildbot, is composed of
%% two parts: the central server, or \emph{buildmaster}, which schedules and
%% coordinates builds between one or more connected clients; and the
%% clients, or \emph{buildslaves}, which execute builds. The buildmaster
%% provides a central location to which to connect, along with
%% configuration information about which clients should execute which
%% commands in what order. Buildslaves connect to the buildmaster and
%% receive detailed instructions. Buildslave configuration consists of
%% installing the software, identifying the master server, and providing
%% connection credentials for the client to connect to the master. Builds
%% are scheduled by the buildmaster, and output is streamed from the
%% buildslaves to the buildmaster and kept on the master server for
%% presentation via the Web and other reporting and notification systems.
中央集権型のアーキテクチャの例として取り上げるBuildbotは、ふたつのパーツで構成されている。中央サーバーである\emph{buildmaster}がそこに接続しているクライアントのビルドスケジュールを管理し、クライアント側の\emph{buildslaves}が実際のビルドを行う。buildmasterはクライアントからの接続先となり、各クライアントがどのコマンドをどの順で実行するのかという設定情報を提供する。buildslaveはbuildmasterに接続詞、詳細な指示を受け取る。buildslaveの設定に含まれるのは、ソフトウェアをインストールすることやマスターサーバーの識別、そしてマスターサーバーに接続するための認証情報などである。ビルド予定をたてるのはbuildmasterで、その出力はbuildslaveからbuildmasterに流される。結果はマスターサーバー上に保持され、ウェブ経由で見たり別のレポートシステムや通知システムで見たりすることができる。

%% On the opposite side of the architecture spectrum lies CDash, which is
%% used for the Visualization Toolkit (VTK)/Insight Toolkit (ITK)
%% projects by Kitware, Inc. CDash is essentially a reporting server,
%% designed to store and present information received from client
%% computers running CMake and CTest. With CDash, the clients initiate
%% the build and test suite, record build and test results, and then
%% connect to the CDash server to deposit the information for central
%% reporting.
アーキテクチャ的にその対極にあるのがCDashで、これはKitware, Inc.のVisualization Toolkit (VTK)/Insight Toolkit (ITK)プロジェクトで使われている。CDashは本質的にレポーティングサーバーで、CMakeおよびCTestを実行するクライアントコンピューターから受け取った情報を蓄積して表示するように作られている。CDashでは、クライアント側がビルドとテストスイートを起動し、ビルドとテストの結果を記録し、それからCDashサーバーに接続して情報をレポーティングサーバーに預ける。

%% Finally, a third system, Jenkins (known as Hudson before a name
%% change in 2011), provides both modes of operation. With Jenkins,
%% builds can either be executed independently with the results sent to
%% the master server; or nodes can be slaved to the Jenkins master
%% server, which then schedules and directs the execution of builds.
最後に、三番目の例として取り上げるJenkins (かつてはHudsonと呼ばれていたが2011年に名前が変わった)は、その両方の操作モードを提供している。Jenkinsの場合、ビルドを個別に実行して結果をマスターサーバーに送ることもできるし、ノードをすべてJenkinsマスターサーバーの支配下においてビルドの予定やその実行をマスターサーバーから指示することもできる。

%% Both the centralized and decentralized models have some features in
%% common, and, as Jenkins shows, both models can co-exist in a single
%% implementation. However, Buildbot and CDash exist in stark contrast to
%% each other: apart from the commonalities of building software and
%% reporting on the builds, essentially every other aspect of the
%% architecture is different. Why?
中央集権型モデルと分散型モデルの両方に共通する機能もあり、Jenkinsを見てもわかるとおり、両方のモデルをひとつの実装に共存させることもできる。しかしBuildbotとCDashはお互い全く正反対の存在である。ソフトウェアをビルドしてその結果を報告するという点は共通しているが、それ以外の面では全く異なるアーキテクチャを採用している。なぜだろう?

%% Further, to what extent does the choice of architecture seem to make
%% certain features easier or harder to implement? Do some features
%% emerge naturally from a centralized model? And how extensible are the
%% existing implementations---can they easily be modified to provide new
%% reporting mechanisms, or scale to many packages, or execute builds and
%% tests in a cloud environment?
アーキテクチャの選択によって、特定の機能の実装しやすさ(しにくさ)はどの程度の影響を受けるのだろう? 中央集権型を採用することで必然的に出てくる機能などがあるのだろうか? 既存の実装の拡張しやすさについてはどうだろう---レポーティングの仕組みに手軽に手を入れたり、多数のパッケージを扱うために規模を拡大したり、あるいはビルドやテストをクラウド環境で実行したりといったことはできるのだろうか?

%% \begin{aosasect2}{What Does Continuous Integration Software Do?}
\begin{aosasect2}{継続的インテグレーションソフトウェアの役割は?}

%% The core functionality of a continuous integration system is simple:
%% build software, run tests, and report the results. The build, test,
%% and reporting can be performed by a script running from a scheduled
%% task or cron job: such a script would just check out a new copy of the
%% source code from the VCS, do a build, and then run the tests. Output
%% would be logged to a file, and either stored in a canonical location
%% or sent out via e-mail in case of a build failure. This is simple to
%% implement: in UNIX, for example, this entire process can be
%% implemented for most Python packages in a seven line script:
継続的インテグレーションシステムの中核となる機能は単純だ。ソフトウェアをビルドしてテストを実行し、その結果を報告するだけである。ビルドやテストそして結果報告はスクリプトで行える。これは、スケジュールを組み込んだタスクやcronジョブとして実行する。スクリプトの仕事は、ソースコードの新たなコピーをVCSから取得してビルドし、そしてテストを実行することだ。出力はログファイルに書き込むことになるだろう。ファイルを所定の場所に保存し、ビルドが失敗したときにはメールを送信することになる。この機能を実装するのは簡単だ。UNIXなら、大半のPythonパッケージについてたった7行のスクリプトでこの機能を実現できる。

\begin{verbatim}
cd /tmp && \
svn checkout http://some.project.url && \
cd project_directory && \
python setup.py build && \
python setup.py test || \
echo build failed | sendmail notification@project.domain
cd /tmp && rm -fr project_directory
\end{verbatim}

%% In \aosafigref{fig.integration.internal}, the unshaded rectangles
%% represent discrete subsystems and functionality within the system.
%% Arrows show information flow between the various components.  The
%% cloud represents potential remote execution of build processes.  The
%% shaded rectangles represent potential coupling between the subsystems;
%% for example, build monitoring may include monitoring of the build
%% process itself and aspects of system health (CPU load, I/O load,
%% memory usage, etc.)
\aosafigref{fig.integration.internal}において影付きでない長方形は、システム内にある個別のサブシステムや機能をを表す。矢印は、コンポーネント間の情報の流れを意味する。雲で囲まれている部分は、おそらくリモートで実行されるであろうビルドプロセスを表す。影付きの長方形は、サブシステム間のつながりを表す。たとえば、ビルドの監視にはビルドプロセス自体の監視とシステムの健康状態(CPUの負荷、入出力の負荷、メモリの使用量など)の監視が含まれる。

%% \aosafigureTop{../images/integration/ci-internal.eps}{Internals of a Continuous Integration System}{fig.integration.internal}
\aosafigureTop{../images/integration/ci-internal.eps}{継続的インテグレーションシステムの内部構造}{fig.integration.internal}

%% But this simplicity is deceptive. Real-world CI systems usually
%% do much more. In addition to initiating or receiving the results of
%% remote build processes, continuous integration software may support
%% any of the following additional features:
しかし、単純そうに見えるのは見かけだけである。実際のCIシステムは、通常はこれ以上のことを行っている。リモートのビルドプロセスを立ち上げてその結果を受け取ったりするだけでなく、継続的インテグレーションソフトウェアはこのような追加機能に対応していることもある。

\begin{aosadescription}

  %% \item{Checkout and update:} For large projects, checking out a
  %% new copy of the source code can be costly in terms of bandwidth
  %% and time. Usually, CI systems update an existing working copy in
  %% place, which communicates only the differences from the previous
  %% update. In exchange for this savings, the system must keep track
  %% of the working copy and know how to update it, which usually means
  %% at least minimal integration with a VCS.
  \item{チェックアウトと更新:} 大規模なプロジェクトでは、ソースコードすべてを新たにチェックアウトするのは帯域的にも時間的にもコストがかかることになる。通常、CIシステムは既存の作業コピーをその場で更新することになる。更新の際にやりとりするのは、前回の更新以降の差分だけである。通信量は節約できるが、その代わりにシステム側で作業コピーの状況をわかっていなければならない、更新方法も知る必要がある。つまり、通常は少なくともVCSとは最小限の統合をすることになる。

  %% \item{Abstract build recipes:} A configure/build/test recipe
  %% must be written for the software in question. The underlying
  %% commands will often be different for different operating systems,
  %% e.g. Mac OS X vs. Windows vs. UNIX, which means either specialized
  %% recipes need to be written (introducing potential bugs or
  %% disconnection from the actual build environment) or some suitable
  %% level of abstraction for recipes must be provided by the CI
  %% configuration system.
  \item{ビルドレシピの抽象化:} 設定やビルドそしてテストのレシピは、対象となるソフトウェア用に書かなければならない。もとになるコマンドは(Mac OS XとWindowsとUNIXなど)OSによって異なることが多い。ということは、それぞれのOSに特化したレシピを書く(これはバグのもとになるし、実際のビルド環境とはかけ離れてしまう可能性もある)か、さもなければ何らかの抽象化をしてレシピをCI構成システムから提供できるようにしなければならない。

  %% \item{Storing checkout/build/test status:} It may be desirable
  %% for details of the checkout (files updated, code version), build
  %% (warnings or errors) and test (code coverage, performance, memory
  %% usage) to be stored and used for later analysis. These results can
  %% be used to answer questions across the build architectures (did
  %% the latest check-in significantly affect performance on any
  %% particular architecture?) or over history (has code coverage
  %% changed dramatically in the last month?) As with the build recipe,
  %% the mechanisms and data types for this kind of introspection are
  %% usually specific to the platform or build system.
  \item{チェックアウト/ビルド/テスト の状態の保存:} チェックアウトの詳細(更新されたファイル、コードのバージョンなど)やビルドの情報(警告やエラー)、そしてテストの結果(コードカバレッジ、パフォーマンス、メモリの使用量)などを保存し、あとで解析に使えるようにしたいという要望もあるだろう。これらの結果を使えば、ビルドアーキテクチャをまたがる質問(最新のチェックインのせいで特定のアーキテクチャのパフォーマンスに問題が出ていないか?)や歴史を超えた質問(コードカバレッジは先月に比べて劇的に上昇したか?)にも答えられるようになる。ビルドレシピと同様、この種の調査の仕組みやデータ形式は、プラットフォームやビルドシステムに依存するものとなる。

  %% \item{Release packages:} Builds may produce binary packages or
  %% other products that need to be made externally available. For
  %% example, developers who don't have direct access to the build
  %% machine may want to test the latest build in a specific
  %% architecture; to support this, the CI system needs to be able to
  %% transfer build products to a central repository.
  \item{パッケージのリリース:} ビルドを実行するとバイナリパッケージあるいはその他外部に公開する必要のある何かができあがるかもしれない。たとえば、ビルドマシンに直接アクセスできない開発者が、最新のビルドを特定のアーキテクチャでテストしたくなることもあるだろう。これをサポートするためには、CIシステムがビルドの成果物を中央リポジトリに転送できるようにしておく必要がある。
  
  %% \item{Multiple architecture builds:} Since one goal
  %% of continuous integration is to build on multiple architectures to
  %% test cross-platform functionality, the CI software may need to
  %% track the architecture for each build machine and link
  %% builds and build outcomes to each client.
  \item{複数のアーキテクチャでのビルド:} 継続的インテグレーションの目的のひとつは複数のアーキテクチャでビルドしてクロスプラットフォームな機能をテストすることなので、CIソフトウェアは各ビルドマシンのアーキテクチャを追跡してビルドやビルド結果を各クライアントにリンクしなければならない。

  %% \item{Resource management:} If a build step is resource
  %% intensive on a single machine, the CI system may want to run
  %% conditionally. For example, builds may wait for the absence of
  %% other builds or users, or delay until a particular CPU or memory
  %% load is reached.
  \item{リソース管理:} ビルドの手順が特定のマシンのリソースの状況に依存する場合、CIシステムはビルドを条件付きで実行させたくなることもある。たとえば、他のビルドやユーザーがいない間はビルドを待ったり、CPUやメモリの使用率が一定に達したらビルドを遅らせたりということがあり得る。

  %% \item{External resource coordination:} Integration tests may
  %% depend on non-local resources such as a staging database or a
  %% remote web service. The CI system may therefore need to coordinate
  %% builds between multiple machines to organize access to these
  %% resources.
  \item{外部リソースとの協調:} インテグレーションテストはローカルにないリソースに依存することがある。ステージング環境のデータベースやリモートウェブサービスなどだ。したがって、CIシステムは複数のマシン間で協調し、これらのリソースへのアクセスを整理する必要がある。

  %% \item{Progress reports:} For long build processes, regular
  %% reporting from the build may also be important. If a user is
  %% primarily interested in the results of the first 30 minutes of a 5
  %% hour build and test, then it would be a waste of time to make them
  %% wait until the end of a run to see any results.
  \item{進捗レポート:} 時間がかかるビルド手順については、ビルド状況の定期的な報告も大切である。5時間におよぶビルドやテストの中で主に知りたいのは最初の30分の結果だったとしよう。最後まで実行しないと何も結果を見られないのは時間の無駄になる。

\end{aosadescription}

%% A high-level view of all of these potential components of a CI system
%% is shown in \aosafigref{fig.integration.internal}. CI software usually
%% implements some subset of these components.
CIシステムで必要となりそうな全コンポーネントの概要は\aosafigref{fig.integration.internal}に示したとおりだ。CIソフトウェアは通常、これらのコンポーネントの一部を実装している。

\end{aosasect2}

%% \begin{aosasect2}{External Interactions}
\begin{aosasect2}{外部とのインタラクション}

%% Continuous integration systems also need to interact with other
%% systems. There are several types of potential interactions:
継続的インテグレーションシステムでは、他のシステムとのやりとりも必要となる。考えうるやりとりには、次のような型がある。

\begin{aosadescription}

  %% \item{Build notification}: The outcomes of builds generally
  %% need to be communicated to interested clients, either via pull
  %% (Web, RSS, RPC, etc.) or push notification (e-mail, Twitter,
  %% PubSubHubbub, etc.) This can include notification of all builds,
  %% or only failed builds, or builds that haven't been executed within
  %% a certain period.
  \item{ビルド通知}: ビルドの結果は、一般的にクライアントとのやりとりを要するだろう。プル形式での取得(ウェブ、RSS、RPCなど)あるいはプッシュによる通知(メール、Twitter、PubSubHubbubなど)のいずれかとなる。すべてのビルド結果を通知することもあれば失敗したビルドだけを通知することもある。あるいは、所定の時間内に実行できなかったビルドだけを通知することもある。

  %% \item{Build information}: Build details and products may need
  %% to be retrieved, usually via RPC or a bulk download system. For
  %% example, it may be desirable to have an external analysis system
  %% do an in-depth or more targeted analysis, or report on code
  %% coverage or performance results. In addition, an external test
  %% result repository may be employed to keep track of failing and
  %% successful tests separately from the CI system.
  \item{ビルド情報}: ビルドの詳細やその成果物を取得しなければならないこともあるだろう。通常は、RPCを使うか一括ダウンロードの仕組みを用意する。たとえば、別の解析システムを使ってより詳細な(あるいはより的を絞った)解析を行ったり、コードカバレッジやパフォーマンスの情報を表示させることがある。さらに、テスト結果のリポジトリを別に用意して、CIシステムでの失敗したテストと成功したテストの記録を保存しておくこともあるかもしれない。

  %% \item{Build requests}: External build requests from users or a
  %% code repository may need to be handled. Most VCSs have post-commit
  %% hooks that can execute an RPC call to initiate a build, for
  %% example. Or, users may request builds manually through a Web
  %% interface or other user-initiated RPC.
  \item{ビルド要求}: ユーザーあるいはコードリポジトリからのビルド要求を受け、それに対応する必要があるかもしれない。大半のVCSにはコミット後に何らかの処理をフックする仕組みがあり、たとえばビルド処理を起動するようなRPCコールを実行することができる。あるいは、ユーザーがウェブインターフェイスあるいはRPCを使って手動でビルド要求を出すかもしれない。

  %% \item{Remote control of the CI system}: More generally, the
  %% entire runtime may be modifiable through a more-or-less
  %% well-defined RPC interface. Either ad hoc extensions or a
  %% more formally specified interface may need to be able to drive
  %% builds on specific platforms, specify alternate source branches in order to
  %% build with various patches, and execute additional builds
  %% conditionally. This is useful in support of more general workflow
  %% systems, e.g. to permit commits only after they have passed the
  %% full set of CI tests, or to test patches across a wide variety of
  %% systems before final integration. Because of the variety of bug
  %% tracker, patch systems, and other external systems in use, it may
  %% not make sense to include this logic within the CI system itself.
  \item{CIシステムのリモート制御}: より一般化して、実行環境全体の変更をある程度うまく作られたRPCインターフェイスで行いたいものだ。アドホックな拡張あるいは正式に決められたインターフェイスを使って、特定のプラットフォーム上でのビルドの実行や別のブランチにさまざまなパッチを適用したビルドの実行、あるいは条件付きでのビルドの実行などを実行できる必要がある。この機能があれば、より一般的なワークフローにも対応できるので便利だ。たとえばCIテストに完全にパスした変更だけをコミットできるようにしたり、パッチをさまざまなシステムでテストしてから最終的に取り込むようにしたりといったことができる。バグ追跡システムやパッチシステムその他外部のシステムにはさまざまなものがあるので、このロジックをCIシステム自体に組み込んでしまうのは意味がない。

\end{aosadescription}

\end{aosasect2}

\end{aosasect1}

%% \begin{aosasect1}{Architectures}
\begin{aosasect1}{アーキテクチャ}

%% Buildbot and CDash have chosen opposite architectures, and implement
%% overlapping but distinct sets of features. Below we examine these
%% feature sets and discuss how features are easier or harder to
%% implement given the choice of architecture.
BuildbotとCDashはまったく正反対のアーキテクチャを選択しており、一部重複する部分もあるが別々の機能群を実装している。個々の機能セットについて以下で吟味し、ある機能の実装しやすさ(しにくさ)がアーキテクチャの選択でどのように変わるのかを確かめる。

%% \begin{aosasect2}{Implementation Model: Buildbot}
\begin{aosasect2}{実装モデル: Buildbot}

%% \aosafigure{../images/integration/buildbot.eps}{Buildbot Architecture}{fig.integration.buildbot}
\aosafigure{../images/integration/buildbot.eps}{Buildbotのアーキテクチャ}{fig.integration.buildbot}

%% Buildbot uses a master/slave architecture, with a single central
%% server and multiple build slaves. Remote execution is entirely
%% scripted by the master server in real time: the master configuration
%% specifies the command to be executed on each remote system, and runs
%% them when each previous command is finished. Scheduling and build
%% requests are not only coordinated through the master but directed
%% entirely \emph{by} the master. No built-in recipe abstraction exists,
%% except for basic version control system integration (``our code is in
%% this repository'') and a distinction between commands that operate on
%% the build directory vs. within the build directory. OS-specific
%% commands are typically specified directly in the configuration.
Buildbotはマスター/スレーブ型のアーキテクチャで、一台の中央サーバーと複数台のビルド用スレーブで構成されている。リモートの実行は、完全にマスターサーバー側からの指示でリアルタイムに行われる。各クライアント上で実行するコマンドをマスター側で設定し、直前のコマンドが終了するとそれを実行する。スケジューリングやビルド要求をサーバー側でとりまとめるだけではなく、実際の指示もマスターが行う。レシピの抽象化機能は組み込まれていない。ただし、基本的なバージョン管理システムとの統合(``我々のコードはこのリポジトリにある'')、そしてビルドディレクトリ上で実行されるコマンドとビルドディレクトリ内で実行されるコマンドの区別は例外である。OS固有のコマンドは、通常は設定で直接指定する。

%% Buildbot maintains a constant connection with each buildslave, and
%% manages and coordinates job execution between them.  Managing remote
%% machines through a persistent connection adds significant practical
%% complexity to the implementation, and has been a long-standing source
%% of bugs.  Keeping robust long-term network connections running is not
%% simple, and testing applications which interact with the local GUI is
%% challenging through a network connection. OS alert windows are
%% particularly difficult to deal with.  However, this constant
%% connection makes resource coordination and scheduling straightforward,
%% because slaves are entirely at the disposal of the master for
%% execution of jobs.
Buildbotは各スレーブとの接続を持続させ、ジョブの管理やスレーブ間での調整を行う。持続的接続を使ったリモートマシンの管理の実装は複雑なものとなり、長年バグの元となり続けている。堅牢なネットワーク接続を長期間保つのは単純なことではなく、ローカルのGUIと対話するアプリケーションのテストをネットワークごしに行うのは大変だ。OSのアラートウィンドウは特に扱いにくいものである。しかし、このように接続を持続させるおかげで、リソースの協調やスケジューリングを直感的に行うことができる。ジョブを実行させるときには、マスターがスレーブを完全に操れるからだ。

%% The kind of tight control designed into the Buildbot model makes
%% centralized build coordination between resources very easy. Buildbot
%% implements both master and slave locks on the buildmaster, so that
%% builds can coordinate system-global and machine-local resources. This
%% makes Buildbot particularly suitable for large installations that run
%% system integration tests, e.g. tests that interact with databases or
%% other expensive resources.
Buildbotのモデルの設計に組み込まれたような密な制御は、中央管理型でビルドを管理してリソース間で協調させることが容易にできる。Buildbotはbuildmaster上でのマスターのロックとスレーブのロックの両方を実装している。これらを使って、システムグローバルなリソースとマシンローカルなリソースを協調させたビルドが可能となる。この点で、Buildbotが特に適しているのは大規模なシステムのインテグレーションテスト(データベースやその他高価なリソースと組み合わせたテスト)であると言える。

%% The centralized configuration causes problems for a distributed use
%% model, however. Each new buildslave must be explicitly allowed for in
%% the master configuration, which makes it impossible for new
%% buildslaves to dynamically attach to the central server and offer
%% build services or build results. Moreover, because each build slave is
%% entirely driven by the build master, build clients are vulnerable to
%% malicious or accidental misconfigurations: the master literally
%% controls the client entirely, within the client OS security
%% restrictions.
しかし、中央集権型の設定は、分散型の利用モデルでは問題の原因となる。新しいbuildslaveを追加するには、マスターの設定で明示的に許可しなければならない。つまり、新しいbuildslaveを動的に中央サーバーにアタッチしてビルドサービスやビルド結果を送るようにするのは不可能だということだ。さらに、個々のスレーブは完全にマスター側からの指示で動いているので、悪意のある設定をされたり設定を間違えてしまったりするといった事故に対して脆弱になる。クライアントOSのセキュリティ制約の範囲内で、マスターはクライアントを文字通り完全に支配する。

%% One limiting feature of Buildbot is that there is no simple way to
%% return build products to the central server. For example, code
%% coverage statistics and binary builds are kept on the remote
%% buildslave, and there is no API to transmit them to the central
%% buildmaster for aggregation and distribution. It is not clear why this
%% feature is absent. It may be a consequence of the limited set of
%% command abstractions distributed with Buildbot, which are focused on
%% executing remote commands on the build slaves. Or, it may be due to
%% the decision to use the connection between the buildmaster and
%% buildslave as a control system, rather than as an RPC mechanism.
Buildbotの機能面での制限のひとつは、ビルドの成果物を中央サーバーに返すシンプルな方法がないことだ。たとえば、コードカバレッジの統計情報やビルド後のバイナリはリモートのbuildslave上に残ったままとなる。中央のbuildmaster側に、それを集約したり配布したりするAPIは用意されていない。この機能が存在しない理由は定かではない。Buildbotとともに配布されているコマンド群の抽象化の制約のためかもしれない。もとのコマンド群が、スレーブ上でのリモートコマンドの実行に焦点を合わせたものだからだ。あるいは、buildmasterとbuildslaveとの間の接続はあくまでも制御システムとして使い、RPCの仕組みとしては使わないように決めた結果かもしれない。

%% Another consequence of the master/slave model and this limited
%% communications channel is that buildslaves do not report system
%% utilization and the master cannot be configured to be aware of high
%% slave load.
マスター/スレーブモデルを採用し、かつこのように制限のある通信チャンネルを用意した結果、buildslave側からはシステムの利用状況を報告できなくなっており、マスター側ではスレーブの負荷対策を組み込むことができない。

%% External CPU notification of build results is handled entirely by the
%% buildmaster, and new notification services need to be implemented
%% within the buildmaster itself. Likewise, new build requests must be
%% communicated directly to the buildmaster.
ビルド結果の外部CPU通知は完全にbuildmasterが処理する。新たな通知サービスを追加するにはbuildmaster自身の中で実装しなければならない。同様に、新たなビルド要求は直接buildmasterにしなければならない。

\end{aosasect2}

%% \begin{aosasect2}{Implementation Model: CDash}
\begin{aosasect2}{実装モデル: CDash}

%% \aosafigure[300pt]{../images/integration/cdash.eps}{CDash Architecture}{fig.integration.cdash}
\aosafigure[300pt]{../images/integration/cdash.eps}{CDashのアーキテクチャ}{fig.integration.cdash}

%% In contrast to Buildbot, CDash implements a reporting server model. In
%% this model, the CDash server acts as a central repository for
%% information on remotely executed builds, with associated reporting on
%% build and test failures, code coverage analysis, and memory
%% usage. Builds run on remote clients on their own schedule, and submit
%% build reports in an XML format. Builds can be submitted both by
%% ``official'' build clients and by non-core developers or users running
%% the published build process on their own machines.
Buildbotとは対照的に、CDashはレポーティングサーバーモデルを実装している。このモデルにおいて、CDashサーバーは中央リポジトリとしてふるまう。リモートで実行したビルドの情報やビルド・テストの失敗報告、コードカバレッジ解析、そしてメモリ使用状況などがここに集まる。ビルドのスケジュールをたてたり実行したりするのはリモートクライアント側で、ビルドレポートはXML形式で送信する。ビルド結果の送信は``公式の''ビルドクライアントで行うこともできるし、コアデベロッパー以外の開発者やユーザーが公開したビルドプロセスを自分のマシンで実行することもできる。

%% This simple model is made possible because of the tight conceptual
%% integration between CDash and other elements of the Kitware build
%% infrastructure: CMake, a build configuration system, CTest, a test
%% runner, and CPack, a packaging system. This software provides a
%% mechanism by which build, test, and packaging recipes can be
%% implemented at a fairly high level of abstraction in an OS-agnostic
%% manner.
このシンプルなモデルが可能が実現できた理由は、CDashとその他のKitwareビルド基盤の要素(ビルド設定システムであるCMake、テストランナーであるCTest、そしてパッケージングシステムであるCPack)が概念的に密結合していたことである。このソフトウェアが提供する仕組みを使うと、ビルドやテストそしてパッケージングのレシピを高度に抽象化して実装できる。その際にOSを意識する必要はない。

%% CDash's client-driven process simplifies many aspects of the
%% client-side CI process. The decision to run a build is made by build
%% clients, so client-side conditions (time of day, high load, etc.) can
%% be taken into account by the client before starting a build. Clients
%% can appear and disappear as they wish, easily enabling volunteer
%% builds and builds ``in the cloud''. Build products can be sent to the
%% central server via a straightforward upload mechanism.
CDashのクライアント主導のプロセスは、クライアント側のCIプロセスを多くの面で単純化する。ビルドを実行するかどうかを決めるのはクライアント側なので、クライアント側の状況(時刻や負荷など)を考慮にいれてビルドを始めることができる。望みに応じてクライアントを増やしたり減らしたりするしてビルドを手伝ったり、ビルドを``クラウドで''行うこともできる。ビルドの成果物を中央サーバーに送るのも、単純なアップロードで済む話だ。

%% However, in exchange for this reporting model, CDash lacks many
%% convenient features of Buildbot. There is no centralized coordination
%% of resources, nor can this be implemented simply in a distributed
%% environment with untrusted or unreliable clients. Progress reports are
%% also not implemented: to do so, the server would have to allow
%% incremental updating of build status. And, of course, there is no way
%% to both globally request a build, and guarantee that anonymous clients
%% \emph{perform} the build in response to a check-in---clients must be
%% considered unreliable.
しかし、このレポーティングモデルを採用した代償として、CDashにはBuildbotが持つ多くの便利な機能が欠けている。中央管理型のリソース制御機能はないし、事前に登録済みでないクライアントを分散環境に投入することもできない。進捗レポート機能も実装されていない。実装するにはビルド状況のインクリメンタルな更新をサーバーが許可しないといけない。そしてもちろん、全体にビルド要求を出したり、チェックインに反応して匿名クライアントにビルドさせることもできない---クライアントはすべて信頼できないものとみなさなければならない。

%% Recently, CDash added functionality to enable an ``@Home'' cloud build
%% system, in which clients offer build services to a CDash server.
%% Clients poll the server for build requests, execute them upon request,
%% and return the results to the server.  In the current implementation
%% (October 2010), builds must be manually requested on the server side,
%% and clients must be connected for the server to offer their services.
%% However, it is straightforward to extend this to a more generic
%% scheduled-build model in which builds are requested automatically by
%% the server whenever a relevant client is available.  The ``@Home''
%% system is very similar in concept to the Pony-Build system described
%% later.
最近、CDashに新機能が追加されてクラウドビルドシステム``@Home''が使えるようになった。これは、クライアントがCDashサーバーに対してビルドサービスを提供する仕組みだ。クライアントがサーバーをポーリングしてビルドリクエストを受けとり、そのリクエストを処理して、結果をサーバーに返す。2010年10月時点の実装では、ビルドのリクエストはサーバー側で手動で行う必要がある。そして、クライアントはサーバーに接続しないとサービスを提供できない。しかし、これを素直に拡張すればより汎用的なビルドモデルが作れる。つまり、サーバー側から自動的にビルドリクエストを送ったら、対応可能なクライアントが処理してくれるというモデルだ。``@Home''システムは、後で説明するPony-Buildシステムと非常に似た概念である。

\end{aosasect2}

%% \begin{aosasect2}{Implementation Model: Jenkins}
\begin{aosasect2}{実装モデル: Jenkins}

%% Jenkins is a widely used continuous integration system implemented in
%% Java; until early 2011, it was known as Hudson. It is capable of
%% acting either as a standalone CI system with execution on a local
%% system, or as a coordinator of remote builds, or even as a passive
%% receiver of remote build information. It takes advantage of the JUnit
%% XML standard for unit test and code coverage reporting to integrate
%% reports from a variety of test tools. Jenkins originated with Sun, but
%% is very widely used and has a robust open-source community associated
%% with it.
Jenkinsは幅広く使われている継続的インテグレーションシステムで、Javaで書かれている。2011年初期まではHudsonという名前で知られていた。スタンドアロンのCIシステムとしてローカルシステム上で動かすこともできるし、リモートビルドの調整役として使うこともできる。あるいは、リモートでのビルドの情報を受け取るだけの役割としても使うことができる。JUnitのユニットテストやコードカバレッジレポートで使われている標準のXMLをうまく活用し、さまざまなテストツールからのレポートを統合する。Jenkinsは元々Sunが作り始めたものだが、さまざまな場所で使われており、しっかりとしたオープンソースコミュニティがついている。

%% Jenkins operates in a hybrid mode, defaulting to master-server build
%% execution but allowing a variety of methods for executing remote
%% builds, including both server- and client-initiated builds. Like
%% Buildbot, however, it is primarily designed for central server
%% control, but has been adapted to support a wide variety of distributed
%% job initiation mechanisms, including virtual machine management.
Jenkinsはハイブリッドモードで動作する。デフォルトはマスターサーバーでビルドを実行するが、さまざまなスタイルのリモートビルドも(サーバー側からでもクライアント側からでも)実行できる。Buildbotと同様、本来は中央サーバーが管理するように作られている。しかし、さまざまな分散ジョブ実行機構に対応するようになった。仮想マシンの管理機能も含む。

%% Jenkins can manage multiple remote machines through a connection
%% initiated by the master via an SSH connection, or from the client via
%% JNLP (Java Web Start). This connection is two-way, and supports the
%% communication of objects and data via serial transport. 
Jenkinsは複数のリモートマシンを管理することができる。接続はマスター側からSSHで確立することもできるし、クライアント側からJNLP (Java Web Start)で確立することもできる。この接続は双方向で、オブジェクトやデータもシリアル化してやりとりすることができる。

%% Jenkins has a robust plugin architecture that abstracts the details of
%% this connection, which has allowed the development of many third-party
%% plugins to support the return of binary builds and more significant
%% result Data.
Jenkinsにはしっかりとしたプラグイン機構が組み込まれており、この接続の詳細を抽象化している。そのおかげで、多くのサードパーティのプラグインがバイナリビルドや結果のデータを扱えるようになっている。

%% For jobs that are controlled by a central server, Jenkins has a
%% ``locks'' plugin to discourage jobs from running in parallel, although
%% as of January 2011 it is not yet fully developed.
中央サーバーが管理するジョブ用として、Jenkinsには``locks''プラグインが用意されている。このプラグインはジョブを並列実行させないようにするものだが、2011年1月の時点では未完成だ。

\end{aosasect2}

%% \begin{aosasect2}{Implementation Model: Pony-Build}
\begin{aosasect2}{実装モデル: Pony-Build}

%% \aosafigure{../images/integration/webhooks.eps}{Pony-Build Architecture}{fig.integration.pb}
\aosafigure{../images/integration/webhooks.eps}{Pony-Buildのアーキテクチャ}{fig.integration.pb}

%% Pony-Build is a proof-of-concept decentralized CI system written in
%% Python. It is composed of three core components, which are illustrated
%% in \aosafigref{fig.integration.pb}. The results server acts as a
%% centralized database containing build results received from individual
%% clients. The clients independently contain all configuration
%% information and build context, coupled with a lightweight client-side
%% library to help with VCS repository access, build process management,
%% and the communication of results to the server. The reporting server
%% is optional, and contains a simple Web interface, both for reporting
%% on the results of builds and potentially for requesting new builds. In
%% our implementation, the reporting server and results server run in a
%% single multithreaded process but are loosely coupled at the API level
%% and could easily be altered to run independently.
Pony-Buildは、分散型のCIシステムの概念実証モデルとしてPythonで作られた。\aosafigref{fig.integration.pb}に示す三つのコアコンポーネントで構成されている。結果サーバーが中央データベースとして働き、個々のクライアントから受け取ったビルド結果を保持する。クライアントはそれぞれ独立してすべての設定情報やビルドコンテキストを保持しており、軽量なクライアントライブラリでVCSのリポジトリにアクセスしたりビルドプロセスを管理したり、結果をサーバーに通信したりといったことができる。レポートサーバーは必須ではない。ここにはシンプルなウェブインターフェイスが組み込まれており、ビルド結果の報告や新しいビルドの要求を行う。我々の実装では、レポートサーバーと結果サーバーは単一のマルチスレッドプロセスで動作する。しかしAPIレベルでの結合は緩く、個別に動くようにも容易に変更できる。

%% This basic model is decorated with a variety of webhooks and RPC
%% mechanisms to facilitate build and change notification and build
%% introspection. For example, rather than tying VCS change notification
%% from the code repository directly into the build system, remote build
%% requests are directed to the reporting system, which communicates them
%% to the results server. Likewise, rather than building push
%% notification of new builds out to e-mail, instant messaging, and other
%% services directly into the reporting server, notification is
%% controlled using the PubSubHubbub (PuSH) active notification
%% protocol. This allows a wide variety of consuming applications to
%% receive notification of ``interesting'' events (currently limited to
%% new builds and failed builds) via a PuSH webhook.
基本モデルに加えてさまざまなWebHookやRPC機構が用意されており、ビルドや変更通知そしてビルドに関する調査を支援する。たとえば、VCSのコードリポジトリの変更通知をビルドシステムと直接結び付けるのではなく、リモートからのビルドリクエストを直接レポートシステムに回し、レポートシステムがそれを結果サーバーに伝えるようにする。同様に、メールやインスタントメッセージングなどを使って新しいビルドを直接レポートサーバーにプッシュ通知するのではなく、通知の制御にはPubSubHubbub (PuSH)を使っている。これにより、さまざまなアプリケーションが``興味のある''イベント(今のところは新しいビルドと失敗したビルドに限られる)の通知をPuSH WebHookで受け取れるようになっている。

%% The advantages of this very decoupled model are substantial:
このような疎結合のモデルを採用する利点は多い。

\begin{aosadescription}

  %% \item{Ease of communication:} The basic architectural
  %% components and webhook protocols are extremely easy to implement,
  %% requiring only a basic knowledge of Web programming.
  \item{通信の容易性:} 基本となるアーキテクチャ上のコンポーネントやWebHookのプロトコルは極めて容易に実装でき、基本的なウェブプログラミングの知識さえあればよい。

  %% \item{Easy modification:} The implementation of new
  %% notification methods, or a new reporting server interface, is
  %% extremely simple.
  \item{変更の容易性:} 新しい通知方式や新しいレポートサーバー用インターフェイスを実装するのは極めて容易である。

  %% \item{Multiple language support:} Since the various components
  %% call each other via webhooks, which are supported by most
  %% programming languages, different components can be implemented in
  %% different languages.
  \item{多言語のサポート:} さまざまなコンポーネントがお互いWebHookで呼び合っている。大半のプログラミング言語はWebHookをサポートしているので、コンポーネントごとに別々の言語で実装することができる。

  %% \item{Testability:} Each component can be completely isolated
  %% and mocked, so the system is very testable.
  \item{テストのしやすさ:} 各コンポーネントは完全に分離していてモックがあるので、システムのテストを簡単に実行できる。

  %% \item{Ease of configuration:} The client-side requirements are
  %% minimal, with only a single library file required beyond Python
  %% itself.
  \item{設定の容易性:} クライアント側の要件は最小限で、Pythonそのもの以外に必要となるライブラリはひとつだけである。

  %% \item{Minimal server load:} Since the central server has
  %% virtually no control responsibilities over the clients, isolated
  %% clients can run in parallel without contacting the server and
  %% placing any corresponding load it, other than at reporting Time.
  \item{サーバーの負荷の最小化:} 中央サーバーにはクライアントを制御する責任が事実上ないと言えるので、隔離されたクライアントをサーバーと連携せずに並列に実行させることができる。報告時を除いて、サーバーに負荷をかけることはない。

  %% \item{VCS integration:} Build configuration is entirely client
  %% side, allowing it to be included within the VCS.
  \item{VCSとの統合:} ビルドの設定は完全にクライアント側で行われるので、VCSに含めることもできる。

  %% \item{Ease of results access:} Applications that wish to
  %% consume build results can be written in any language capable of an
  %% XML-RPC request. Users of the build system can be granted access
  %% to the results and reporting servers at the network level, or via
  %% a customized interface at the reporting server. The build clients
  %% only need access to post results to the results server.
  \item{結果へのアクセスの容易性:} ビルド結果を取得したいアプリケーションを書くためのプログラミング言語は、XML-RPCリクエストを扱えるものなら何でもよい。ビルドシステムのユーザーには結果サーバーやレポートサーバーへのネットワークレベルでのアクセス権限を与えることもできるし、レポートサーバーの独自インターフェイスを使ってもよい。ビルドクライアントに必要なアクセス権限は、結果サーバーへの結果の送信権だけである。

\end{aosadescription}

%% Unfortunately, there are also many serious \emph{disadvantages}, as
%% with the CDash model:
残念ながら、深刻な\emph{弱点}も多い。これはCDashのモデルと同様である。

\begin{aosadescription}

  %% \item{Difficulty requesting builds:} This difficulty is
  %% introduced by having the build clients be entirely independent of
  %% the results server. Clients may poll the results server, to see if
  %% a build request is operative, but this introduces high load and
  %% significant latency. Alternatively, command and control
  %% connections need to be established to allow the server to notify
  %% clients of build requests directly. This introduces more
  %% complexity into the system and eliminates the advantages of
  %% decoupled build clients.
  \item{ビルド要求を送るのが難しい:} この弱点の原因は、ビルドクライアントが結果サーバーから完全に独立しているということだ。クライアント側からサーバー側にビルド要求があるかどうかを確認することはできるかもしれない。しかしこれは負荷が高く、待ち時間も長くなる。それ以外の手段をとるなら、指揮・制御のための接続を確立してサーバー側からクライアント側にビルドリクエストを直接通知しなければならない。システムはさらに複雑になり、分散型ビルドクライアントの利点を損ねてしまう。

  %% \item{Poor support for resource locking:} It is easy to
  %% provide an RPC mechanism for holding and releasing resource locks,
  %% but much more difficult to enforce client policies. While CI
  %% systems like CDash assume good faith on the client side, clients
  %% may fail unintentionally and badly, e.g. without releasing
  %% locks. Implementing a robust distributed locking system is hard
  %% and adds significant undesired complexity. For example, in order
  %% to provide master resource locks with unreliable clients, the
  %% master lock controller must have a policy in place for clients
  %% that take a lock and never release it, either because they crash
  %% or because of a deadlock Situation.
  \item{リソースロックのサポートが貧弱:} リソースロックを管理するためにRPCの仕組みを提供するのは簡単だが、クライアントポリシーを強制するのはずっと難しい。CDashのようなCIシステムはクライアント側を信頼することを前提としているが、クライアント側はそのつもりがなくても失敗してしまうかもしれない(ロックの解放を忘れるなど)。堅牢な分散型ロックシステムを実装するのは困難であり、余計な複雑さを持ち込んでしまう。たとえば、マスターリソースのロックを信頼できないクライアントに提供するには、ロックをつかんで離さないクライアントに対するポリシーをマスターロックコントローラー側で決めておく必要がある。これは、クライアントがクラッシュしたり、デッドロックが発生したりした場合の対策となる。

  %% \item{Poor support for real-time monitoring:} Real-time
  %% monitoring of the build, and control of the build process
  %% itself, is challenging to implement in a system without a constant
  %% connection. One significant advantage of Buildbot over the
  %% client-driven model is that intermediate inspection of long builds
  %% is easy, because the results of the build are communicated to the
  %% master interface incrementally. Moreover, because Buildbot retains
  %% a control connection, if a long build goes bad in the middle
  %% due to misconfiguration or a bad check-in, it can be interrupted
  %% and aborted. Adding such a feature into Pony-Build, in which the
  %% results server has no guaranteed ability to contact the clients,
  %% would require either constant polling by the clients, or the
  %% addition of a standing connection to the clients.
  \item{リアルタイムの監視処理が貧弱:} リアルタイムでビルドを監視したりビルドプロセス自体を制御したりする仕組みを実装するのは、常に接続が持続しているシステムでないと困難である。Buildbotがクライアント主導のモデルに比べてはるかに優れている点のひとつは、時間のかかるビルドの途中経過を調べやすいというところだ。これは、ビルドの結果がマスター側のインターフェイスにインクリメンタルに送られてくるからである。さらに、Buildbotは制御用の接続を保持しているので、もし時間のかかるビルドが途中で(設定ミスや間違ったチェックインなどで)失敗した場合は、そこでビルドを中止できる。そのような機能をPony-Build(結果サーバーからクライアント側への連絡ができない)に追加するには、クライアント側から定期的にポーリングさせるかクライアント側への接続を確立するかのどちらかの仕組みが必要となる。

\end{aosadescription}

%% Two other aspects of CIs that were raised by Pony-Build were how best
%% to implement \emph{recipes}, and how to manage \emph{trust}. These are
%% intertwined issues, because recipes execute arbitrary code on build
%% clients.
Pony-Buildが提起するCIのその他ふたつの側面は、\emph{レシピ}をいかに実装するかということ、そして\emph{信頼}をどのように管理するのかということだ。これらはともに関連する問題である。というのも、レシピはビルドクライアント上の任意のコードを実行するからである。

\end{aosasect2}

\begin{aosasect2}{Build Recipes}

Build recipes add a useful level of abstraction, especially for
software built in a cross-platform language or using a multi-platform
build system. For example, CDash relies on a strict kind of
recipe; most, or perhaps all, software that uses CDash is built
with CMake, CTest, and CPack, and these tools are built to handle
multi-platform issues. This is the ideal situation from the viewpoint
of a continuous integration system, because the CI system can simply
delegate all issues to the build tool chain.

However, this is not true for all languages and build environments. In
the Python ecosystem, there has been increasing standardization around
distutils and distutils2 for building and packaging software, but as
yet no standard has emerged for discovering and running tests, and
collating the results. Moreover, many of the more complex Python
packages add specialized build logic into their system, through a
distutils extension mechanism that allows the execution of arbitrary
code. This is typical of most build tool chains: while there may be a
fairly standard set of commands to be run, there are always exceptions
and extensions.

Recipes for building, testing, and packaging are therefore
problematic, because they must solve two problems: first, they should
be specified in a platform independent way, so that a single recipe
can be used to build software on multiple systems; and second, they
must be customizable to the software being built.

\end{aosasect2}

\begin{aosasect2}{Trust}

This raises a third problem.  Widespread use of recipes by a CI system
introduces a second party that must be trusted by the system: not only
must the software itself be trustworthy (because the CI clients are
executing arbitrary code), but the recipes must also be trustworthy
(because they, too, must be able to execute arbitrary code).

These trust issues are easy to handle in a tightly controlled
environment, e.g. a company where the build clients and CI system are
part of an internal process. In other development environments,
however, interested third parties may want to offer build services,
for example to open source projects. The ideal solution would be to
support the inclusion of standard build recipes in software on a
community level, a direction that the Python community is taking with
distutils2. An alternative solution would be to allow for the use of
digitally signed recipes, so that trusted individuals could write and
distribute signed recipes, and CI clients could check to see if they
should trust the recipes.

\end{aosasect2}

\begin{aosasect2}{Choosing a Model}

In our experience, a loosely coupled RPC or webhook callback-based
model for continuous integration is extremely easy to implement, as
long as one ignores any requirements for tight coordination that would
involve complex coupling. Basic execution of remote checkouts and
builds has similar design constraints whether the build is being
driven locally or remotely; collection of information about the build
(success/failure, etc.) is primarily driven by client-side
requirements; and tracking information by architecture and result
involves the same basic requirements.  Thus a basic CI system can be
implemented quite easily using the reporting model.

We found the loosely coupled model to be very flexible and expandable,
as well. Adding new results reporting, notification mechanisms, and
build recipes is easy because the components are clearly separated and
quite independent. Separated components have clearly delegated tasks
to perform, and are also easy to test and easy to modify.

The only challenging aspect of remote builds in a CDash-like
loosely-coupled model is build coordination: starting and stopping
builds, reporting on ongoing builds, and coordinating resource locks
between different clients is technically demanding compared to the
rest of the implementation.

It is easy to reach the conclusion that the loosely coupled model is
``better'' all around, but obviously this is only true if build
coordination is not needed.  This decision should be made based on the
needs of projects using the CI system.

\end{aosasect2}

\end{aosasect1}

\begin{aosasect1}{The Future}

While thinking about Pony-Build, we came up with a few features that
we would like to see in future continuous integration systems.

\begin{aosadescription}

  \item{A language-agnostic set of build recipes:} Currently,
  each continuous integration system reinvents the wheel by
  providing its own build configuration language, which is
  manifestly ridiculous; there are fewer than a dozen commonly used
  build systems, and probably only a few dozen test
  runners. Nonetheless, each CI system has a new and different way
  of specifying the build and test commands to be run. In fact, this
  seems to be one of the reasons why so many basically identical CI
  systems exist: each language and community implements their own
  configuration system, tailored to their own build and test
  systems, and then layers on the same set of features above that
  system. Therefore, building a domain-specific language (DSL)
  capable of representing the options used by the few dozen commonly
  used build and test tool chains would go a long way toward
  simplifying the CI landscape.

  \item{Common formats for build and test reporting:} There is
  little agreement on exactly what information, in what format, a
  build and test system needs to provide. If a common format or
  standard could be developed it would make it much easier for
  continuous integration systems to offer both detailed and summary
  views across builds. The Test Anywhere Protocol, TAP (from the
  Perl community) and the JUnit XML test output format (from the
  Java community) are two interesting options that are capable of
  encoding information about number of tests run, successes and
  failures, and per-file code coverage details.

  \item{Increased granularity and introspection in reporting:}
  Alternatively, it would be convenient if different build platforms
  provided a well-documented set of hooks into their configuration,
  compilation, and test systems. This would provide an API (rather
  than a common format) that CI systems could use to extract more
  detailed information about builds.

\end{aosadescription}

\begin{aosasect2}{Concluding Thoughts}

The continuous integration systems described above implemented
features that fit their architecture, while the hybrid Jenkins system
started with a master/slave model but added features from the more
loosely coupled reporting architecture.

It is tempting to conclude that architecture dictates function. This
is nonsense, of course. Rather, the choice of architecture seems to
canalize or direct development towards a particular set of
features. For Pony-Build, we were surprised at the extent to which our
initial choice of a CDash-style reporting architecture drove later
design and implementation decisions. Some implementation choices, such
as the avoidance of a centralized configuration and scheduling system
in Pony-Build were driven by our use cases: we needed to allow dynamic
attachment of remote build clients, which is difficult to support with
Buildbot. Other features we didn't implement, such as progress reports
and centralized resource locking in Pony-Build, were desirable but
simply too complicated to add without a compelling requirement.

Similar logic may apply to Buildbot, CDash, and Jenkins. In each case
there are useful features that are absent, perhaps due to
architectural incompatibility. However, from discussions with members
of the Buildbot and CDash communities, and from reading the Jenkins
website, it seems likely that the desired features were chosen first,
and the system was then developed using an architecture that permitted
those features to be easily implemented. For example, CDash serves a
community with a relatively small set of core developers, who develop
software using a centralized model. Their primary consideration is to
keep the software working on a core set of machines, and secondarily
to receive bug reports from tech-savvy users. Meanwhile, Buildbot is
increasingly used in complex build environments with many clients that
require coordination to access shared resources. Buildbot's more
flexible configuration file format with its many options for
scheduling, change notification, and resource locks fits that need
better than the other options. Finally, Jenkins seems aimed at ease of
use and simple continuous integration, with a full GUI for configuring
it and configuration options for running on the local server.

The sociology of open source development is another confounding factor
in correlating architecture with features: suppose developers choose
open source projects based on how well the project architecture and
features fit their use case?  If so, then their contributions will
generally reflect an extension of a use case that already fits the
project well. Thus projects may get locked into a certain feature set,
since contributors are self-selected and may avoid projects with
architectures that don't fit their own desired features. This was
certainly true for us in choosing to implement a new system,
Pony-Build, rather than contributing to Buildbot: the Buildbot
architecture was simply not appropriate for building hundreds or
thousands of packages.

Existing continuous integration systems are generally built around one
of two disparate architectures, and generally implement only a subset
of desirable features.  As CI systems mature and their user
populations grow, we would expect them to grow additional features;
however, implementation of these features may be constrained by the
base choice of architecture.  It will be interesting to see how the
field evolves.

\end{aosasect2}

\begin{aosasect2}{Acknowledgments}

We thank Greg Wilson, Brett Cannon, Eric Holscher, Jesse Noller, and
Victoria Laidler for interesting discussions on CI systems in general,
and Pony-Build in particular.  Several students contributed to
Pony-Build development, including Jack Carlson, Fatima Cherkaoui, Max
Laite, and Khushboo Shakya.

\end{aosasect2}

\end{aosasect1}

\end{aosachapter}
