\begin{aosachapter}{The Bourne-Again Shell}{s:bash}{Chet Ramey}
%% Based on EN-Revision r229

%% \begin{aosasect1}{Introduction}
\begin{aosasect1}{導入}

%% A Unix shell provides an interface that lets the user interact
%% with the operating system by running commands.
%% But a shell is also a fairly rich
%% programming language: there are constructs for flow control,
%% alternation, looping, conditionals, basic mathematical operations,
%% named functions, string variables, and two-way communication between
%% the shell and the commands it invokes.
Unixのシェルは、ユーザーとOSとの間のコマンドによるインターフェイスを提供する。しかし、シェルはまた、リッチなプログラミング言語でもある。フロー制御やループそして条件分岐といった制御構造もあるし、基本的な数学演算や関数、文字列変数などもあり、シェルとコマンドの間の双方向の通信もある。

%% Shells can be used interactively, from a terminal or terminal emulator
%% such as xterm, and non-interactively, reading commands from a file.
%% Most modern shells, including bash, provide command-line editing, in
%% which the command line can be manipulated using emacs- or vi-like
%% commands while it's being entered, and various forms of a saved
%% history of commands.
シェルは、ターミナルあるいはターミナルエミュレータ(xtermなど)から対話的に使うこともできるし、コマンドをファイルから読み込むこともできる。bashを含むモダンなシェルにはコマンドラインの編集機能があり、コマンドの入力中にemacs風あるいはvi風の操作でコマンドラインをいじることができる。また、さまざまな形式でコマンド履歴を記録する。

%% Bash processing is much like a shell pipeline: after being read from
%% the terminal or a script, data is passed through a number of stages,
%% transformed at each step, until the shell finally executes a command
%% and collects its return status.
Bashの処理はシェルのパイプラインとそっくりだ。ターミナルあるいはスクリプトから読み込んだデータはいくつかのステージを通過し、各ステージで変換され、シェルが最終的にコマンドを実行してその返り値を受け取る。

%% This chapter will explore bash's major components: input processing,
%% parsing, the various word expansions and other command processing, and
%% command execution, from the pipeline perspective.  These components
%% act as a pipeline for data read from the keyboard or from a file,
%% turning it into an executed command.
本章では、bashの主要なコンポーネントである入力処理やパース、さまざまなワードの展開、その他のコマンド処理、そしてコマンドの実行について、パイプラインの観点から探求する。これらのコンポーネントはキーボードやファイルから読み込んだデータのパイプラインとして働き、それを実行されるコマンドに変える。

%% \aosafigure{../images/bash/bash-article-diagram.eps}{Bash Component Architecture}{fig.bash.fig1}
\aosafigure{../images/bash/bash-article-diagram.eps}{Bashのコンポーネントのアーキテクチャ}{fig.bash.fig1}

%% \begin{aosasect2}{Bash}
\begin{aosasect2}{Bash}

%% Bash is the shell that appears in the GNU operating system, commonly
%% implemented atop the Linux kernel, and several other common operating
%% systems, most notably Mac OS X\@.  It offers functional improvements
%% over historical versions of sh for both interactive and programming
%% use.
BashはGNUオペレーティングシステムで使われているシェルであり、一般的にはLinuxカーネル上で実装されている。また、Mac OS X\@などその他の主要OS上でも動く。過去の歴史上のバージョンであるshに対して、対話的な操作においてもプログラミング機能においても改良が施されている。

%% The name is an acronym for Bourne-Again SHell, a pun combining the
%% name of Stephen Bourne (the author of the direct ancestor of the
%% current Unix shell \code{/bin/sh}, which appeared in the Bell Labs
%% Seventh Edition Research version of Unix) with the notion of rebirth
%% through reimplementation.
%% The original author of bash was Brian Fox, an employee of the Free
%% Software Foundation.  I am the current developer and maintainer, a
%% volunteer who works at Case Western Reserve University in Cleveland,
%% Ohio.
名前の由来はBourne-Again SHellの頭文字をとったもので、Stephen Bourne(現在のUnixシェルの先祖である\code{/bin/sh}の作者。このシェルはベル研のVersion 7 Unixで登場した)の名前と再実装によって生まれ変わったことをかけている。bashの最初の作者はBrian Foxで、彼はFree Software Foundationのメンバーだった。私は現在の開発者兼メンテナ—であり、オハイオ州クリーブランドにあるケースウエスタンリザーブ大学に勤務している。

%% Like other GNU software, bash is quite portable.  It currently runs on
%% nearly every version of Unix and a few other operating
%% systems---independently-supported ports exist for hosted Windows
%% environments such as Cygwin and MinGW, and ports to Unix-like systems
%% such as QNX and Minix are part of the distribution.  It only requires
%% a Posix environment to build and run, such as one provided by
%% Microsoft's Services for Unix (SFU).
他のGNUソフトウェアと同様、bashも移植性がきわめて高い。Unixのほぼすべてのバージョンで動作するし、その他のOSでも動作する---独自にサポートしている移植版にはWindows上のCygwinやMinGWといった環境もあるし、QNXやMinixといったUnixライクなシステムへの移植版は配布物に含まれている。ビルドして実行するために必要なのはPosix環境だけである。つまり、MicrosoftのServices for Unix (SFU)などでもよい。

\end{aosasect2}

\end{aosasect1}

%% \begin{aosasect1}{Syntactic Units and Primitives}
\begin{aosasect1}{構文単位およびプリミティブ}

%% \begin{aosasect2}{Primitives}
\begin{aosasect2}{プリミティブ}

%% To bash, there are basically three kinds of tokens: reserved
%% words, words, and operators.  Reserved words are those that have
%% meaning to the shell and its programming language; usually these words
%% introduce flow control constructs, like \code{if} and \code{while}.
%% Operators are composed of one or more metacharacters: characters that
%% have special meaning to the shell on their own, such as \code{|} and
%% \code{{\textgreater}}.  The rest of the shell's input consists of
%% ordinary words, some of which have special meaning---assignment
%% statements or numbers, for instance---depending on where they appear
%% on the command line.
bashには、基本的に三種類のトークンがある。予約語(reserved word)、単語(word)、そして演算子(operator)だ。予約語とはシェルやそのプログラミング言語に対して何らかの意味を持つ単語のことで、フロー制御構文に使われることが多い。たとえば\code{if}や\code{while}がそれにあたる。演算子とはメタ文字を組み合わせたもののことで、メタ文字とはシェル自身に対して特別な意味を持つ文字を指す。\code{|}や\code{{\textgreater}}などだ。それ以外のシェルへの入力は普通の単語で、その中にはコマンドライン内での登場位置によって特殊な意味を持つもの---代入文や数値など---もある。

\end{aosasect2}

%% \begin{aosasect2}{Variables and Parameters}
\begin{aosasect2}{変数およびパラメータ}

%% As in any programming language, shells provide variables: names to
%% refer to stored data and operate on it.  The shell provides basic
%% user-settable variables and some built-in variables referred to as
%% parameters.  Shell parameters generally reflect some aspect of the
%% shell's internal state, and are set automatically or as a side effect
%% of another operation.
他のプログラミング言語と同様、シェルにも変数の機能があり、保存したデータを後で参照したり演算に使ったりすることができる。シェルが提供している変数には、ユーザーが設定可能な基本的な変数と、パラメータとして参照できる組み込みの変数がある。シェルのパラメータは一般的にシェルの内部状態を反映するもので、自動的に設定されたり別の操作の副作用として設定されたりする。

%% Variable values are strings.  Some values are treated specially
%% depending on context; these will be explained later.  Variables are
%% assigned using statements of the form \code{name=value}.  The
%% \code{value} is optional; omitting it assigns the empty string to
%% \code{name}.  If the value is supplied, the shell expands the value
%% and assigns it to \code{name}.  The shell can perform different
%% operations based on whether or not a variable is set, but assigning a
%% value is the only way to set a variable.  Variables that have not been
%% assigned a value, even if they have been declared and given
%% attributes, are referred to as \emph{unset}.
変数の値は文字列である。値の中には状況によって特別な意味を持つものもあるが、それについては後で説明する。変数への代入は、\code{name=value}形式の文を使う。\code{value}は必須ではなく、省略した場合は空の文字列を\code{name}に代入する。valueを指定すると、シェルはその内容を展開して\code{name}に代入する。シェルは、変数が設定されているかどうかによって処理を変えることがある。しかし、変数に値を設定するには値を代入する以外の方法はない。値を代入されていない変数は、たとえ事前に宣言されていたとしても参照すると\emph{unset}となる。

%% A word beginning with a dollar sign introduces a variable or parameter
%% reference.  The word, including the dollar sign, is replaced with the
%% value of the named variable.  The shell provides a rich set of
%% expansion operators, from simple value replacement to changing or
%% removing portions of a variable's value that match a pattern.
ドル記号で始まる単語は、変数あるいはパラメータへの参照を意味する。ドル記号を含めた単語が、その名前の変数の値に置きかえられる。シェルには豊富な展開演算子が用意されており、単純な値の置換だけではなくパターンにマッチする部分を変更したり削除したりすることもできる。

%% There are provisions for local and global variables.  By default, all
%% variables are global.  Any simple command (the most familiar type of
%% command---a command name and optional set of arguments and
%% redirections) may be prefixed by a set of assignment statements to
%% cause those variables to exist only for that command.  The shell
%% implements stored procedures, or shell functions, which can have
%% function-local variables.
変数には、ローカルとグローバルの二種類がある。デフォルトでは、すべての変数はグローバルとなる。単純なコマンド(最も見なれた形式のコマンド---コマンド名の後にオプションで引数やリダイレクトが続く形式)の前には代入文がくることもあり、そのコマンドのためだけに変数が存在することになる。シェルはストアドプロシージャやシェル関数を実装しており、それぞれ関数ローカルな変数を持つことができる。

%% Variables can be minimally typed: in addition to simple string-valued
%% variables, there are integers and arrays.  Integer-typed variables are
%% treated as numbers: any string assigned to them is expanded as an
%% arithmetic expression and the result is assigned as the variable's
%% value.  Arrays may be indexed or associative; indexed arrays use
%% numbers as subscripts, while associative arrays use arbitrary strings.
%% Array elements are strings, which can be treated as integers if
%% desired.  Array elements may not be other arrays.
変数には最低限の型をつけることができる。単純な文字列値の変数に加えて、静数値と配列が使える。整数型の変数は数値として扱われる。文字列を代入するとそれを計算式とみなして展開し、計算結果を変数の値として代入する。配列は、インデックス型と連想型のどちらかになる。インデックス型の配列は数値を添字として使い、連想配列は任意の文字列を添字として使う。配列の要素は文字列であり、望むなら静数値として扱うこともできる。配列の要素に別の配列が入ることはない。

%% Bash uses hash tables to store and retrieve shell variables, and
%% linked lists of these hash tables to implement variable scoping.
%% There are different variable scopes for shell function calls and
%% temporary scopes for variables set by assignment statements preceding
%% a command.  When those assignment statements precede a command that is
%% built into the shell, for instance, the shell has to keep track of the
%% correct order in which to resolve variable references, and the linked
%% scopes allow bash to do that.  There can be a surprising number of
%% scopes to traverse depending on the execution nesting level.
Bashは、ハッシュテーブルを使ってシェル変数の格納や取得を行う。また、そのハッシュテーブルの連結リストで変数のスコープを実装する。シェル関数の呼び出し用にさまざまなスコープがあり、コマンドの前にある代入文で設定した変数用のテンポラリスコープもある。代入文の後にシェルの組み込みコマンドが続くときは、シェルは変数の参照の解決順序を覚えておく必要がある。また、連結したスコープがbashにそれを許可しなければならない。実行のネストレベルによっては、走査するスコープの数が驚くほど多くなることもありえる。

\end{aosasect2}

%% \begin{aosasect2}{The Shell Programming Language}
\begin{aosasect2}{シェルプログラミング言語}

%% A \emph{simple} shell command, one with which most readers are most
%% familiar, consists of a command name, such as \code{echo} or
%% \code{cd}, and a list of zero or more arguments and redirections.
%% Redirections allow the shell user to control the input to and output
%% from invoked commands.  As noted above, users can define variables
%% local to simple commands.
\emph{単純な}シェルコマンド、つまり読者の多くが最も見なれているであろうコマンドは、まず\code{echo}や\code{cd}のようなコマンド名があってその後にゼロ個以上の引数やリダイレクトが続く。リダイレクトを使うと、起動するコマンドへの入力やコマンドからの出力をシェルのユーザーが制御できるようになる。先ほど説明したように、単純なコマンド内のローカル変数を定義することができる。

%% Reserved words introduce more complex shell commands.  There are
%% constructs common to any high-level programming language, such as
%% \code{if-then-else}, \code{while}, a \code{for} loop that iterates
%% over a list of values, and a C-like arithmetic \code{for} loop.
%% These more complex commands allow the shell to execute a
%% command or otherwise test a condition and perform different operations
%% based on the result, or execute commands multiple times.
予約語を使えば、より複雑なシェルコマンドを実行できる。他の高級言語にもよくある制御構造である\code{if-then-else}や\code{while}も使えるし、\code{for}ループで値のリストを順に処理することもできる。また、C言語風にカウンタを用いた\code{for}ループも使える。これらの複雑なコマンドを使えば、ある条件を調べてその結果によって処理を切り替えるようなコマンドを実行することもできるし、あるコマンドを複数回実行することもできる。

%% One of the gifts Unix brought the computing world is the pipeline: a
%% linear list of commands, in which the output of one command in the
%% list becomes the input of the next.  Any shell construct can be used
%% in a pipeline, and it's not uncommon to see pipelines in which a
%% command feeds data to a loop.
Unixが計算機界にもたらした贈り物のひとつがパイプラインである。これを使えば、一連のコマンド群でひとつのコマンドの出力を次のコマンドへの入力とすることができる。シェルの制御構造はすべてパイプラインの中でも使え、あるコマンドがデータをループに送るようなパイプラインを見ることも珍しくない。

%% Bash implements a facility that allows the standard input, standard
%% output, and standard error streams for a command to be redirected to
%% another file or process when the command is invoked.  Shell
%% programmers can also use redirection to open and close files in the
%% current shell environment.
Bashには、あるコマンドの実行時に標準入力や標準出力そして標準エラー出力をリダイレクトして別のファイルやプロセスに送る機能がある。シェルプログラマーは、リダイレクトを使って現在のシェル環境でファイルを開いたり閉じたりすることができる。

%% Bash allows shell programs to be stored and used more than once.
%% Shell functions and shell scripts are both ways to name a group of
%% commands and execute the group, just like executing any other command.
%% Shell functions are declared using a special syntax and stored and
%% executed in the same shell's context; shell scripts are created by
%% putting commands into a file and executing a new instance of the shell
%% to interpret them.  Shell functions share most of the execution
%% context with the shell that calls them, but shell scripts, since they
%% are interpreted by a new shell invocation, share only what is passed
%% between processes in the environment.
Bashでは、シェルのプログラムを保存して再利用することができる。シェル関数やシェルスクリプトは、どちらもコマンド群に名前をつけて実行できるようにしたものであり、他のコマンドと同じように実行できる。シェル関数の宣言は特別な構文で行い、同じシェルのコンテキストで使うことができる。シェルスクリプトはコマンドを書いたファイルとして作り、実行するときにはそれを解釈する新たなシェルのインスタンスを立ち上げる。シェル関数は大半の実行時コンテキストを呼び出し元のシェルと共有するが、シェルスクリプトは新たなシェルを立ち上げて動作するので、環境変数で渡された内容しか共有できない。

\end{aosasect2}

%% \begin{aosasect2}{A Further Note}
\begin{aosasect2}{さらなる注意}

%% As you read further, keep in mind that the shell implements its
%% features using only a few data structures: arrays, trees,
%% singly-linked and doubly-linked lists, and hash tables.  Nearly all of
%% the shell constructs are implemented using these primitives.
さらに読み進めていくうえで覚えておいてほしいのは、シェルがその機能を実装するために使っているデータ構造はほんのわずかであるということだ。配列、ツリー、片方向連結リスト、双方向連結リスト、そしてハッシュテーブル。これだけである。シェルのほぼすべての構造が、これらのプリミティブを用いて実装されている。

%% The basic data structure the shell uses to pass information from one
%% stage to the next, and to operate on data units within each processing
%% stage, is the \code{WORD\_DESC}:
あるステージから次のステージに情報を渡したり各処理ステージでデータを操作したりするときに使う基本的なデータ構造が\code{WORD\_DESC}だ。

\begin{verbatim}
typedef struct word_desc {
  char *word;           /* Zero terminated string. */
  int flags;            /* Flags associated with this word. */
} WORD_DESC;
\end{verbatim}

%% \noindent Words are combined into, for example, argument lists, using simple
%% linked lists:
\noindent 単語を組み合わせて引数リストなどを作るときには、単純な連結リストを使う。

\begin{verbatim}
typedef struct word_list {
  struct word_list *next;
  WORD_DESC *word;
} WORD_LIST;
\end{verbatim}

%% \code{WORD\_LIST}s are pervasive throughout the shell.  A simple
%% command is a word list, the result of expansion is a word list, and
%% the built-in commands each take a word list of arguments.
\code{WORD\_LIST}はシェル全体に広がる。単純なコマンドは単語のリストだし、その展開結果も単語のリスト、そして組み込みのコマンドも引数の一覧を単語のリストで受け取る。

\end{aosasect2}

\end{aosasect1}

%% \begin{aosasect1}{Input Processing}
\begin{aosasect1}{入力の処理}

%% The first stage of the bash processing pipeline is input processing:
%% taking characters from the terminal or a file, breaking them into
%% lines, and passing the lines to the shell parser to transform into
%% commands.  As you would expect, the lines are
%% sequences of characters terminated by newlines.
bashのパイプライン処理における最初のステージは、入力の処理である。ターミナルあるいはファイルから文字を受け取り、それを行単位に分け、各行をパーサに渡してコマンドに変換する。想像がつくだろうが、行とは改行文字で終わる文字列のことだ。

%% \begin{aosasect2}{Readline and Command Line Editing}
\begin{aosasect2}{Readlineおよびコマンドラインの編集}

%% Bash reads input from the terminal when interactive, and from the
%% script file specified as an argument otherwise.  When interactive,
%% bash allows the user to edit command lines as they are typed in, using
%% familiar key sequences and editing commands similar to the Unix emacs
%% and vi editors.
Bashは、対話モードのときにはターミナルから入力を読み込み、それ以外の場合は引数で指定したスクリプトファイルから入力を読み込む。対話モードのときは、ユーザーが入力したコマンドラインを編集することができる。編集時には、Unixのエディタemacsやviとよく似たキーシーケンスや編集コマンドが使える。

%% Bash uses the readline library to implement command line editing.
%% This provides a set of functions allowing users to edit command lines,
%% functions to save command lines as they are entered, to recall
%% previous commands, and to perform csh-like history expansion.  Bash is
%% readline's primary client, and they are developed together, but there
%% is no bash-specific code in readline.  Many other projects have
%% adopted readline to provide a terminal-based line editing interface.
Bashはreadlineライブラリを使ってコマンドラインの編集を実装している。このライブラリが提供する関数を使うと、コマンドラインの編集や入力内容の保存、過去のコマンドの呼び出し、そしてcsh風の履歴の展開ができるようになる。Readlineはもともとbash用に開発されたものであり、今でも一緒に開発が進められているが、readlineにはbash固有のコードは一切含まれていない。多くのプロジェクトが、readlineを使ってターミナルベースの行編集インターフェイスを提供している。

%% Readline also allows users to bind key sequences of unlimited length
%% to any of a large number of readline commands.  Readline has commands
%% to move the cursor around the line, insert and remove text, retrieve
%% previous lines, and complete partially-typed words.  On top of this,
%% users may define macros, which are strings of characters that are
%% inserted into the line in response to a key sequence, using the same
%% syntax as key bindings.  Macros afford readline users a simple string
%% substitution and shorthand facility.
Readlineには、任意の長さのキーシーケンスをreadlineコマンドにバインドする機能もある。Readlineには、カーソルの移動やテキストの挿入・削除、前の行の取得、そして途中まで入力した単語の補完などに対応するコマンドがある。これらのコマンドを使い、ユーザーはキーバインドと同じ構文でマクロを定義できる。マクロとは、キーシーケンスに対応して挿入される文字列のことである。マクロのおかげで、readlineのユーザーはちょっとした文字列置換や作業の短縮をできるようになる。

%% \begin{aosasect3}{Readline Structure}
\begin{aosasect3}{Readlineの構造}

Readline is structured as a basic read/dispatch/execute/redisplay
loop.  It reads characters from the keyboard using \code{read} or
equivalent, or obtains input from a macro.  Each character is used as
an index into a keymap, or dispatch table.  Though indexed by a single
eight-bit character, the contents of each element of the keymap can be
several things.  The characters can resolve to additional keymaps,
which is how multiple-character key sequences are possible.  Resolving
to a readline command, such as \code{beginning-of-line}, causes that
command to be executed.
A character bound to the \code{self-insert} command is stored into the
editing buffer.
It's also possible to bind a key sequence to
a command while simultaneously binding subsequences to different
commands (a relatively recently-added feature); there is a special
index into a keymap to indicate that this is done.  Binding a key
sequence to a macro provides a great deal of flexibility, from
inserting arbitrary strings into a command line to creating
keyboard shortcuts for complex editing sequences.  Readline stores
each character bound to \code{self-insert} in the
editing buffer, which when displayed may occupy one or more lines on
the screen.

Readline manages only character buffers and strings using C
\code{char}s, and builds multibyte characters out of them if
necessary.  It does not use \code{wchar\_t} internally for both speed
and storage reasons, and because the editing code existed before
multibyte character support became widespread.  When in a locale that
supports multibyte characters, readline automatically reads an entire
multibyte character and inserts it into the editing buffer.  It's
possible to bind multibyte characters to editing commands, but one has
to bind such a character as a key sequence; this is possible, but
difficult and usually not wanted.  The existing emacs and vi command
sets do not use multibyte characters, for instance.

Once a key sequence finally resolves to an editing command,
readline updates the terminal display to reflect the
results.
This happens regardless of whether the command
results in characters being inserted into the buffer, the editing
position being moved, or the line being partially or completely
replaced.
Some bindable editing commands, such as those that modify
the history file, do not cause any change to the contents of the
editing buffer.

Updating the terminal display, while seemingly simple, is quite
involved.  Readline has to keep track of three things: the current
contents of the buffer of characters displayed on the screen, the
updated contents of that display buffer, and the actual characters
displayed.  In the presence of multibyte characters, the characters
displayed do not exactly match the buffer, and the redisplay engine
must take that into account.  When redisplaying, readline must compare
the current display buffer's contents with the updated buffer, figure
out the differences, and decide how to most efficiently modify the
display to reflect the updated buffer.  This problem has been the
subject of considerable research through the years (the
\emph{string-to-string correction problem}).  Readline's approach is to
identify the beginning and end of the portion of the buffer that
differs, compute the cost of updating just that portion, including
moving the cursor backward and forward (e.g., will it take more effort
to issue terminal commands to delete characters and then insert new
ones than to simply overwrite the current screen contents?), perform
the lowest-cost update, then clean up by removing any characters
remaining at the end of the line if necessary and position the cursor
in the correct spot.

The redisplay engine is without question the one piece of readline
that has been modified most heavily.  Most of the changes have been to
add functionality---most significantly, the ability to have
non-displaying characters in the prompt (to change colors, for instance)
and to cope with characters
that take up more than a single byte.

Readline returns the contents of the editing buffer to the calling
application, which is then responsible for saving the
possibly-modified results in the history list.

\end{aosasect3}

\begin{aosasect3}{Applications Extending Readline}

Just as readline offers users a variety of ways to customize and
extend readline's default behavior, it provides a number of mechanisms
for applications to extend its default feature set.  First, bindable
readline functions accept a standard set of arguments and return a
specified set of results, making it easy for applications to extend
readline with application-specific functions.  Bash, for instance,
adds more than thirty bindable commands, from bash-specific word
completions to interfaces to shell built-in commands.

The second way readline allows applications to modify its behavior is
through the pervasive use of pointers to hook functions with
well-known names and calling interfaces.  Applications can replace
some portions of readline's internals, interpose functionality in
front of readline, and perform application-specific
transformations.

\end{aosasect3}

\end{aosasect2}

\begin{aosasect2}{Non-interactive Input Processing}

When the shell is not using readline, it uses either \code{stdio} or its own
buffered input routines to obtain input.  The bash buffered input
package is preferable to \code{stdio} when the shell is not interactive
because of the somewhat peculiar restrictions Posix imposes on input
consumption: the shell must consume only the input necessary to parse
a command and leave the rest for executed programs.  This is
particularly important when the shell is reading a script from the
standard input.  The shell is allowed to buffer input as much as it
wants, as long as it is able to roll the file offset back to just
after the last character the parser consumes.  As a practical matter,
this means that the shell must read scripts a character at a time when
reading from non-seekable devices such as pipes, but may buffer as
many characters as it likes when reading from files.

These idiosyncrasies aside, the output of the non-interactive input
portion of shell processing is the same as readline: a buffer of
characters terminated by a newline.

\end{aosasect2}

\begin{aosasect2}{Multibyte Characters}

Multibyte character processing was added to the shell a long time
after its initial implementation, and it was done in a way designed to
minimize its impact on the existing code.  When in a locale that
supports multibyte characters, the shell stores its input in a buffer
of bytes (C \code{char}s), but treats these bytes as potentially
multibyte characters.  Readline understands how to display multibyte
characters (the key is knowing how many screen positions a multibyte
character occupies, and how many bytes to consume from a buffer when
displaying a character on the screen), how to move forward and
backward in the line a character at a time, as opposed to a byte at a
time, and so on.  Other than that, multibyte characters don't have
much effect on shell input processing.  Other parts of the shell,
described later, need to be aware of multibyte characters and take
them into account when processing their input.

\end{aosasect2}

\end{aosasect1}

\begin{aosasect1}{Parsing}

The initial job of the parsing engine is lexical analysis: to separate
the stream of characters into words and apply meaning to the result.
The word is the basic unit on which the parser operates.  Words are
sequences of characters separated by metacharacters, which include
simple separators like spaces and tabs, or characters that are special
to the shell language, like semicolons and ampersands.

One historical problem with the shell, as Tom Duff said in his paper
about \code{rc}, the Plan 9 shell, is that nobody really knows what
the Bourne shell grammar is.  The Posix shell committee deserves
significant credit for finally publishing a definitive grammar for a
Unix shell, albeit one that has plenty of context dependencies.  That
grammar isn't without its problems---it disallows some constructs that
historical Bourne shell parsers have accepted without error---but it's
the best we have.

The bash parser is derived from an early version of the Posix grammar,
and is, as far as I know, the only Bourne-style shell parser
implemented using Yacc or Bison.  This has presented its own set of
difficulties---the shell grammar isn't really well-suited to
yacc-style parsing and requires some complicated lexical analysis and
a lot of cooperation between the parser and the lexical analyzer.

In any event, the lexical analyzer takes lines of input from readline
or another source, breaks them into tokens at metacharacters,
identifies the tokens based on context, and passes them on to the
parser to be assembled into statements and commands.  There is a lot
of context involved---for instance, the word \code{for} can be a
reserved word, an identifier, part of an assignment statement, or
other word, and the following is a perfectly valid command:

\begin{verbatim}
for for in for; do for=for; done; echo $for
\end{verbatim}

\noindent that displays \code{for}.

At this point, a short digression about aliasing is in order.  Bash
allows the first word of a simple command to be replaced with
arbitrary text using aliases.  Since they're completely lexical,
aliases can even be used (or abused) to change the shell grammar: it's
possible to write an alias that implements a compound command that
bash doesn't provide.  The bash parser implements aliasing completely
in the lexical phase, though the parser has to inform the analyzer
when alias expansion is permitted.

Like many programming languages, the shell allows characters to be
escaped to remove their special meaning, so that metacharacters such as
\code{\&} can appear in commands.  There are three types of quoting,
each of which is slightly different and permits slightly different
interpretations of the quoted text: the backslash, which escapes the
next character; single quotes, which prevent interpretation of all
enclosed characters; and double quotes, which prevent some
interpretation but allow certain word expansions (and treats
backslashes differently).  The lexical analyzer interprets quoted
characters and strings and prevents them from being recognized by the
parser as reserved words or metacharacters.  There are also two
special cases, \code{\$'...'} and \code{\$"..."}, that expand
backslash-escaped characters in the same fashion as ANSI C strings and
allow characters to be translated using standard internationalization
functions, respectively.  The former is widely used; the latter,
perhaps because there are few good examples or use cases, less so.

The rest of the interface between the parser and lexical analyzer is
straightforward.  The parser encodes a certain amount of state and
shares it with the analyzer to allow the sort of context-dependent
analysis the grammar requires.  For example, the lexical analyzer
categorizes words according to the token type: reserved word (in the
appropriate context), word, assignment statement, and so on.  In order
to do this, the parser has to tell it something about how far it has
progressed parsing a command, whether it is processing a
multiline string (sometimes called a ``here-document''),
whether it's in a case statement or a conditional
command, or whether it is processing an extended shell pattern or compound
assignment statement.

Much of the work to recognize the end of the command substitution
during the parsing stage is encapsulated into a single function
(\code{parse\_comsub}), which knows an uncomfortable amount of shell
syntax and duplicates rather more of the token-reading code than is
optimal.  This function has to know about here documents, shell
comments, metacharacters and word boundaries, quoting, and when
reserved words are acceptable (so it knows when it's in a \code{case}
statement); it took a while to get that right.

When expanding a
command substitution during word expansion, bash uses the parser to
find the correct end of the construct.  This is similar to turning a
string into a command for \code{eval}, but in this
case the command isn't terminated by the end of the string.  In order
to make this work, the parser must recognize a right parenthesis as a
valid command terminator, which leads to special cases in a number of
grammar productions and requires the lexical analyzer to flag a right
parenthesis (in the appropriate context) as denoting EOF\@.  The parser
also has to save and restore parser state before recursively invoking
\code{yyparse}, since a command substitution can be parsed and
executed as part of expanding a prompt string in the middle of reading
a command.  Since the input functions implement read-ahead, this
function must finally take care of rewinding the bash input pointer to
the right spot, whether bash is reading input from a string, a file,
or the terminal using readline.  This is important not only so that
input is not lost, but so the command substitution expansion functions
construct the correct string for execution.

Similar problems are posed by programmable word completion, which allows
arbitrary commands to be executed while parsing another command,
and solved by saving and restoring parser state around invocations.

Quoting is also a source of incompatibility and debate.  Twenty years
after the publication of the first Posix shell standard, members of
the standards working group are still debating the proper behavior of
obscure quoting.  As before, the Bourne shell is no help other than as
a reference implementation to observe behavior.

The parser returns a single C structure representing a command (which,
in the case of compound commands like loops, may include other
commands in turn) and passes it to the next stage of the shell's
operation: word expansion.  The command structure is composed of
command objects and lists of words.  Most of the word lists are
subject to various transformations, depending on their context, as
explained in the following sections.

\end{aosasect1}

\begin{aosasect1}{Word Expansions}

After parsing, but before execution, many of the words produced by the
parsing stage are subjected to one or more word expansions, so that
(for example) \code{\$OSTYPE} is replaced with the string
\code{"linux-gnu"}.

\begin{aosasect2}{Parameter and Variable Expansions}

Variable expansions are the ones users find most familiar.  Shell
variables are barely typed, and, with few exceptions, are treated as
strings.  The expansions expand and transform these strings into new
words and word lists.

There are expansions that act on the variable's value itself.
Programmers can use these to produce substrings of a variable's
value, the value's length, remove portions that match a specified
pattern from the beginning or end, replace portions of the value
matching a specified pattern with a new string, or modify the case of
alphabetic characters in a variable's value.

In addition, there are expansions that depend on the state of a
variable: different expansions or assignments happen based on whether
or not the variable is set.  For instance,
\code{\$\{parameter:-word\}} will expand to \code{parameter} if it's
set, and \code{word} if it's not set or set to the empty string.

\end{aosasect2}

\begin{aosasect2}{And Many More}

Bash does many other kinds of expansion, each of which has its own
quirky rules.  The first in processing order is brace expansion, which
turns:

\begin{verbatim}
pre{one,two,three}post
\end{verbatim}

\noindent into:

\begin{verbatim}
preonepost pretwopost prethreepost
\end{verbatim}

There is also command substitution, which is a nice marriage of the
shell's ability to run commands and manipulate variables.  The shell
runs a command, collects the output, and uses that output as the value
of the expansion.

One of the problems with command substitution is that it runs the
enclosed command immediately and waits for it to complete: there's
no easy way for the shell to send input to it.  Bash uses a feature
named process substitution, a sort of combination of command
substitution and shell pipelines, to compensate for these
shortcomings.  Like command substitution, bash runs a command, but
lets it run in the background and doesn't wait for it to complete.
The key is that bash opens a pipe to the command for reading or
writing and exposes it as a filename, which becomes the result of the
expansion.

Next is tilde expansion.  Originally intended to turn
\code{{\textasciitilde}alan} into a
reference to Alan's home directory, it has grown over the years into a
way to refer to a large number of different directories.

Finally, there is arithmetic expansion.  \code{\$((expression))}
causes \code{expression} to be evaluated according to the same rules
as C language expressions.  The result of the expression becomes the
result of the expansion.

Variable expansion is where the difference between single and double
quotes becomes most apparent.  Single quotes inhibit all
expansions---the characters enclosed by the quotes pass through the
expansions unscathed---whereas double quotes permit some expansions
and inhibit others.  The word expansions and command, arithmetic, and
process substitution take place---the double quotes only affect how
the result is handled---but brace and tilde expansion do not.

\end{aosasect2}

\begin{aosasect2}{Word Splitting}

The results of the word expansions are split using the characters in
the value of the shell variable \code{IFS} as delimiters.  This is how
the shell transforms a single word into more than one.  Each time one
of the characters in \code{\$IFS}\footnote{In most cases, a sequence of one
of the characters.} appears in the result, bash splits the word into
two.  Single and double quotes both inhibit word splitting.

\end{aosasect2}

\begin{aosasect2}{Globbing}

After the results are split, the shell interprets each word resulting
from the previous expansions as a potential pattern and tries to match
it against an existing filename, including any leading directory path.

\end{aosasect2}

\begin{aosasect2}{Implementation}

If the basic architecture of the shell parallels a pipeline, the word
expansions are a small pipeline unto themselves.  Each stage of word
expansion takes a word and, after possibly transforming it, passes it
to the next expansion stage.  After all the word expansions have been
performed, the command is executed.

The bash implementation of word expansions builds on the basic data
structures already described.  The words output by the parser are
expanded individually, resulting in one or more words for each input
word.  The \code{WORD\_DESC} data structure has proved versatile
enough to hold all the information required to encapsulate the
expansion of a single word. The flags are used to encode information
for use within the word expansion stage and to pass information from
one stage to the next. For instance, the parser uses a flag to tell
the expansion and command execution stages that a particular word is a
shell assignment statement, and the word expansion code uses flags
internally to inhibit word splitting or note the presence of a quoted
null string (\code{"\$x"}, where \code{\$x} is unset or has a null
value).  Using a single character string for each word being expanded,
with some kind of character encoding to represent additional
information, would have proved much more difficult.

As with the parser, the word expansion code handles characters whose
representation requires more than a single byte.  For example, the
variable length expansion (\code{\${\#variable}}) counts the length in
characters, rather than bytes, and the code can correctly identify the
end of expansions or characters special to expansions in the presence
of multibyte characters.

\end{aosasect2}

\end{aosasect1}

\begin{aosasect1}{Command Execution}

The command execution stage of the internal bash pipeline is where the
real action happens.  Most of the time, the set of expanded words is
decomposed into a command name and set of arguments, and passed to the
operating system as a file to be read and executed with the remaining
words passed as the rest of the elements of \code{argv}.

The description thus far has deliberately concentrated on what Posix
calls simple commands---those with a command name and a set of
arguments.  This is the most common type of command, but bash provides
much more.

The input to the command execution stage is the command structure
built by the parser and a set of possibly-expanded words.  This is
where the real bash programming language comes into play.  The
programming language uses the variables and expansions discussed
previously, and implements the constructs one would expect in a
high-level language: looping, conditionals, alternation, grouping,
selection, conditional execution based on pattern matching, expression
evaluation, and several higher-level constructs specific to the shell.

\begin{aosasect2}{Redirection}

One reflection of the shell's role as an interface to the operating
system is the ability to redirect input and output to and from the
commands it invokes.  The redirection syntax is one of the things that
reveals the sophistication of the shell's early users: until very
recently, it required users to keep track of the file descriptors they
were using, and explicitly specify by number any other than standard
input, output, and error.

A recent addition to the redirection syntax allows users to direct the
shell to choose a suitable file descriptor and assign it to a
specified variable, instead of having the user choose one.  This
reduces the programmer's burden of keeping track of file descriptors,
but adds extra processing: the shell has to duplicate file descriptors
in the right place, and make sure they are assigned to the specified
variable.  This is another example of how information is passed from
the lexical analyzer to the parser through to command execution: the
analyzer classifies the word as a redirection containing a variable
assignment; the parser, in the appropriate grammar production, creates
the redirection object with a flag indicating assignment is required;
and the redirection code interprets the flag and ensures that the file
descriptor number is assigned to the correct variable.

The hardest part of implementing redirection is remembering how to
undo redirections.  The shell deliberately blurs the distinction
between commands executed from the filesystem that cause the creation
of a new process and commands the shell executes itself (builtins),
but, no matter how the command is implemented, the effects of
redirections should not persist beyond the command's completion\footnote{The
\code{exec} builtin is an exception to this rule.}. The shell therefore has
to keep track of how to undo the effects of each redirection,
otherwise redirecting the output of a shell builtin would change the
shell's standard output.  Bash knows how to undo each type of
redirection, either by closing a file descriptor that it allocated, or
by saving file descriptor being duplicated to and restoring it later
using \code{dup2}.  These use the same redirection objects as those
created by the parser and are processed using the same functions.

Since multiple redirections are implemented as simple lists of
objects, the redirections used to undo are kept in a separate list.
That list is processed when a command completes, but the shell has to
take care when it does so, since redirections attached to a shell
function or the ``\code{.}'' builtin must stay in effect until that
function or builtin completes.  When it doesn't invoke a command, the
\code{exec} builtin causes the undo list to simply be discarded,
because redirections associated with \code{exec} persist in the shell
environment.

The other complication is one bash brought on itself.  Historical
versions of the Bourne shell allowed the user to manipulate only file
descriptors 0-9, reserving descriptors 10 and above for the shell's
internal use.  Bash relaxed this restriction, allowing a user to
manipulate any descriptor up to the process's open file limit.  This
means that bash has to keep track of its own internal file
descriptors, including those opened by external libraries and not
directly by the shell, and be prepared to move them around on demand.
This requires a lot of bookkeeping, some heuristics involving the
close-on-exec flag, and yet another list of redirections to be
maintained for the duration of a command and then either processed or
discarded.

\end{aosasect2}

\begin{aosasect2}{Builtin Commands}

Bash makes a number of commands part of the shell itself.  These
commands are executed by the shell, without creating a new process.

The most common reason to make a command a builtin is to maintain or
modify the shell's internal state.  \code{cd} is a good example; one
of the classic exercises for introduction to Unix classes is to
explain why \code{cd} can't be implemented as an external command.

Bash builtins use the same internal primitives as the rest of the
shell.  Each builtin is implemented using a C language function that
takes a list of words as arguments.  The words are those output by the
word expansion stage; the builtins treat them as command names and
arguments.  For the most part, the builtins use the same standard
expansion rules as any other command, with a couple of exceptions: the
bash builtins that accept assignment statements as arguments (e.g.,
\code{declare} and \code{export}) use the same expansion rules for the
assignment arguments as those the shell uses for variable assignments.
This is one place where the \code{flags} member of the
\code{WORD\_DESC} structure is used to pass information between one
stage of the shell's internal pipeline and another.

\end{aosasect2}

\begin{aosasect2}{Simple Command Execution}

Simple commands are the ones most commonly encountered.  The search
for and execution of commands read from the filesystem, and
collection of their exit status, covers many of the shell's remaining
features.

Shell variable assignments (i.e., words of the form \code{var=value}) are a
kind of simple command themselves.  Assignment statements can either
precede a command name or stand alone on a command line.  If they
precede a command, the variables are passed to the executed command in
its environment (if they precede a built-in command or shell function,
they persist, with a few exceptions, only as long as the builtin or
function executes).  If they're not followed by a command name, the
assignment statements modify the shell's state.

When presented a command name that is not the name of a shell function
or builtin, bash searches the filesystem for an executable file with
that name.  The value of the \code{PATH} variable is used as a
colon-separated list of directories in which to search.  Command names
containing slashes (or other directory separators) are not looked up,
but are executed directly.

When a command is found using a \code{PATH} search, bash saves the
command name and the corresponding full pathname in a hash table,
which it consults before conducting subsequent \code{PATH} searches.
If the command is not found, bash executes a specially-named function,
if it's defined, with the command name and arguments as arguments to
the function.  Some Linux distributions use this facility to offer to
install missing commands.

If bash finds a file to execute, it forks and creates a new execution
environment, and executes the program in this new environment.  The
execution environment is an exact duplicate of the shell environment,
with minor modifications to things like signal disposition and files
opened and closed by redirections.

\end{aosasect2}

\begin{aosasect2}{Job Control}

The shell can execute commands in the foreground, in which it waits
for the command to finish and collects its exit status, or the
background, where the shell immediately reads the next command.  Job
control is the ability to move processes (commands being executed)
between the foreground and background, and to suspend and resume their
execution.  To implement this, bash introduces the concept of a job,
which is essentially a command being executed by one or more
processes. A pipeline, for instance, uses one process for each of its elements.
The process group is a way to join separate processes
together into a single job.  The terminal has a process group ID
associated with it, so the foreground process group is the one whose
process group ID is the same as the terminal's.

The shell uses a few simple data structures in its job control
implementation.  There is a structure to represent a child process,
including its process ID, its state, and the status it returned when
it terminated.  A pipeline is just a simple linked list of these
process structures.  A job is quite similar: there is a list of
processes, some job state (running, suspended, exited, etc.), and the
job's process group ID\@.  The process list usually consists of a single
process; only pipelines result in more than one process being
associated with a job.  Each job has a unique process group ID, and
the process in the job whose process ID is the same as the job's
process group ID is called the process group leader.  The current set
of jobs is kept in an array, conceptually very similar to how it's
presented to the user.  The job's state and exit status are assembled
by aggregating the state and exit statuses of the constituent
processes.

Like several other things in the shell, the complex part about
implementing job control is bookkeeping.  The shell must take care to
assign processes to the correct process groups, make sure that child
process creation and process group assignment are synchronized, and
that the terminal's process group is set appropriately, since the
terminal's process group determines the foreground job (and, if it's
not set back to the shell's process group, the shell itself won't be
able to read terminal input).  Since it's so process-oriented, it's
not straightforward to implement compound commands such as
\code{while} and \code{for} loops so an entire loop can be stopped and
started as a unit, and few shells have done so.

\end{aosasect2}

\begin{aosasect2}{Compound Commands}

Compound commands consist of lists of one or more simple commands and
are introduced by a keyword such as \code{if} or \code{while}.  This
is where the programming power of the shell is most visible and
effective.

The implementation is fairly unsurprising.  The parser constructs
objects corresponding to the various compound commands, and interprets
them by traversing the object.  Each compound command is implemented
by a corresponding C function that is responsible for performing the
appropriate expansions, executing commands as specified, and altering
the execution flow based on the command's return status.  The function
that implements the \code{for} command is illustrative.  It must first
expand the list of words following the \code{in} reserved word.  The
function must then iterate through the expanded words, assigning each
word to the appropriate variable, then executing the list of commands
in the \code{for} command's body.  The for command doesn't have to
alter execution based on the return status of the command, but it does
have to pay attention to the effects of the \code{break} and
\code{continue} builtins.  Once all the words in the list have been
used, the \code{for} command returns.  As this shows, for the most
part, the implementation follows the description very closely.

\end{aosasect2}

\end{aosasect1}

\begin{aosasect1}{Lessons Learned}

\begin{aosasect2}{What I Have Found Is Important}

I have spent over twenty years working on bash, and I'd like to think
I have discovered a few things.
The most important---one that I can't stress enough---is that it's
vital to have detailed change logs.  It's good when you can go back to
your change logs and remind yourself about why a particular change was
made. It's even better when you can tie that change to a particular
bug report, complete with a reproducible test case, or a suggestion.

If it's appropriate, extensive regression testing is something I would
recommend building into a project from the beginning.  Bash has
thousands of test cases covering virtually all of its non-interactive
features. I have considered building tests for interactive
features---Posix has them in its conformance test suite---but did not
want to have to distribute the framework I judged it would need.

Standards are important.  Bash has benefited from being an
implementation of a standard.  It's important to participate in the
standardization of the software you're implementing.  In addition to
discussions about features and their behavior, having a standard to
refer to as the arbiter can work well.  Of course, it can also work poorly---it
depends on the standard.

External standards are important, but it's good to have
internal standards as well.  I was lucky enough to fall into the GNU
Project's set of standards, which provide plenty of good, practical
advice about design and implementation.

Good documentation is another essential.  If you expect a program to
be used by others, it's worth having comprehensive, clear
documentation.  If software is successful, there will end up being
lots of documentation for it, and it's important that the developer
writes the authoritative version.

There's a lot of good software out there.  Use what you can: for
instance, gnulib has a lot of convenient library functions (once you
can unravel them from the gnulib framework).  So do the BSDs and Mac
OS X\@.  Picasso said "Great artists steal" for a reason.

Engage the user community, but be prepared for occasional criticism,
some that will be head-scratching.  An active user community can be a
tremendous benefit, but one consequence is that people will become
very passionate.  Don't take it personally.

\end{aosasect2}

\begin{aosasect2}{What I Would Have Done Differently}

Bash has millions of users.  I've been educated about the importance
of backwards compatibility.  In some sense, backwards compatibility
means never having to say you're sorry.  The world, however, isn't
quite that simple.  I've had to make incompatible changes from time to
time, nearly all of which generated some number of user complaints,
though I always had what I considered to be a valid reason, whether
that was to correct a bad decision, to fix a design misfeature, or to
correct incompatibilities between parts of the shell.  I would have
introduced something like formal bash compatibility levels
earlier.

Bash's development has never been particularly open.  I have become
comfortable with the idea of milestone releases (e.g., bash-4.2) and
individually-released patches.  There are reasons for doing this: I
accommodate vendors with longer release timelines than the free
software and open source worlds, and I've had trouble in the past with
beta software becoming more widespread than I'd like.  If I had to
start over again, though, I would have considered more frequent
releases, using some kind of public repository.

No such list would be complete without an implementation
consideration.  One thing I've considered multiple times, but never
done, is rewriting the bash parser using straight recursive-descent
rather than using \code{bison}.  I once thought I'd have to do this in
order to make command substitution conform to Posix, but I was able to
resolve that issue without changes that extensive.  Were I starting
bash from scratch, I probably would have written a parser by hand.  It
certainly would have made some things easier.

\end{aosasect2}

\end{aosasect1}

\begin{aosasect1}{Conclusions}

Bash is a good example of a large, complex piece of free software.  It
has had the benefit of more than twenty years of development, and is
mature and powerful.  It runs nearly everywhere, and is used by
millions of people every day, many of whom don't realize it.

Bash has been influenced by many sources, dating back to the original
7th Edition Unix shell, written by Stephen Bourne.  The most
significant influence is the Posix standard, which dictates a
significant portion of its behavior.  This combination of backwards
compatibility and standards compliance has brought its own challenges.

Bash has profited by being part of the GNU Project, which has provided
a movement and a framework in which bash exists.  Without GNU, there
would be no bash.  Bash has also benefited from its active, vibrant
user community.  Their feedback has helped to make bash what it is
today---a testament to the benefits of free software.

\end{aosasect1}

\end{aosachapter}
