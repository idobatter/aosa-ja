\begin{aosachaptertoc}{VisTrails}{s:vistrails}{Juliana Freire, David Koop, Emanuele Santos, \\ Carlos Scheidegger, Claudio Silva, and Huy T.\ Vo}{Juliana Freire, David Koop, Emanuele Santos, \\ \hspace*{0.9cm} Carlos Scheidegger, Claudio Silva, and Huy T.\ Vo}
%% Based on EN-Revision r272

%% VisTrails\footnote{\url{http://www.vistrails.org}} is an open-source
%% system that supports data exploration and visualization.  It includes
%% and substantially extends useful features of scientific workflow and
%% visualization systems.  Like scientific workflow systems such as
%% Kepler and Taverna, VisTrails allows the specification of
%% computational processes which integrate existing applications,
%% loosely-coupled resources, and libraries according to a set of
%% rules. Like visualization systems such as AVS and ParaView, VisTrails makes
%% advanced scientific and information visualization techniques available
%% to users, allowing them to explore and compare different visual
%% representations of their data.  As a result, users can create complex
%% workflows that encompass important steps of scientific discovery, from
%% data gathering and manipulation to complex analyses and
%% visualizations, all integrated in one system.
VisTrails\footnote{\url{http://www.vistrails.org}}は、データの調査や可視化をサポートする
オープンソースのシステムである。科学ワークフローシステムやビジュアライゼーションシステムの便利な機能を
含んでおり、今も成長を続けている。KeplerやTavernaといった科学ワークフローシステムと同様、
VisTrailsでも計算プロセスの指定ができ、既存のアプリケーションや
疎結合のリソースそしてライブラリなどをルールに沿って統合できる。
AVSやParaViewなどのビジュアライゼーションシステムのように、
VisTrailsでも科学的な情報の可視化ができるようになっている。
それを使ってデータを調べたり、さまざまな可視化表現でデータを比較したりできる。
結果としてユーザーは複雑なワークフローを作れるようになる。これは科学的な発見のための重要なステップだ。
データの収集から複雑な解析や可視化まで、すべてが一つのシステムに統合されているのだ。

%% A distinguishing feature of VisTrails is its provenance
%% infrastructure~\cite{bib:freire:vistrails}.  VisTrails captures and maintains a detailed history 
%% of the steps followed and data derived in the course of an exploratory
%% task.  Workflows have traditionally been used to automate repetitive
%% tasks, but in applications that are exploratory in nature, such as
%% data analysis and visualization, very little is repeated---change is
%% the norm. As a user generates and evaluates hypotheses about their
%% data, a series of different, but related, workflows are created as
%% they are adjusted iteratively.
VisTrailsの特徴は、その履歴管理基盤だ~\cite{bib:freire:vistrails}。
VisTrailsは、調査作業中に発生したデータや作業手順を記録して
その履歴を管理する。繰り返すタスクを自動化するときに従来使われてきたのは
ワークフローだが、自然界の調査用のアプリケーション(データ解析や可視化など)
では繰り返しはほとんど発生しない。変化するのが普通だ。
ユーザーは、データに基づいた仮設をたててそれを検証する。
それぞれ異なるけれども関連している一連のワークフローを作り、
繰り返しながら調整していく。

%% VisTrails was designed to manage these rapidly-evolving workflows: it
%% maintains provenance of data products (e.g., visualizations, plots),
%% of the workflows that derive these products, and their executions.
%% The system also provides annotation capabilities so users can enrich
%% the automatically-captured provenance.
VisTrailsは、こういった常に成長し続けるワークフローを管理するために作られた。
生成したデータ(可視化やプロットなど)、そこから派生するワークフロー、
そしてその実行などの履歴を管理する。
またアノテーション機能も提供しており、自動的にキャプチャした履歴を飾ることもできる。

%% Besides enabling reproducible results, VisTrails leverages provenance
%% information through a series of operations and intuitive user
%% interfaces that help users to collaboratively analyze data.  Notably,
%% the system supports reflective reasoning by storing temporary results,
%% allowing users to examine the actions that led to a result and to
%% follow chains of reasoning backward and forward.  Users can navigate
%% workflow versions in an intuitive way, undo changes without losing
%% results, visually compare multiple workflows and show their results
%% side-by-side in a visualization spreadsheet.
結果を再生できるようにするだけでなく、
VisTrailsはその履歴情報を一連の操作として活用する。
直感的なユーザーインターフェイスを使って、共同でデータを解析できるのだ。
特筆すべきなのは、このシステムでは一時的な結果を保存しており、
それを反映した推論にも対応しているという点だ。
これを利用すると、何かの結果に至った操作を吟味して前の推論や次の推論につなげられる。
ユーザーはワークフローの各バージョンを直感的に移動でき、変更を取り消しても一切結果を失わずに済む。
また、複数のワークフローをビジュアルに比較して、その結果をビジュアライゼーションスプレッドシートに並べて表示できる。

%% VisTrails addresses important usability issues that have hampered a
%% wider adoption of workflow and visualization systems.  To cater to a
%% broader set of users, including many who do not have programming
%% expertise, it provides a series of operations and user interfaces that
%% simplify workflow design and use~\cite{bib:freire:vistrails}, including the ability to create and
%% refine workflows by analogy, to query workflows by example, and to
%% suggest workflow completions as users interactively construct their
%% workflows using a recommendation
%% system~\cite{bib:scheidegger:analogy}. We have also developed a new
%% framework that allows the creation of custom applications that can be
%% more easily deployed to (non-expert) end users.
VisTrailsは、ワークフローシステムやビジュアライゼーションシステムを広く取り入れるときの障害になる
重大な使い勝手問題にも対応している。プログラミングの専門知識がない人も含めた幅広いユーザー層に対応するため、
ワークフローの設計や利用を簡単にする一連の操作やユーザーインターフェイスを用意したのだ\cite{bib:freire:vistrails}。
ワークフローの作成や修正にアナロジーを使えるようにしたり、実例でワークフローを問い合わせられるようにしたり、
リコメンデーションシステムで対話的にワークフローを作っていけるようにしたりといったものだ\cite{bib:scheidegger:analogy}。
また、新たなフレームワークも作った。これを使えば、プログラミングに詳しくないエンドユーザーでも
カスタムアプリケーションを作って配布できるようになる。

%% The extensibility of VisTrails comes from an infrastructure that makes
%% it simple for users to integrate tools and libraries, as well as to
%% quickly prototype new functions. This has been instrumental in
%% enabling the use of the system in a wide range of application areas,
%% including environmental sciences, psychiatry, astronomy, cosmology,
%% high-energy physics, quantum physics, and molecular modeling.
VisTrailsの拡張性の元になっているのはその基盤で、ユーザーにとってシンプルになるように作られている。
ツールやライブラリを組み込んだり、お手軽に新機能のプロトタイプを作ったりできるようになっている。
これは、このシステムをさまざまな分野に適用するのに役立っており、
環境科学や精神医学、天文学、宇宙論、高エネルギー物理学、量子物理学、分子モデリングなどで使われ散る。

%% To keep the system open-source and free for all, we have built
%% VisTrails using only free, open-source packages. VisTrails is written
%% in Python and uses Qt as its GUI toolkit (through PyQt Python
%% bindings).  Because of the broad range of users and applications, we
%% have designed the system from the ground up with portability in
%% mind. VisTrails runs on Windows, Mac and Linux.
システムをオープンソースで自由に使えるようにするため、私たちはVisTrails
を作るときに、自由なオープンソースのパッケージだけを使うようにした。
VisTrailsはPythonで書かれており、GUIツールキットにはQtを(そのPythonバインディングであるPyQtと通して)使っている。
さまざまなユーザーがいるしさまざまなアプリケーションがあるので、
システムを設計するときにはポータビリティを重視した。
VisTrailsはWindowsでもMacでもLinuxでも動く。

%% \aosafigure{../images/vistrails/overview.eps}{Components of the VisTrails User Interface}{fig.vistrails.overview}
\aosafigure{../images/vistrails/overview.eps}{VisTrailsのユーザーインターフェイスのコンポーネント}{fig.vistrails.overview}

%% \begin{aosasect1}{System Overview}
\begin{aosasect1}{システムの概要}

%% Data exploration is an inherently creative process that requires users
%% to locate relevant data, to integrate and visualize this data, to
%% collaborate with peers while exploring different solutions, and to
%% disseminate results.  Given the size of data and complexity of
%% analyses that are common in scientific exploration, tools are needed
%% that better support creativity.
データの調査は本質的にクリエイティブな作業であり、いろんなことをしなければならない。
関連するデータを特定したり、そのデータを組み合わせて可視化したり、
別の解法を追求する仲間と共同作業をしたり、自分の成果を広めたりといったことだ。
科学調査の世界での一般的なデータ量や分析の複雑性を考慮すると、
そのクリエイティビティをサポートするためには何らかのツールが必要となる。

%% There are two basic requirements for these tools that go hand in
%% hand. First, it is important to be able to specify the exploration
%% processes using formal descriptions, which ideally, are
%% executable. Second, to reproduce the results of these
%% processes as well as reason about the different steps followed to
%% solve a problem, these tools must have the ability to systematically
%% capture provenance. VisTrails was designed with these requirements in
%% mind.
こういったツールがうまく協調できるようにするためには次の二つの基本要件を満たさないといけない。
まず、調査の手順をフォーマルな形式で指定できること。それが実行可能ならさらに望ましい。
次に、その調査手順をそのまま再生したり、すこし変更して実行したりできること。
作業の履歴を体系的に記録するための仕組みが必要だ。
VisTrailsは、こういった要件を念頭に置いて作られた。

%% \begin{aosasect2}{Workflows and Workflow-Based Systems}
\begin{aosasect2}{ワークフローおよびワークフローベースのシステム}

%% Workflow systems support the creation of pipelines (workflows) that
%% combine multiple tools. As such, they enable the automation of
%% repetitive tasks and result reproducibility. Workflows are rapidly
%% replacing primitive shell scripts in a wide range of tasks, as
%% evidenced by a number of workflow-based applications, both commercial
%% (e.g., Apple's Mac OS X Automator and Yahoo!  Pipes) and academic
%% (e.g., NiPype, Kepler, and Taverna).
ワークフローシステムがサポートするのは、複数のツールを組み合わせたパイプライン(ワークフロー)づくりだ。
繰り返し作業を自動化したり、作業を再生できるようにしたりする。
さまざまなタスクに関するワークフローはすぐにプリミティブなシェルスクリプトに置き換えることができ、
さまざまなワークフローベースのアプリケーションで確かめることができる。
商用ツール(AppleのMac OS X AutomatorやYahoo! Pipesなど)でも学術ツール(NiPypeやKeplerそしてTavernaなど)
でもかまわない。

%% Workflows have a number of advantages compared to scripts and programs
%% written in high-level languages. They provide a simple programming
%% model whereby a sequence of tasks is composed by connecting the
%% outputs of one task to the inputs of another.
%% \aosafigref{fig.vistrails.overview} shows a workflow which reads a CSV
%% file that contains weather observations and creates a scatter plot of
%% the values.
ワークフローにはさまざまな利点があり、スクリプトや高級言語でのプログラムよりも優れている。
シンプルなプログラミングモデルが用意されており、一連のタスクを作るのは
単にあるタスクの出力と別のタスクの入力をつなぐだけのことだ。
\aosafigref{fig.vistrails.overview}は、気象情報を含むCSVファイルを読み込んで
散布図を作るワークフローの例だ。

%% This simpler programming model allows workflow systems to provide
%% intuitive visual programming interfaces, which make them more
%% \emph{suitable for users who do not have substantial programming
%%   expertise}.  Workflows also have an \emph{explicit structure}: they
%% can be viewed as graphs, where nodes represent processes (or modules)
%% along with their parameters and edges capture the flow of data between
%% the processes. In the example of \aosafigref{fig.vistrails.overview},
%% the module \code{CSVReader} takes as a parameter a filename
%% (\code{/weather/temp\_precip.dat}), reads the file, and feeds its
%% contents into the modules \code{GetTemperature} and
%% \code{GetPrecipitation}, which in turn send the temperature and
%% precipitation values to a matplotlib function that generates a scatter
%% plot.
このシンプルなプログラミングモデルのおかげで直感的なビジュアルプログラミングインターフェイスを提供でき、
\emph{プログラミングの経験が浅いユーザーにもより使いやすい}ものとなった。
また、ワークフローは\emph{明確な構造}を持っている。
グラフとして表すことができ、このとき各ノードはプロセス(あるいはモジュール)とそのパラメータを表す。
またエッジはプロセス間のデータの流れを捕らえたものとなる。
\aosafigref{fig.vistrails.overview}の例で考えると、\code{CSVReader}モジュールが
ファイル名(\code{/weather/temp\_precip.dat})をパラメータとして受け取ってそのファイルを読み、
ファイルの中身を\code{GetTemperature}モジュールと\code{GetPrecipitation}
モジュールに渡すことになる。これらがそれぞれ気温と降水量の値をmatplotlibの関数に送って、
散布図を生成する。

%% Most workflow systems are designed for a specific application
%% area. For example, Taverna targets bioinformatics workflows, and
%% NiPype allows the creation of neuroimaging workflows.  While VisTrails
%% supports much of the functionality provided by other workflow systems,
%% it was designed to support general exploratory tasks in a broad range
%% of areas, integrating multiple tools, libraries, and services.
ほとんどのワークフローシステムは、特定の領域に特化した作りになっている。
たとえばTavernaのターゲットはバイオインフォマティクスのワークフローだし、
NiPypeは神経画像のワークフローを作れるようになっている。
VisTrailsは他のワークフローシステムが持つ機能の多くに対応しているが、
より幅広い分野の汎用的な調査タスクに対応するよう作られている。
複数のツールやライブラリ、サービスの統合にも対応している。

\end{aosasect2}

%% \begin{aosasect2}{Data and Workflow Provenance}
\begin{aosasect2}{データおよびワークフローの履歴}

%% The importance of keeping provenance information for results (and data
%% products) is well recognized in the scientific community.  The
%% provenance (also referred to as the audit trail, lineage, and
%% pedigree) of a data product contains information about the process and
%% data used to derive the data product. Provenance provides important
%% documentation that is key to preserving the data, to determining the
%% data's quality and authorship, and to reproducing as well as
%% validating the results~\cite{bib:freire:provenance}.
結果(や生成データ)の履歴情報を保持しておくことの重要性は、自然科学の世界ではよく知られている。
生成データの履歴(あるいは監査証跡や系統、血統などとも呼ばれる)には、
プロセスに関する情報やデータを生成するために利用したデータなどの情報が含まれる。
履歴に含まれる情報は、データを永続化させたりその品質や所有者を調べたり、
結果を再現して検証したりするための重要な鍵となる\cite{bib:freire:provenance}。

%% An important component of provenance is information about
%% \emph{causality}, i.e., a description of a process (sequence of steps)
%% which, together with input data and parameters, caused the creation of
%% a data product. Thus, the structure of provenance mirrors the
%% structure of the workflow (or set of workflows) used to derive a given
%% result set.
履歴の中で重要となる情報の一つが、\emph{因果関係}だ。
これは、プロセス(一連の手順)に関する説明と入力データおよびパラメータをまとめたものであり、
生成データを作るもとになったものだ。
従って、履歴の構造は、指定した結果セットを作るもとになったワークフロー(あるいはワークフロー群)の構造を反映したものとなる。

%% In fact, a catalyst for the widespread use of workflow systems in
%% science has been that they can be easily used to automatically capture
%% provenance.  While early workflow systems have been \emph{extended} to
%% capture provenance, VisTrails was \emph{designed} to support
%% provenance.
実際、自然科学の分野でワークフローシステムが幅広く使われるようになったきっかけは、
履歴情報の取得を簡単に自動化できることだった。初期のワークフローシステムは、
履歴情報を取り込むように\emph{拡張して}使っていた。
一方VisTrailsは、\emph{はじめから}履歴の取得に対応するように作られている。

%% \aosafigure{../images/vistrails/overview-tree.eps}{Provenance of Exploration Enhanced by Annotations}{fig.vistrails.tree}
\aosafigure{../images/vistrails/overview-tree.eps}{探索の履歴のアノテーションによる拡張}{fig.vistrails.tree}

\end{aosasect2}

%% \begin{aosasect2}{User Interface and Basic Functionality}
\begin{aosasect2}{ユーザーインターフェイスおよび基本機能}

%% The different user interface components of the system are illustrated
%% in \aosafigref{fig.vistrails.overview} and
%% \aosafigref{fig.vistrails.tree}. Users create and edit workflows using
%% the Workflow Editor.
%% %
%% To build the workflow graphs, users can drag modules from the Module
%% Registry and drop them into the Workflow Editor canvas. VisTrails
%% provides a series of built-in modules, and users can also add their
%% own (see \aosasecref{sec.vistrails.packages} for details). When a
%% module is selected, VisTrails displays its parameters (in the
%% Parameter Edits area) where the user can set and modify their values.
システムで使われているさまざまなユーザーインターフェイスコンポーネントを
\aosafigref{fig.vistrails.overview}と\aosafigref{fig.vistrails.tree}にまとめた。
ユーザーは、ワークフローを作ったり編集したりするときにワークフローエディタを使う。
ワークフローグラフを作るには、モジュールレジストリからモジュールをドラッグして、
それをワークフローエディタのキャンバスにドロップする。
VisTrailsには組み込みのモジュールが用意されており、
それ以外にユーザーが自分でモジュールを追加することもできる
(詳細は\aosasecref{sec.vistrails.packages}を参照)。
モジュールを選ぶと、VisTrailsはそのパラメータをパラメータ編集領域に表示する。
ユーザーは、その値を設定したり変更したりできる。

%% As a workflow specification is refined, the system captures the
%% changes and presents them to the user in the Version Tree View
%% described below.  Users may interact with the workflows and their
%% results in the VisTrails Spreadsheet.  Each cell in the spreadsheet
%% represents a view that corresponds to a workflow instance.  In
%% \aosafigref{fig.vistrails.overview}, the results of the workflow shown
%% in the Workflow Editor are displayed on the top-left cell of the
%% spreadsheet.  Users can directly modify the parameters of a workflow
%% as well as synchronize parameters across different cells in the
%% spreadsheet.
ワークフローを修正すると、システムがその変更を捕捉して、後述するバージョンツリービューで確認できるようになる。
また、Vistrails Spreadsheetを使えば、ワークフローやその結果を対話的に操作することもできる。
スプレッドシートの各セルが、ワークフローのインスタンスを表す。
\aosafigref{fig.vistrails.overview}では、ワークフローエディタの例で示したワークフローが
スプレッドシートの左上のセルに表示されている。ワークフローのパラメータを直接変更することもできるし、
そのパラメータをスプレッドシート内の別のセルと同期させることもできる。

%% The Version Tree View helps users to navigate through the different
%% workflow versions. As shown in \aosafigref{fig.vistrails.tree}, by
%% clicking on a node in the version tree, users can view a workflow, its
%% associated result (Visualization Preview), and metadata. Some of the
%% metadata is automatically captured, e.g., the id of the user who
%% created a particular workflow and the creation date, but users may
%% also provide additional metadata, including a tag to identify the
%% workflow and a written description.
バージョンツリービューを使えば、一つのワークフローのさまざまなバージョンを渡り歩ける。
\aosafigref{fig.vistrails.tree}に示すように、バージョンツリーのノードをクリックすれば、
ワークフローやそれに関連する結果(ビジュアライゼーションプレビュー)およびメタデータを確認できる。
メタデータの中には自動的に記録されるものもある(ワークフローを作ったユーザーのidや作成日など)が、
自分でメタデータを追加することもできる。たとえば、ワークフローを識別するためのタグやワークフローについての説明などを追加できる。

%% \aosafigureTop[400pt]{../images/vistrails/system-interaction.eps}{VisTrails Architecture}{fig.vistrails.arch}
\aosafigureTop[400pt]{../images/vistrails/system-interaction.eps}{VisTrailsのアーキテクチャ}{fig.vistrails.arch}

\end{aosasect2}

\end{aosasect1}

%% \begin{aosasect1}{Project History}
\begin{aosasect1}{プロジェクトの歴史}

Initial versions of versions of VisTrails were written in Java and
C++~\cite{bib:bavoil:vistrails}.  The C++ version was distributed to a few early adopters, whose 
feedback was instrumental in shaping our requirements for the system.

Having observed a trend in the increase of the number of Python-based
libraries and tools in multiple scientific communities, we opted to
use Python as the basis for VisTrails. Python is quickly becoming a
universal modern glue language for scientific software. Many libraries
written in different languages such as Fortran, C, and C++ use Python
bindings as a way to provide scripting capabilities. Since VisTrails
aims to facilitate the orchestration of many different software
libraries in workflows, a pure Python implementation makes this much
easier. In particular, Python has dynamic code loading features
similar to the ones seen in LISP environments, while having a much
bigger developer community, and an extremely rich standard
library. Late in 2005, we started the development of the current
system using Python/PyQt/Qt. This choice has greatly simplified
extensions to the system, in particular, the addition of new modules
and packages.

A beta version of the VisTrails system was first released in January
2007. Since then, the system has been downloaded over twenty-five
thousand times.

\end{aosasect1}

%% \begin{aosasect1}{Inside VisTrails}
\begin{aosasect1}{インサイドVisTrails}

The internal components that support the user-interface functionality
described above are depicted in the high-level architecture of
VisTrails, shown in \aosafigref{fig.vistrails.arch}.  Workflow
execution is controlled by the Execution Engine, which keeps track of
invoked operations and their respective parameters and captures the
provenance of workflow execution (Execution Provenance). As part of
the execution, VisTrails also allows the caching of intermediate
results both in memory and on disk. As we discuss in
\aosasecref{sec.vistrails.exec-cache}, only new combinations of
modules and parameters are re-run, and these are executed by invoking
the appropriate functions from the underlying libraries (e.g.,
matplotlib).  Workflow results, connected to their provenance, can
then be included in electronic documents
(\aosasecref{sec.vistrails.publish}).

Information about changes to workflows is captured in a Version Tree,
which can be persisted using different storage back ends, including an
XML file store in a local directory and a relational database.
VisTrails also provides a query engine that allows users to explore
the provenance information.

We note that, although VisTrails was designed as an interactive tool,
it can also be used in server mode.  Once workflows are created, they
can be executed by a VisTrails server. This feature is useful in a
number of scenarios, including the creation of Web-based interfaces
that allows users to interact with workflows and the ability to run
workflows in high-performance computing environments.

%% \begin{aosasect2}{The Version Tree: Change-Based Provenance}
\begin{aosasect2}{バージョンツリー: 変更ベースの履歴}

%% \aosafigure{../images/vistrails/change-based-prov.eps}{Change-Based Provenance Model}{fig.vistrails.changeprov}
\aosafigure{../images/vistrails/change-based-prov.eps}{変更ベースの履歴モデル}{fig.vistrails.changeprov}

A new concept we introduced with VisTrails is the notion of provenance
of workflow evolution~\cite{bib:freire:vistrails}. In contrast to previous workflow and 
workflow-based visualization systems, which maintain provenance only
for derived data products, VisTrails treats the workflows as
first-class data items and also captures their provenance. The
availability of workflow-evolution provenance supports reflective
reasoning. Users can explore multiple chains of reasoning without
losing any results, and because the system stores intermediate
results, users can reason about and make inferences from this
information.  It also enables a series of operations which simplify
exploratory processes. For example, users can easily navigate through
the space of workflows created for a given task, visually compare the
workflows and their results (see
\aosafigref{fig.vistrails.changeprov}), and explore (large) parameter
spaces. In addition, users can query the provenance information and
learn by example.

The workflow evolution is captured using the change-based provenance
model.  As illustrated in \aosafigref{fig.vistrails.changeprov},
VisTrails stores the operations or changes that are applied to
workflows (e.g., the addition of a module, the modification of a
parameter, etc.), akin to a database transaction log.  This
information is modeled as a tree, where each node corresponds to a
workflow version, and an edge between a parent and a child node
represents the change applied to the parent to obtain the child. We
use the terms version tree and vistrail (short for \emph{visual
trail}) interchangeably to refer to this tree.  Note that the
change-based model uniformly captures both changes to parameter values
and to workflow definitions. This sequence of changes is sufficient to
determine the provenance of data products and it also captures
information about how a workflow evolves over time.  The model is both
simple and compact---it uses substantially less space than the
alternative of storing multiple \emph{versions} of a workflow.

There are a number of benefits that come from the use of this model.
\aosafigref{fig.vistrails.changeprov} shows the visual difference
functionality that VisTrails provides for comparing two
workflows. Although the workflows are represented as graphs, using the
change-based model, comparing two workflows becomes very simple: it
suffices to navigate the version tree and identify the series of
actions required to transform one workflow into the other.

Another important benefit of the change-based provenance model is that
the underlying version tree can serve as a mechanism to support
collaboration. Because designing workflows is a notoriously difficult
task, it often requires multiple users to collaborate.  Not only does
the version tree provide an intuitive way to visualize the
contribution of different users (e.g., by coloring nodes according to
the user who created the corresponding workflow), but the monotonicity
of the model allows for simple algorithms for synchronizing changes
performed by multiple users.

Provenance information can be easily captured while a workflow
is being executed. Once the execution completes, it is also important
to maintain \emph{strong} links between a data product and its
provenance, i.e., the workflow, parameters and input files used to
derive the data product. When data files or provenance are moved or
modified, it can be difficult to find the data associated with the
provenance or to find the provenance associated with the data.
VisTrails provides a persistent storage mechanism that manages input,
intermediate, and output data files, strengthening the links between
provenance and data.  This mechanism provides better support for
reproducibility because it ensures the data referenced in provenance
information can be readily (and correctly) located.  Another important
benefit of such management is that it allows caching of intermediate
data which can then be shared with other users.

\end{aosasect2}

%% \begin{aosasect2}{Workflow Execution and Caching}
\begin{aosasect2}{ワークフローの実行とキャッシュ}
\label{sec.vistrails.exec-cache}

The execution engine in VisTrails was designed to allow the
integration of new and existing tools and libraries.  We tried to
accommodate different styles commonly used for wrapping third-party
scientific visualization and computation software. In particular,
VisTrails can be integrated with application libraries that exist
either as pre-compiled binaries that are executed on a shell and use
files as input/outputs, or as C++/Java/Python class libraries that
pass internal objects as input/output.

VisTrails adopts a dataflow execution model, where each module
performs a computation and the data produced by a module flows through
the connections that exist between modules. Modules are executed in a
bottom-up fashion; each input is generated on-demand by recursively
executing upstream modules (we say module A is \emph{upstream} of B
when there is a sequence of connections that goes from A to B).  The
intermediate data is temporarily stored either in memory (as a Python
object) or on disk (wrapped by a Python object that contains
information on accessing the data).

To allow users to add their own functionality to VisTrails, we built
an extensible package system (see
\aosasecref{sec.vistrails.packages}).  Packages allow users to include
their own or third-party modules in VisTrails workflows.  A package
developer must identify a set of computational modules and for each,
identify the input and output ports as well as define the computation.
For existing libraries, a compute method needs to specify the
translation from input ports to parameters for the existing function
and the mapping from result values to output ports.

In exploratory tasks, similar workflows, which share common
sub-structures, are often executed in close succession. To improve the
efficiency of workflow execution, VisTrails caches intermediate
results to minimize recomputation.  Because we reuse previous
execution results, we implicitly assume that cacheable modules are
functional: given the same inputs, modules will produce the same
outputs. This requirement imposes definite behavior restrictions on
classes, but we believe they are reasonable. 

There are, however, obvious situations where this behavior
is unattainable.  For example, a module that uploads a file to a
remote server or saves a file to disk has a significant side effect
while its output is relatively unimportant.  Other modules might use
randomization, and their non-determinism might be desirable; such
modules can be flagged as non-cacheable. However, some modules that
are not naturally functional can be converted; a function that writes
data to two files might be wrapped to output the contents of the
files.

\end{aosasect2}

%% \begin{aosasect2}{Data Serialization and Storage}
\begin{aosasect2}{データのシリアライズと格納}

One of the key components of any system supporting provenance is the
serialization and storage of data.  VisTrails originally stored data
in XML via simple \code{fromXML} and \code{toXML} methods embedded in
its internal objects (e.g., the version tree, each module).  To
support the evolution of the schema of these objects, these functions
encoded any translation between schema versions as well.  As the
project progressed, our user base grew, and we decided to support
different serializations, including relational stores.  In
addition, as schema objects evolved, we needed to maintain better
infrastructure for common data management concerns like versioning
schemas, translating between versions, and supporting entity
relationships.  To do so, we added a new database (db) layer.

The db layer is composed of three core components: the domain objects,
the service logic, and the persistence methods.  The domain and
persistence components are versioned so that each schema version has
its own set of classes.  This way, we maintain code to read each
version of the schema.  There are also classes that define
translations for objects from one schema version to those of another.
The service classes provide methods to interface with data and deal
with detection and translation of schema versions.

Because writing much of this code is tedious and repetitive, we use
templates and a meta-schema to define both the object layout (and any
in-memory indices) and the serialization code.  The meta-schema is
written in XML, and is extensible in that serializations other than
the default XML and relational mappings VisTrails defines can be
added.  This is similar to object-relational mappings and frameworks
like Hibernate\footnote{\url{http://www.hibernate.org}} and
SQLObject\footnote{\url{http://www.sqlobject.org}}, but adds some
special routines to automate tasks like re-mapping identifiers and
translating objects from one schema version to the next.  In addition,
we can also use the same meta-schema to generate serialization code
for many languages.  After originally writing meta-Python, where the
domain and persistence code was generated by running Python code with
variables obtained from the meta-schema, we have recently migrated to
Mako templates\footnote{\url{http://www.makotemplates.org}}.

Automatic translation is key for users that need to migrate their data
to newer versions of the system.  Our design adds hooks to make this
translation slightly less painful for developers.  Because we maintain
a copy of code for each version, the translation code just needs to
map one version to another.  At the root level, we define a map to
identify how any version can be transformed to any other.  For distant
versions, this usually involves a chain through multiple intermediate
versions.  Initially, this was a forward-only map, meaning new
versions could not be translated to old versions, but reverse mappings
have been added for more-recent schema mappings.

Each object has an \code{update\_version} method that takes a different
version of an object and returns the current version.  By default, it
does a recursive translation where each object is upgraded by mapping
fields of the old object to those in a new version.  This mapping
defaults to copying each field to one with the same name, but it is
possible to define a method to "override" the default behavior for any
field.  An override is a method that takes the old object and returns
a new version.  Because most changes to the schema only affect a small
number of fields, the default mappings cover most cases, but the
overrides provide a flexible means for defining local changes.

\end{aosasect2}

%% \begin{aosasect2}{Extensibility Through Packages and Python}
\begin{aosasect2}{パッケージやPythonによる拡張性}
\label{sec.vistrails.packages}

The first prototype of VisTrails had a fixed set of modules. It was an
ideal environment to develop basic ideas about the VisTrails version
tree and the caching of multiple execution runs, but it severely
limited long-term utility.

We see VisTrails as infrastructure for computational science, and that
means, literally, that the system should provide scaffolding for other
tools and processes to be developed. An essential requirement of this
scenario is extensibility.  A typical way to achieve this involves
defining a target language and writing an appropriate
interpreter. This is appealing because of the intimate control it
offers over execution. This appeal is amplified in light of our
caching requirements.  However, implementing a full-fledged
programming language is a large endeavor that has never been our
primary goal. More importantly, forcing users who are just trying to
use VisTrails to learn an entirely new language was out of the
question.

We wanted a system which made it easy for a user to add custom
functionality. At the same time, we needed the system to be powerful
enough to express fairly complicated pieces of software. As an
example, VisTrails supports the VTK visualization library\footnote{\url{http://www.vtk.org}}. VTK
contains about 1000 classes, which change depending on compilation,
configuration, and operating system. Since it seems counterproductive
and ultimately hopeless to write different code paths for all these
cases, we decided it was necessary to dynamically determine the set of
VisTrails modules provided by any given package, and VTK naturally
became our model target for a complex package.

Computational science was one of the areas we originally targeted, and
at the time we designed the system, Python was becoming popular as
"glue code" among these scientists. By specifying the behavior of
user-defined VisTrails modules using Python itself, we would all but
eliminate a large barrier for adoption.  As it turns out, Python
offers a nice infrastructure for dynamically-defined classes and
reflection. Almost every definition in Python has an equivalent form
as a first-class expression. The two important reflection features of
Python for our package system are:

\begin{aosaitemize}

  \item Python classes can be defined dynamically via function calls to
  the \code{type} callable. The return value is a representation of a
  class that can be used in exactly the same way that a
  typically-defined Python class can.

  \item Python modules can be imported via function calls to
  \code{\_\_import\_\_}, and the resulting value behaves in the same
  way as the identifier in a standard \code{import} statement. The
  path from which these modules come from can also be specified at runtime.

\end{aosaitemize}

Using Python as our target has a few disadvantages, of course. First
of all, this dynamic nature of Python means that while we would like
to ensure some things like type safety of VisTrails packages, this is
in general not possible.  More importantly, some of the requirements
for VisTrails modules, notably the ones regarding referential
transparency (more on that later) cannot be enforced in Python. Still,
we believe that it is worthwhile to restrict the allowed constructs in
Python via cultural mechanisms, and with this caveat, Python is an
extremely attractive language for software extensibility.

\end{aosasect2}

%% \begin{aosasect2}{VisTrails Packages and Bundles}
\begin{aosasect2}{VisTrailsのパッケージおよびバンドル}

A VisTrails package encapsulates a set of modules. Its most
common representation in disk is the same representation as a Python
package (in a possibly unfortunate naming clash). A Python package
consists of a set of Python files which define Python values such as
functions and classes. A VisTrails package is a Python package that
respects a particular interface. It has files that define specific
functions and variables. In its simplest form, a VisTrails package
should be a directory containing two files: \code{\_\_init\_\_.py} and
\code{init.py}.

The first file \code{\_\_init\_\_.py} is a requirement of Python
packages, and should only contain a few definitions which should be
constant. Although there is no way to guarantee that this is the case,
VisTrails packages failing to obey this are considered buggy. The
values defined in the file include a globally unique identifier for
the package which is used to distinguish modules when workflows are
serialized, and package versions (package versions become important
when handling workflow and package upgrades, see
\aosasecref{sec.vistrails.upgrades}). This file can also include
functions called \code{package\_dependencies} and
\code{package\_requirements}. Since we allow VisTrails modules to
subclass from other VisTrails modules beside the root \code{Module}
class, it is conceivable for one VisTrails package to extend the
behavior of another, and so one package needs to be initialized before
another. These inter-package dependencies are specified by
\code{package\_dependencies}. The \code{package\_requirements}
function, on the other hand, specifies system-level library
requirements which VisTrails, in some cases, can try to automatically
satisfy, through its bundle abstraction.

A bundle is a system-level package that VisTrails manages via
system-specific tools such as RedHat's RPM or Ubuntu's APT\@.  When
these properties are satisfied, VisTrails can determine the package
properties by directly importing the Python module and accessing the
appropriate variables.

The second file, \code{init.py}, contains the entry points for all the
actual VisTrails module definitions. The most important feature of
this file is the definition of two functions, \code{initialize} and
\code{finalize}. The \code{initialize} function is called when a
package is enabled, after all the dependent packages have themselves
been enabled. It performs setup tasks for all of the modules in a
package. The \code{finalize} function, on the other hand, is usually
used to release runtime resources (for example, temporary files
created by the package can be cleaned up).

Each VisTrails module is represented in a package by one Python
class. To register this class in VisTrails, a package developer calls
the \code{add\_module} function once for each VisTrails module.  These
VisTrails modules can be arbitrary Python classes, but they must
respect a few requirements. The first of these is that each must be a
subclass of a basic Python class defined by VisTrails called, perhaps
boringly, \code{Module}.  VisTrails modules can use multiple
inheritance, but only one of the classes should be a VisTrails
module---no diamond hierarchies in the VisTrails module tree are
allowed. Multiple inheritance becomes useful in particular to define
class mix-ins: simple behaviors encoded by parent classes which can be
composed together to create more complicated behaviors.

The set of available ports determine the interface of a VisTrails
module, and so impact not only the display of these modules but also
their connectivity to other modules. These ports, then, must be
explicitly described to the VisTrails infrastructure. This can be done
either by making appropriate calls to \code{add\_input\_port} and
\code{add\_output\_port} during the call to \code{initialize}, or by
specifying the per-class lists \code{\_input\_ports} and
\code{\_output\_ports} for each VisTrails module.

Each module specifies the computation to be performed by overriding
the \code{compute} method. Data is passed between modules through
ports, and accessed through the \code{get\_input\_from\_port} and
\code{set\_result} methods.  In traditional dataflow environments,
execution order is specified on-demand by the data requests. In our
case, the execution order is specified by the topological sorting of the workflow
modules. Since the caching algorithm requires an acyclic graph, we
schedule the execution in reverse topological sorted order, so the
calls to these functions do not trigger executions of upstream
modules. We made this decision deliberately: it makes it simpler to
consider the behavior of each module separately from all the others,
which makes our caching strategy simpler and more robust.

As a general guideline, VisTrails modules should refrain from using
functions with side-effects during the evaluation of the
\code{compute} method. As discussed in
\aosasecref{sec.vistrails.exec-cache}, this requirement makes caching
of partial workflow runs possible: if a module respects this property,
then its behavior is a function of the outputs of upstream
modules. Every acyclic subgraph then only needs to be computed once,
and the results can be reused.

\end{aosasect2}

%% \begin{aosasect2}{Passing Data as Modules}
\begin{aosasect2}{データをモジュールとして渡す}

One peculiar feature of VisTrails modules and their communication is
that the data that is passed between VisTrails modules are themselves
VisTrails modules. In VisTrails, there is a single hierarchy for
module and data classes. For example, a module can provide
\emph{itself} as an output of a computation (and, in fact, every
module provides a default "self" output port). The main disadvantage
is the loss of conceptual separation between computation and data that
is sometimes seen in dataflow-based architectures. There are, however,
two big advantages.  The first is that this closely mimics the object
type systems of Java and C++, and the choice was not accidental: it
was very important for us to support automatic wrapping of large class
libraries such as VTK.  These libraries allow objects to produce
other objects as computational results, making a wrapping that
distinguishes between computation and data more complicated.

The second advantage this decision brings is that defining constant
values and user-settable parameters in workflows becomes easier and
more uniformly integrated with the rest of the system. Consider, for
example, a workflow that loads a file from a location on the Web
specified by a constant. This is currently specified by a GUI in which
the URL can be specified as a parameter (see the Parameter Edits area
in \aosafigref{fig.vistrails.overview}). A natural modification of
this workflow is to use it to fetch a URL that is \emph{computed}
somewhere upstream. We would like the rest of the workflow to change
as little as possible. By assuming modules can output themselves, we
can simply connect a string with the right value to the port
corresponding to the parameter. Since the output of a constant
evaluates to itself, the behavior is exactly the same as if the value
had actually been specified as a constant.

%% \aosafigure{../images/vistrails/python-source.eps}{Prototyping New Functionality with the \code{PythonSource} Module}{fig.vistrails.pysource}
\aosafigure{../images/vistrails/python-source.eps}{新機能のプロトタイピングを\code{PythonSource}モジュールで行う}{fig.vistrails.pysource}

There are other considerations involved in designing constants. 
Each constant type has a different ideal GUI interface for
specifying values. For example, in VisTrails, a file constant module
provides a file chooser dialog; a Boolean value is specified by a
checkbox; a color value has a color picker native to each operating
system.  To achieve this generality, a developer must subclass a
custom constant from the \code{Constant} base class and provide
overrides which define an appropriate GUI widget and a string
representation (so that arbitrary constants can be serialized to
disk).

We note that, for simple prototyping tasks, VisTrails provides a
built-in \code{PythonSource} module.  A PythonSource module can be
used to directly insert scripts into a workflow. The configuration
window for PythonSource (see \aosafigref{fig.vistrails.pysource})
allows multiple input and output ports to be specified along with the
Python code that is to be executed.

\end{aosasect2}

\end{aosasect1}

%% \begin{aosasect1}{Components and Features}
\begin{aosasect1}{コンポーネントおよび機能}

As discussed above, VisTrails provides a set of functionalities and
user interfaces that simplify the creation and execution of
exploratory computational tasks. Below, we describe some of these. We
also briefly discuss how VisTrails is being used as the basis for an
infrastructure that supports the creation of provenance-rich
publications. For a more comprehensive description of VisTrails and its features,
see VisTrails' online documentation\footnote{\url{http://www.vistrails.org/usersguide}}.

%% \aosafigure{../images/vistrails/spreadsheet.pdf}{The Visual Spreadsheet}{fig.vistrails.spreadsheet}
\aosafigure{../images/vistrails/spreadsheet.eps}{Visual Spreadsheet}{fig.vistrails.spreadsheet}

%% \begin{aosasect2}{Visual Spreadsheet}
\begin{aosasect2}{Visual Spreadsheet}

VisTrails allows users to explore and compare results from multiple
workflows using the Visual Spreadsheet (see
\aosafigref{fig.vistrails.spreadsheet}). The spreadsheet is a
VisTrails package with its own interface composed of sheets and
cells. Each sheet contains a set of cells and has a customizable
layout.  A cell contains the visual representation of a result
produced by a workflow, and can be customized to display diverse types
of data.

To display a cell on the spreadsheet, a workflow must contain a module
that is derived from the base \code{SpreadsheetCell} module.  Each
\code{SpreadsheetCell} module corresponds to a cell in the
spreadsheet, so one workflow can generate multiple cells.  The
\code{compute} method of the \code{SpreadsheetCell} module handles the
communication between the Execution Engine
(\aosafigref{fig.vistrails.arch}) and the spreadsheet.  During
execution, the spreadsheet creates a cell according to its type
on-demand by taking advantage of Python's dynamic class instantiation.
Thus, custom visual representations can be achieved by creating a
subclass of \code{SpreadsheetCell} and having its \code{compute}
method send a custom cell type to the spreadsheet. For example, the
workflow in \aosafigref{fig.vistrails.overview}, \code{MplFigureCell}
is a \code{SpreadsheetCell} module designed to display images created
by matplotlib.

Since the spreadsheet uses PyQt as its GUI back end, custom cell
widgets must be subclassed from PyQt's \code{QWidget}.  They must also
define the \code{updateContents} method, which is invoked by the
spreadsheet to update the widget when new data arrives. Each cell
widget may optionally define a custom toolbar by implementing the
\code{toolbar} method; it will be displayed in the spreadsheet
toolbar area when the cell is selected.

\aosafigref{fig.vistrails.spreadsheet} shows the spreadsheet when a
VTK cell is selected, in this case, the toolbar provides specific
widgets to export PDF images, save camera positions back to the
workflow, and create animations.  The spreadsheet package defines a
customizable \code{QCellWidget}, which provides common features such
as history replay (animation) and multi-touch events forwarding. This
can be used in place of \code{QWidget} for faster development of new
cell types.

Even though the spreadsheet only accepts PyQt widgets as cell types,
it is possible to integrate widgets written with other GUI toolkits.
To do so, the widget must export its elements to the native platform,
and PyQt can then be used to grab it. We use this approach for the
\code{VTKCell} widget because the actual widget is written in C++. At
run-time, the \code{VTKCell} grabs the window id, a Win32, X11, or
Cocoa/Carbon handle depending on the system, and maps it to the
spreadsheet canvas.

Like cells, sheets may also be customized. By default, each sheet
lives in a tabbed view and has a tabular layout.  However, any sheet
can be undocked from the spreadsheet window, allowing multiple sheets
to be visible at once.  It is also possible to create a different
sheet layout by subclassing the \code{StandardWidgetSheet}, also a
PyQt widget. The \code{StandardWidgetSheet} manages cell layouts as
well as interactions with the spreadsheet in editing mode. In editing
mode, users can manipulate the cell layout and perform advanced
actions on the cells, rather than interacting with cell contents.  Such
actions include applying analogies (see
\aosasecref{sec.vistrails.diff}) and creating new workflow versions
from parameter explorations.

\end{aosasect2}

%% \begin{aosasect2}{Visual Differences and Analogies}
\begin{aosasect2}{Visual DifferencesおよびAnalogies}
\label{sec.vistrails.diff}

As we designed VisTrails, we wanted to enable the \emph{use} of
provenance information in addition to its capture.  First, we wanted
users to see the exact differences between versions, but we then realized that
a more helpful feature was being able to apply these differences to other
workflows.  Both of these tasks are possible because VisTrails tracks
the evolution of workflows.

Because the version tree captures all of the changes and we can invert
each action, we can find a complete sequence of actions that transform one
version to another.  Note that some changes will cancel each other
out, making it possible to compress this sequence. For example, the
addition of a module that was later deleted need not be examined when
computing the difference.  Finally, we have some heuristics to further
simplify the sequence: when the same module occurs in both workflows
but was added through separate actions, we we cancel the adds and
deletes.

From the set of changes, we can create a visual representation
that shows similar and different modules, connections, and parameters.
This is illustrated in 
\aosafigref{fig.vistrails.changeprov}. Modules and connections that
appear in both workflows are colored gray, and those appearing in only
one are colored according to the workflow they appear in.  Matching
modules with different parameters are shaded a lighter gray and a user
can inspect the parameter differences for a specific module in a table
that shows the values in each workflow.

The analogy operation allows users to take these differences and apply
them to other workflows.  If a user has made a set of changes to an
existing workflow (e.g., changing the resolution and file format of an
output image), he can apply the same changes to other workflows via an
analogy.  To do so, the user selects a source and a target workflow,
which delimits the set of desired changes, as well as the workflow
they wish to apply the analogy to.  VisTrails computes the difference
between the first two workflows as a template, and then determines 
how to remap this difference in order to apply it to the third
workflow.  Because it is possible to apply differences to workflows
that do not exactly match the starting workflow, we need a soft
matching that allows correspondences between similar modules.  With
this matching, we can remap the difference so the sequence of changes
can be applied to the selected
workflow~\cite{bib:scheidegger:analogy}.  The method is not foolproof
and may generate new workflows that are not exactly what was desired.
In such cases, a user may try to fix any introduced mistakes, or go
back to the previous version and apply the changes manually.

To compute the soft matching used in analogies, we want to balance
local matches (identical or very similar modules) with the overall
workflow structure.  Note that the computation of even the identical
matching is inefficient due to the hardness of subgraph isomorphism,
so we need to employ a heuristic.  In short, if two somewhat-similar
modules in the two workflows share similar neighbors, we might
conclude that these two modules function similarly and should be
matched as well.  More formally, we construct a product graph where
each node is a possible pairing of modules in the original workflows
and an edge denotes shared connections.  Then, we run steps diffusing
the scores at each node across the edges to neighboring nodes.  This
is a Markov process similar to Google's PageRank, and will eventually
converge leaving a set of scores that now includes some global
information.  From these scores, we can determine the best matching,
using a threshold to leave very dissimilar modules unpaired.

\end{aosasect2}

%% \begin{aosasect2}{Querying Provenance}
\begin{aosasect2}{履歴の問い合わせ}

The provenance captured by VisTrails includes a set of workflows, each
with its own structure, 
metadata, and execution logs. It is important that users can access
and explore these data.  VisTrails provides both text-based and visual
(WYSIWYG) query interfaces.  For information like tags, annotations,
and dates, a user can use keyword search with optional markup. For
example, look for all workflows with the keyword \code{plot} that were
created by \code{user:{\textasciitilde}dakoop}.  However, queries for specific subgraphs of
a workflow are more easily represented through a visual, query-by-example interface, where users can
either build the query from scratch or copy and modify an existing
piece of a pipeline.

In designing this query-by-example interface, we kept most of the code
from the existing Workflow Editor, with a few changes to parameter
construction.  For parameters, it is often useful to search for ranges
or keywords rather than exact values.  Thus, we added modifiers to the
parameter value fields; when a user adds or edits a parameter value,
they may choose to select one of these modifiers which default to
exact matches.  In addition to visual query construction, query
results are shown visually.  Matching versions are highlighted in the
version tree, and any selected workflow is displayed with the matching
portion highlighted.  The user can exit query results mode by
initiating another query or clicking a reset button.

\end{aosasect2}

%% \begin{aosasect2}{Persistent Data}
\begin{aosasect2}{永続データ}

VisTrails saves the provenance of how results were derived and the
specification of each step. However, reproducing a workflow run can be
difficult if the data needed by the workflow is no longer available.
In addition, for long-running workflows, it may be useful to store
intermediate data as a persistent cache across sessions in order to
avoid recomputation.

Many workflow systems store filesystem paths to data as provenance,
but this approach is problematic.  A user might rename a file, move
the workflow to another system without copying the data, or change the
data contents.  In any of these cases, storing the path as provenance
is not sufficient.  Hashing the data and storing
the hash as provenance helps to determine whether the data might have
changed, but does not help one locate the data if it exists.  To solve
this problem, we created the Persistence Package, a VisTrails package
that uses version control infrastructure to store data that can be
referenced from provenance.  Currently we use Git to manage the data,
although other systems could easily be employed.

We use universally unique identifiers (UUIDs) to identify data, and
commit hashes from git to reference versions.  If the data changes
from one execution to another, a new version is checked in to the
repository.  Thus, the \code{(uuid, version)} tuple is a compound identifier
to retrieve the data in any state.  In addition, we store the hash of
the data as well as the signature of the upstream portion of the
workflow that generated it (if it is not an input).  This allows one
to link data that might be identified differently as well as reuse
data when the same computation is run again.

The main concern when designing this package was the way users were
able to select and retrieve their data.  Also, we wished to keep all
data in the same repository, regardless of whether it is used as
input, output, or intermediate data (an output of one workflow might
be used as the input of another).  There are two main modes a user
might employ to identify data: choosing to create a new reference or
using an existing one.  Note that after the first execution, a new
reference will become an existing one as it has been persisted during
execution; a user may later choose to create another reference if
they wish but this is a rare case.  Because a user often wishes to
always use the latest version of data, a reference identified without
a specific version will default to the latest version.

Recall that before executing a module, we recursively update all of
its inputs.  A persistent data module will not update its inputs if
the upstream computations have already been run.  To determine this,
we check the signature of the upstream subworkflow against the
persistent repository and retrieve the precomputed data if the
signature exists.  In addition, we record the data identifiers and
versions as provenance so that a specific execution can be reproduced.

\end{aosasect2}

%% \begin{aosasect2}{Upgrades}
\begin{aosasect2}{アップグレード}
\label{sec.vistrails.upgrades}

With provenance at the core of VisTrails, the ability to upgrade old
workflows so they will run with new versions of packages is a key
concern.  Because packages can be created by third-parties, we need
both the infrastructure for upgrading workflows as well as the hooks
for package developers to specify the upgrade paths.  The core action
involved in workflow upgrades is the replacement of one module with a
new version.  Note that this action is complicated because we must
replace all of the connections and parameters from the old module.  In
addition, upgrades may need to reconfigure, reassign, or rename these
parameters or connections for a module, e.g., when the module
interface changes.

Each package (together with its associated modules) is tagged by a
version, and if that version changes, we assume that the modules in
that package may have changed.  Note that some, or even most, may not
have changed, but without doing our own code analysis, we cannot check
this.  We, however, attempt to automatically upgrade any module
whose interface has not changed.  To do this, we try replacing the
module with the new version and throw an exception if it does not
work.  When developers have changed the interface of a module or
renamed a module, we allow them to specify these changes explicitly.
To make this more manageable, we have created a \code{remap\_module}
method that allows developers to define only the places where the
default upgrade behavior needs to be modified.  For example, a
developer that renamed an input port `file' to `value' can specify
that specific remapping so when the new module is created, any
connections to `file' in the old module will now connect to
`value'. Here is an example of an upgrade path for a built-in
VisTrails module:

% JF: avoid spreading this over 2 pages
% Commenting minipage out for now because code is now in the middle of a page
% \begin{minipage}{\textwidth}
\begin{verbatim}
def handle_module_upgrade_request(controller, module_id, pipeline):
   module_remap = {'GetItemsFromDirectory':
                       [(None, '1.6', 'Directory',
                         {'dst_port_remap':
                              {'dir': 'value'},
                          'src_port_remap':
                              {'itemlist': 'itemList'},
                          })],
                   }
  return UpgradeWorkflowHandler.remap_module(controller, module_id, pipeline,
                                             module_remap)
\end{verbatim}
% \end{minipage}

This piece of code upgrades workflows that use the old
\code{GetItemsFromDirectory} (any version up to 1.6) module to use the
\code{Directory} module instead.  It maps the \code{dir} port from the
old module to \code{value} and the \code{itemlist} port to
\code{itemList}.

Any upgrade creates a new version in the version tree so that
executions before and after upgrades can be differentiated and
compared.  It is possible that the upgrades change the execution of
the workflow (e.g., if a bug is fixed by a package developer), and we
need to track this as provenance information.  Note that in older
vistrails, it may be necessary to upgrade every version in the tree.
In order to reduce clutter, we only upgrade versions that a user has
navigated to.  In addition, we provide a preference that allows a user
to delay the persistence of any upgrade until the workflow is modified
or executed; if a user just views that version, there is no need to
persist the upgrade.

\end{aosasect2}

%% \begin{aosasect2}{Sharing and Publishing Provenance-Rich Results}
\begin{aosasect2}{Sharing and Publishing Provenance-Rich Results}
\label{sec.vistrails.publish}

While reproducibility is the cornerstone of the scientific method,
current publications that describe computational experiments 
often fail to provide enough information to enable the results
to be repeated or generalized. Recently, there has been a renewed
interest in the publication of reproducible results. A major roadblock
to the more widespread adoption of this practice is the fact that it
is hard to create a bundle that includes all of the components (e.g.,
data, code, parameter settings) needed to reproduce a result as well
as verify that result.

By capturing detailed provenance, and through many of the features
described above, VisTrails simplifies this process for computational
experiments that are carried out within the system. However,
mechanisms are needed to both link documents to and share the
provenance information.

We have developed VisTrails packages that enable results present in
papers to be linked to their provenance, like a deep caption. Using
the LaTeX package we developed, users can include figures that link to
VisTrails workflows.  The following LaTeX code will generate a figure
that contains a workflow result:

\begin{verbatim}
\begin{figure}[t]
\centering{
\vistrail[wfid=119,buildalways=false]{width=0.9\linewidth}
}
\caption{Visualizing a binary star system simulation. This is an image
  that was generated by embedding a workflow directly in the text.}
\label{fig:astrophysics}
\end{figure}
\end{verbatim}

When the document is compiled using pdflatex, the
\code{{\textbackslash}vistrail} command will invoke a Python script
with the parameters received, which sends an XML-RPC message to a
VisTrails server to execute the workflow with \code{id 119}. This same
Python script downloads the results of the workflow from the server
and includes them in the resulting PDF document by generating
hyperlinked LaTeX \code{{\textbackslash}includegraphics} commands
using the specified layout options
(\code{width=0.9{\textbackslash}linewidth}).

It is also possible to include VisTrails results into Web pages,
wikis, Word documents and PowerPoint presentations. The linking
between Microsoft PowerPoint and VisTrails was done through the
Component Object Model (COM) and Object Linking and Embedding (OLE)
interface. In order for an object to interact with PowerPoint, at
least the \code{IOleObject}, \code{IDataObject} and
\code{IPersistStorage} interface of COM must be implemented. As we use
the \code{QAxAggregated} class of Qt, which is an abstraction for
implementing COM interfaces, to build our OLE object, both
\code{IDataObject} and \code{IPersistStorage} are automatically
handled by Qt. Thus, we only need to implement the \code{IOleObject}
interface. The most important call in this interface is
\code{DoVerb}. It lets VisTrails react to certain actions from
PowerPoint, such as object activation. In our implementation, when the
VisTrails object is activated, we load the VisTrails application and
allow users to open, interact with and select a pipeline that they
want to insert. After they close VisTrails, the pipeline result will
be shown in PowerPoint. Pipeline information is also stored with the
OLE object.

To enable users to freely share their results together with the
associated provenance, we have created
crowdLabs.\footnote{\url{http://www.crowdlabs.org}} crowdLabs is a
social Web site that integrates a set of usable tools and a scalable
infrastructure to provide an environment for scientists to
collaboratively analyze and visualize data.  crowdLabs is tightly
integrated with VisTrails. If a user wants to share any results
derived in VisTrails, she can connect to the crowdLabs server directly
from VisTrails to upload the information. Once the information is
uploaded, users can interact with and execute the workflows through a
Web browser---these workflows are executed by a VisTrails server that
powers crowdLabs. For more details on how VisTrails is used to created reproducible publications, see \url{http://www.vistrails.org}.

\end{aosasect2}

\end{aosasect1}

%% \begin{aosasect1}{Lessons Learned}
\begin{aosasect1}{教訓}

Luckily, back in 2004 when we started thinking about building a data
exploration and visualization system that supported provenance, we
never envisioned how challenging it would be, or how long it would
take to get to the point we are at now. If we had, we probably would
never have started.

Early on, one strategy that worked well was quickly prototyping new
features and showing them to a select set of users. The initial
feedback and the encouragement we received from these users was
instrumental in driving the project forward. It would have been
impossible to design VisTrails without user feedback. If there is one
aspect of the project that we would like to highlight is that most
features in the system were designed as direct response to user
feedback. However, it is worthy to note that many times what a user
asks for is not the best solution for his/her need---being responsive
to users does not necessarily mean doing exactly what they ask for.
Time and again, we have had to design and re-design features to make
sure they would be useful and properly integrated in the system.

Given our user-centric approach, one might expect that every feature
we have developed would be heavily used. Unfortunately this has not
been the case.  Sometimes the reason for this is that the feature is
highly "unusual", since it is not found in other tools. For instance,
analogies and even the version tree are not concepts that most users
are familiar with, and it takes a while for them to get comfortable
with them. Another important issue is documentation, or lack
thereof. As with many other open source projects, we have been much
better at developing new features than at documenting the existing
ones. This lag in documentation leads not only to the underutilization
of useful features, but also to many questions on our mailing lists.

One of the challenges of using a system like VisTrails is that it is
very general. Despite our best efforts to improve usability, VisTrails
is a complex tool and requires a steep learning curve for some
users. We believe that over time, with improved documentation, further
refinements to the system, and more application- and domain-specific
examples, the adoption bar for any given field will get lower. Also,
as the concept of provenance becomes more widespread, it will be
easier for users to understand the philosophy that we have adopted in
developing VisTrails.

%% \begin{aosasect2}{Acknowledgments}
\begin{aosasect2}{謝辞}

We would like to thank all the talented developers that contributed to
VisTrails: Erik Anderson, Louis Bavoil, Clifton Brooks, Jason
Callahan, Steve Callahan, Lorena Carlo, Lauro Lins, Tommy Ellkvist,
Phillip Mates, Daniel Rees, and Nathan Smith. Special thanks to
Antonio Baptista who was instrumental in helping us develop the vision
for the project; and Matthias Troyer, whose collaboration has helped
us to improve the system, and in particular has provided much of the
impetus for the development and release of the provenance-rich
publication functionality. The research and development of the
VisTrails system has been funded by the National Science Foundation under grants IIS 1050422, IIS-0905385, IIS 0844572, ATM-0835821,
IIS-0844546, IIS-0746500, CNS-0751152, IIS-0713637, OCE-0424602,
IIS-0534628, CNS-0514485, IIS-0513692, CNS-0524096, CCF-0401498,
OISE-0405402, CCF-0528201, CNS-0551724,
the Department of Energy SciDAC (VACET and SDM centers), and IBM Faculty Awards.

\end{aosasect2}

\end{aosasect1}

\end{aosachaptertoc}
