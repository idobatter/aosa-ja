\begin{aosachapter}{Violet}{s:violet}{Cay Horstmann}
%% Based on EN-Revision r367

In 2002, I wrote an undergraduate textbook on object-oriented design
and patterns \cite{bib:horstmann:oodp}.  As with so many books, this one
was motivated by frustration with the canonical curriculum.
Frequently, computer science students learn how to design a single
class in their first programming course, and then have no further
training in object-oriented design until their senior level software
engineering course. In that course, students rush through a couple of
weeks of UML and design patterns, which gives no more than an illusion
of knowledge. My book supports a semester-long course for students
with a background in Java programming and basic data structures
(typically from a Java-based CS1/CS2 sequence). The book covers object-oriented
design principles and design patterns in the context of familiar
situations. For example, the Decorator design pattern is introduced
with a Swing \code{JScrollPane}, in the hope that this example is more
memorable than the canonical Java streams example.

\aosafigure[250pt]{../images/violet/object-sample.eps}{A Violet Object Diagram}{fig.vio.diagram}

I needed a light subset of UML for the book: class diagrams, sequence
diagrams, and a variant of object diagrams that shows Java object
references (\aosafigref{fig.vio.diagram}).  I also wanted students to
draw their own diagrams. However, commercial offerings such as
Rational Rose were not only expensive but also cumbersome to learn and
use~\cite{bib:shumba:ratrose}, and the open source alternatives
available at the time were too limited or buggy to be
useful\footnote{At the time, I was not aware of Diomidis Spinellis'
admirable UMLGraph program~\cite{bib:spinellis:umlgraph}, in which
diagrams are specified by textual declarations rather than the more
common point-and-click interface.}.  In particular, sequence
diagrams in ArgoUML were seriously broken.

I decided to try my hand at implementing the simplest editor that is
(a) useful to students and (b) an example of an extensible framework
that students can understand and modify. Thus, Violet was born.

\begin{aosasect1}{Introducing Violet}

Violet is a lightweight UML editor, intended for students, teachers,
and authors who need to produce simple UML diagrams quickly. It is
very easy to learn and use. It draws class, sequence, state, object
and use-case diagrams. (Other diagram types have since been
contributed.) It is open-source and cross-platform software. In its
core, Violet uses a simple but flexible graph framework that takes
full advantage of the Java 2D graphics API.

The Violet user interface is purposefully simple. You don't have to go
through a tedious sequence of dialogs to enter attributes and
methods. Instead, you just type them into a text field. With a few
mouse clicks, you can quickly create attractive and useful diagrams.

Violet does not try to be an industrial-strength UML program. Here are
some features that Violet does \emph{not} have:

\begin{aosaitemize}

\item Violet does not generate source code from UML diagrams or UML
  diagrams from source code.

\item Violet does not carry out any semantic checking of models; you
  can use Violet to draw contradictory diagrams.

\item Violet does not generate files that can be imported into other
  UML tools, nor can it read model files from other tools.

\item Violet does not attempt to lay out diagrams automatically,
  except for a simple ``snap to grid'' facility.

\end{aosaitemize}

\noindent (Attempting to address some of these limitations makes good student
projects.)

When Violet developed a cult following of designers who wanted
something more than a cocktail napkin but less than an
industrial-strength UML tool, I published the code on SourceForge
under the GNU General Public License. Starting in 2005, Alexandre de Pellegrin
joined the project by providing an Eclipse plugin and a prettier user
interface. He has since made numerous architectural changes and is now
the primary maintainer of the project.

In this article, I discuss some of the original architectural choices
in Violet as well as its evolution. A part of the article is focused
on graph editing, but other parts---such as the use of JavaBeans
properties and persistence, Java WebStart and plugin
architecture---should be of general interest.

\end{aosasect1}

\begin{aosasect1}{The Graph Framework}

Violet is based on a general graph editing framework that can render
and edit nodes and edges of arbitrary shapes. The Violet UML editor
has nodes for classes, objects, activation bars (in sequence
diagrams), and so on, and edges for the various edge shapes in UML
diagrams. Another instance of the graph framework might display
entity-relationship diagrams or railroad diagrams.

\aosafigure{../images/violet/Ch8-06.eps}{A Simple Instance of the Editor Framework}{fig.vio.editor}

In order to illustrate the framework, let us consider an editor for
very simple graphs, with black and white circular nodes and straight
edges (\aosafigref{fig.vio.editor}).
The \code{SimpleGraph} class specifies prototype objects for the node
and edge types, illustrating the prototype pattern:

\pagebreak

\begin{verbatim}
public class SimpleGraph extends AbstractGraph
{
  public Node[] getNodePrototypes()
  {
    return new Node[] 
    {
      new CircleNode(Color.BLACK),
      new CircleNode(Color.WHITE)
    };
  }
  public Edge[] getEdgePrototypes()
  {
    return new Edge[] 
    {
      new LineEdge()
    };
  }
}
\end{verbatim}

Prototype objects are used to draw the node and edge buttons at the
top of \aosafigref{fig.vio.editor}. They are cloned whenever the
user adds a new node or edge instance to the graph.
\code{Node} and \code{Edge} are interfaces with the following
key methods:

\begin{aosaitemize}

\item Both interfaces have a \code{getShape} method that returns a
  Java2D \code{Shape} object of the node or edge shape.

\item The \code{Edge} interface has methods that yield the nodes at
  the start and end of the edge.

\item The \code{getConnectionPoint} method in the Node interface type
  computes an optimal attachment point on the boundary of a node (see
  \aosafigref{fig.vio.connection}).

\item The \code{getConnectionPoints} method of the \code{Edge}
  interface yields the two end points of the edge. This method is
  needed to draw the ``grabbers'' that mark the currently selected edge.

\item A node can have children that move together with the parent. A number
  of methods are provided for enumerating and managing children.

\end{aosaitemize}

\aosafigure[200pt]{../images/violet/Ch8-07.eps}{Finding a Connection Point on the Boundary of the Node Shape}{fig.vio.connection}

Convenience classes \code{AbstractNode} and \code{AbstractEdge}
implement a number of these methods, and classes
\code{RectangularNode} and \code{SegmentedLineEdge} provide complete
implementations of rectangular nodes with a title string and edges
that are made up of line segments.

In the case of our simple graph editor, we would need to supply
subclasses \code{CircleNode} and \code{LineEdge} that provide a
\code{draw} method, a \code{contains} method, and the
\code{getConnectionPoint} method that describes the shape of the node
boundary.
The code is given below,
and \aosafigref{fig.vio.classdiag} shows a class diagram of these classes
(drawn, of course, with Violet).

\begin{verbatim}
public class CircleNode extends AbstractNode
{
  public CircleNode(Color aColor)
  {
    size = DEFAULT_SIZE;
    x = 0;
    y = 0;
    color = aColor;
  }

  public void draw(Graphics2D g2)
  {
    Ellipse2D circle = new Ellipse2D.Double(x, y, size, size);
    Color oldColor = g2.getColor();
    g2.setColor(color);
    g2.fill(circle);
    g2.setColor(oldColor);
    g2.draw(circle);
  }
\end{verbatim}

\begin{verbatim}
  public boolean contains(Point2D p)
  {
    Ellipse2D circle = new Ellipse2D.Double(x, y, size, size);
    return circle.contains(p);
  }

  public Point2D getConnectionPoint(Point2D other)
  {
    double centerX = x + size / 2;
    double centerY = y + size / 2;
    double dx = other.getX() - centerX;
    double dy = other.getY() - centerY;
    double distance = Math.sqrt(dx * dx + dy * dy);
    if (distance == 0) return other;
    else return new Point2D.Double(
      centerX + dx * (size / 2) / distance,
      centerY + dy * (size / 2) / distance);
  }

  private double x, y, size, color;
  private static final int DEFAULT_SIZE = 20;
}

public class LineEdge extends AbstractEdge
{
  public void draw(Graphics2D g2)
  { g2.draw(getConnectionPoints()); }

  public boolean contains(Point2D aPoint)
  {
    final double MAX_DIST = 2;
    return getConnectionPoints().ptSegDist(aPoint) < MAX_DIST;
  }
}
\end{verbatim}

\aosafigure{../images/violet/SimpleGraph-in-Violet.eps}{Class Diagram for a Simple Graph}{fig.vio.classdiag}

In summary, Violet provides a simple framework for producing graph
editors.  To obtain an editor instance, define node and edge classes
and provide methods in a graph class that yield prototype node and
edge objects.

Of course, there are other graph frameworks available, such as
JGraph~\cite{bib:alder:jgraph} and
JUNG\footnote{\url{http://jung.sourceforge.net}}. However, those
frameworks are considerably more complex, and they provide frameworks
for drawing graphs, not for applications that draw graphs.

\end{aosasect1}

\begin{aosasect1}{Use of JavaBeans Properties}

In the golden days of client-side Java, the JavaBeans specification
was developed in order to provide portable mechanisms for editing GUI
components in visual GUI builder environments. The vision was that a
third-party GUI component could be dropped into any GUI builder, where
its properties could be configured in the same way as the standard
buttons, text components, and so on.

Java does not have native properties. Instead, JavaBeans properties
can be discovered as pairs of getter and setter methods, or specified
with companion BeanInfo classes. Moreover, \emph{property editors} can
be specified for visually editing property values. The JDK even
contains a few basic property editors, for example for the type
\code{java.awt.Color}.

The Violet framework makes full use of the 
JavaBeans specification. For example, the \code{CircleNode} class can
expose a color property simply by providing two methods:

\begin{verbatim}
public void setColor(Color newValue)
public Color getColor()
\end{verbatim}

\noindent No further work is necessary. The graph editor can now edit node
colors of circle nodes (\aosafigref{fig.vio.circlecolors}).

\aosafigure{../images/violet/Ch8-11.eps}{Editing Circle Colors with the default JavaBeans Color Editor}{fig.vio.circlecolors}

\end{aosasect1}

\begin{aosasect1}{Long-Term Persistence}

Just like any editor program, Violet must save the user's creations in
a file and reload them later. I had a look at the XMI
specification\footnote{\url{http://www.omg.org/technology/documents/formal/xmi.htm}}
which was designed as a common interchange format for UML models. I
found it cumbersome, confusing, and hard to consume. I don't think I
was the only one---XMI had a reputation for poor interoperability even
with the simplest models~\cite{bib:persson:osstools}.

I considered simply using Java serialization, but it is difficult to
read old versions of a serialized object whose implementation has
changed over time.  This problem was also anticipated by the JavaBeans
architects, who developed a standard XML format for long-term
persistence\footnote{\url{http://jcp.org/en/jsr/detail?id=57}}.  A
Java object---in the case of Violet, the UML diagram---is serialized
as a sequence of statements for constructing and modifying it. Here is
an example:

\begin{verbatim}
<?xml version="1.0" encoding="UTF-8"?>
<java version="1.0" class="java.beans.XMLDecoder">
 <object class="com.horstmann.violet.ClassDiagramGraph"> 
  <void method="addNode"> 
   <object id="ClassNode0" class="com.horstmann.violet.ClassNode"> 
    <void property="name">...</void> 
   </object> 
   <object class="java.awt.geom.Point2D$Double"> 
    <double>200.0</double> 
    <double>60.0</double> 
   </object> 
  </void> 
  <void method="addNode"> 
   <object id="ClassNode1" class="com.horstmann.violet.ClassNode"> 
    <void property="name">...</void> 
   </object> 
   <object class="java.awt.geom.Point2D$Double"> 
    <double>200.0</double> 
    <double>210.0</double> 
   </object> 
  </void> 
  <void method="connect"> 
   <object class="com.horstmann.violet.ClassRelationshipEdge"> 
    <void property="endArrowHead"> 
     <object class="com.horstmann.violet.ArrowHead" field="TRIANGLE"/> 
    </void> 
   </object> 
   <object idref="ClassNode0"/> 
   <object idref="ClassNode1"/> 
  </void> 
 </object> 
</java>
\end{verbatim}

When the \code{XMLDecoder} class reads this file, it executes these
statements (package names are omitted for simplicity).

\begin{verbatim}
ClassDiagramGraph obj1 = new ClassDiagramGraph();
ClassNode ClassNode0 = new ClassNode();
ClassNode0.setName(...);
obj1.addNode(ClassNode0, new Point2D.Double(200, 60));
ClassNode ClassNode1 = new ClassNode();
ClassNode1.setName(...);
obj1.addNode(ClassNode1, new Point2D.Double(200, 60));
ClassRelationShipEdge obj2 = new ClassRelationShipEdge();
obj2.setEndArrowHead(ArrowHead.TRIANGLE);
obj1.connect(obj2, ClassNode0, ClassNode1);
\end{verbatim}

As long as the semantics of the constructors, properties, and methods
has not changed, a newer version of the program can read a file that
has been produced by an older version.

Producing such files is quite straightforward. The encoder
automatically enumerates the properties of each object and writes
setter statements for those property values that differ from the
default. Most basic datatypes are handled by the Java platform;
however, I had to supply special handlers for \code{Point2D},
\code{Line2D}, and \code{Rectangle2D}. Most importantly,
the encoder must know that a graph can be serialized as a sequence of
\code{addNode} and \code{connect} method calls:

\begin{verbatim}
encoder.setPersistenceDelegate(Graph.class, new DefaultPersistenceDelegate()
{
  protected void initialize(Class<?> type, Object oldInstance, 
    Object newInstance, Encoder out)
  {
    super.initialize(type, oldInstance, newInstance, out);
    AbstractGraph g = (AbstractGraph) oldInstance;
    for (Node n : g.getNodes())
      out.writeStatement(new Statement(oldInstance, "addNode", new Object[]
      {
        n, 
        n.getLocation()
      }));
    for (Edge e : g.getEdges())
      out.writeStatement(new Statement(oldInstance, "connect", new Object[]
      {
        e, e.getStart(), e.getEnd()
      }));
   } 
 });
\end{verbatim}

Once the encoder has been configured, saving a graph is as simple as:

\begin{verbatim}
encoder.writeObject(graph);
\end{verbatim}

\noindent Since the decoder simply executes statements, it requires no
configuration.  Graphs are simply read with:

\begin{verbatim}
Graph graph = (Graph) decoder.readObject();
\end{verbatim}

This approach has worked exceedingly well over numerous versions of
Violet, with one exception. A recent refactoring changed
some package names and thereby broke backwards compatibility.
One option would have been to keep the classes in the original packages,
even though they no longer matched the new package structure.
Instead, the maintainer provided an XML
transformer for rewriting the package names when reading a legacy
file.

\end{aosasect1}

\begin{aosasect1}{Java WebStart}

Java WebStart is a technology for launching an application from a web
browser. The deployer posts a JNLP file that triggers a helper
application in the browser which downloads and runs the Java
program. The application can be digitally signed, in which case the
user must accept the certificate, or it can be unsigned, in which case
the program runs in a sandbox that is slightly more permissive than
the applet sandbox.

I do not think that end users can or should be trusted to judge the
validity of a digital certificate and its security implications. One
of the strengths of the Java platform is its security, and I feel it
is important to play to that strength.

The Java WebStart sandbox is sufficiently powerful to enable users to
carry out useful work, including loading and saving files and
printing.  These operations are handled securely and conveniently from
the user perspective. The user is alerted that the application wants
to access the local filesystem and then chooses the file to be read
or written. The application merely receives a stream object, without
having an opportunity to peek at the filesystem during the file
selection process.

It is annoying that the developer must write
custom code to interact with a \code{FileOpenService} and a
\code{FileSaveService} when the application is running under WebStart,
and it is even more annoying that there is no WebStart API call to
find out whether the application was launched by WebStart.

Similarly, saving user preferences must be implemented in two ways:
using the Java preferences API when the application runs normally, or
using the WebStart preferences service when the application is under
WebStart. Printing, on the other hand, is entirely transparent to the
application programmer.

Violet provides simple abstraction layers over these services to
simplify the lot of the application programmer. For example, here is
how to open a file:

\begin{verbatim}
FileService service = FileService.getInstance(initialDirectory);
  // detects whether we run under WebStart
FileService.Open open = fileService.open(defaultDirectory, defaultName, 
  extensionFilter);
InputStream in = open.getInputStream();
String title = open.getName();
\end{verbatim}

\noindent The \code{FileService.Open} interface is implemented by two classes: a
wrapper over \code{JFileChooser} or the JNLP \code{FileOpenService}.

No such convenience is a part of the JNLP API itself, but that API has
received little love over its lifetime and has been widely
ignored. Most projects simply use a self-signed certificate for their
WebStart application, which gives users no security. This is a
shame---open source developers should embrace the JNLP sandbox as a
risk-free way to try out a project.

\end{aosasect1}

\begin{aosasect1}{Java 2D}

Violet makes intensive use of the Java2D library, one of the lesser
known gems in the Java API\@. Every node and edge has a method
\code{getShape} that yields a \code{java.awt.Shape}, the common
interface of all Java2D shapes. This interface is implemented by
rectangles, circles, paths, and their unions, intersections, and
differences. The \code{GeneralPath} class is useful for making shapes
that are composed of arbitrary line and quadratic/cubic curve
segments, such as straight and curved arrows.

To appreciate the flexibility of the Java2D API, consider the
following code for drawing a
shadow in the \code{AbstractNode.draw} method:

\begin{verbatim}
Shape shape = getShape();
if (shape == null) return;
g2.translate(SHADOW_GAP, SHADOW_GAP);
g2.setColor(SHADOW_COLOR);
g2.fill(shape);
g2.translate(-SHADOW_GAP, -SHADOW_GAP);
g2.setColor(BACKGROUND_COLOR);
g2.fill(shape);
\end{verbatim}

\noindent A few lines of code produce a shadow for any shape, even shapes that a
developer may add at a later point.

Of course, Violet saves bitmap images in any format that the
\code{javax.imageio} package supports; that is, GIF, PNG, JPEG, and so
on. When my publisher asked me for vector images, I noted another
advantage of the Java 2D library. When you print to a PostScript
printer, the Java2D operations are translated into PostScript vector
drawing operations. If you print to a file, the result can be consumed
by a program such as \code{ps2eps} and then imported into Adobe
Illustrator or Inkscape. Here is the code, where \code{comp} is the
Swing component whose \code{paintComponent} method paints the graph:

\begin{verbatim}
DocFlavor flavor = DocFlavor.SERVICE_FORMATTED.PRINTABLE;
String mimeType = "application/postscript";
StreamPrintServiceFactory[] factories;
StreamPrintServiceFactory.lookupStreamPrintServiceFactories(flavor, mimeType);
FileOutputStream out = new FileOutputStream(fileName);
PrintService service = factories[0].getPrintService(out);            
SimpleDoc doc = new SimpleDoc(new Printable() {
  public int print(Graphics g, PageFormat pf, int page) {
      if (page >= 1) return Printable.NO_SUCH_PAGE;
      else {
        double sf1 = pf.getImageableWidth() / (comp.getWidth() + 1);
        double sf2 = pf.getImageableHeight() / (comp.getHeight() + 1);
        double s = Math.min(sf1, sf2);
        Graphics2D g2 = (Graphics2D) g;
        g2.translate((pf.getWidth() - pf.getImageableWidth()) / 2, 
            (pf.getHeight() - pf.getImageableHeight()) / 2);
        g2.scale(s, s);
                        
        comp.paint(g);
        return Printable.PAGE_EXISTS;
      }
  }
}, flavor, null);
DocPrintJob job = service.createPrintJob();
PrintRequestAttributeSet attributes = new HashPrintRequestAttributeSet();
job.print(doc, attributes);
\end{verbatim}

At the beginning, I was concerned that there might be a performance
penalty when using general shapes, but that has proven not to be the
case. Clipping works well enough that only those shape operations
that are required for updating the current viewport are actually
executed.

\end{aosasect1}

\begin{aosasect1}{No Swing Application Framework}

Most GUI frameworks have some notion of an application that manages a
set of documents that deals with menus, toolbars, status bars,
etc. However, this was never a part of the Java
API\@. JSR~296\footnote{\url{http://jcp.org/en/jsr/detail?id=296}} was
supposed to supply a basic framework for Swing applications, but it is
currently inactive.  Thus, a Swing application author has two choices:
reinvent a good number of wheels or base itself on a third party
framework. At the time that Violet was written, the primary choices
for an application framework were the Eclipse and NetBeans platform,
both of which seemed too heavyweight at the time. (Nowadays, there are
more choices, among them JSR~296 forks such as
GUTS\footnote{\url{http://kenai.com/projects/guts}}.)  Thus, Violet
had to reinvent mechanisms for handling menus and internal frames.

In Violet, you specify menu items in property files, like this:

\begin{verbatim}
file.save.text=Save
file.save.mnemonic=S
file.save.accelerator=ctrl S
file.save.icon=/icons/16x16/save.png
\end{verbatim}

\noindent A utility method creates the menu item from the prefix (here
\code{file.save}). The suffixes \code{.text}, \code{.mnemonic}, and so
on, are what nowadays would be called ``convention over
configuration''.  Using resource files for describing these settings
is obviously far superior to setting up menus with API calls because
it allows for easy localization. I reused that mechanism in another
open source project, the GridWorld environment for high school
computer science
education\footnote{\url{http://horstmann.com/gridworld}}.

An application such as Violet allows users to open multiple
``documents'', each containing a graph. When Violet was first written,
the multiple document interface (MDI) was still commonly used.  With
MDI, the main frame has a menu bar, and each view of a document is
displayed in an internal frame with a title but no menu bar. Each
internal frame is contained in the main frame and can be resized or
minimized by the user. There are operations for cascading and tiling
windows.

Many developers disliked MDI, and so this style user interface has 
gone out of fashion. For a while, a single document interface (SDI), in which an
application displays multiple top level frames, was considered
superior, presumably because those frames can be manipulated with the
standard window management tools of the host operating system. When it
became clear that having lots of top level windows isn't so great
after all, tabbed interfaces started to appear, in which multiple
documents are again contained in a single frame, but now all displayed
at full size and selectable with tabs. This does not allow users
to compare two documents side by side, but seems to have won out.

The original version of Violet used an MDI interface. The Java API has
a internal frames feature, but I had to add support for tiling and
cascading.  Alexandre switched to a tabbed interface, which is
somewhat better-supported by the Java API\@. It would be desirable to
have an application framework where the document display policy was
transparent to the developer and perhaps selectable by the user.

Alexandre also added support for sidebars, a status bar, a welcome
panel, and a splash screen. All this should ideally be a part of a
Swing application framework.

\end{aosasect1}

\begin{aosasect1}{Undo/Redo}

Implementing multiple undo/redo seems like a daunting task, but the
Swing undo package (\cite{bib:topley:coreswing}, Chapter 9) gives good
architectural guidance. An \code{UndoManager} manages a stack of
\code{UndoableEdit} objects. Each of them has an \code{undo} method
that undoes the effect of the edit operation, and a \code{redo} method
that undoes the undo (that is, carries out the original edit
operation). A \code{CompoundEdit} is a sequence of \code{UndoableEdit}
operations that should be undone or redone in their entirety. You are
encouraged to define small, atomic edit operations (such as adding or
removing a single edge or node in the case of a graph) that are
grouped into compound edits as necessary.

A challenge is to define a small set of atomic operations, each of
which can be undone easily. In Violet, they are:

\begin{aosaitemize}

  \item adding or removing a node or edge

  \item attaching or detaching a node's child

  \item moving a node

  \item changing a property of a node or edge

\end{aosaitemize}

Each of these operations has an obvious undo. For example the undo of
adding a node is the node's removal. The undo of moving a node is to
move it by the opposite vector.

\aosafigure{../images/violet/undo.eps}{An Undo Operation Must Undo Structural Changes in the Model}{fig.vio.undo}

Note that these atomic operations are \emph{not} the same as the
actions in the user interface or the methods of the \code{Graph}
interface that the user interface actions invoke. For example,
consider the sequence diagram in \aosafigref{fig.vio.undo},
and suppose the user drags the mouse from the activation bar to the
lifeline on the right. When the mouse button is released, the method:

\begin{verbatim}
public boolean addEdgeAtPoints(Edge e, Point2D p1, Point2D p2)
\end{verbatim}

\noindent
is invoked. That method adds an edge, but it may also carry out other
operations, as specified by the participating \code{Edge} and
\code{Node} subclasses. In this case, an activation bar will be added
to the lifeline on the right. Undoing the operation needs to remove
that activation bar as well. Thus, the \emph{model} (in our case, the
graph) needs to record the structural changes that need to be
undone. It is not enough to collect controller operations.

As envisioned by the Swing undo package, the graph, node, and edge
classes should send \code{UndoableEditEvent} notifications to an
\code{UndoManager} whenever a structural edit occurs. Violet has a
more general design where the graph itself manages listeners for the
following interface:

\begin{verbatim}
public interface GraphModificationListener
{
  void nodeAdded(Graph g, Node n);
  void nodeRemoved(Graph g, Node n);
  void nodeMoved(Graph g, Node n, double dx, double dy);
  void childAttached(Graph g, int index, Node p, Node c);
  void childDetached(Graph g, int index, Node p, Node c);
  void edgeAdded(Graph g, Edge e);
  void edgeRemoved(Graph g, Edge e);
  void propertyChangedOnNodeOrEdge(Graph g, PropertyChangeEvent event);
}
\end{verbatim}

The framework installs a listener into each graph that is a bridge to
the undo manager. For supporting undo, adding generic listener support to 
the model is overdesigned---the graph operations could directly interact
with the undo manager. However, I also wanted to support an experimental
collaborative editing feature.

If you want to support undo/redo in your application, think carefully
about the atomic operations in your model (and not your user
interface).  In the model, fire events when a structural change
happens, and allow the Swing undo manager to collect and group these
events.

\end{aosasect1}

\begin{aosasect1}{Plugin Architecture}

For a programmer familiar with 2D graphics, it is not difficult to add
a new diagram type to Violet. For example, the activity diagrams were
contributed by a third party. When I needed to create railroad
diagrams and ER diagrams, I found it faster to write Violet extensions
instead of fussing with Visio or Dia. (Each diagram type took a day to
implement.)

These implementations do not require knowledge of the full Violet
framework.  Only the graph, node, and edge interfaces and convenience
implementations are needed. In order to make it easier for
contributors to decouple themselves from the evolution of the
framework, I designed a simple plugin architecture.

Of course, many programs have a plugin architecture, many quite
elaborate.  When someone suggested that Violet should support OSGi, I
shuddered and instead implemented the simplest thing that works.

Contributors simply produce a JAR file with their graph, node, and
edge implementations and drop it into a \code{plugins} directory. When
Violet starts, it loads those plugins, using the Java
\code{ServiceLoader} class. That class was designed to load services
such as JDBC drivers. A \code{ServiceLoader} loads JAR files that
promise to provide a class implementing a given interface (in our
case, the \code{Graph} interface.)

Each JAR file must have a subdirectory \code{META-INF/services}
containing a file whose name is the fully qualified classname of the
interface (such as \code{com.horstmann.violet.Graph}), and that
contains the names of all implementing classes, one per line.
The \code{ServiceLoader} constructs a class loader for the plugin
directory, and loads all plugins:

\begin{verbatim}
ServiceLoader<Graph> graphLoader = ServiceLoader.load(Graph.class, classLoader);
for (Graph g : graphLoader) // ServiceLoader<Graph> implements Iterable<Graph>
  registerGraph(g); 
\end{verbatim}

This is a simple but useful facility of standard Java that you might
find valuable for your own projects.

\end{aosasect1}

\begin{aosasect1}{Conclusion}

Like so many open source projects, Violet was born of an unmet
need---to draw simple UML diagrams with a minimum of fuss. Violet was
made possible by the amazing breadth of the Java SE platform, and it
draws from a diverse set of technologies that are a part of that
platform. In this article, I described how Violet makes use of Java
Beans, Long-Term Persistence, Java Web Start, Java 2D, Swing
Undo/Redo, and the service loader facility.  These technologies are
not always as well understood as the basics of Java and Swing, but
they can greatly simplify the architecture of a desktop
application. They allowed me, as the initial sole developer, to
produce a successful application in a few months of part-time work.
Relying on these standard mechanisms also made it easier for others to
improve on Violet and to extract pieces of it into their own projects.

\end{aosasect1}

\end{aosachapter}
