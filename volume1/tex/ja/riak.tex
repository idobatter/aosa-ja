%\begin{aosachapter}{Riak and Erlang/OTP}{s:riak}{Francesco Cesarini, Andy Gross, and Justin Sheehy}
\begin{aosachapter}{RiakとErlang/OTP}{s:riak}{Francesco Cesarini, Andy Gross, and Justin Sheehy}
%% Based on EN-Revision r236

%% Riak is a distributed, fault tolerant, open source database that
%% is a distributed, fault tolerant, open source database that
%% illustrates how to build large scale systems using Erlang/OTP\@.  Thanks
%% in large part to Erlang's support for massively scalable distributed
%% systems, Riak offers features that are uncommon in databases, such as
%% high-availability and linear scalability of both capacity and
%% throughput.

Riakは分散型の耐障害性を備えたオープンソースのデータベースで、
Erlang/OTP\@を使って大規模システムを作るよい例にもなっている。大規模な
分散システムをErlangがサポートしてくれているおかげで、Riakは高可用性や、
容量とスループットの両面における線形スケーラビリティといった、データベー
スではあまり見られない機能を備えている。

%% Erlang/OTP provides an ideal platform for developing systems like Riak
%% because it provides inter-node communication, message queues, failure
%% detectors, and client-server abstractions out of the box.  What's
%% more, most frequently-used patterns in Erlang have been implemented
%% in library modules, commonly referred to as OTP behaviors. They
%% contain the generic code framework for concurrency and error handling,
%% simplifying concurrent programming and protecting
%% the developer from many common pitfalls. Behaviors are monitored by
%% supervisors, themselves a behavior, and grouped together in
%% supervision trees. A supervision tree is packaged in an application,
%% creating a building block of an Erlang program.

Erlang/OTPにはノード間通信、メッセージ・キュー、失敗の検出、クライアン
ト/サーバの抽象化がもともとあるので、Riakのようなシステムを開発するのに
理想的なプラットフォームだ。さらに、Erlangでよく使われるパターンは(OTP
ビヘイビアと呼ばれる)ライブラリ・モジュールとして実装されている。並行プ
ログラミングを簡単にしたり、開発者がよくある間違いを犯さないようにする
ため、ビヘイビアには並行実行やエラー処理のための汎用コード・フレームワー
クが入っている。ビヘイビアはスーパバイザによって監視されていて(スーパバ
イザ自体もビヘイビアだ)、監視ツリー/*supervision trees*/の形でまと
められる。監視ツリーはアプリケーションとしてパッケージ化され、Erlangプ
ログラムを構築する際の1つのブロックになる。

%% A complete Erlang system such as Riak is a set of loosely coupled
%% applications that interact with each other. Some of these applications
%% have been written by the developer, some are part of the standard
%% Erlang/OTP distribution, and some may be other open source
%% components. They are sequentially loaded and started by a boot script
%% generated from a list of applications and versions.

Riakのように完結したErlangシステムは、互いにやり取りし合う疎結合のアプ
リケーションが集まってできている。このようなアプリケーションには、開発
者自身が書いたものや、Erlang/OTPの標準配布物に入っているもの、他のオー
プンソースのものがある。アプリケーションを順番にロードするのはブート・
スクリプトの役目で、ブート・スクリプトはアプリケーションとそのバージョ
ンのリストから生成される。

%% What \emph{differs} among systems are the applications that are part
%% of the release which is started. In the standard Erlang distribution,
%% the boot files will start the \emph{Kernel} and \emph{StdLib}
%% (Standard Library) applications. In some installations, the
%% \emph{SASL} (Systems Architecture Support Library) application is also
%% started. SASL contains release and software upgrade tools together
%% with logging capabilities. Riak is no different, other than starting
%% the Riak specific applications as well as their runtime dependencies,
%% which include \emph{Kernel}, \emph{StdLib} and \emph{SASL}. A complete and
%% ready-to-run build of Riak actually embeds these standard elements of
%% the Erlang/OTP distribution and starts them all in unison when
%% \code{riak start} is invoked on the command line. Riak consists of
%% many complex applications, so this chapter should not be
%% interpreted as a complete guide. It should be seen as an introduction
%% to OTP where examples from the Riak source code are used. The figures
%% and examples have been abbreviated and shortened for demonstration
%% purposes.


個々のシステムを特徴付けるのは、リリース物/*which is started*/に含
まれているアプリケーションたちだ。Erlangの標準配布物では、ブート・ファ
イルから\emph{Kernel}アプリケーションと\emph{StdLib} (Standard
Library)アプリケーションが起動する。インストール環境によっては、
\emph{SASL} (Systems Architecture Support Library)アプリケーションも起
動する。SASLにはリリースやソフトウェア・アップグレードのためのツールと、
ログ機能が入っている。Riakも同様だが、Riak固有のアプリケーション群と、
それらが実行時に依存する\emph{Kernel}、\emph{StdLib}、\emph{SASL}も起動
する。完成して実行可能な状態になっているRiakのビルドにはこれらの
Erlang/OTP配布物の標準要素が埋め込まれていて、コマンド・ラインから
\code{riak start}を実行すると順番に起動される。Riakは多数の複雑なアプリ
ケーションで構成されているので、この章ですべてがわかるとは思わないで、
Riakのソース・コードに例を取ったOTPの紹介と考えてほしい。説明をわかりや
すくするため、図や例は省略したり短縮してある。

%% \begin{aosasect1}{An Abridged Introduction to Erlang}
\begin{aosasect1}{Erlangの簡単な紹介}

%% Erlang is a concurrent functional programming language that compiles
%% to byte code and runs in a virtual machine. Programs consist of
%% functions that call each other, often resulting in side effects such
%% as inter-process message passing, I/O and database operations. Erlang
%% variables are single assignment, i.e., once they have been given
%% values, they cannot be updated.  The language makes extensive use of
%% pattern matching, as shown in the factorial example below:

Erlangは並行実行をサポートした関数型プログラミング言語で、コンパイルし
たバイト・コードを仮想マシンで実行する。プログラムは関数の呼び出しで構
成されるが、プロセス間メッセージ・パッシングやI/O、データベース操作など
の副作用を伴なうこともある。Erlangにおける変数は単一代入だ。つまり、変
数にいったん値を与えたら、その値を変更することはできない。次の階乗の例
のように、Erlangではパターン・マッチングを多用する。

\begin{verbatim}
-module(factorial).
-export([fac/1]).
fac(0) -> 1;
fac(N) when N>0 ->
   Prev = fac(N-1),
   N*Prev.
\end{verbatim}

%% \noindent Here, the first clause gives the factorial of zero, the second
%% factorials of positive numbers. The body of each clause is a sequence
%% of expressions, and the final expression in the body is the result of
%% that clause. Calling the function with a negative number will result
%% in a run time error, as none of the clauses match. Not handling this
%% case is an example of non-defensive programming, a practice encouraged
%% in Erlang.

\noindent ここでは、最初の節では0の階乗を定義し、2番目の節では正の数の
階乗を定義している。各節の本体(ボディ部)には式を並べるが、節の結果とし
ては最後の式が使われる。負の数に対してこの関数を呼び出すと、どの節にも
マッチしないため実行時エラーになる。このような場合についての処理をあえ
て用意しておかないのは、Erlangで推奨される非防御的プログラミング/*
non-defensive programming*/の1つの例だ。

%% Within the module, functions are called in the usual way; outside, the
%% name of the module is prepended, as in \code{factorial:fac(3)}. It
%% is possible to define functions with the same name but different
%% numbers of arguments---this is called their \emph{arity}. In the export
%% directive in the \code{factorial} module the \code{fac} function of
%% arity one is denoted by \code{fac/1}.

モジュール内での関数の呼び出し方法は特に変ったところはないが、モジュー
ルを越えて関数を呼び出す場合は、\code{factorial:fac(3)}のようにモジュー
ル名を先頭に付ける。名前が同じだが引数の個数(\emph{arity}と呼ばれる)が
異なる関数を定義することもできる。\code{factorial} モジュールのexportディ
レクティブを見るとわかるように、アリティが1の\code{fac}関数を
\code{fac/1}と表記している。

%% Erlang supports tuples (also called product types) and lists. Tuples
%% are enclosed in curly brackets, as in \code{\{ok,37\}}. In tuples, we
%% access elements by position. Records are another data type; they allow
%% us to store a fixed number of elements which are then accessed and
%% manipulated by name. We define a record using the \code{-record(state,
%% \{id, msg\_list=[]\}).}  To create an instance, we use the
%% expression \code{Var = \#state\{id=1\}}, and we examine its contents
%% using \code{Var\#state.id}.  For a variable number of elements, we use
%% lists defined in square brackets such as in \code{{[}23,34{]}}. The
%% notation \code{{[}X|Xs{]}} matches a non-empty list with head \code{X}
%% and tail \code{Xs}. Identifiers beginning with a lower case letter
%% denote atoms, which simply stand for themselves; the \code{ok} in the
%% tuple \code{\{ok,37\}} is an example of an atom. Atoms used in this
%% way are often used to distinguish between different kinds of function
%% result: as well as \code{ok} results, there might be results of the
%% form \code{\{error, "Error String"\}}.

Erlangはタプル(積型とも呼ばれる)とリストをサポートしている。タプルは
\code{\{ok,37\}}のように波括弧でくくる。タプルの各要素は位置を指定して
アクセスする。レコードというデータ型もある。レコードは決まった個数の要
素を持ち、各要素は名前を指定してアクセスしたり操作したりできる。レコー
ドを定義するには\code{-record(state, \{id, msg\_list=[]\}).}を使う。レ
コードのインスタンスを作るには\code{Var = \#state\{id=1\}}という式を使
い、その要素を取り出すには\code{Var\#state.id}とする。要素の個数が一定
でない場合はリストを使う。リストは\code{{[}23,34{]}}のように角括弧で定
義する。\code{{[}X|Xs{]}}という表記はヘッドが\code{X}でテールが
\code{Xs}の空でないリストにマッチする。小文字で始まる識別子はアトム(そ
れ自身を表わすもの)を示す。たとえば、タプル\code{\{ok,37\}}の\code{ok}
はアトムだ。このように、アトムは関数が返す様々な結果を区別するのに使う。
たとえば、関数の結果として\code{ok}の他に\code{\{error, "Error
  String"\}}というのも考えられる。

%% Processes in Erlang systems run concurrently in separate memory, and
%% communicate with each other by message passing. Processes can be used
%% for a wealth of applications, including gateways to databases, as
%% handlers for protocol stacks, and to manage the logging of trace
%% messages from other processes. Although these processes handle
%% different requests, there will be similarities in how these requests
%% are handled.

Erlangシステムにおけるプロセスはそれぞれ独立したメモリ空間で並行に動作
し、メッセージ・パッシングを使って通信する。プロセスの用途は広く、デー
タベースへのゲートウェイ、プロトコル・スタックのハンドラ、他のプロセス
からのトレース・メッセージのログ管理などに使える。プロセスは用途に応じ
て様々な要求を処理するが、要求の処理方法には似た部分がある。

%% As processes exist only within the virtual machine, a single VM can
%% simultaneously run millions of processes, a feature Riak exploits
%% extensively. For example, each request to the database---reads, writes, and
%% deletes---is modeled as a separate process, an approach that would
%% not be possible with most OS-level threading implementations.

プロセスは仮想マシン(VM)内にしか存在しないため、1つのVMは数百万ものプロ
セスを同時に実行できる。Riakはこの機能を多いに利用している。たとえば、
データベースへの要求の1つ1つ(読み出し、書き込み、削除)をそれぞれ独立し
たプロセスとしてモデル化することができる。このような手法はOSレベルのス
レッドを用いた実装では非常に困難だろう。

%% Processes are identified by process identifiers, called PIDs, but
%% they can also be registered under an alias; this should only be used
%% for long-lived ``static'' processes. Registering a process with its
%% alias allows other processes to send it messages without knowing its
%% PID. Processes are created using the \code{spawn(Module, Function,
%%  Arguments)} built-in function (BIF). BIFs are functions integrated
%% in the VM and used to do what is impossible or slow to execute in pure
%% Erlang. The \code{spawn/3} BIF takes a \code{Module}, a
%% \code{Function} and a list of \code{Arguments} as parameters. The call
%% returns the PID of the newly spawned process and as a side effect,
%% creates a new process that starts executing the function in the module
%% with the arguments mentioned earlier.

プロセスを識別するにはプロセス識別子(PID)を使うが、PIDは別名を付けて登
録することもできる。ただし、PIDの登録は寿命の長い静的なプロセスだけにし
ておくべきだ。プロセスを別名付きで登録しておくと、他のプロセスはそのプ
ロセスのPIDを知らなくてもメッセージを送れるようになる。プロセスを生成す
るには\code{spawn(Module, Function, Arguments)}組み込み関数(BIF)を使う。
BIFとはVM内に組み込まれている関数で、純粋なErlangコードでは実現不可能だっ
たり、速度的に問題がある場合に使われる。\code{spawn/3} BIFはパラメータ
として\code{Module}、\code{Function}、リストの\code{Arguments}を受け取
る。この関数は副作用として新しいプロセスを生成し、そのプロセスのPIDを返
す。生成されたプロセスは、指定されたモジュールにある指定された関数を、
指定された引数を渡して実行する。

%% A message \code{Msg} is sent to a process with process id \code{Pid}
%% using \code{Pid ! Msg}. A process can find out its PID by calling the
%% BIF \code{self}, and this can then be sent to other processes for them
%% to use to communicate with the original process. Suppose that a
%% process expects to receive messages of the form \code{\{ok, N\}} and
%% \code{\{error, Reason\}}. To process these it uses a receive
%% statement:

プロセスIDが\code{Pid}のプロセスにメッセージ\code{Msg}を送るには
\code{Pid ! Msg}を使う。自分自身のPIDを知るには、BIFの\code{self}を呼び
出す。自分自身のPIDを他のプロセスに送ることによって、送信先のプロセスは
送信元のプロセスとさらに通信できる。たとえば、\code{\{ok, N\}}と
\code{\{error, Reason\}}という形のメッセージを受信することを期待してい
るプロセスがあるとする。このようなメッセージを処理するには、
\code{receive}文を使う。

\begin{verbatim}
receive
   {ok, N} ->
      N+1;
   {error, _} ->
      0
end
\end{verbatim}

%% \noindent The result of this is a number determined by the pattern-matched
%% clause. When the value of a variable is not needed in the pattern
%% match, the underscore wild-card can be used as shown above.

\noindent この文の結果はパターンがマッチした節によって決まる数になる。
変数の値がパターン・マッチに不要な場合は、上記の例のようにワイルド・カー
ドとしてアンダースコアを使う。

%% Message passing between processes is asynchronous, and the messages
%% received by a process are placed in the process's mailbox in the order
%% in which they arrive. Suppose that now the \code{receive} expression
%% above is to be executed: if the first element in the mailbox is either
%% \code{\{ok, N\}} or \code{\{error, Reason\}} the corresponding result
%% will be returned. If the first message in the mailbox is not of this
%% form, it is retained in the mailbox and the second is processed in a
%% similar way. If no message matches, the receive will wait for a
%% matching message to be received.

プロセス間のメッセージ・パッシングは非同期に行なわれ、プロセスが受信す
るメッセージは到着順にプロセスのメールボックスに置かれる。たとえば上記
の\code{receive}式を実行するとしよう。メールボックスの最初の要素が
\code{\{ok, N\}}または\code{\{error, Reason\}}ならば、それに対応する結
果が返される。メールボックスの最初のメッセージがこの形式でなければ、そ
のメッセージはメールボックスに留め置かれ、次のメッセージが同様に処理さ
れる。どのメッセージもマッチしなければ、マッチするメッセージを受信する
までreceiveは待機する。

%% Processes terminate for two reasons. If there is no more code to
%% execute, they are said to terminate with reason \emph{normal}. If a
%% process encounters a run-time error, it is said to terminate with a
%% \emph{non-normal} reason. A process terminating will not affect other
%% processes unless they are linked to it. Processes can link to each
%% other through the \code{link(Pid)} BIF or when calling the
%% \code{spawn\_link(Module, Function, Arguments)}. If a process
%% terminates, it sends an EXIT signal to processes in its link set. If
%% the termination reason is non-normal, the process terminates itself,
%% propagating the EXIT signal further. By calling the
%% \code{process\_flag(trap\_exit, true)} BIF, processes can receive the
%% EXIT signals as Erlang messages in their mailbox instead of
%% terminating.

プロセスが終了する理由には2つある。これ以上実行するコードがない場合は、
\emph{通常}終了と呼ばれる。実行時エラーが発生した場合は、\emph{異常/*
  non-normal*/}終了と呼ばれる。プロセスが他のプロセスにリンクしていな
ければ、プロセスが終了しても他のプロセスには影響を与えない。プロセス同
士をリンクするには、\code{link(Pid)} BIFを使うか、
\code{spawn\_link(Module, Function, Arguments)}を使ってプロセスを生成す
る。プロセスは終了するときに自分のリンク・セットに含まれるプロセス群に
EXITシグナルを送信する。終了理由が異常の場合、プロセスは自分自身を終了
し、EXITシグナルをさらに伝播させる。\code{process\_flag(trap\_exit,
  true)} BIFを呼び出しておくことによって、EXITシグナルが発生した場合に
プロセスは終了せずにEXITシグナルをメールボックスへのErlangメッセージと
して受信できる。

%% Riak uses EXIT signals to monitor the well-being of helper processes
%% performing non-critical work initiated by the request-driving finite
%% state machines. When these helper processes terminate abnormally, the
%% EXIT signal allows the parent to either ignore the error or restart
%% the process.

Riakでは、要求を生成する有限状態マシン群が起動したクリティカルでない作
業を実行するヘルパー・プロセスの状態を監視するためにEXITシグナルを利用
している。/*長いので分割すべき?*/このようなヘルパー・プロセスが異常
終了した場合、親プロセスはEXITシグナルを利用して、エラーを無視したりプ
ロセスを起動するといった対処ができる。

\end{aosasect1}

%% \begin{aosasect1}{Process Skeletons}
\begin{aosasect1}{プロセスの骨格}

%% We previously introduced the notion that processes follow a common
%% pattern regardless of the particular purpose for which the process was
%% created. To start off, a process has to be spawned and then,
%% optionally, have its alias registered. The first action of the newly
%% spawned process is to initialize the process loop data. The loop data
%% is often the result of arguments passed to the \code{spawn} built-in
%% function at the initialization of the process. Its loop data is stored
%% in a variable we refer to as the process state. The state, often
%% stored in a record, is passed to a receive-evaluate function, running
%% a loop which receives a message, handles it, updates the state, and
%% passes it back as an argument to a tail-recursive call. If one of the
%% messages it handles is a `stop' message, the receiving process will
%% clean up after itself and then terminate.

ここまでで、プロセスはその用途に関わらずある共通のパターンに従うことを
学んだ。まず、プロセスは生成する必要があり、必要ならばその別名を登録し
ておくこともできる。新しく生成されたプロセスの最初の動作は、プロセスの
ループ・データの初期化だ。プロセスの初期化時に\code{spawn}組み込み関数
に渡された引数の結果をループ・データとして使うことが多い。このループ・
データはプロセス状態と呼ばれる変数に格納する。この状態はレコー
ドに格納することが多く、受信/評価関数/*receive-evalutate function*/
に渡される。受信/評価関数が実行するループはメッセージを受信して処理し、
状態を更新してその結果を末尾再帰呼び出しに渡す。「停止(stop)」メッ
セージを受信したプロセスは後始末をしてから終了する。

%% This is a recurring theme among processes that will occur regardless
%% of the task the process has been assigned to perform. With this in
%% mind, let's look at the differences between the processes that conform
%% to this pattern:

プロセスに与えられた役割に関わらず、この処理パターンはよく使われる。こ
のことを念頭に、この処理パターンに従ったプロセス同士で何が違うか見てみ
よう。

\begin{aosaitemize}

  %% \item The arguments passed to the \code{spawn} BIF calls will differ
  %% from one process to another.

  \item \code{spawn} BIFに渡された引数はプロセスごとに異なる。

  %% \item You have to decide whether you should register a process under
  %% an alias, and if you do, what alias should be used.

  \item プロセスを別名で登録するかどうか、登録するならばどのような名前
    にするかを決める必要がある。

  %% \item In the function that initializes the process state, the actions
  %% taken will differ based on the tasks the process will perform.

  \item プロセス状態を初期化する関数は、プロセスが実行する作業に応じて
    異なる動作を行なう。

  %% \item The state of the system is represented by the loop data in every
  %% case, but the contents of the loop data will vary among processes.

  \item システムの状態はどの場合でもループ・データで表わされるが、ルー
    プ・データの内容はプロセスによって異なる。

  %% \item When in the body of the receive-evaluate loop, processes will
  %% receive different messages and handle them in different ways.

  \item 受信/評価ループの本体部分では、プロセスは異なるメッセージを受信
    して異なる処理を行なう。

  %% \item Finally, on termination, the cleanup will vary from process to
  %% process.

  \item 最後に、終了時の後始末はプロセスによって異なる。

\end{aosaitemize}

%% So, even if a skeleton of generic actions exists, these actions are
%% complemented by specific ones that are directly related to the tasks
%% assigned to the process. Using this skeleton as a template,
%% programmers can create Erlang processes that act as servers, finite
%% state machines, event handlers and supervisors. But instead of
%% re-implementing these patterns every time, they have been placed in
%% library modules referred to as behaviors. They come as part as the OTP
%% middleware.

このように、汎用の動作の骨格はあるが、プロセスの役割に直接関わる特定の
動作で補うことになる。この骨格をテンプレートとして使えば、サーバーや有
限状態マシン、イベント・ハンドラ、スーパバイザなどとして動作するErlang
プロセスを作ることができる。しかし、これらのパターンは、毎度実装しなお
さなくてよいように、ビヘイビアと呼ばれるライブラリ・モジュールになって
いる。ビヘイビアはOTPミドルウェアに入っている。

\end{aosasect1}

%% \begin{aosasect1}{OTP Behaviors}
\begin{aosasect1}{OTPビヘイビア}

%% The core team of developers committing to Riak is spread across nearly
%% a dozen geographical locations.  Without very tight coordination and
%% templates to work from, the result would consist of different
%% client/server implementations not handling special borderline cases
%% and concurrency-related errors. There would probably be no uniform way
%% to handle client and server crashes or guaranteeing that a response
%% from a request is indeed the response, and not just any message that
%% conforms to the internal message protocol.

Riakにコミットしている開発者のコア・チームは、十数もの場所に散らばって
いる。緊密な調整とベースになるテンプレートがなければ、特殊な境界条件や
並行実行関連のエラーをまともに扱えないクライアント/サーバ実装がばらばら
にできてしまうだろう。クライアントやサーバのクラッシュを扱う方法も一貫
しないだろうし、要求に対する応答が単に内部メッセージ・プロトコルに準拠
しているだけではなくて、期待したものであることを保証する方法も1つに定ま
らないだろう。

%% OTP is a set of Erlang libraries and design principles providing
%% ready-made tools with which to develop robust systems. Many of these
%% patterns and libraries are provided in the form of ``behaviors.''

OTPはErlangのライブラリ群といくつかの基本原理で構成されていて、頑健なシ
ステムを開発するための既製ツールとして利用できる。OTPのパターンやライブ
ラリの多くは``behaviors''として用意されている。

%% OTP behaviors address these issues by providing library modules that
%% implement the most common concurrent design patterns. Behind the
%% scenes, without the programmer having to be aware of it, the library
%% modules ensure that errors and special cases are handled in a
%% consistent way. As a result, OTP behaviors provide a set of
%% standardized building blocks used in designing and building
%% industrial-grade systems.

上記のような問題は、よくある並行デザイン・パターンを実装したライブラリ・
モジュールであるOTPビヘイビアによって解決できる。プログラマが気にかけな
くても、エラーや特別な場合が一貫した方法で処理されるように裏でライブラ
リ・モジュールが対処してくれる。このため、OTPビヘイビアは実用/*
industrial-grade*/システムの設計や構築で利用できる、標準化されたブロッ
クの集合となっている。

%% \begin{aosasect2}{Introduction}
\begin{aosasect2}{はじめに}

%% OTP behaviors are provided as library modules in the \code{stdlib}
%% application which comes as part of the Erlang/OTP distribution. The
%% specific code, written by the programmer, is placed in a separate
%% module and called through a set of predefined callback functions
%% standardized for each behavior. This callback module will contain all
%% of the specific code required to deliver the desired functionality.

OTPビヘイビアは、Erlang/OTP配布物に含まれている\code{stdlib}アプリケー
ションのライブラリ・モジュールとして提供されている。プログラマが書くア
プリケーションごとのコードは別のモジュールに置かれ、各ビヘイビアごとに
標準化されている定義済みコールバック関数を通じて呼び出される。このコー
ルバック・モジュールには、必要な機能を実現するために必要なコードをすべ
て入れておく。

%% OTP behaviors include worker processes, which do the actual
%% processing, and supervisors, whose task is to monitor workers and
%% other supervisors. Worker behaviors, often denoted in diagrams as
%% circles, include servers, event handlers, and finite state
%% machines. Supervisors, denoted in illustrations as squares, monitor
%% their children, both workers and other supervisors, creating what is
%% called a supervision tree.

OTPビヘイビアには、実際の処理を行なうワーカ・プロセスと、ワーカや他のスー
パバイザを監視するためのスーパバイザが含まれている。ワーカ・ビヘイビア
(図では円で示す場合が多い)には、サーバ、イベント・ハンドラ、有限状態マ
シンがある。スーパバイザ(図では四角で示す)は監視ツリーを構成する子プロ
セス群(ワーカや他のスーパバイザ)を監視する。

\aosafigure{../images/riak/supervision-tree.eps}{OTP Riak Supervision Tree}{fig.erlang.supervision}

%% Supervision trees are packaged into a behavior called an
%% application. OTP applications are not only the building blocks of
%% Erlang systems, but are also a way to package reusable
%% components. Industrial-grade systems like Riak consist of a set of
%% loosely coupled, possibly distributed applications. Some of these
%% applications are part of the standard Erlang distribution and some are
%% the pieces that make up the specific functionality of Riak.

監視ツリーはアプリケーションと呼ばれるビヘイビアにまとめられる。OTPアプ
リケーションはErlangシステムの構築単位になるだけでなく、再利用可能なコ
ンポーネントをパッケージ化するための方法としても使われる。Riakのような
実用システムは、疎結合の、場合によっては分散したアプリケーションの集合
で構成される。このようなアプリケーションには、Erlangの標準配布物に入っ
ているものと、Riak特有の機能を実現するためのものがある。

%% Examples of OTP applications include the Corba ORB or the Simple
%% Network Management Protocol (SNMP) agent. An OTP application is a
%% reusable component that packages library modules together with
%% supervisor and worker processes. From now on, when we refer to an
%% application, we will mean an OTP application.

OTPアプリケーションの例としては、Corba ORBやSNMP (Simple Network
Management Protocol)エージェントがある。OTPアプリケーションは、スーパバ
イザとワーカ・プロセス群をまとめたライブラリ・モジュールの集合を再利用
可能なコンポーネントとしてパッケージ化したものだ。以降では、アプリケー
ションとはOTPアプリケーションのことを指すものとする。

%% The behavior modules contain all of the generic code for each given
%% behavior type. Although it is possible to implement your own behavior
%% module, doing so is rare because the ones that come with the
%% Erlang/OTP distribution will cater to most of the design patterns you
%% would use in your code. The generic functionality provided in a
%% behavior module includes operations such as:

ビヘイビアのモジュールには、そのビヘイビア・タイプに必要な汎用コードが
すべて入っている。自分でビヘイビア・モジュールを実装することもできるが、
必要になりそうなデザイン・パターンのほとんどはErlang/OTPの配布物に入っ
ているもので間に合うため、その必要はほとんどないだろう。ビヘイビア・モ
ジュールが用意する汎用の機能には以下のようなものがある:

\begin{aosaitemize}

  %% \item spawning and possibly registering the process;
  \item プロセスの生成と、場合によっては登録

  %% \item sending and receiving client messages as synchronous or
  %% asynchronous calls, including defining the internal message protocol;
  \item 同期または非同期の呼び出しによるクライアント・メッセージの送受
    信。内部メッセージ・プロトコルの定義も含む

  %% \item storing the loop data and managing the process loop; and
  \item ループ・データの格納とプロセス・ループの管理

  \item プロセスの停止

\end{aosaitemize}

%% The loop data is a variable that will contain the data the behavior
%% needs to store in between calls. After the call, an updated variant of
%% the loop data is returned. This updated loop data, often referred to
%% as the new loop data, is passed as an argument in the next call. Loop
%% data is also often referred to as the behavior state.

ループ・データとは、ビヘイビアが呼び出しを越えて覚えておく必要があるデー
タを格納する変数だ。呼び出しの後、更新されたループ・データが返される。
更新されたループ・データ(新ループ・データと呼ばれることが多い)は次の呼
び出しの引数として渡される。ループ・データはビヘイビアの状態とも呼ばれ
る。

%% The functionality to be included in the callback module for the
%% generic server application to deliver the specific required behavior
%% includes the following:

あるビヘイビアを実現するための汎用サーバ・アプリケーションのコールバッ
ク・モジュールに入れる機能には以下のようなものがある:

\begin{aosaitemize}

  %% \item Initializing the process loop data, and, if the process is
  %% registered, the process name.
  \item プロセスのループ・データの初期化。プロセスを登録する場合はプロ
    セス名の初期化も

  %% \item Handling the specific client requests, and, if synchronous, the
  %% replies sent back to the client.
  \item 特定のクライアント要求の処理。同期型の場合はクライアントへの応答の送信

  %% \item Handling and updating the process loop data in between the
  %% process requests.
  \item プロセスへの要求の合間に行なう、プロセス・ループ・データの処理と更新

  \item 終了時のプロセス・ループ・データの後始末

\end{aosaitemize}

\end{aosasect2}

%% \begin{aosasect2}{Generic Servers}
\begin{aosasect2}{汎用サーバ}

%% Generic servers that implement client/server behaviors are defined in
%% the \code{gen\_server} behavior that comes as part of the standard
%% library application. In explaining generic servers, we will use the
%% \code{riak\_core\_node\_watcher.erl} module from the \code{riak\_core}
%% application. It is a server that tracks and reports on which
%% sub-services and nodes in a Riak cluster are available. The module
%% headers and directives are as follows:

クライアント/サーバ型のビヘイビアを実装する汎用サーバは、標準ライブラリ・
アプリケーションの一部である\code{gen\_server}ビヘイビアで定義されてい
る。この汎用サーバの解説では、\code{riak\_core}アプリケーションの
\code{riak\_core\_node\_watcher.erl}モジュールを使う。このモジュールは、
Riakクラスタ内のどのサブサービスやノードが利用可能か追跡して報告するサー
バだ。モジュールのヘッダとディレクティブは以下のようになっている:

\begin{verbatim}
-module(riak_core_node_watcher).
-behavior(gen_server).
%% API
-export([start_link/0,service_up/2,service_down/1,node_up/0,node_down/0,services/0,
         services/1,nodes/1,avsn/0]).
%% gen_server callbacks
-export([init/1,handle_call/3,handle_cast/2,handle_info/2,terminate/2, code_change/3]).

-record(state, {status=up, services=[], peers=[], avsn=0, bcast_tref,
                bcast_mod={gen_server, abcast}}).
\end{verbatim}

%% We can easily recognize generic servers through the
%% \code{-behavior(gen\_server).} directive. This directive is used by
%% the compiler to ensure all callback functions are properly
%% exported. The record state is used in the server loop data.

これが汎用サーバであることは\code{-behavior(gen\_server)}ディレクティブ
から容易にわかる。このディレクティブは、ールバック関数をすべて適切にエ
クスポートされているかコンパイラがチェックするために使われる。stateレコー
ドはサーバのループ・データで使われる。

\end{aosasect2}

%% \begin{aosasect2}{Starting Your Server}
\begin{aosasect2}{サーバの起動}

%% With the \code{gen\_server} behavior, instead of using the
%% \code{spawn} and \code{spawn\_link} BIFs, you will use the
%% \code{gen\_server:start} and \code{gen\_server:start\_link}
%% functions. The main difference between \code{spawn} and \code{start}
%% is the synchronous nature of the call. Using \code{start} instead of
%% \code{spawn} makes starting the worker process more deterministic and
%% prevents unforeseen race conditions, as the call will not return the
%% PID of the worker until it has been initialized. You call the
%% functions with either of:

\code{gen\_server}ビヘイビアでは、\code{spawn} BIFと\code{spawn\_link}
BIFを使う代わりに、\code{gen\_server:start}関数と
\code{gen\_server:start\_link}関数を使う。\code{spawn}と\code{start}の
主な違いは、呼び出しが同期型かどうかだ。\code{spawn}ではなく
\code{start}を使うことによって、ワーカ・プロセスのPIDが初期化されるまで
呼び出しが戻らなくなるため、ワーカ・プロセスの起動をより決定的にするこ
とができ、予期しない競合条件による問題を予防できる。\code{start}は次の
いずれかの形で呼び出す:

\begin{verbatim}
gen_server:start_link(ServerName, CallbackModule, Arguments, Options)
gen_server:start_link(CallbackModule, Arguments, Options)
\end{verbatim}

%% \noindent \code{ServerName} is a tuple of the format \code{\{local, Name\}} or
%% \code{\{global, Name\}}, denoting a local or global \code{Name} for the
%% process alias if it is to be registered. Global names allow servers
%% to be transparently accessed across a cluster of distributed Erlang
%% nodes. If you do not want to register the process and instead
%% reference it using its PID, you omit the argument and use a
%% \code{start\_link/3} or \code{start/3} function call
%% instead. \code{CallbackModule} is the name of the module in which
%% the specific callback functions are placed, \code{Arguments} is a
%% valid Erlang term that is passed to the \code{init/1} callback
%% function, while \code{Options} is a list that allows you to set the
%% memory management flags \code{fullsweep\_after} and \code{heapsize},
%% as well as other tracing and debugging flags.

\noindent \code{ServerName}は\code{\{local, Name\}}または\code{\{global, Name\}}という形のタプルで、プロセスを登録する際のプロセス別名を示すローカルまたはグローバルの\code{Name}になる。グローバルな名前を使うと、分散Erlangノードのクラスタからサーバを透過的にアクセスできる。プロセスを登録せずにPIDでプロセスを参照する場合は、この引数を持たない\code{start\_link/3}または\code{start/3}を使う。\code{CallbackModule}はコールバック関数が置かれているモジュールの名前、\code{Arguments}は\code{init/1}コールバック関数に渡される有効なErlang項だ。また、\code{Options}には、メモリ管理関連の\code{fullsweep\_after}フラグや\code{heapsize}フラグ、その他トレース用やデバッグ用のフラグのリストを指定できる。

%% In our example, we call \code{start\_link/4}, registering the process
%% with the same name as the callback module, using the \code{?MODULE}
%% macro call. This macro is expanded to the name of the module it is
%% defined in by the preprocessor when compiling the code. It is always
%% good practice to name your behavior with an alias that is the same as
%% the callback module it is implemented in. We don't pass any arguments,
%% and as a result, just send the empty list. The options list is kept
%% empty:

この例では、\code{?MODULE}を利用して\code{start\_link/4}を呼び出し、コールバック・モジュールと同じ名前でプロセスを登録する。このマクロは、コードをコンパイルする際にプリプロセッサによってモジュール名に展開される。ビヘイビアの名前と、そのビヘイビアを実装しているコールバック・モジュールはいつも同じにしておいたほうがよい。引数は渡さないので、結果的に空のリストを送信することになる。オプション・リストは空のままだ:

\begin{verbatim}
start_link() ->
    gen_server:start_link({local, ?MODULE}, ?MODULE, [], []).
\end{verbatim}

%% \noindent The obvious difference between the \code{start\_link} and \code{start}
%% functions is that \code{start\_link} links to its parent, most often a
%% supervisor, while \code{start} doesn't. This needs a special mention
%% as it is an OTP behavior's responsibility to link itself to the
%% supervisor. The \code{start} functions are often used when testing
%% behaviors from the shell, as a typing error causing the shell process
%% to crash would not affect the behavior. All variants of the
%% \code{start} and \code{start\_link} functions return \code{\{ok, Pid\}}.

\noindent \code{start\_link}関数と\code{start}関数の明らかな違いは、
\code{start\_link}は親(たいていはスーパバイザ)にリンクするのに対し、
\code{start}はリンクしない点だ。自分自身をスーパバイザにリンクするのは
OTPビヘイビアの責任になるため、この点は注意が必要だ。\code{start}関数は
ビヘイビアをシェルからテストする際によく使う。入力ミスでシェル・プロセ
スをクラッシュさせてしまってもビヘイビアには影響がないからだ。
\code{start}関数や\code{start\_link}関数とそれらの変種はすべて
\code{\{ok, Pid\}}を返す。

%% The \code{start} and \code{start\_link} functions will spawn a new
%% process that calls the \code{init(Arguments)} callback function in the
%% \code{CallbackModule}, with the \code{Arguments} supplied. The
%% \code{init} function must initialize the \code{LoopData} of the server
%% and has to return a tuple of the format \code{\{ok,
%% LoopData\}}. \code{LoopData} contains the first instance of the loop
%% data that will be passed between the callback functions. If you want
%% to store some of the arguments you passed to the \code{init} function, you
%% would do so in the \code{LoopData} variable. The \code{LoopData} in
%% the Riak node watcher server is the result of the
%% \code{schedule\_broadcast/1} called with a record of type \code{state}
%% where the fields are set to the default values:

\code{start}関数と\code{start\_link}関数は、\code{CallbackModule}にある\code{init(Arguments)}コールバック関数を呼び出すプロセスを生成する。この関数には\code{Arguments}が渡される。\code{init}関数はサーバの\code{LoopData}を初期化し、\code{\{ok, LoopData\}}という形のタプルを返す必要がある。\code{LoopData}にはループ・データの最初のインスタンスを格納する。このループ・データはコールバック関数の間で受け渡される。\code{init}関数に渡された引数の一部を取っておきたい場合は、\code{LoopData}変数に入れておけばよい。Riakのノード監視サーバの\code{LoopData}は、\code{state}型のレコードを渡して呼び出された\code{schedule\_broadcast/1}の結果だ。各フィールドにはデフォルト値が設定される:

%% \begin{verbatim}
%% init([]) ->

%%     %% Watch for node up/down events
%%     net_kernel:monitor_nodes(true),

%%     %% Setup ETS table to track node status
%%     ets:new(?MODULE, [protected, named_table]),

%%     {ok, schedule_broadcast(#state{})}.
%% \end{verbatim}

\begin{verbatim}
init([]) ->

    %% ノードのアップ/ダウン・イベントを監視する
    net_kernel:monitor_nodes(true),

    %% ノードのステータスを追跡するためのETSテーブルを用意する
    ets:new(?MODULE, [protected, named_table]),

    {ok, schedule_broadcast(#state{})}.
\end{verbatim}

%% Although the supervisor process might call the \code{start\_link/4}
%% function, a different process calls the \code{init/1} callback: the
%% one that was just spawned.  As the purpose of this server is to
%% notice, record, and broadcast the availability of sub-services within
%% Riak, the initialization asks the Erlang runtime to notify it of such
%% events, and sets up a table to store this information in.  This needs
%% to be done during initialization, as any calls to the server would
%% fail if that structure did not yet exist. Do only what is necessary
%% and minimize the operations in your \code{init} function, as the call
%% to \code{init} is a synchronous call that prevents all of the other
%% serialized processes from starting until it returns.

\code{start\_link/4}関数を呼び出すのはスーパバイザ・プロセスだが、
\code{init/1}コールバックを呼び出すのは別のプロセス、つまり生成されたば
かりのプロセスだ。このサーバの役割はRiakの各サブサービスの利用可能状態
を監視、記録、ブロードキャストすることなので、初期化処理ではそのような
イベントが通知されるようにErlangランタイムに依頼し、その情報を記録する
ためのテーブルを用意する。このデータ構造が存在しないとサーバの呼び出し
はすべて失敗してしまうため、この準備は初期化中にやっておかなければなら
ない。\code{init}の呼び出しは同期型で、この呼び出しが戻るまでほかの逐次
実行プロセスはどれも実行できないので、\code{init}関数での処理は必要最小
限に抑えるべきだ。

\end{aosasect2}

%% \begin{aosasect2}{Passing Messages}
\begin{aosasect2}{メッセージの受け渡し}

%% If you want to send a synchronous message to your server, you use the
%% \code{gen\_server:call/2} function. Asynchronous calls are made using
%% the \code{gen\_server:cast/2} function. Let's start by taking two
%% functions from Riak's service API; we will provide the rest of the
%% code later. They are called by the client process and result in a
%% synchronous message being sent to the server process registered with
%% the same name as the callback module. Note that validating the data
%% sent to the server should occur on the client side. If the client
%% sends incorrect information, the server should terminate.

サーバに同期型メッセージを送信する必要があるときは、
\code{gen\_server:call/2}関数を使う。非同期型呼び出しには
\code{gen\_server:cast/2}関数を使う。まず、RiakのサービスAPIの2つの関数
を例に見てみよう。残りのコードは後で説明する。クライアント・プロセスが
これらの関数を呼び出すと、コールバック・モジュールと同じ名前で登録され
ているサーバ・プロセスに同期型メッセージが送信される。ただし、サーバに
送信されるデータの検証/*validating*/はクライアント側で行なわれる。
クライアントが不正な情報を送ってきた場合、サーバは終了する。

\begin{verbatim}
service_up(Id, Pid) ->
    gen_server:call(?MODULE, {service_up, Id, Pid}).

service_down(Id) ->
    gen_server:call(?MODULE, {service_down, Id}).
\end{verbatim}

%% \noindent Upon receiving the messages, the \code{gen\_server} process calls the
%% \code{handle\_call/3} callback function dealing with the messages in
%% the same order in which they were sent:

\noindent \code{gen\_server}プロセスはメッセージを受信すると\code{handle\_call/3}コールバック関数を呼び出す。この関数は、メッセージを送信順に/*受信順?*/処理する:

%% \begin{verbatim}
%% handle_call({service_up, Id, Pid}, _From, State) ->
%%     %% Update the set of active services locally
%%     Services = ordsets:add_element(Id, State#state.services),
%%     S2 = State#state { services = Services },

%%     %% Remove any existing mrefs for this service
%%     delete_service_mref(Id),

%%     %% Setup a monitor for the Pid representing this service
%%     Mref = erlang:monitor(process, Pid),
%%     erlang:put(Mref, Id),
%%     erlang:put(Id, Mref),

%%     %% Update our local ETS table and broadcast
%%     S3 = local_update(S2),
%%     {reply, ok, update_avsn(S3)};

%% handle_call({service_down, Id}, _From, State) ->
%%     %% Update the set of active services locally
%%     Services = ordsets:del_element(Id, State#state.services),
%%     S2 = State#state { services = Services },

%%     %% Remove any existing mrefs for this service
%%     delete_service_mref(Id),

%%     %% Update local ETS table and broadcast
%%     S3 = local_update(S2),
%%     {reply, ok, update_avsn(S3)};
%% \end{verbatim}

\begin{verbatim}
handle_call({service_up, Id, Pid}, _From, State) ->
    %% アクティブなサービスの集合をローカルに更新する/*要確認*/
    Services = ordsets:add_element(Id, State#state.services),
    S2 = State#state { services = Services },

    %% このサービスの既存のmrefをすべて削除する
    delete_service_mref(Id),

    %% このサービスを示すPidのモニタを用意する
    Mref = erlang:monitor(process, Pid),
    erlang:put(Mref, Id),
    erlang:put(Id, Mref),

    %% ローカルなETSを更新しブロードキャストする
    S3 = local_update(S2),
    {reply, ok, update_avsn(S3)};

handle_call({service_down, Id}, _From, State) ->
    %% アクティブなサービスの集合をローカルに更新する
    Services = ordsets:del_element(Id, State#state.services),
    S2 = State#state { services = Services },

    %% このサービスの既存のmrefをすべて削除する
    delete_service_mref(Id),

    %% ローカルなETSを更新しブロードキャストする
    S3 = local_update(S2),
    {reply, ok, update_avsn(S3)};
\end{verbatim}

\noindent Note the return value of the callback function. The tuple contains the
control atom \code{reply}, telling the \code{gen\_server} generic code
that the second element of the tuple (which in both of these cases is
the atom \code{ok}) is the reply sent back to the client. The third
element of the tuple is the new \code{State}, which, in a new
iteration of the server, is passed as the third argument to the
\code{handle\_call/3} function; in both cases here it is updated to
reflect the new set of available services. The argument \code{\_From}
is a tuple containing a unique message reference and the client
process identifier. The tuple as a whole is used in library functions
that we will not be discussing in this chapter. In the majority of
cases, you will not need it.

The \code{gen\_server} library module has a number of mechanisms and
safeguards built in that operate behind the scenes. If your client
sends a synchronous message to your server and you do not get a
response within five seconds, the process executing the \code{call/2}
function is terminated. You can override this by using
\code{gen\_server:call(Name, Message, Timeout)} where \code{Timeout}
is a value in milliseconds or the atom \code{infinity}.

The timeout mechanism was originally put in place for deadlock
prevention purposes, ensuring that servers that accidentally call each
other are terminated after the default timeout. The crash report would
be logged, and hopefully would result in the error being debugged and
fixed. Most applications will function appropriately with a timeout of
five seconds, but under very heavy loads, you might have to fine-tune
the value and possibly even use \code{infinity}; this choice is
application-dependent. All of the critical code in Erlang/OTP uses
\code{infinity}.  Various places in Riak use different values for the
timeout: \code{infinity} is common between coupled pieces of the
internals, while \code{Timeout} is set based on a user-passed
parameter in cases where the client code talking to Riak has specified
that an operation should be allowed to time out.

Other safeguards when using the \code{gen\_server:call/2} function
include the case of sending a message to a nonexistent server and
the case of a server crashing before sending its reply. In
both cases, the calling process will terminate. In raw Erlang, sending
a message that is never pattern-matched in a receive clause is a bug
that can cause a memory leak. Two different strategies are used in
Riak to mitigate this, both of which involve ``catchall'' matching
clauses.  In places where the message might be user-initiated, an
unmatched message might be silently discarded.  In places where such a
message could only come from Riak's internals, it represents a bug and
so will be used to trigger an error-alerting internal crash report,
restarting the worker process that received it.

Sending asynchronous messages works in a similar way. Messages are
sent asynchronously to the generic server and handled in the
\code{handle\_cast/2} callback function. The function has to return a
tuple of the format \code{\{reply, NewState\}}. Asynchronous calls are
used when we are not interested in the request of the server and are
not worried about producing more messages than the server can
consume. In cases where we are not interested in a response but want
to wait until the message has been handled before sending the next
request, we would use a \code{gen\_server:call/2}, returning the atom
\code{ok} in the reply. Picture a process generating database entries
at a faster rate than Riak can consume. By using asynchronous calls,
we risk filling up the process mailbox and make the node run out of
memory.  Riak uses the message-serializing properties of synchronous
\code{gen\_server} calls to regulate load, processing the next request
only when the previous one has been handled.  This approach eliminates
the need for more complex throttling code: in addition to enabling
concurrency, \code{gen\_server} processes can also be used to
introduce serialization points.

\end{aosasect2}

\begin{aosasect2}{Stopping the Server}

How do you stop the server? In your \code{handle\_call/3} and
\code{handle\_cast/2} callback functions, instead of returning
\code{\{reply, Reply, NewState\}} or \code{\{noreply, NewState\}}, you
can return \code{\{stop, Reason, Reply, NewState\}} or \code{\{stop,
Reason, NewState\}}, respectively. Something has to trigger this
return value, often a stop message sent to the server. Upon
receiving the stop tuple containing the \code{Reason} and
\code{State}, the generic code executes the \code{terminate(Reason,
State)} callback.

The \code{terminate} function is the natural place to insert the code
needed to clean up the \code{State} of the server and any other
persistent data used by the system. In our example, we send out one
last message to our peers so that they know that this node watcher is
no longer up and watching. In this example, the variable \code{State}
contains a record with the fields \code{status} and \code{peers}:

\begin{verbatim}
terminate(_Reason, State) ->
    %% Let our peers know that we are shutting down
    broadcast(State#state.peers, State#state { status = down }).
\end{verbatim}

Use of the behavior callbacks as library functions and invoking them
from other parts of your program is an extremely bad practice. For
example, you should never call
\path{riak_core_node_watcher:init(Args)} from another module to
retrieve the initial loop data. Such retrievals should be done through
a synchronous call to the server. Calls to behavior callback functions
should originate only from the behavior library modules as a result of
an event occurring in the system, and never directly by the user.

\end{aosasect2}

\end{aosasect1}

\begin{aosasect1}{Other Worker Behaviors}

A large number of other worker behaviors can and have been implemented
using these same ideas.

\begin{aosasect2}{Finite State Machines}

Finite state machines (FSMs), implemented in the \code{gen\_fsm} behavior
module, are a crucial component when implementing protocol stacks in
telecom systems (the problem domain Erlang was originally invented
for). States are defined as callback functions named after the state
that return a tuple containing the next \code{State} and the updated
loop data. You can send events to these states synchronously and
asynchronously. The finite state machine callback module should also
export the standard callback functions such as \code{init},
\code{terminate}, and \code{handle\_info}.

Of course, finite state machines are not telecom specific. In Riak,
they are used in the request handlers. When a client issues a request
such as \code{get}, \code{put}, or \code{delete}, the process
listening to that request will spawn a process implementing the
corresponding \code{gen\_fsm} behavior. For instance, the
\code{riak\_kv\_get\_fsm} is responsible for handling a \code{get}
request, retrieving data and sending it out to the client process. The
FSM process will pass through various states as it determines which
nodes to ask for the data, as it sends out messages to those nodes, and as
it receives data, errors, or timeouts in response.

\end{aosasect2}

\begin{aosasect2}{Event Handlers}

Event handlers and managers are another behavior implemented in the
\code{gen\_event} library module. The idea is to create a centralized
point that receives events of a specific kind. Events can be sent
synchronously and asynchronously with a predefined set of actions
being applied when they are received. Possible responses to events
include logging them to file, sending off an alarm in the form of an
SMS, or collecting statistics. Each of these actions is defined in a
separate callback module with its own loop data, preserved between
calls. Handlers can be added, removed, or updated for every specific
event manager. So, in practice, for every event manager there could
be many callback modules, and different instances of these callback
modules could exist in different managers. Event handlers include
processes receiving alarms, live trace data, equipment related events
or simple logs.

One of the uses for the \code{gen\_event} behavior in Riak is for
managing subscriptions to ``ring events'', i.e., changes to the
membership or partition assignment of a Riak cluster.  Processes on a
Riak node can register a function in an instance of
\code{riak\_core\_ring\_events}, which implements the
\code{gen\_event} behavior.  Whenever the central process managing the
ring for that node changes the membership record for the overall
cluster, it fires off an event that causes each of those callback
modules to call the registered function.  In this fashion, it is
easy for various parts of Riak to respond to changes in one of Riak's
most central data structures without having to add complexity to the
central management of that structure.

Most common concurrency and communication patterns are handled with
the three primary behaviors we've just discussed: \code{gen\_server},
\code{gen\_fsm}, and \code{gen\_event}.  However, in large systems,
some application-specific patterns emerge over time that warrant the
creation of new behaviors.  Riak includes one such behavior,
\code{riak\_core\_vnode}, which formalizes how virtual nodes are
implemented.  Virtual nodes are the primary storage abstraction in
Riak, exposing a uniform interface for key-value storage to the
request-driving FSMs.  The interface for callback modules is specified
using the \code{behavior\_info/1} function, as follows:


\begin{verbatim}
behavior_info(callbacks) ->
    [{init,1},
     {handle_command,3},
     {handoff_starting,2},
     {handoff_cancelled,1},
     {handoff_finished,2},
     {handle_handoff_command,3},
     {handle_handoff_data,2},
     {encode_handoff_item,2},
     {is_empty,1},
     {terminate,2},
     {delete,1}];
\end{verbatim}

\noindent The above example shows the \code{behavior\_info/1} function from
\code{riak\_core\_vnode}.  The list of \code{\{CallbackFunction,
Arity\}} tuples defines the contract that callback modules must
follow.  Concrete virtual node implementations must export these
functions, or the compiler will emit a warning. Implementing your own
OTP behaviors is relatively straightforward. Alongside defining your
callback functions, using the \code{proc\_lib} and \code{sys} modules,
you need to start them with particular functions, handle system
messages and monitor the parent in case it terminates.

\end{aosasect2}

\end{aosasect1}

\begin{aosasect1}{Supervisors}

The supervisor behavior's task is to monitor its children and, based
on some preconfigured rules, take action when they terminate. Children
consist of both supervisors and worker processes. This allows the Riak
codebase to focus on the correct case, which enables the supervisor to
handle software bugs, corrupt data or system errors in a consistent
way across the whole system. In the Erlang world, this non-defensive
programming approach is often referred to the ``let it crash''
strategy. The children that make up the supervision tree can include
both supervisors and worker processes. Worker processes are OTP
behaviors including the \code{gen\_fsm}, \code{gen\_server}, and
\code{gen\_event}. The Riak team, not having to handle borderline
error cases, get to work with a smaller code base. This code base,
because of its use of behaviors, is smaller to start off with, as it
only deals with specific code. Riak has a top-level supervisor like
most Erlang applications, and also has sub-supervisors for groups
of processes with related responsibilities.  Examples include Riak's
virtual nodes, TCP socket listeners, and query-response managers.

\begin{aosasect2}{Supervisor Callback Functions}

To demonstrate how the supervisor behavior is implemented, we will use
the \code{riak\_core\_sup.erl} module. The Riak core supervisor is the
top level supervisor of the Riak core application. It starts a set of
static workers and supervisors, together with a dynamic number of
workers handling the HTTP and HTTPS bindings of the node's RESTful API
defined in application specific configuration files. In a similar way
to \code{gen\_servers}, all supervisor callback modules must include
the \code{-behavior(supervisor).} directive. They are started using
the \code{start} or \code{start\_link} functions which take the
optional \code{ServerName}, the \code{CallBackModule}, and an
\code{Argument} which is passed to the \code{init/1} callback
function.

Looking at the first few lines of code in the
\code{riak\_core\_sup.erl} module, alongside the behavior directive
and a macro we will describe later, we notice the \code{start\_link/3}
function:

\begin{verbatim}
-module(riak_core_sup).
-behavior(supervisor).
%% API
-export([start_link/0]).
%% Supervisor callbacks
-export([init/1]).
-define(CHILD(I, Type), {I, {I, start_link, []}, permanent, 5000, Type, [I]}).
start_link() ->
    supervisor:start_link({local, ?MODULE}, ?MODULE, []).
\end{verbatim}

\noindent Starting a supervisor will result in a new process being spawned, and
the \code{init/1} callback function being called in the callback
module \code{riak\_core\_sup.erl}. The \code{ServerName} is a tuple of
the format \code{\{local, Name\}} or \code{\{global, Name\}}, where
\code{Name} is the supervisor's registered name. In our example, both
the registered name and the callback module are the atom
\code{riak\_core\_sup}, originating form the \code{?MODULE} macro. We
pass the empty list as an argument to \code{init/1}, treating it as a
null value. The \code{init} function is the only supervisor callback
function. It has to return a tuple with format:

\begin{verbatim}
{ok,  {SupervisorSpecification, ChildSpecificationList}}
\end{verbatim}

\noindent where \code{SupervisorSpecification} is a 3-tuple 
\code{\{RestartStrategy, AllowedRestarts, \linebreak MaxSeconds\}} containing
information on how to handle process crashes and
restarts. \code{Restart\-Strategy} is one of three configuration
parameters determining how the behavior's siblings are affected upon
abnormal termination:

\begin{aosaitemize}

  \item \code{one\_for\_one}: other processes in the supervision tree
  are not affected.

  \item \code{rest\_for\_one}: processes started after the terminating
  process are terminated and restarted.

  \item \code{one\_for\_all}: all processes are terminated and restarted.

\end{aosaitemize}

\code{AllowedRestarts} states how many times any of the supervisor
children may terminate in \code{MaxSeconds} before the supervisor
terminates itself (and its children).  When ones terminates,
it sends an EXIT signal to its
supervisor which, based on its restart strategy, handles the termination
accordingly. The supervisor terminating after reaching the maximum
allowed restarts ensures that cyclic restarts and other issues that
cannot be resolved at this level are escalated. Chances are that the
issue is in a process located in a different sub-tree, allowing the
supervisor receiving the escalation to terminate the affected sub-tree
and restart it.

Examining the last line of the \code{init/1} callback function in the
\code{riak\_core\_sup.erl} module, we notice that this particular
supervisor has a one-for-one strategy, meaning that the processes are
independent of each other. The supervisor will allow a maximum of ten
restarts before restarting itself.

\code{ChildSpecificationList} specifies which children the supervisor
has to start and monitor, together with information on how to
terminate and restart them. It consists of a list of tuples of the
following format:

\begin{verbatim}
{Id, {Module, Function, Arguments}, Restart, Shutdown, Type, ModuleList}
\end{verbatim}

\code{Id} is a unique identifier for that particular
supervisor. \code{Module}, \code{Function}, and \code{Arguments} is an
exported function which results in the behavior \code{start\_link}
function being called, returning the tuple of the format \code{\{ok,
Pid\}}. The \code{Restart} strategy dictates what happens
depending on the termination type of the process, which can be:

\begin{aosaitemize}

\item \code{transient} processes, which are never restarted;

  \item \code{temporary} processes, are restarted only if they terminate
  abnormally; and

  \item \code{permanent} processes, which are always restarted, regardless of
  the termination being normal or abnormal.

\end{aosaitemize}

\code{Shutdown} is a value in milliseconds referring to the time the
behavior is allowed to execute in the \code{terminate} function when
terminating as the result of a restart or shutdown. The atom
\code{infinity} can also be used, but for behaviors other than
supervisors, it is highly discouraged. \code{Type} is either the atom
\code{worker}, referring to the generic servers, event handlers and
finite state machines, or the atom \code{supervisor}. Together with
\code{ModuleList}, a list of modules implementing the behavior, they
are used to control and suspend processes during the runtime software
upgrade procedures. Only existing or user implemented behaviors may be
part of the child specification list and hence included in a
supervision tree.

With this knowledge at hand, we should now be able to formulate a
restart strategy defining inter-process dependencies, fault tolerance
thresholds and escalation procedures based on a common
architecture. We should also be able to understand what is going on in
the \code{init/1} example of the \code{riak\_core\_sup.erl}
module. First of all, study the \code{CHILD} macro. It creates the
child specification for one child, using the callback module name as
\code{Id}, making it permanent and giving it a shut down time of 5
seconds. Different child types can be workers or supervisors. Have a
look at the example, and see what you can make out of it:

\begin{verbatim}
-define(CHILD(I, Type), {I, {I, start_link, []}, permanent, 5000, Type, [I]}).

init([]) ->
    RiakWebs = case lists:flatten(riak_core_web:bindings(http),
                                  riak_core_web:bindings(https)) of
                   [] ->
                       %% check for old settings, in case app.config
                       %% was not updated
                       riak_core_web:old_binding();
                   Binding ->
                       Binding
               end,

    Children =
                 [?CHILD(riak_core_vnode_sup, supervisor),
                  ?CHILD(riak_core_handoff_manager, worker),
                  ?CHILD(riak_core_handoff_listener, worker),
                  ?CHILD(riak_core_ring_events, worker),
                  ?CHILD(riak_core_ring_manager, worker),
                  ?CHILD(riak_core_node_watcher_events, worker),
                  ?CHILD(riak_core_node_watcher, worker),
                  ?CHILD(riak_core_gossip, worker) |
                  RiakWebs
                 ],
    {ok, {{one_for_one, 10, 10}, Children}}.
\end{verbatim}

Most of the \code{Children} started by this supervisor are statically
defined workers (or in the case of the \code{vnode\_sup}, a
supervisor).  The exception is the \code{RiakWebs} portion, which is
dynamically defined depending on the HTTP portion of Riak's
configuration file.

With the exception of library applications, every OTP application,
including those in Riak, will have their own supervision tree. In
Riak, various top-level applications are running in the Erlang node,
such as \code{riak\_core} for distributed systems algorithms,
\code{riak\_kv} for key/value storage semantics, \code{webmachine} for
HTTP, and more.  We have shown the expanded tree under
\code{riak\_core} to demonstrate the multi-level supervision going on.
One of the many benefits of this structure is that a given subsystem
can be crashed (due to bug, environmental problem, or intentional
action) and only that subtree will in a first instance be terminated.

The supervisor will restart the needed processes and the overall
system will not be affected. In practice we have seen this work
well for Riak.  A user might figure out how to crash a virtual node,
but it will just be restarted by \code{riak\_core\_vnode\_sup}.  If
they manage to crash that, the \code{riak\_core} supervisor will
restart it, propagating the termination to the top-level supervisor.
This failure isolation and recovery mechanism allows Riak (and Erlang)
developers to straightforwardly build resilient systems.

The value of the supervisory model was shown when one large industrial
user created a very abusive environment in order to find out where
each of several database systems would fall apart.  This environment
created random huge bursts of both traffic and failure conditions.
They were confused when Riak simply wouldn't stop running, even under
the worst such arrangement.  Under the covers, of course, they were
able to make individual processes or subsystems crash in multiple
ways---but the supervisors would clean up and restart things to put
the whole system back into working order every time.

\end{aosasect2}

\begin{aosasect2}{Applications}

The \code{application} behavior we previously introduced is used to
package Erlang modules and resources into reusable components. In OTP,
there are two kinds of applications. The most common form, called
normal applications, will start a supervision tree and all of the
relevant static workers. Library applications such as the Standard
Library, which come as part of the Erlang distribution, contain
library modules but do not start a supervision tree. This is not to
say that the code may not contain processes or supervision trees. It
just means they are started as part of a supervision tree belonging to
another application.

An Erlang system will consist of a set of loosely coupled
applications. Some are written by the developers, some are available
as open source, and others are be part of the Erlang/OTP
distribution. The Erlang runtime system and its tools treat all
applications equally, regardless of whether they are part of the
Erlang distribution or not.

\end{aosasect2}

\end{aosasect1}

\begin{aosasect1}{Replication and Communication in Riak}

Riak was designed for extreme reliability and availability at a
massive scale, and was inspired by Amazon's Dynamo storage system
\cite{bib:amazon:dynamo}.  Dynamo and Riak's architectures combine
aspects of both Distributed Hash Tables (DHTs) and traditional
databases.  Two key techniques that both Riak and Dynamo use are
\emph{consistent hashing} for replica placement and a \emph{gossip
protocol} for sharing common state.

Consistent hashing requires that all nodes in the system know about
each other, and know what partitions each node owns.  This assignment
data could be maintained in a centrally managed configuration file,
but in large configurations, this becomes extremely difficult. Another
alternative is to use a central configuration server, but this
introduces a single point of failure in the system. Instead, Riak uses
a gossip protocol to propagate cluster membership and partition
ownership data throughout the system.

Gossip protocols, also called epidemic protocols, work exactly as they
sound.  When a node in the system wishes to change a piece of shared
data, it makes the change to its local copy of the data and gossips
the updated data to a random peer.  Upon receiving an update, a node
merges the received changes with its local state and gossips again to
another random peer.

When a Riak cluster is started, all nodes must be configured with the
same partition count. The consistent hashing ring is then divided by
the partition count and each interval is stored locally as a
\code{\{HashRange, Owner\}} pair. The first node in a cluster simply
claims all the partitions.  When a new node joins the cluster, it
contacts an existing node for its list of \code{\{HashRange, Owner\}}
pairs.  It then claims (partition count)/(number of nodes) pairs,
updating its local state to reflect its new ownership. The updated
ownership information is then gossiped to a peer. This updated state
then spread throughout the entire cluster using the above algorithm.

By using a gossip protocol, Riak avoids introducing a single point of
failure in the form of a centralized configuration server, relieving
system operators from having to maintain critical cluster
configuration data.  Any node can then use the gossiped partition
assignment data in the system to route requests.  When used together,
the gossip protocol and consistent hashing enable Riak to function as
a truly decentralized system, which has important consequences for
deploying and operating large-scale systems.

\end{aosasect1}

\begin{aosasect1}{Conclusions and Lessons Learned}

Most programmers believe that smaller and simpler codebases are not
only easier to maintain, they often have fewer bugs.  By using
Erlang's basic distribution primitives for communication in a cluster,
Riak can start out with a fundamentally sound asynchronous messaging
layer and build its own protocols without having to worry about that
underlying implementation. As Riak grew into a mature system, some
aspects of its networked communication moved away from use of Erlang's
built-in distribution (and toward direct manipulation of TCP sockets)
while others remained a good fit for the included primitives.  By
starting out with Erlang's native message passing for everything, the
Riak team was able to build out the whole system very quickly.  These
primitives are clean and clear enough that it was still easy later to
replace the few places where they turned out to not be the best fit in
production.

Also, due to the nature of Erlang messaging and the lightweight core
of the Erlang VM, a user can just as easily run 12 nodes on 1 machine
or 12 nodes on 12 machines. This makes development and testing much
easier when compared to more heavyweight messaging and clustering
mechanisms. This has been especially valuable due to Riak's
fundamentally distributed nature. Historically, most distributed
systems are very difficult to operate in a ``development mode'' on a
single developer's laptop. As a result, developers often end up
testing their code in an environment that is a subset of their full
system, with very different behavior. Since a many-node Riak cluster
can be trivially run on a single laptop without excessive resource
consumption or tricky configuration, the development process can more
easily produce code that is ready for production deployment.

The use of Erlang/OTP supervisors makes Riak much more resilient in
the face of subcomponent crashes. Riak takes this further; inspired by
such behaviors, a Riak cluster is also able to easily keep functioning
even when whole nodes crash and disappear from the system. This can
lead to a sometimes-surprising level of resilience.  One example of
this was when a large enterprise was stress-testing various databases
and intentionally crashing them to observe their edge conditions.
When they got to Riak, they became confused.  Each time they would
find a way (through OS-level manipulation, bad IPC, etc) to crash a
subsystem of Riak, they would see a very brief dip in performance and
then the system returned to normal behavior. This is a direct result
of a thoughtful ``let it crash'' approach. Riak was cleanly restarting
each of these subsystems on demand, and the overall system simply
continued to function. That experience shows exactly the sort of
resilience enabled by Erlang/OTP's approach to building programs.

\begin{aosasect2}{Acknowledgments}

This chapter is based on Francesco Cesarini and Simon Thompson's 2009
lecture notes from the central European Functional Programming School
held in Budapest and Kom\'{a}rno. Major contributions were made by
Simon Thompson of the University of Kent in Canterbury, UK. A special
thank you goes to all of the reviewers, who at different stages in the
writing of this chapter provided valuable feedback.

\end{aosasect2}

\end{aosasect1}

\end{aosachapter}
