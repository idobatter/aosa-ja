\begin{aosachapter}{Telepathy}{s:telepathy}{Danielle Madeley}
%% Based on EN-Revision r272

%% Telepathy\footnote{\url{http://telepathy.freedesktop.org/}, or see the
%% developers' manual at \url{http://telepathy.freedesktop.org/doc/book/}} is a
%% modular framework for real-time communications that handles voice, video, text, file
%% transfer, and so on. What's unique about Telepathy is not that it abstracts the
%% details of various instant messaging protocols, but that it provides the idea
%% of communications as a service, in much the same way that printing is a
%% service, available to many applications at once. To achieve this Telepathy
%% makes extensive use of the D-Bus messaging bus and a modular design.
Telepathy\footnote{\url{http://telepathy.freedesktop.org/}、あるいは
\url{http://telepathy.freedesktop.org/doc/book/}にある開発者向けマニュアルを参照}
はリアルタイム通信のためのモジュラーフレームワークで、音声や動画、テキスト、
ファイル転送などを扱える。Telepathyが他のフレームワークと違う点は、
さまざまなインスタントメッセージングプロトコルの詳細を抽象化しているというところではない。
サービスとしての通信(communications as a service)を提供するというアイデアこそが他との違いで、
これはちょうど印刷をサービスとして提供して多くのアプリケーションから使えるようにするという考え方と同じである。
このアイデアを実現するために、TelepathyはD-Busメッセージングバスと
モジュラー設計を幅広く活用した。

%% Communications as a service is incredibly useful, because it allows us
%% to break communications out of a single application. This enables lots
%% of interesting use cases: being able to see a contact's
%% presence in your email application; start communicating with
%% her; launching a file transfer to a contact straight from your file
%% browser; or providing contact-to-contact collaboration within
%% applications, known in Telepathy as \emph{Tubes}.
通信をサービスとして提供できると非常に便利だ。
単一のアプリケーションの枠を超えた通信ができるようになるからだ。
いろんな使い道が考えられる。たとえば、メールアプリケーション上で相手が在席中であることを確認してから
相手とのやりとりを始めたり、ファイルを転送も、ファイルブラウザから相手に直接送れたりする。
アプリケーション内でお互いに共同作業することもできる。これはTelepathy as \emph{Tubes}
(Telepathyを土管として使う)と呼ばれている。

%% Telepathy was created by Robert McQueen in 2005 and since that time
%% has been developed and maintained by several companies and individual
%% contributors including Collabora, the company co-founded by McQueen.
TelepathyはRobert McQueenが2005年に作ったものだ。
それ以降はいくつかの企業や個人で開発と保守を進めている。
その中の一社であるCollaboraは、McQueenが共同設立者の一人となっている企業である。

%% \begin{aosabox}{The D-Bus Message Bus}
\begin{aosabox}{D-Busメッセージバス}

%% D-Bus is an asynchronous message bus for interprocess communication
%% that forms the backbone of most GNU/Linux systems including the GNOME
%% and KDE desktop environments.  D-Bus is a primarily a shared bus
%% architecture: applications connect to a bus (identified by a socket
%% address) and can either transmit a targeted message to another
%% application on the bus, or broadcast a signal to all bus
%% members. Applications on the bus have a bus address, similar to an IP
%% address, and can claim a number of well-known names, like DNS names,
%% for example \code{org.freedesktop.Telepathy.AccountManager}.  All
%% processes communicate via the D-Bus daemon, which handles message
%% passing, and name registration.
D-Busはプロセス間通信用の非同期メッセージバスで、GNOMEやKDEといったデスクトップ環境を含む
ほとんどのGNU/Linuxシステムのバックボーンとなっている。
D-Busはそもそも共有バスアーキテクチャだった。
アプリケーションはソケットアドレスを指定してバスに接続し、
ターゲットメッセージをバス上の別のアプリケーションに送信したり
バス上の全メンバーにシグナルをブロードキャストしたりできる。
バス上のアプリケーションは、IPアドレスと同じような感じのバスアドレスを持つ。
また、名前を付けることもできる。DNSと同じような感じで、たとえば\code{org.freedesktop.Telepathy.AccountManager}
のようになる。すべてのプロセスがD-Busデーモン経由で通信を行う。
このデーモンが、メッセージの受け渡しや名前の登録を処理する。

%% From the user's perspective, there are two buses available on every
%% system.  The system bus is a bus that allows the user to communicate
%% with system-wide components (printers, bluetooth, hardware management,
%% etc.)  and is shared by all users on the system.  The session bus is
%% unique to that user---i.e., there is a session bus per logged-in
%% user---and is used for the user's applications to communicate with
%% each other.  When a lot of traffic is to be transmitted over the bus,
%% it's also possible for applications to create their own private bus,
%% or to create a peer-to-peer, unarbitrated bus with no
%% \code{dbus-daemon}.
ユーザーの視点で見ると、すべてのシステムにはふたつのバスが存在する。
ひとつはシステムバスで、これはユーザーがシステム全体のコンポーネント
(プリンタやBluetooth、ハードウェア管理など)と通信できるようにするためのバスである。
システム上のすべてのユーザーが共有する。
もうひとつはセッションバスで、これは各ユーザーに固有のものである
(つまり、システムにログインしているユーザーの数だけセッションバスが存在する)。
これは、そのユーザーが使うアプリケーションどうしがお互いに通信するために使う。
セッションバス上に大量のトラフィックが流れている場合は、
アプリケーションが自前のプライベートバスを作ることもできる。
あるいは、\code{dbus-daemon}を介さないピアツーピアのバスを作ることもできる。

%% Several libraries implement the D-Bus protocol and can communicate
%% with the D-Bus daemon, including libdbus, GDBus, QtDBus, and
%% python-dbus. These libraries are responsible for sending and receiving
%% D-Bus messages, marshalling types from the language's type system into
%% D-Bus' type format and publishing objects on the bus.  Usually, the
%% libraries also provide convenience APIs for listing connected
%% applications and activatable applications, and requesting well-known
%% names on the bus.  At the D-Bus level, all of these are done by making
%% method calls on an object published by \code{dbus-daemon} itself.
D-Busプロトコルを実装してD-Busデーモンと通信できるようにしたライブラリがいくつか存在する。
libdbusやGDBus、QtDBus、そしてpython-dbusなどだ。これらのライブラリの役割は、
D-Busメッセージの送受信だけでなく各言語の型システムからD-Busの型フォーマットへの変換や
バス上のオブジェクトの公開などもある。
これらのライブラリは、便利なAPIを提供していることも多い。
接続中のアプリケーションの一覧やアクティベート可能なアプリケーションの一覧を表示したり、
バス上での名前を取得したりするためのAPIである。
D-Busレベルでは、これらすべての操作は\code{dbus-daemon}自身が公開するオブジェクト上での
メソッド呼び出しで行う。

%% For more information on D-Bus, see
%% \url{http://www.freedesktop.org/wiki/Software/dbus}.
D-Busについての詳細は\url{http://www.freedesktop.org/wiki/Software/dbus}
を参照してほしい。

\end{aosabox}

%% \begin{aosasect1}{Components of the Telepathy Framework}
\begin{aosasect1}{Telepathy Frameworkのコンポーネント}

Telepathy is modular, with each module communicating with the others
via a D-Bus messaging bus. Most usually via the user's session bus. This
communication is detailed in the Telepathy
specification\footnote{\url{http://telepathy.freedesktop.org/spec/}}.
The components of the Telepathy framework are as shown in
\aosafigref{fig.telepathy.components}:

\begin{aosaitemize}

  \item A Connection Manager provides the interface between Telepathy
    and the individual communication services. For instance, there is
    a Connection Manager for XMPP, one for SIP, one for IRC, and so
    on.  Adding support for a new protocol to Telepathy is simply a
    matter of writing a new Connection Manager.

  \item The Account Manager service is responsible for storing the
    user's communications accounts and establishing a connection to
    each account via the appropriate Connection Manager when
    requested.

  \item The Channel Dispatcher's role is to listen for incoming
    channels signalled by each Connection Manager and dispatch them to
    clients that indicate their ability to handle that type of
    channel, such as text, voice, video, file transfer, tubes.  The
    Channel Dispatcher also provides a service so that applications,
    most importantly applications that are not Telepathy clients, can
    request outgoing channels and have them handled locally by the
    appropriate client. This allows an application, such as an email
    application, to request a text chat with a contact, and have your
    IM client show a chat window.

  \item Telepathy clients handle or observe communications
    channels. They include both user interfaces like IM and VoIP
    clients and services such the chat logger. Clients register
    themselves with the Channel Dispatcher, giving a list of channel
    types they wish to handle or observe.

\end{aosaitemize}

Within the current implementation of Telepathy, the Account Manager
and the Channel Dispatcher are both provided by a single process known
as Mission Control.

%% \aosafigureTop[250pt]{../images/telepathy/telepathy-components.eps}{Example Telepathy Components}{fig.telepathy.components}
\aosafigureTop[250pt]{../images/telepathy/telepathy-components.eps}{Telepathyコンポーネントの例}{fig.telepathy.components}

This modular design was based on Doug McIlroy's philosophy, ``Write
programs that do one thing and do it well,'' and has several important
advantages:

\pagebreak

\begin{aosadescription}

  \item{Robustness:} a fault in one component won't crash the
  entire service.

  \item{Ease of development:} components can be replaced within
  a running system without affecting others. It's possible to test a
  development version of one module against another known to be
  good.

  \item{Language independence:} components can be written in any
  language that has a D-Bus binding. If the best
  implementation of a given communications protocol is in a certain
  language, you are able to write your Connection Manager in that
  language, and still have it available to all Telepathy clients.
  Similarly, if you wish to develop your user interface in a certain
  language, you have access to all available protocols.

  \item{License independence:} components can be under different
  software licenses that would be incompatible if everything was
  running as one process.

  \item{Interface independence:} multiple user interfaces can be
  developed on top of the same Telepathy components. This allows native
  interfaces for desktop environments and hardware devices
  (e.g., GNOME, KDE, Meego, Sugar).

  \item{Security:} Components run in separate address spaces and
  with very limited privileges.  For example, a typical
  Connection Manager only needs access to the network and the D-Bus
  session bus, making it possible to use something like SELinux to
  limit what a component can access.

\end{aosadescription}

The Connection Manager manages a number of Connections, where each
Connection represents a logical connection to a communications
service. There is one Connection per configured account.
A Connection will contain multiple Channels. Channels are the
mechanism through which communications are carried out. A channel
might be an IM conversation, voice or video call, file transfer or
some other stateful operation.  Connections and channels are discussed
in detail in \aosasecref{sec.telepathy.ccc}.

\end{aosasect1}

%% \begin{aosasect1}{How Telepathy uses D-Bus}
\begin{aosasect1}{TelepathyによるD-Busの利用法}

Telepathy components communicate via a D-Bus messaging bus, which is
usually the user's session bus.  D-Bus provides features common to
many IPC systems: each service publishes objects which have
a strictly namespaced object path, like
\code{/org/freedesktop/Telepathy/AccountManager}\footnote{From here
on, \code{/org/freedesktop/Telepathy/} and
\code{org.freedesktop.Telepathy} will be abbreviated to \code{ofdT}
to save space.}.  Each object implements a number of
interfaces. Again strictly namespaced, these have forms like
\code{org.freedesktop.DBus.Properties} and \code{ofdT.Connection}.
Each interface provides methods, signals and properties that you can
call, listen to, or request.

%% \aosafigure{../images/telepathy/bus-hierarchy-conceptual.eps}{Conceptual Representation of Objects Published by a D-Bus Service}{fig.telepathy.conceptual}
\aosafigure{../images/telepathy/bus-hierarchy-conceptual.eps}{D-Busサービスが公開するオブジェクトの概念表現}{fig.telepathy.conceptual}

%% \begin{aosabox}{Publishing D-Bus Objects}
\begin{aosabox}{D-Busオブジェクトの公開}

Publishing D-Bus objects is handled entirely by the D-Bus library
being used. In effect it is a mapping from a D-Bus object path to the
software object implementing those interfaces.  The paths of objects
being published by a service are exposed by the optional
\code{org.freedesktop.DBus.Introspectable} interface.

When a service receives an incoming method call with a given
destination path (e.g., \code{/ofdT/AccountManager}), the D-Bus
library is responsible for locating the software object providing that
D-Bus object and then making the appropriate method call on that
object.

\end{aosabox}

The interfaces, methods, signal and properties provided by Telepathy
are detailed in an XML-based D-Bus IDL that has been expanded to
include more information.  The specification can be parsed to generate
documentation and language bindings.

Telepathy services publish a number of objects onto the bus. Mission
Control publishes objects for the Account Manager and Channel
Dispatcher so that their services can be accessed. Clients publish a
Client object that can be accessed by the Channel Dispatcher. Finally,
Connection Managers publish a number of objects: a service object that
can be used by the Account Manager to request new connections, an
object per open connection, and an object per open channel.

Although D-Bus objects do not have a type (only interfaces), Telepathy
simulates types several ways. The object's path tells us whether the
object is a connection, channel, client, and so on, though generally
you already know this when you request a proxy to it. Each object
implements the base interface for that type, e.g.,
\code{ofdT.Connection} or \code{ofdT.Channel}.  For channels this is
sort of like an abstract base class.  Channel objects then have a
concrete class defining their channel type.  Again, this is
represented by a D-Bus interface. The channel type can be learned by
reading the \code{ChannelType} property on the Channel interface.

Finally, each object implements a number of optional interfaces
(unsurprisingly also represented as D-Bus interfaces), which depend on
the capabilities of the protocol and the Connection Manager.  The
interfaces available on a given object are available via the
\code{Interfaces} property on the object's base class.

For Connection objects of type \code{ofdT.Connection}, the optional
interfaces have names like \code{ofdT.Connection.Interface.Avatars}
(if the protocol has a concept of avatars),
\path{odfT.Connection.Interface.ContactList} (if the protocol provides
a contact roster---not all do) and
\path{odfT.Connection.Interface.Location} (if a protocol provides
geolocation information).  For Channel objects, of type
\path{ofdT.Channel}, the concrete classes have interface names of the
form \path{ofdT.Channel.Type.Text}, \path{odfT.Channel.Type.Call} and
\path{odfT.Channel.Type.FileTransfer}. Like Connections, optional
interface have names likes \path{odfT.Channel.Interface.Messages} (if
this channel can send and receive text messages) and
\path{odfT.Channel.Interface.Group} (if this channel is to a group
containing multiple contacts, e.g., a multi-user chat).  So, for
example, a text channel implements at least the \path{ofdT.Channel},
\path{ofdT.Channel.Type.Text} and \path{Channel.Interface.Messages}
interfaces.  If it's a multi-user chat, it will also implement
\path{odfT.Channel.Interface.Group}.

%% \begin{aosabox}{Why an Interfaces Property and not D-Bus Introspection?}
\begin{aosabox}{なぜインターフェイスプロパティであってD-Busイントロスペクションではないのか}

You might wonder why each base class implements an \code{Interfaces}
property, instead of relying on D-Bus' introspection capabilities to
tell us what interfaces are available.  The answer is that different
channel and connection objects may offer different interfaces to each
other, depending on the capabilities of the channel or connection, but
that most of the implementations of D-Bus introspection assume that
all objects of the same object class will have the same interfaces.
For example, in \code{telepathy-glib}, the D-Bus interfaces listed by
D-Bus introspection are retrieved from the object interfaces a class
implements, which is statically defined at compile time.  We work
around this by having D-Bus introspection provide data for all the
interfaces that could exist on an object, and use the
\code{Interfaces} property to indicate which ones actually do.

\end{aosabox}

Although D-Bus itself provides no sanity checking that connection
objects only have connection-related interfaces and so forth (since
D-Bus has no concept of types, only arbitrarily named interfaces), we
can use the information contained within the Telepathy specification
to provide sanity checking within the Telepathy language bindings.

%% \begin{aosabox}{Why and How the Specification IDL was Expanded}
\begin{aosabox}{なぜ、そしてどうやってSpecification IDLを展開するのか}

The existing D-Bus specification IDL defines the names, arguments,
access restrictions and D-Bus type signatures of methods, properties
and signals. It provides no support for documentation, binding hints
or named types.

To resolve these limitations, a new XML namespace was added to provide
the required information. This namespace was designed to be generic so
that it could be used by other D-Bus APIs. New elements were added to
include inline documentation, rationales, introduction and deprecation
versions and potential exceptions from methods.

D-Bus type signatures are the low-level type notation of what is
serialized over the bus. A D-Bus type signature may look like
\code{(ii)} (which is a structure containing two int32s), or it may be
more complex.  For example, \code{a\{sa(usuu)\}}, is a map from string
to an array of structures containing uint32, string, uint32, uint32
(\aosafigref{fig.telepathy.dbustypes}).  These types, while
descriptive of the data format, provide no semantic meaning to the
information contained in the type.

In an effort to provide semantic clarity for programmers and
strengthen the typing for language bindings, new elements were added
to name simple types, structs, maps, enums, and flags, providing their
type signature, as well as documentation.  Elements were also added in
order to simulate object inheritance for D-Bus objects.

\end{aosabox}

%% \aosafigure[250pt]{../images/telepathy/telepathy-types-unpacked.eps}{D-Bus Types (ii) and a\{sa(usuu)\}}{fig.telepathy.dbustypes}
\aosafigure[250pt]{../images/telepathy/telepathy-types-unpacked.eps}{D-Bus Types (ii) and a\{sa(usuu)\}}{fig.telepathy.dbustypes}

%% \begin{aosasect2}{Handles}
\begin{aosasect2}{ハンドル}

Handles are used in Telepathy to represent identifiers (e.g., contacts
and room names). They are an unsigned integer value assigned by the
connection manager, such that the tuple (connection, handle type,
handle) uniquely refers to a given contact or room.

\pagebreak

Because different communications protocols normalize identifiers in
different ways (e.g., case sensitivity, resources), handles provide a
way for clients to determine if two identifiers are the same. They can
request the handle for two different identifiers, and if the handle
numbers match, then the identifiers refer to the same contact or room.

Identifier normalization rules are different for each protocol, so it
is a mistake for clients to compare identifier strings to compare
identifiers. For example, \code{escher@tuxedo.cat/bed} and
\code{escher@tuxedo.cat/litterbox} are two instances of the same
contact (\code{escher@tuxedo.cat}) in the XMPP protocol, and therefore
have the same handle. It is possible for clients to request channels
by either identifier or handle, but they should only ever use handles
for comparison.

\end{aosasect2}

%% \begin{aosasect2}{Discovering Telepathy Services}
\begin{aosasect2}{Telepathyサービスの検出}

Some services, such as the Account Manager and the Channel Dispatcher,
which always exist, have well known names that are defined in the
Telepathy specification. However, the names of Connection Managers and
clients are not well-known, and must be discovered.

There's no service in Telepathy responsible for the registration of
running Connection Managers and Clients. Instead, interested parties
listen on the D-Bus for the announcement of a new service.  The D-Bus
bus daemon will emit a signal whenever a new named D-Bus service
appears on the bus. The names of Clients and Connection Managers begin
with known prefixes, defined by the specification, and new names can
be matched against these.

The advantage of this design is that it's completely stateless. When a
Telepathy component is starting up, it can ask the bus daemon (which
has a canonical list, based on its open connections) what services are
currently running.  For instance, if the Account Manager crashes, it
can look to see what connections are running, and reassociate those
with its account objects.

%% \begin{aosabox}{Connections are Services Too}
\begin{aosabox}{コネクションもまたサービス}

As well as the Connection Managers themselves, the connections are
also advertised as D-Bus services. This hypothetically allows for the
Connection Manager to fork each connection off as a separate process,
but to date no Connection Manager like this has been implemented.
More practically, it allows all running connections to be discovered
by querying the D-Bus bus daemon for all services beginning with
\code{ofdT.Connection}.

\end{aosabox}

The Channel Dispatcher also uses this method to discover Telepathy
clients. These begin with the name \code{ofdT.Client}, e.g.,
\code{ofdT.Client.Logger}.

\end{aosasect2}

%% \begin{aosasect2}{Reducing D-Bus Traffic}
\begin{aosasect2}{D-Busトラフィックの軽減}

Original versions of the Telepathy specification created an excessive
amount of D-Bus traffic in the form of method calls requesting
information desired by lots of consumers on the bus. Later versions of
the Telepathy have addressed this through a number of optimizations.

Individual method calls were replaced by D-Bus properties. The
original specification included separate method calls for object
properties: \code{GetInterfaces}, \code{GetChannelType},
etc. Requesting all the properties of an object required several
method calls, each with its own calling overhead. By using D-Bus
properties, everything can be requested at once using the standard
\code{GetAll} method.

Furthermore, quite a number of properties on a channel are immutable
for the lifetime of the channel. These include things like the
channel's type, interfaces, who it's connected to and the requestor.
For a file transfer channel, for example, it also includes things like
the file size and its content type.

A new signal was added to herald the creation of channels (both
incoming and in response to outgoing requests) that includes a hash
table of the immutable properties.  This can be passed directly to the
channel proxy constructor (see \aosasecref{sec.telepathy.readiness}),
which saves interested clients from having to request this information
individually.

User avatars are transmitted across the bus as byte arrays. Although
Telepathy already used tokens to refer to avatars, allowing clients to
know when they needed a new avatar and to save downloading unrequired
avatars, each client had to individually request the avatar via a
\code{RequestAvatar} method that returned the avatar as its reply.
Thus, when the Connection Manager signalled that a contact had updated
its avatar, several individual requests for the avatar would be made,
requiring the avatar to be transmitted over the message bus several
times.

This was resolved by adding a new method which did not return the
avatar (it returns nothing). Instead, it placed the avatar in a
request queue.  Retrieving the avatar from the network would result in
a signal, \code{AvatarRetrieved}, that all interested clients could
listen to. This means the avatar data only needs to be transmitted
over the bus once, and will be available to all the interested
clients. Once the client's request was in the queue, all further
client requests can be ignored until the emission of the
\code{AvatarRetrieved}.

Whenever a large number of contacts need to be loaded (i.e., when
loading the contact roster), a significant amount of information needs
to be requested: their aliases, avatars, capabilities, and group
memberships, and possibly their location, address, and telephone numbers.
Previously in Telepathy this would require one method call per
information group (most API calls, such as \code{GetAliases} already
took a list of contacts), resulting in half a dozen or more method calls.

To solve this, the \code{Contacts} interface was introduced.  It
allowed information from multiple interfaces to be returned via a
single method call. The Telepathy specification was expanded to
include Contact Attributes: namespaced properties returned by the
\code{GetContactAttributes} method that shadowed method calls used to
retrieve contact information. A client calls
\code{GetContactAttributes} with a list of contacts and interfaces it
is interested in, and gets back a map from contacts to a map of
contact attributes to values.

A bit of code will make this clearer.  The request looks like this:

\begin{verbatim}
connection[CONNECTION_INTERFACE_CONTACTS].GetContactAttributes(
  [ 1, 2, 3 ], # contact handles
  [ "ofdT.Connection.Interface.Aliasing",
    "ofdT.Connection.Interface.Avatars",
    "ofdT.Connection.Interface.ContactGroups",
    "ofdT.Connection.Interface.Location"
  ],
  False # don't hold a reference to these contacts
)
\end{verbatim}

\noindent and the reply might look like this:

\begin{verbatim}
{ 1: { 'ofdT.Connection.Interface.Aliasing/alias': 'Harvey Cat',
       'ofdT.Connection.Interface.Avatars/token': hex string,
       'ofdT.Connection.Interface.Location/location': location,
       'ofdT.Connection.Interface.ContactGroups/groups': [ 'Squid House' ],
       'ofdT.Connection/contact-id': 'harvey@nom.cat'
     },
  2: { 'ofdT.Connection.Interface.Aliasing/alias': 'Escher Cat',
       'ofdT.Connection.Interface.Avatars/token': hex string,
       'ofdT.Connection.Interface.Location/location': location,
       'ofdT.Connection.Interface.ContactGroups/groups': [],
       'ofdT.Connection/contact-id': 'escher@tuxedo.cat'
     },
  3: { 'ofdT.Connection.Interface.Aliasing/alias': 'Cami Cat',
        ...
     }
}
\end{verbatim}

\end{aosasect2}

\end{aosasect1}

%% \begin{aosasect1}{Connections, Channels and Clients}
\begin{aosasect1}{コネクション、チャネル、そしてクライアント}
\label{sec.telepathy.ccc}

%% \begin{aosasect2}{Connections}
\begin{aosasect2}{コネクション}

A Connection is created by the Connection Manager to establish a
connection to a single protocol/account. For example, connecting to
the XMPP accounts \code{escher@tuxedo.cat} and \code{cami@egg.cat}
would result in two Connections, each represented by a D-Bus
object. Connections are typically set up by the Account Manager, for
the currently enabled accounts.

The Connection provides some mandatory functionality for managing and
monitoring the connection status and for requesting channels. It can
then also provide a number of optional features, depending on the
features of the protocol. These are provided as optional D-Bus
interfaces (as discussed in the previous section) and listed by the
Connection's \code{Interfaces} property.

Typically Connections are managed by the Account Manager, created
using the properties of the respective accounts. The Account Manager
will also synchronize the user's presence for each account to its
respective connection and can be asked to provide the connection path
for a given account.

\end{aosasect2}

%% \begin{aosasect2}{Channels}
\begin{aosasect2}{チャネル}

Channels are the mechanism through which communications are carried
out.  A channel is typically an IM conversation, voice or video call
or file transfer, but channels are also used to provide some stateful
communication with the server itself, (e.g., to search for chat rooms
or contacts). Each channel is represented by a D-Bus object.

Channels are typically between two or more users, one of whom is
yourself. They typically have a target identifier, which is either
another contact, in the case of one-to-one communication; or a room
identifier, in the case of multi-user communication (e.g., a chat
room). Multi-user channels expose the \code{Group} interface, which
lets you track the contacts who are currently in the channel.

Channels belong to a Connection, and are requested from the Connection
Manager, usually via the Channel Dispatcher; or they are created by
the Connection in response to a network event (e.g., incoming chat),
and handed to the Channel Dispatcher for dispatching.

The type of channel is defined by the channel's \code{ChannelType}
property. The core features, methods, properties, and signals that are
needed for this channel type (e.g., sending and receiving text
messages) are defined in the appropriate \code{Channel.Type} D-Bus
interface, for instance \code{Channel.Type.Text}. Some channel types
may implement optional additional features (e.g., encryption) which
appear as additional interfaces listed by the channel's
\code{Interfaces} property.  An example text channel that connects the
user to a multi-user chatroom might have the interfaces shown in
\aosatblref{tbl.telepathy.textchannel}.

\begin{table}[h]\centering
  \begin{tabular}{ |lp{3.0in}| }
   \hline
    \code{odfT.Channel} & Features common to all channels \\
    \code{odfT.Channel.Type.Text} & The Channel Type, includes features common to text channels \\
    \code{odfT.Channel.Interface.Messages} & Rich-text messaging \\
    \code{odfT.Channel.Interface.Group} & List, track, invite and approve members in this channel \\
    \code{odfT.Channel.Interface.Room} & Read and set properties such as the chatroom's subject \\
   \hline
  \end{tabular}
  \caption{Example Text Channel}
  \label{tbl.telepathy.textchannel}
\end{table}

%% \begin{aosabox}{Contact List Channels: A Mistake}
\begin{aosabox}{コンタクトリストチャネル: 失敗例}

In the first versions of the Telepathy specification, contact lists
were considered a type of channel. There were several server-defined
contact lists (subscribed users, publish-to users, blocked users),
that could be requested from each Connection. The members of the list
were then discovered using the \code{Group} interface, like for a
multi-user chat.

Originally this would allow for channel creation to occur only once
the contact list had been retrieved, which takes time on some
protocols. A client could request the channel whenever it liked, and
it would be delivered once ready, but for users with lots of contacts
this meant the request would occasionally time out.  Determining the
subscription/publish/blocked status of a client required checking
three channels.

Contact Groups (e.g., Friends) were also exposed as channels, one
channel per group. This proved extremely difficult for client
developers to work with.  Operations like getting the list of groups a
contact was in required a significant amount of code in the client.
Further, with the information only available via channels, properties
such as a contact's groups or subscription state could not be
published via the Contacts interface.

Both channel types have since been replaced by interfaces on the
Connection itself which expose contact roster information in ways more
useful to client authors, including subscription state of a contact
(an enum), groups a contact is in, and contacts in a group.  A signal
indicates when the contact list has been prepared.

\end{aosabox}

\end{aosasect2}

%% \begin{aosasect2}{Requesting Channels, Channel Properties and Dispatching}
\begin{aosasect2}{チャネルのリクエスト、チャネルのプロパティそしてディスパッチ}

Channels are requested using a map of properties you wish the desired
channel to possess. Typically, the channel request will include the
channel type, target handle type (contact or room) and target.
However, a channel request may also include properties such as the
filename and filesize for file transfers, whether to initially include
audio and video for calls, what existing channels to combine into a
conference call, or which contact server to conduct a contact search
on.

The properties in the channel request are properties defined by
interfaces of the Telepathy spec, such as the \code{ChannelType}
property (\aosatblref{tbl.telepathy.channelrequest}). They are
qualified with the namespace of the interface they come from
Properties which can be included in channel requests are marked as
\emph{requestable} in the Telepathy spec.

\begin{table}[h]\centering
  \begin{tabular}{ |ll| }
    \hline
    Property & Value \\
    \hline
    \code{ofdT.Channel.ChannelType} & \code{ofdT.Channel.Type.Text} \\
    \code{ofdT.Channel.TargetHandleType} & \code{Handle\_Type\_Contact} (1) \\
    \code{ofdT.Channel.TargetID} & \code{escher@tuxedo.cat} \\
    \hline
  \end{tabular}
  \caption{Example Channel Requests}
  \label{tbl.telepathy.channelrequest}
\end{table}

The more complicated example in \aosatblref{tbl.telepathy.transfer}
requests a file transfer channel. Notice how the requested properties
are qualified by the interface from which they come.  (For brevity,
not all required properties are shown.)

\begin{table}[h]\centering
\begin{tabular}{ |ll| }
    \hline
    Property & Value \\
    \hline
    \code{ofdT.Channel.ChannelType} & \code{ofdT.Channel.Type.FileTransfer} \\
    \code{ofdT.Channel.TargetHandleType} & \code{Handle\_Type\_Contact} (1) \\
    \code{ofdT.Channel.TargetID} & \code{escher@tuxedo.cat} \\
    \code{ofdT.Channel.Type.FileTransfer.Filename} & \code{meow.jpg} \\
    \code{ofdT.Channel.Type.FileTransfer.ContentType} & \code{image/jpeg} \\
    \hline
  \end{tabular}
  \caption{File Transfer Channel Request}
  \label{tbl.telepathy.transfer}
\end{table}

Channels can either be \emph{created} or \emph{ensured}. Ensuring a
channel means creating it only if it does not already exist. Asking to
create a channel will either result in a completely new and separate
channel being created, or in an error being generated if multiple
copies of such a channel cannot exist. Typically you wish to ensure
text channels and calls (i.e., you only need one conversation open with
a person, and in fact many protocols do not support multiple separate
conversations with the same contact), and wish to create file
transfers and stateful channels.

Newly created channels (requested or otherwise) are announced by a
signal from the Connection. This signal includes a map of the
channel's \emph{immutable} properties. These are the properties which
are guaranteed not to change throughout the channel's lifetime.
Properties which are considered immutable are marked as such in the
Telepathy spec, but typically include the channel's type, target
handle type, target, initiator (who created the channel) and
interfaces.  Properties such as the channel's state are obviously not
included.

%% \begin{aosabox}{Old-School Channel Requesting}
\begin{aosabox}{昔ながらのチャネルリクエスト}

Channels were originally requested simply by type, handle type and
target handle.  This wasn't sufficiently flexible because not all
channels have a target (e.g., contact search channels), and some
channels require additional information included in the initial
channel request (e.g., file transfers, requesting voicemails and
channels for sending SMSes).

It was also discovered that two different behaviors might be desired
when a channel was requested (either to create a guaranteed unique
channel, or simply ensure a channel existed), and until this time the
Connection had been responsible for deciding which behavior would
occur.  Hence, the old method was replaced by the newer, more
flexible, more explicit ones.

\end{aosabox}

\pagebreak

Returning a channel's immutable properties when you create or ensure
the channel makes it much faster to create a proxy object for the
channel. This is information we now don't have to request.  The map in
\aosatblref{tbl.telepathy.immutable} shows the immutable properties
that might be included when we request a text channel (i.e., using the
channel request in \aosatblref{tbl.telepathy.transfer}). Some
properties (including \code{TargetHandle} and \code{InitiatorHandle})
have been excluded for brevity.

\begin{table}[h]\centering
\begin{tabular}{ |p{6cm} p{6cm} | }
    \hline
    Property & Value \\
    \hline
    \code{ofdT.Channel.ChannelType} & \code{Channel.Type.Text} \\
    \code{ofdT.Channel.Interfaces} & \code{{[} Channel.Interface.Messages,\newline Channel.Interface.Destroyable,\newline Channel.Interface.ChatState {]}}  \\
    \code{ofdT.Channel.TargetHandleType} & \code{Handle\_Type\_Contact} (1) \\
    \code{ofdT.Channel.TargetID} & \code{escher@tuxedo.cat} \\
    \code{ofdT.Channel.InitiatorID} & \code{danielle.madeley@collabora.co.uk} \\
    \code{ofdT.Channel.Requested} & \code{True} \\
    \code{ofdT.Channel.Interface.Messages.}{\newline}\hspace*{1em}\code{SupportedContentTypes} & \code{{[} text/html, text/plain {]}} \\
    \hline
  \end{tabular}
  \caption{Example Immutable Properties Returned by a New Channel}
  \label{tbl.telepathy.immutable}
\end{table}

The requesting program typically makes a request for a channel to the
Channel Dispatcher, providing the account the request is for, the
channel request, and optionally the name of a the desired handler
(useful if the program wishes to handle the channel itself).  Passing
the name of an account instead of a connection means that the Channel
Dispatcher can ask the Account Manager to bring an account online if
required.

Once the request is complete, the Channel Dispatcher will either pass
the channel to the named Handler, or locate an appropriate Handler
(see below for discussion on Handlers and other clients). Making the
name of the desired Handler optional makes it possible for programs
that have no interest in communication channels beyond the initial
request to request channels and have them handled by the best program
available (e.g., launching a text chat from your email client).

%% \aosafigure[300pt]{../images/telepathy/dispatching-model.eps}{Channel Request and Dispatching}{fig.telepathy.request}
\aosafigure[300pt]{../images/telepathy/dispatching-model.eps}{チャネルリクエストとディスパッチ}{fig.telepathy.request}

The requesting program makes a channel request to the Channel
Dispatcher, which in turn forwards the request to the appropriate
Connection. The Connection emits the NewChannels signal which is
picked up by the Channel Dispatcher, which then finds the appropriate
client to handle the channel.  Incoming, unrequested channels are
dispatched in much the same way, with a signal from the Connection
that is picked up by the Channel Dispatcher, but obviously without the
initial request from a program.

\end{aosasect2}

%% \begin{aosasect2}{Clients}
\begin{aosasect2}{クライアント}

Clients handle or observe incoming and outgoing communications
channels. A client is anything that is registered with the Channel
Dispatcher.  There are three types of clients (though a single client
may be two, or all three, types if the developer wishes):

\begin{aosadescription}

  \item{Observers}: Observe channels without interacting with
    them. Observers tend to be used for chat and activity logging
    (e.g., incoming and outgoing VoIP calls).

  \item{Approvers}: Responsible for giving users an opportunity to
    accept or reject an incoming channel.

  \item{Handlers}: Actually interact with the channel. That might be
    acknowledging and sending text messages, sending or receiving a
    file, etc. A Handler tends to be associated with a user interface.

\end{aosadescription}

Clients offer D-Bus services with up to three interfaces:
\code{Client.Observer}, \code{Client.Approver}, and
\code{Client.Handler}. Each interface provides a method that the
Channel Dispatcher can call to inform the client about a channel to
observe, approve or handle.

The Channel Dispatcher dispatches the channel to each group of clients
in turn. First, the channel is dispatched to all appropriate
Observers.  Once they have all returned, the channel is dispatched to
all the appropriate Approvers. Once the first Approver has approved or
rejected the channel, all other Approvers are informed and the channel
is finally dispatched to the Handler.  Channel dispatching is done in
stages because Observers might need time to get set up before the
Handler begins altering the channel.

Clients expose a channel filter property which is a list of filters
read by the Channel Dispatcher so that it knows what sorts of channels
a client is interested in. A filter must include at least the channel
type, and target handle type (e.g., contact or room) that the client
is interested in, but it can contain more properties. Matching is done
against the channel's immutable properties, using simple equality for
comparison.  The filter in \aosatblref{tbl.telepathy.filter} matches
all one-to-one text channels.

\begin{table}\centering
\begin{tabular}{ |ll| }
  \hline
    Property & Value \\
  \hline
    \code{ofdT.Channel.ChannelType} & \code{Channel.Type.Text} \\
    \code{ofdT.Channel.TargetHandleType} & \code{Handle\_Type\_Contact} (1) \\
  \hline
  \end{tabular}
  \caption{Example Channel Filter}
  \label{tbl.telepathy.filter}
\end{table}

Clients are discoverable via D-Bus because they publish services
beginning with the well-known name \code{ofdT.Client} (for example
\code{ofdT.Client.Empathy.Chat}).  They can also optionally install a
file which the Channel Dispatcher will read specifying the channel
filters. This allows the Channel Dispatcher to start a client if it is
not already running.  Having clients be discoverable in this way makes
the choice of user interface configurable and changeable at any time
without having to replace any other part of Telepathy.

%% \begin{aosabox}{All or Nothing}
\begin{aosabox}{オールオアナッシング}

It is possible to provide a filter indicating you are interested in
all channels, but in practice this is only useful as an example of
observing channels. Real clients contain code that is specific to
channel types.

An empty filter indicates a Handler is not interested in any channel
types. However it is still possible to dispatch a channel to this
handler if you do so by name.  Temporary Handlers which are created on
demand to handle a specific channel use such a filter.

\end{aosabox}

\end{aosasect2}

\end{aosasect1}

%% \begin{aosasect1}{The Role of Language Bindings}
\begin{aosasect1}{言語バインディングの役割}

As Telepathy is a D-Bus API, and thus can driven by any programming
language that supports D-Bus.  Language bindings are not required for
Telepathy, but they can be used to provide a convenient way to use it.

Language bindings can be split into two groups: low-level bindings
that include code generated from the specification, constants, method
names, etc.; and high-level bindings, which are hand-written code that
makes it easier for programmers to do things using Telepathy.
Examples of high-level bindings are the GLib and Qt4 bindings.
Examples of low-level bindings are the Python bindings and the
original libtelepathy C bindings, though the GLib and Qt4 bindings
include a low-level binding.

%% \begin{aosasect2}{Asynchronous Programming}
\begin{aosasect2}{非同期プログラミング}

Within the language bindings, all method calls that make requests over
D-Bus are asynchronous: the request is made, and the reply is given in
a callback. This is required because D-Bus itself is asynchronous.

Like most network and user interface programming, D-Bus requires the
use of an event loop to dispatch callbacks for incoming signals and
method returns. D-Bus integrates well with the GLib mainloop used by
the GTK+ and Qt toolkits.

Some D-Bus language bindings (such as dbus-glib) provide a
pseudo-synchronous API, where the main loop is blocked until the
method reply is returned.  Once upon a time this was exposed via the
telepathy-glib API bindings. Unfortunately using pseudo-synchronous
API turns out to be fraught with problems, and was eventually removed
from telepathy-glib.

%% \begin{aosabox}{Why Pseudo-Synchronous D-Bus Calls Don't Work}
\begin{aosabox}{疑似同期D-Bus呼び出しが失敗する理由}

The pseudo-synchronous interface offered by dbus-glib and other D-Bus
bindings is implemented using a request-and-block technique. While
blocking, only the D-Bus socket is polled for new I/O and any D-Bus
messages that are not the response to the request are queued for later
processing.

This causes several major and inescapable problems:

\begin{aosaitemize}

  \item The caller is blocked while waiting for the request to be
    answered.  It (and its user interface, if any) will be completely
    unresponsive. If the request requires accessing the network, that
    takes time; if the callee has locked up, the caller will be
    unresponsive until the call times out.

    Threading is not a solution here because threading is just another
    way of making your calling asynchronous. Instead you may as well
    make asynchronous calls where the responses come in via the
    existing event loop.

  \item Messages may be reordered. Any messages received before the
    watched-for reply will be placed on a queue and delivered to the
    client after the reply.

    This causes problems in situations where a signal indicating a
    change of state (i.e., the object has been destroyed) is now
    received after the method call on that object fails (i.e., with
    the exception \code{UnknownMethod}).  In this situation, it is
    hard to know what error to display to the user.  Whereas if we
    receive a signal first, we can cancel pending D-Bus method calls,
    or ignore their responses.

  \item Two processes making pseudo-blocking calls on each other can
    deadlock, with each waiting for the other to respond to its query.
    This scenario can occur with processes that are both a D-Bus
    service and call other D-Bus services (for example, Telepathy
    clients). The Channel Dispatcher calls methods on clients to
    dispatch channels, but clients also call methods on the Channel
    Dispatcher to request the opening of new channels (or equally they
    call the Account Manager, which is part of the same process).

\end{aosaitemize}
\end{aosabox}

Method calls in the first Telepathy bindings, generated in C, simply
used typedef callback functions. Your callback function simply had to
implement the same type signature.

\begin{verbatim}
typedef void (*tp_conn_get_self_handle_reply) (
    DBusGProxy *proxy,
    guint handle,
    GError *error,
    gpointer userdata
);
\end{verbatim}

\noindent This idea is simple, and works for C, so was continued into
the next generation of bindings.

In recent years, people have developed a way to use scripting
languages such as Javascript and Python, as well as a C\#-like
language called Vala, that use GLib/GObject-based APIs via a tool
called GObject-Introspection.  Unfortunately, it's extremely difficult
to rebind these types of callbacks into other languages, so newer
bindings are designed to take advantage of the asynchronous callback
features provided by the languages and GLib.

\end{aosasect2}

%% \begin{aosasect2}{Object Readiness}
\begin{aosasect2}{オブジェクトの準備}
\label{sec.telepathy.readiness}

In a simple D-Bus API, such as the low-level Telepathy bindings, you
can start making method calls or receive signals on a D-Bus object
simply by creating a proxy object for it.  It's as simple as giving an
object path and interface name and getting started.

However, in Telepathy's high-level API, we want our object proxies to
know what interface are available, we want common properties for the
object type to be retrieved (e.g., the channel type, target,
initiator), and we want to determine and track the object's state or
status (e.g., the connection status).

Thus, the concept of \emph{readiness} exists for all proxy objects. By
making a method call on a proxy object, you are able to asynchronously
retrieve the state for that object and be notified when state is
retrieved and the object is ready for use.

Since not all clients implement, or are interested in, all the
features of a given object, readiness for an object type is separated
into a number of possible features.  Each object implements a
\emph{core} feature, which will prepare crucial information about the
object (i.e., its \code{Interfaces} property and basic state), plus a
number of optional features for additional state, which might include
extra properties or state-tracking.  Specific examples of additional
features you can ready on various proxies are contact info,
capabilities, geolocation information, chat states (such as ``Escher
is typing{\ldots}'') and user avatars.

For example, connection object proxies have:

\begin{aosaitemize}

  \item a core feature which retrieves the interface and connection
    status;

  \item features to retrieve the requestable channel classes and
    support contact info; and

  \item a feature to establish a connection and return ready when
    connected.

\end{aosaitemize}

The programmer requests that the object is readied, providing a list
of features in which they are interested and a callback to call when
all of those features are ready. If all the features are already
ready, the callback can be called immediately, else the callback is
called once all the information for those features is retrieved.

\end{aosasect2}

\end{aosasect1}

%% \begin{aosasect1}{Robustness}
\begin{aosasect1}{ロバストネス}

One of the key advantages of Telepathy is its robustness. The
components are modular, so a crash in one component should not bring
down the whole system.  Here are some of the features that make
Telepathy robust:

\begin{aosaitemize}

  \item The Account Manager and Channel Dispatcher can recover their
    state.  When Mission Control (the single process that includes the
    Account Manager and Channel Dispatcher) starts, it looks at the
    names of services currently registered on the user's session bus.
    Any Connections it finds that are associated with a known account
    are reassociated with that account (rather than a new connection
    being established), and running clients are queried for the list
    of channels they're handling.

  \item If a client disappears while a channel it's handling is open,
    the Channel Dispatcher will respawn it and reissue the channel.

    If a client repeatedly crashes the Channel Dispatcher can attempt
    to launch a different client, if available, or else it will close
    the channel (to prevent the client repeatedly crashing on data it
    can't handle).

    Text messages require acknowledgment before they will disappear
    from the list of pending messages. A client is only meant to
    acknowledge a message once it is sure the user has seen it (that
    is, displayed the message in a focused window). This way if the
    client crashes trying to render the message, the channel will
    still have the previously undisplayed message in the pending
    message queue.

  \item If a Connection crashes, the Account Manager will respawn
    it. Obviously the content of any stateful channels will be lost,
    but it will only affect the Connections running in that process
    and no others. Clients can monitor the state of the connections
    and simply re-request information like the contact roster and any
    stateless channels.

\end{aosaitemize}

\end{aosasect1}

%% \begin{aosasect1}{Extending Telepathy: Sidecars}
\begin{aosasect1}{Telepathyの拡張: サイドカー}

Although the Telepathy specification tries to cover a wide range of
features exported by communication protocols, some protocols are
themselves extensible\footnote{E.g., the Extensible Messaging and
 Presence Protocol (XMPP).}.  Telepathy's developers wanted to make
it possible extend your Telepathy connections to make use of such
extensions without having to extend the Telepathy specification
itself. This is done through the use of \emph{sidecars}.

Sidecars are typically implemented by plugins in a Connection Manager.
Clients call a method requesting a sidecar that implements a given
D-Bus interface.  For example, someone's implementation of XEP-0016
privacy lists might implement an interface named
\code{com.example.PrivacyLists}. The method then returns a D-Bus
object provided by the plugin, which should implement that interface
(and possibly others). The object exists alongside the main Connection
object (hence the name sidecar, like on a motorcycle).

%% \begin{aosabox}{The History of Sidecars}
\begin{aosabox}{サイドカーの歴史}

In the early days of Telepathy, the One Laptop Per Child project
needed to support custom XMPP extensions (XEPs) to share information
between devices. These were added directly to Telepathy-Gabble (the
XMPP Connection Manager), and exposed via undocumented interfaces on
the Connection object.  Eventually, with more developers wanting
support for specific XEPs which have no analogue in other
communications protocols, it was agreed that a more generic interface
for plugins was needed.

\end{aosabox}

\end{aosasect1}

%% \begin{aosasect1}{A Brief Look Inside a Connection Manager}
\begin{aosasect1}{コネクションマネージャーの内部構造の概要}

Most Connection Managers are written using the C/GLib language
binding, and a number of high-level base classes have been developed
to make writing a Connection Manager easier.  As discussed previously,
D-Bus objects are published from software objects that implement a
number of software interfaces that map to D-Bus
interfaces. Telepathy-GLib provides base objects to implement the
Connection Manager, Connection and Channel objects. It also provides
an interface to implement a Channel Manager. Channel Managers are
factories that can be used by the \code{BaseConnection} to instantiate
and manage channel objects for publishing on the bus.

The bindings also provide what are known as \emph{mixins}.  These can
be added to a class to provide additional functionality, abstract the
specification API and provide backwards compatibility for new and
deprecated versions of an API through one mechanism. The most commonly
used mixin is one that adds the D-Bus properties interface to an
object. There are also mixins to implement the
\code{ofdT.Connection.Interface.Contacts} and
\code{ofdT.Channel.Interface.Group} interfaces and mixins making it
possible to implement the old and new presence interfaces, and old and
new text message interfaces via one set of methods.

%% \aosafigure{../images/telepathy/cm.eps}{Example Connection Manager Architecture}{fig.telepathy.manager}
\aosafigure{../images/telepathy/cm.eps}{コネクションマネージャーのアーキテクチャの例}{fig.telepathy.manager}

%% \begin{aosabox}{Using Mixins to Solve API Mistakes}
\begin{aosabox}{Mixinによる、APIの間違いの解決}

One place where mixins have been used to solve a mistake in the
Telepathy specification is the \code{TpPresenceMixin}.  The original
interface exposed by Telepathy
(\code{odfT.Connection.Interface.Presence}) was incredibly
complicated, hard to implement for both Connections and Clients, and
exposed functionality that was both nonexistent in most communications
protocols, and very rarely used in others. The interface was replaced
by a much simpler interface
(\code{odfT.Connection.Interface.SimplePresence}), which exposed all
the functionality that users cared about and had ever actually been
implemented in the connection managers.

The presence mixin implements both interfaces on the Connection so
that legacy clients continue to work, but only at the functionality
level of the simpler interface.

\end{aosabox}

\end{aosasect1}

%% \begin{aosasect1}{Lessons Learned}
\begin{aosasect1}{教訓}

Telepathy is an excellent example of how to build a modular, flexible
API on top of D-Bus. It shows how you can develop an extensible,
decoupled framework on top of D-Bus. One which requires no central
management daemon and allows components to be restartable, without
loss of data in any other component.  Telepathy also shows how you can
use D-Bus efficiently and effectively, minimizing the amount of
traffic you transmit on the bus.

Telepathy's development has been iterative, improving its use of D-Bus
as time goes on. Mistakes were made, and lessons have been learned.
Here are some of the important things we learned in designing the
architecture of Telepathy:

\begin{aosadescription}

  \item{Use D-Bus properties; don't require dozens of small D-Bus
    method calls to look up information.}  Every method call has a
    round-trip time. Rather than making lots of individual calls
    (e.g., \code{GetHandle}, \code{GetChannelType},
    \code{GetInterfaces}) use D-Bus properties and return all the
    information via a single call to \code{GetAll}.

  \item{Provide as much information as you can when announcing new
    objects.}  The first thing clients used to do when they learned
    about a new object was to request all of its properties to learn
    whether they were even interested in the object. By including the
    immutable properties of an object in the signal announcing the
    object, most clients can determine their interest in the object
    without making any method calls. Furthermore, if they are
    interested in the object, they do not have to bother requesting
    any of its immutable properties.

  \item{The \code{Contacts} interface allows requesting information
    from multiple interfaces at once.}  Rather than making numerous
    \code{GetAll} calls to retrieve all the information for a contact,
    the \code{Contacts} interface lets us request all the information
    at once, saving a number of D-Bus round trips.

  \item{Don't use abstractions that don't quite fit.}  Exposing the
    contact roster and contact groups as channels implementing the
    \code{Group} interface seemed like a good idea because it used
    existing abstractions rather than requiring additional
    interfaces. However, it made implementing clients difficult and
    was ultimately not suitable.

  \item{Ensure your API will meet your future needs.}  The original
    channel requesting API was very rigid, only permitting very basic
    channel requests. This did not meet our needs when needing to
    request channels that required more information. This API had to
    be replaced with one that had significantly more flexibility.

\end{aosadescription}

\end{aosasect1}

\end{aosachapter}
