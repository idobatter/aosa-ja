\begin{aosachapter}{Selenium WebDriver}{s:selenium}{Simon Stewart}
%% Based on EN-Revision r229

%% Selenium is a browser automation tool, commonly used for writing
%% end-to-end tests of web applications. A browser automation tool
%% does exactly what you would expect: automate the control of a
%% browser so that repetitive tasks can be automated. It sounds like
%% a simple problem to solve, but as we will see, a lot has to happen
%% behind the scenes to make it work.
Seleniumはブラウザの自動化ツールで、ウェブアプリケーションのエンドツーエンドテストを書くときによく使われる。ブラウザの自動化ツールが行うのは、その名前から想像できるとおりのことだ。ブラウザを制御して、繰り返し行われるタスクを自動化できる。解決しようとしている問題は単純なものだが、本章で説明するとおり、その裏側ではさまざまなことが起こっている。

%% Before describing the architecture of Selenium it helps to understand
%% how the various related pieces of the
%% project fit together. At a very high level, Selenium is a suite of
%% three tools. The first of these tools, Selenium
%% IDE, is an extension for Firefox that allows users to record and
%% playback tests. The record/playback paradigm can be limiting and isn't
%% suitable for many users, so the second tool in the suite, Selenium
%% WebDriver, provides APIs in a variety of languages to allow for more
%% control and the application of standard software development
%% practices. The final tool, Selenium Grid, makes it possible to use the
%% Selenium APIs to control browser instances distributed over a grid of
%% machines, allowing more tests to run in parallel. Within the project,
%% they are referred to as ``IDE'', ``WebDriver'' and ``Grid''. This
%% chapter explores the architecture of Selenium WebDriver.
Seleniumのアーキテクチャについて語る前に、プロジェクト内のさまざまなパーツがどのように関連するのかを説明しておこう。上位レベルから見ると、Seleniumは三つのツールを組み合わせたものである。そのうちの一つであるSelenium IDEはFirefoxの拡張で、これを使うとテストの記録や再生ができる。この「記録/再生」のパラダイムは限定的なものであり、多くのユーザーにとっては物足りないものだ。そこで登場するのが第二のツールであるSelenium WebDriverだ。さまざまな言語向けのAPIを提供しており、より細やかな制御を行うことができるし、標準のソフトウェア開発の流れに組み込むこともできる。最後のツールはSelenium Gridだ。これを使うと、Selenium APIで分散環境にあるブラウザのインスタンスを制御でき、テストを並列に実行できるようになる。Seleniumプロジェクト内では、これらのツールはそれぞれ``IDE''、``WebDriver''そして``Grid''と呼ばれている。本章で扱うのはSelenium WebDriverのアーキテクチャである。

%% This chapter was written during the betas of Selenium 2.0 in late
%% 2010. If you're reading the book after then, then things will have
%% moved forward, and you'll be able to see how the architectural choices
%% described here have unfolded. If you're reading before that date:
%% Congratulations! You have a time machine. Can I have some winning
%% lottery numbers?
本章の内容はSelenium 2.0のベータ版に基づいており、2010年の後半に執筆された。もしそれより後にこれを読んでいるのなら、世間はさらに進んでいることだろう。ここで取り上げたアーキテクチャに関する選択が実際にどのような形で実装されたのかを見られるかもしれない。もし2010年後半よりも前にこれを読んでいるのなら…おめでとう!あなたはついにタイムマシンを手に入れたんだね!頼むから、宝くじの当選番号を教えてくれないかな?

\begin{aosasect1}{History}

Jason Huggins started the Selenium project in 2004 while working at
ThoughtWorks on their in-house Time and Expenses (T\&E) system, which
made extensive use of Javascript. Although Internet Explorer was the
dominant browser at the time, ThoughtWorks used a number of
alternative browsers (in particular Mozilla variants) and would file
bug reports when the T\&E app wouldn't work on their browser of
choice. Open Source testing tools at the time were either focused on a
single browser (typically IE) or were simulations of a browser (like
HttpUnit).  The cost of a license for a commercial tool would have
exhausted the limited budget for a small in-house project, so they
weren't even considered as viable testing choices.

Where automation is difficult, it's common to rely on manual
testing. This approach doesn't scale when the team is very small or
when releases are extremely frequent. It's also a waste of humanity to
ask people to step through a script that could be automated. More
prosaically, people are slower and more error prone than a machine for
dull repetitive tasks.  Manual testing wasn't an option.

Fortunately, all the browsers being tested supported Javascript. It
made sense to Jason and the team he was working with to write a
testing tool in that language which could be used to verify the
behavior of the application. Inspired by work being done on
FIT\footnote{\url{http://fit.c2.com}}, a table-based syntax was placed
over the raw Javascript and this allowed tests to be written by people
with limited programming experience using a keyword-driven approach in
HTML files. This tool, originally called ``Selenium'' but later
referred to as ``Selenium Core'', was released under the Apache 2
license in 2004.

The table format of Selenium is structured similarly to the
ActionFixture from FIT\@. Each row of the table is split into three
columns. The first column gives the name of the command to execute,
the second column typically contains an element identifier and the
third column contains an optional value. For example, this is how to
type the string ``Selenium WebDriver'' into an element identified with
the name ``q'':

\begin{verbatim}
type       name=q       Selenium WebDriver
\end{verbatim}

Because Selenium was written in pure Javascript,
its initial design required developers to host Core and
their tests on the same server as the application under test
(AUT) in order to avoid falling foul of the browser's security
policies and the Javascript sandbox. This was not always
practical or possible.  Worse, although a
developer's IDE gives them the ability to swiftly manipulate code and
navigate a large codebase, there is no such tool for HTML\@. It rapidly
became clear that maintaining even a medium-sized suite of tests was
an unwieldy and painful proposition.\footnote{This is very similar to
FIT, and James Shore, one of that project's coordinators, helps
explain some of the drawbacks at
\url{http://jamesshore.com/Blog/The-Problems-With-Acceptance-Testing.html}.}


To resolve this and other issues, an HTTP proxy was written so that
every HTTP request could be intercepted by Selenium. Using this
proxy made it possible to side-step many of the constraints of the ``same host
origin'' policy, where a browser won't allow Javascript to make calls
to anything other than the server from which the current page has been
served, allowing the first weakness to be mitigated. The design opened
up the possibility of writing Selenium bindings in multiple languages:
they just needed to be able to send HTTP requests to a particular
URL\@. The wire format was closely modeled on the table-based syntax of
Selenium Core and it, along with the table-based syntax, became
known as ``Selenese''. Because the language bindings were controlling
the browser at a distance, the tool was called ``Selenium Remote
Control'', or ``Selenium RC''.

While Selenium was being developed, another browser automation
framework was brewing at ThoughtWorks: WebDriver. The initial code for
this was released early in 2007. WebDriver was derived from work on
projects which wanted to isolate their end-to-end tests from the
underlying test tool.  Typically, the way that this isolation is done
is via the Adapter pattern.  WebDriver grew out of insight
developed by applying this approach consistently over numerous
projects, and initially was a wrapper around HtmlUnit. Internet
Explorer and Firefox support followed rapidly after release.

When WebDriver was released there were significant differences between
it and Selenium RC, though they sat in the same software niche of an
API for browser automation.  The most obvious difference to a user was
that Selenium RC had a dictionary-based API, with all methods exposed
on a single class, whereas WebDriver had a more object-oriented API\@.
In addition, WebDriver only supported Java, whereas Selenium RC
offered support for a wide-range of languages.  There were also strong
technical differences: Selenium Core (on which RC was based) was
essentially a Javascript application, running inside the browser's
security sandbox. WebDriver attempted to bind natively to the browser,
side-stepping the browser's security model at the cost of
significantly increased development effort for the framework itself.

In August, 2009, it was announced that the two projects would merge,
and Selenium WebDriver is the result of those merged projects.
As I write this, WebDriver supports language bindings for Java, C\#,
Python and Ruby. It offers support for Chrome, Firefox, Internet
Explorer, Opera, and the Android and iPhone browsers. There are sister
projects, not kept in the same source code repository but working
closely with the main project, that provide Perl bindings, an
implementation for the BlackBerry browser, and for ``headless''
WebKit---useful for those times where tests need to run on a
continuous integration server without a proper display. The original
Selenium RC mechanism is still maintained and allows WebDriver to
provide support for browsers that would otherwise be unsupported.

\end{aosasect1}

\begin{aosasect1}{A Digression About Jargon}

Unfortunately, the Selenium project uses a lot of jargon.  To recap
what we've already come across:

\begin{aosaitemize}

  \item \emph{Selenium Core} is the heart of the original Selenium implementation,
  and is a set of Javascript scripts that control the browser. This is
  sometimes referred to as ``Selenium'' and sometimes as ``Core''.

  \item \emph{Selenium RC} was the name given to the language bindings for Selenium
  Core, and is commonly, and confusingly, referred to as just
  ``Selenium'' or ``RC''.  It has now been replaced by Selenium
  WebDriver, where RC's API is referred to as the ``Selenium 1.x API''.

  \item \emph{Selenium WebDriver} fits in the same niche as RC did, and has subsumed
  the original 1.x bindings.  It refers to both the language bindings
  and the implementations of the individual browser controlling
  code. This is commonly referred to as just ``WebDriver'' or sometimes
  as Selenium 2. Doubtless, this will be contracted to ``Selenium'' over
  time.

\end{aosaitemize}

The astute reader will have noticed that ``Selenium'' is used in a
fairly general sense. Fortunately, context normally makes it clear
which particular Selenium people are referring to.

Finally, there's one more phrase which I'll be using, and there's no
graceful way of introducing it: ``driver'' is the name given to a
particular implementation of the WebDriver API\@. For example, there is
a Firefox driver, and an Internet Explorer driver.

\end{aosasect1}

\begin{aosasect1}{Architectural Themes}

Before we start looking at the individual pieces to understand how
they're wired together, it's useful to understand the the overarching
themes of the architecture and development of the project. Succinctly
put, these are:

\begin{aosaitemize}

\item Keep the costs down.

\item Emulate the user.

\item Prove the drivers work{\ldots}

\item {\ldots}but you shouldn't need to understand how everything works.

\item Lower the bus factor.

\item Have sympathy for a Javascript implementation.

\item Every method call is an RPC call.

\item We are an Open Source project.

\end{aosaitemize}

\begin{aosasect2}{Keep the Costs Down}

Supporting X browsers on Y platforms is inherently an expensive
proposition, both in terms of initial development and maintenance. If
we can find some way to keep the quality of the product high without
violating too many of the other principles, then that's the route we
favor. This is most clearly seen in our adoption of Javascript where
possible, as you'll read about shortly.

\end{aosasect2}

\begin{aosasect2}{Emulate the User}

WebDriver is designed to accurately simulate the way that a user will
interact with a web application. A common approach for simulating user
input is to make use of Javascript to synthesize and fire the series
of events that an app would see if a real user were to perform the
same interaction. This ``synthesized events'' approach is fraught with
difficulties as each browser, and sometimes different versions of the
same browser, fire slightly different events with slightly different
values. To complicate matters, most browsers won't allow a user to
interact in this way with form elements such as file input
elements for security reasons.

Where possible WebDriver uses the alternative approach of firing
events at the OS level. As these ``native events'' aren't generated by
the browser this approach circumvents the security restrictions placed
on synthesized events and, because they are OS specific, once they are
working for one browser on a particular platform reusing the code in
another browser is relatively easy. Sadly, this approach is only
possible where WebDriver can bind closely with the browser
and where the development team have determined how best to send
native events without requiring the browser window to be focused
(as Selenium tests take a long time to run, and it's useful to be
able to use the machine for other tasks as they run). At the time
of writing, this means that native events can be used on Linux and
Windows, but not Mac OS X.

No matter how WebDriver is emulating user input, we try hard to mimic
user behavior as closely as possible. This in contrast to RC, which
provided APIs that operated at a level far lower than that which a
user works at.

\end{aosasect2}

\begin{aosasect2}{Prove the Drivers Work}

It may be an idealistic, ``motherhood and apple pie'' thing, but I
believe there's no point in writing code if it doesn't work. The way
we prove the drivers work on the Selenium project is to have an
extensive set of automated test cases. These are typically
``integration tests'', requiring the code to be compiled and making
use of a browser interacting with a web server, but where possible we
write ``unit tests'', which, unlike an integration test can be run
without a full recompilation. At the time of writing, there are about
500 integration tests and about 250 unit tests that could be run
across each and every browser. We add more as we fix issues and write
new code, and our focus is shifting to writing more unit tests.

Not every test is run against every browser. Some test specific
capabilities that some browsers don't support, or which are handled in
different ways on different browsers. Examples would include the tests
for new HTML5 features which aren't supported on all browsers. Despite
this, each of the major desktop browsers have a significant subset of
tests run against them. Understandably, finding a way to run 500+
tests per browser on multiple platforms is a significant challenge,
and it's one that the project continues to wrestle with.

\end{aosasect2}

\begin{aosasect2}{You Shouldn't Need to Understand How Everything Works}

Very few developers are proficient and comfortable in every language
and technology we use. Consequently, our architecture needs to allow
developers to focus their talents where they can do the most good,
without needing them to work on pieces of the codebase where they are
uncomfortable.

\end{aosasect2}

\begin{aosasect2}{Lower the Bus Factor}

There's a (not entirely serious) concept in software development
called the ``bus factor''. It refers to the number of key developers
who would need to meet some grisly end---presumably by being hit by a
bus---to leave the project in a state where it couldn't
continue. Something as complex as browser automation could be
especially prone to this, so a lot of our architectural decisions are
made to raise this number as high as possible.

\end{aosasect2}

\begin{aosasect2}{Have Sympathy for a Javascript Implementation}

WebDriver falls back to using pure Javascript to drive the browser if
there is no other way of controlling it. This means that any API we
add should be ``sympathetic'' to a Javascript implementation.
As a concrete example, HTML5 introduces LocalStorage, an API for
storing structured data on the client-side. This is typically
implemented in the browser using SQLite. A natural implementation
would have been to provide a database connection to the underlying
data store, using something like JDBC\@.
Eventually, we settled on an API that closely models the underlying
Javascript implementation because something that modeled typical
database access APIs wasn't sympathetic to a Javascript implementation.

\end{aosasect2}

\begin{aosasect2}{Every Call Is an RPC Call}

WebDriver controls browsers that are running in other processes. Although
it's easy to overlook it, this means that every call that is made
through its API is an RPC call and therefore the performance of the
framework is at the mercy of network latency. In normal operation,
this may not be terribly noticeable---most OSes optimize routing to
localhost---but as the network latency between the browser and the
test code increases, what may have seemed efficient becomes less so to
both API designers and users of that API.

This introduces some tension into the design of APIs. A larger API,
with coarser functions would help reduce latency by collapsing
multiple calls, but this must be balanced by keeping the API
expressive and easy to use. For example, there are several checks that
need to be made to determine whether an element is visible to an
end-user. Not only do we need to take into account various CSS
properties, which may need to be inferred by looking at parent
elements, but we should probably also check the dimensions of the
element. A minimalist API would require each of these checks to be
made individually. WebDriver collapses all of them into a single
\code{isDisplayed} method.

\end{aosasect2}

\begin{aosasect2}{Final Thought: This Is Open Source}

Although it's not strictly an architectural point, Selenium is an Open
Source project. The theme that ties all the above points together is
that we'd like to make it as easy as possible for a new developer to
contribute. By keeping the depth of knowledge required as shallow as
possible, using as few languages as necessary and by relying on
automated tests to verify that nothing has broken, we hopefully enable
this ease of contribution.

Originally the project was split into a series of modules, with each
module representing a particular browser with additional modules for
common code and for support and utility code. Source trees for each
binding were stored under these modules. This approach made a lot of
sense for languages such as Java and C\#, but was painful to work with
for Rubyists and Pythonistas. This translated almost directly into
relative contributor numbers, with only a handful of people able and
interested to work on the Python and Ruby bindings. To address this,
in October and November of 2010 the source code was reorganized with
the Ruby and Python code stored under a single top-level directory per
language. This more closely matched the expectations of Open Source
developers in those languages, and the effect on contributions from
the community was noticeable almost immediately.

\end{aosasect2}

\end{aosasect1}

\begin{aosasect1}{Coping with Complexity}

Software is a lumpy construct. The lumps are complexity, and as
designers of an API we have a choice as where to push that
complexity. At one extreme we could spread the complexity as evenly as
possible, meaning that every consumer of the API needs to be party to
it. The other extreme suggests taking as much of the complexity as
possible and isolating it in a single place. That single place would
be a place of darkness and terror for many if they have to venture
there, but the trade-off is that users of the API, who need not delve
into the implementation, have that cost of complexity paid up-front
for them.

The WebDriver developers lean more towards finding and isolating the
complexity in a few places rather than spreading it out. One reason
for this is our users. They're exceptionally good at finding problems
and issues, as a glance at our bug list shows, but because many of
them are not developers a complex API isn't going to work well. We
sought to provide an API that guides people in the right direction. As
an example, consider the following methods from the original Selenium
API, each of which can be used to set the value of an input element:

\begin{aosaitemize}

\item type

\item typeKeys

\item typeKeysNative

\item keydown

\item keypress

\item keyup

\item keydownNative

\item keypressNative

\item keyupNative

\item attachFile

\end{aosaitemize}

Here's the equivalent in the WebDriver API:

\begin{aosaitemize}

\item sendKeys

\end{aosaitemize}

As discussed earlier, this highlights one of the major philosophical
differences between RC and WebDriver in that WebDriver is striving to
emulate the user, whereas RC offers APIs that deal at a lower level
that a user would find hard or impossible to reach. The distinction
between \code{typeKeys} and \code{typeKeysNative} is that the former
always uses synthetic events, whereas the latter attempts to use the
AWT Robot to type the keys. Disappointingly, the AWT Robot sends the
key presses to whichever window has focus, which may not necessarily
be the browser. WebDriver's native events, by contrast, are sent
directly to the window handle, avoiding the requirement that the
browser window have focus.

\begin{aosasect2}{The WebDriver Design}

The team refers to WebDriver's API as being ``object-based''. The
interfaces are clearly defined and try to adhere to having only a
single role or responsibility, but rather than modeling every single
possible HTML tag as its own class we only have a single
\code{WebElement} interface. By following this approach developers
who are using an IDE which supports auto-completion can be led towards
the next step to take. The result is that coding sessions may
look like this (in Java):

\begin{verbatim}
WebDriver driver = new FirefoxDriver();
driver.<user hits space>
\end{verbatim}

\noindent At this point, a relatively short list of 13 methods to pick from
appears. The user selects one:

\begin{verbatim}
driver.findElement(<user hits space>)
\end{verbatim}

\noindent Most IDEs will now drop a hint about the type of the argument
expected, in this case a ``By''. There are a number of preconfigured
factory methods for ``By'' objects declared as static methods on the
By itself. Our user will quickly end up with a line of code that looks
like:

\begin{verbatim}
driver.findElement(By.id("some_id"));
\end{verbatim}

\begin{aosabox}{Role-based Interfaces}

Think of a simplified \code{Shop} class. Every day, it needs to be
restocked, and it collaborates with a \code{Stockist} to deliver this
new stock. Every month, it needs to pay staff and taxes. For the sake
of argument, let's assume that it does this using an
\code{Accountant}. One way of modeling this looks like:

\begin{verbatim}
public interface Shop {
    void addStock(StockItem item, int quantity);
    Money getSalesTotal(Date startDate, Date endDate);
}
\end{verbatim}

We have two choices about where to draw the boundaries when defining
the interface between the Shop, the Accountant and the Stockist. We
could draw a theoretical line as shown in \aosafigref{fig.sel.int1}.

This would mean that both \code{Accountant} and \code{Stockist} would
accept a \code{Shop} as an argument to their respective methods. The
drawback here, though, is that it's unlikely that the Accountant
really wants to stack shelves, and it's probably not a great idea for
the Stockist to realize the vast mark-up on prices that the Shop is
adding. So, a better place to draw the line is shown in
\aosafigref{fig.sel.int2}.

We'll need two interfaces that the Shop needs to implement, but these
interfaces clearly define the role that the Shop fulfills for both the
Accountant and the Stockist. They are role-based interfaces:

\begin{verbatim}
public interface HasBalance {
    Money getSalesTotal(Date startDate, Date endDate);
}

public interface Stockable {
    void addStock(StockItem item, int quantity);
}

public interface Shop extends HasBalance, Stockable {
}
\end{verbatim}
\end{aosabox}

I find \code{UnsupportedOperationExceptions} and their ilk deeply
displeasing, but there needs to be something that allows functionality
to be exposed for the subset of users who might need it without
cluttering the rest of the APIs for the majority of users. To this
end, WebDriver makes extensive use of role-based interfaces. For
example, there is a \code{JavascriptExecutor} interface that provides
the ability to execute arbitrary chunks of Javascript in the context
of the current page. A successful cast of a WebDriver instance to that
interface indicates that you can expect the methods on it to work.

\aosafigure[235pt]{../images/selenium/role-based-interfaces-one-interface.eps}{Accountant and Stockist Depend on Shop}{fig.sel.int1}

\aosafigure[235pt]{../images/selenium/role-based-interfaces-two-interfaces.eps}{Shop Implements HasBalance and Stockable}{fig.sel.int2}

\end{aosasect2}

\begin{aosasect2}{Dealing with the Combinatorial Explosion}

One of the first things that is apparent from a moment's thought about
the wide range of browsers and languages that WebDriver supports is
that unless care is taken it would quickly face an escalating cost of
maintenance. With X browsers and Y languages, it would be very easy to
fall into the trap of maintaining X${\times}$Y implementations.

Reducing the number of languages that WebDriver supports would be one
way to reduce this cost, but we don't want to go down this route for
two reasons. Firstly, there's a cognitive load to be paid when
switching from one language to another, so it's advantageous to users
of the framework to be able to write their tests in the same language
that they do the majority of their development work in. Secondly,
mixing several languages on a single project is something that teams
may not be comfortable with, and corporate coding standards and
requirements often seem to demand a technology monoculture (although,
pleasingly, I think that this second point is becoming less true over
time), therefore reducing the number of supported languages isn't an
available option.

Reducing the number of supported browsers also isn't an option---there
were vociferous arguments when we phased out support for Firefox 2 in
WebDriver, despite the fact that when we made this choice it
represented less than 1\% of the browser market.

The only choice we have left is to try and make all the browsers look
identical to the language bindings: they should offer a uniform
interface that can be addressed easily in a wide variety of languages.
What is more, we want the language bindings themselves to be as easy
to write as possible, which suggests that we want to keep them as slim
as possible. We push as much logic as we can into the underlying
driver in order to support this: every piece of functionality we fail
to push into the driver is something that needs to be implemented in
every language we support, and this can represent a significant amount
of work.

As an example, the IE driver has successfully pushed the
responsibility for locating and starting IE into the main driver
logic. Although this has resulted in a surprising number of lines of
code being in the driver, the language binding for creating a new
instance boils down to a single method call into that driver. For
comparison, the Firefox driver has failed to make this change. In the
Java world alone, this means that we have three major classes that
handle configuring and starting Firefox weighing in at around 1300
lines of code. These classes are duplicated in every language binding
that wants to support the FirefoxDriver without relying on starting a
Java server. That's a lot of additional code to maintain.

\end{aosasect2}

\begin{aosasect2}{Flaws in the WebDriver Design}

The downside of the decision to expose capabilities in this way is
that until someone knows that a particular interface exists they may
not realize that WebDriver supports that type of functionality;
there's a loss of explorability in the API\@. Certainly when WebDriver
was new we seemed to spend a lot of time just pointing people towards
particular interfaces. We've now put a lot more effort into our
documentation and as the API gets more widely used it becomes easier
and easier for users to find the information they need.

There is one place where I think our API is particularly poor. We have
an interface called \code{RenderedWebElement} which has a strange
mish-mash of methods to do with querying the rendered state of the
element (\code{isDisplayed}, \code{getSize} and \code{getLocation}),
performing operations on it (\code{hover} and drag and drop methods),
and a handy method for getting the value of a particular CSS
property. It was created because the HtmlUnit driver didn't expose the
required information, but the Firefox and IE drivers did. It
originally only had the first set of methods but we added the other
methods before I'd done hard thinking about how I wanted the API to
evolve. The interface is well known now, and the tough choice is
whether we keep this unsightly corner of the API given that it's
widely used, or whether we attempt to delete it. My preference is not
to leave a ``broken window'' behind, so fixing this before we release
Selenium 2.0 is important.
As a result, by the time you read this chapter,
\code{RenderedWebElement} may well be gone.

From an implementor's point of view, binding tightly to a browser is
also a design flaw, albeit an inescapable one. It takes significant
effort to support a new browser, and often several attempts need to be
made in order to get it right. As a concrete example, the Chrome
driver has gone through four complete rewrites, and the IE driver has
had three major rewrites too. The advantage of binding tightly to a
browser is that it offers more control.

\end{aosasect2}

\end{aosasect1}

\begin{aosasect1}{Layers and Javascript}
\label{sec.selenium.layers}

A browser automation tool is essentially built of three moving parts:

\begin{aosaitemize}

\item A way of interrogating the DOM.

\item A mechanism for executing Javascript.

\item Some means of emulating user input.

\end{aosaitemize}

This section focuses on the first part: providing a mechanism to
interrogate the DOM\@. The lingua franca of the browser is Javascript,
and this seems like the ideal language to use when interrogating the
DOM\@. Although this choice seems obvious, making it leads to some
interesting challenges and competing requirements that need balancing
when thinking about Javascript.

Like most large projects, Selenium makes use of a layered set of
libraries.  The bottom layer is Google's Closure Library, which
supplies primitives and a modularization mechanism allowing source
files to be kept focused and as small as possible. Above this, there
is a utility library providing functions that range from simple tasks
such as getting the value of an attribute, through determining whether
an element would be visible to an end user, to far more complex
actions such as simulating a click using synthesized events. Within
the project, these are viewed as offering the smallest units of
browser automation, and so are called Browser Automation Atoms or
atoms. Finally, there are adapter layers that compose atoms in order
to meet the API contracts of both WebDriver and Core.

\aosafigure[150pt]{../images/selenium/atoms.eps}{Layers of Selenium Javascript Library}{fig.sel.atoms}

The Closure Library was chosen for several reasons. The main one was
that the Closure Compiler understands the modularization technique the
Library uses. The Closure Compiler is a compiler targeting Javascript
as the output language. ``Compilation'' can be as simple as ordering
input files in dependency order, concatenating and pretty printing
them, or as complex as doing advanced minification and dead code
removal. Another undeniable advantage was that several members of the
team doing the work on the Javascript code were very familiar with
Closure Library.

This ``atomic'' library of code is used pervasively throughout the
project when there is a requirement to interrogate the DOM\@.  For RC
and those drivers largely composed of Javascript, the library is used
directly, typically compiled as a monolithic script. For drivers
written in Java, individual functions from the WebDriver adapter layer
are compiled with full optimization enabled, and the generated
Javascript included as resources in the JARs. For drivers written in C
variants, such as the iPhone and IE drivers, not only are the
individual functions compiled with full optimization, but the
generated output is converted to a constant defined in a header which
is executed via the driver's normal Javascript execution mechanism on
demand. Although this seems like a strange thing to do, it allows the
Javascript to be pushed into the underlying driver without needing to
expose the raw source in multiple places.

Because the atoms are used pervasively it's possible to ensure
consistent behavior between the different browsers, and because the
library is written in Javascript and doesn't require elevated
privileges to execute the development cycle, is easy and fast. The
Closure Library can load dependencies dynamically, so the Selenium
developer need only write a test and load it in a browser, modifying
code and hitting the refresh button as required. Once the test is
passing in one browser, it's easy to load it in another browser and
confirm that it passes there. Because the Closure Library does a good
job of abstracting away the differences between browsers, this is
often enough, though it's reassuring to know that there are continuous
builds that will run the test suite in every supported browser.

Originally Core and WebDriver had many areas of congruent code---code
that performed the same function in slightly different ways. When we
started work on the atoms, this code was combed through to try and
find the ``best of breed'' functionality. After all, both projects had
been used extensively and their code was very robust so throwing away
everything and starting from scratch would not only have been wasteful
but foolish.  As each atom was extracted, the sites at which it would
be used were identified and switched to using the atom. For example,
the Firefox driver's \code{getAttribute} method shrunk from
approximately 50 lines of code to 6 lines long, including blank lines:

\begin{verbatim}
FirefoxDriver.prototype.getElementAttribute =
  function(respond, parameters) {
  var element = Utils.getElementAt(parameters.id,
                                   respond.session.getDocument());
  var attributeName = parameters.name;

  respond.value = webdriver.element.getAttribute(element, attributeName);
  respond.send();
};
\end{verbatim}

\noindent That second-to-last line, where \code{respond.value} is assigned to,
is using the atomic WebDriver library.

The atoms are a practical demonstration of several of the
architectural themes of the project. Naturally they enforce the
requirement that an implementation of an API be sympathetic to a
Javascript implementation. What's even better is that the same library
is shared throughout the codebase; where once a bug had to be verified
and fixed across multiple implementations, it is now enough to fix the
bug in one place, which reduces the cost of change while improving
stability and effectiveness. The atoms also make the bus factor of the
project more favorable. Since a normal Javascript unit test can be
used to check that a fix works the barrier to joining the Open Source
project is considerably lower than it was when knowledge of how each
driver was implemented was required.

There is another benefit to using the atoms. A layer emulating the
existing RC implementation but backed by WebDriver is an important
tool for teams looking to migrate in a controlled fashion to the newer
WebDriver APIs. As Selenium Core is atomized it becomes possible to
compile each function from it individually, making the task of writing
this emulating layer both easier to implement and more accurate.

It goes without saying that there are downsides to the approach taken.
Most importantly, compiling Javascript to a C \code{const} is a very
strange thing to do, and it always baffles new contributors to the
project who want to work on the C code. It is also a rare developer
who has every version of every browser and is dedicated enough to run
every test in all of those browsers---it is possible for someone to
inadvertently cause a regression in an unexpected place, and it can
take some time to identify the problem, particularly if the continuous
builds are being flaky.

Because the atoms normalize return values between browsers, there can
also be unexpected return values. For example, consider this HTML:

\begin{verbatim}
<input name="example" checked>
\end{verbatim}

\noindent The value of the \code{checked} attribute will depend on the browser
being used. The atoms normalize this, and other Boolean attributes
defined in the HTML5 spec, to be ``true'' or ``false''. When this atom
was introduced to the code base, we discovered many places where
people were making browser-dependent assumptions about what the return
value should be. While the value was now consistent there was an
extended period where we explained to the community what had happened
and why.

\end{aosasect1}

\begin{aosasect1}{The Remote Driver, and the Firefox Driver in Particular}

The remote WebDriver was originally a glorified RPC mechanism. It has
since evolved into one of the key mechanisms we use to reduce the cost
of maintaining WebDriver by providing a uniform interface that
language bindings can code against. Even though we've pushed as much
of the logic as we can out of the language bindings and into the
driver, if each driver needed to communicate via a unique protocol we
would still have an enormous amount of code to repeat across all the
language bindings.

The remote WebDriver protocol is used wherever we need to communicate
with a browser instance that's running out of process.  Designing this
protocol meant taking into consideration a number of concerns. Most of
these were technical, but, this being open source, there was also the
social aspect to consider.

Any RPC mechanism is split into two pieces: the transport and the
encoding. We knew that however we implemented the remote WebDriver
protocol, we would need support for both pieces in the languages we
wanted to use as clients. The first iteration of the design was
developed as part of the Firefox driver.

Mozilla, and therefore Firefox, was always seen as being a
multi-platform application by its developers. In order to facilitate
the development, Mozilla created a framework inspired by Microsoft's
COM that allowed components to be built and bolted together called
XPCOM (cross-platform COM). An XPCOM interface is declared using IDL,
and there are
language bindings for C and Javascript as well as other languages. Because
XPCOM is used to construct Firefox, and because XPCOM has Javascript
bindings, it's possible to make use of XPCOM objects in Firefox
extensions.

Normal Win32 COM allows interfaces to be accessed remotely. There were
plans to add the same ability to XPCOM too, and Darin Fisher added an
XPCOM ServerSocket implementation to facilitate this. Although the
plans for D-XPCOM never came to fruition, like an appendix, the
vestigial infrastructure is still there.  We took advantage of this to
create a very basic server within a custom Firefox extension
containing all the logic for controlling Firefox. The protocol used
was originally text-based and line-oriented, encoding all strings as
UTF-2. Each request or response began with a number, indicating how
many newlines to count before concluding that the request or reply had
been sent. Crucially, this scheme was easy to implement in Javascript
as SeaMonkey (Firefox's Javascript engine at the time) stores
Javascript strings internally as 16 bit unsigned integers.

Although futzing with custom encoding protocols over raw sockets is a
fun way to pass the time, it has several drawbacks. There were no
widely available libraries for the custom protocol, so it needed to be
implemented from the ground up for every language that we wanted to
support. This requirement to implement more code would make it less
likely that generous Open Source contributors would participate in the
development of new language bindings. Also, although a line-oriented
protocol was fine when we were only sending text-based data around, it
brought problems when we wanted to send images (such as screenshots)
around.

It became very obvious, very quickly that this original RPC mechanism
wasn't practical. Fortunately, there was a well-known transport that
has widespread adoption and support in almost every language that
would allow us to do what we wanted: HTTP.

Once we had decided to use HTTP for a transport mechanism, the next
choice that needed to be made was whether to use a single end-point
(\`{a} la SOAP) or multiple end points (in the style of REST) The
original Selenese protocol used a single end-point and had encoded
commands and arguments in the query string. While this approach worked
well, it didn't ``feel'' right: we had visions of being able to
connect to a remote WebDriver instance in a browser to view the state
of the server. We ended up choosing an approach we call ``REST-ish'':
multiple end-point URLs using the verbs of HTTP to help provide
meaning, but breaking a number of the constraints required for a truly
RESTful system, notably around the location of state and cacheability,
largely because there is only one location for the application state
to meaningfully exist.

Although HTTP makes it easy to support multiple ways of encoding data
based on content type negotiation, we decided that we needed a
canonical form that all implementations of the remote WebDriver
protocol could work with. There were a handful of obvious choices:
HTML, XML or JSON\@. We quickly ruled out XML: although it's a
perfectly reasonable data format and there are libraries that support
it for almost every language, my perception of how well-liked it is in
the Open Source community was that people don't enjoy working with
it. In addition, it was entirely possible that although the returned
data would share a common ``shape'' it would be easy for additional
fields to be added\footnote{For example, the remote server returns a
base64-encoded screen grab with every exception as a debugging aid
but the Firefox driver doesn't.}. Although these extensions could be
modeled using XML namespaces this would start to introduce Yet More
Complexity into the client code: something I was keen to avoid.  XML
was discarded as an option.  HTML wasn't really a good choice, as we
needed to be able to define our own data format, and though an
embedded micro-format could have been devised and used that seems like
using a hammer to crack an egg.

The final possibility considered was Javascript Object Notation
(JSON). Browsers can transform a string into an object using either a
straight call to \code{eval} or, on more recent browsers, with
primitives designed to transform a Javascript object to and from a
string securely and without side-effects. From a practical
perspective, JSON is a popular data format with libraries for handling
it available for almost every language and all the cool kids like
it. An easy choice.

The second iteration of the remote WebDriver protocol therefore used
HTTP as the transport mechanism and UTF-8 encoded JSON as the default
encoding scheme. UTF-8 was picked as the default encoding so that
clients could easily be written in languages with limited support for
Unicode, as UTF-8 is backwardly compatible with ASCII\@. Commands sent
to the server used the URL to determine which command was being sent,
and encoded the parameters for the command in an array.

For example a call to \code{WebDriver.get("http://www.example.com")}
mapped to a POST request to a URL encoding the session ID and ending
with ``/url'', with the array of parameters looking like
\code{{[}'http://www.example.com'{]}}.  The returned result was a
little more structured, and had place-holders for a returned value and
an error code.  It wasn't long until the third iteration of remote
protocol, which replaced the request's array of parameters with a
dictionary of named parameters. This had the benefit of making
debugging requests significantly easier, and removed the possibility
of clients mistakenly mis-ordering parameters, making the system as a
whole more robust.  Naturally, it was decided to use normal HTTP error
codes to indicate certain return values and responses where they were
the most appropriate way to do so; for example, if a user attempts to
call a URL with nothing mapped to it, or when we want to indicate the
``empty response''.

The remote WebDriver protocol has two levels of error handling, one
for invalid requests, and one for failed commands. An example of an
invalid request is for a resource that doesn't exist on the server, or
perhaps for a verb that the resource doesn't understand (such as
sending a DELETE command to the the resource used for dealing with the
URL of the current page) In those cases, a normal HTTP 4xx response is
sent. For a failed command, the responses error code is set to 500
(``Internal Server Error'') and the returned data contains a more
detailed breakdown of what went wrong.

When a response containing data is sent from the server, it takes the
form of a JSON object:\\

\begin{tabular}{|lp{4in}|}
\hline
Key & Description \\
\hline
sessionId & An opaque handle used by the server to determine where to route session-specific commands. \\
status & A numeric status code summarizing the result of the command. A non-zero value indicates that the command failed. \\
value & The response JSON value. \\
\hline
\end{tabular}

~\\

\noindent
An example response would be:

\begin{verbatim}
{
  sessionId: 'BD204170-1A52-49C2-A6F8-872D127E7AE8',
  status: 7,
  value: 'Unable to locate element with id: foo'
}
\end{verbatim}

\noindent As can be seen, we encode status codes in the response, with a
non-zero value indicating that something has gone horribly awry. The
IE driver was the first to use status codes, and the values used in
the wire protocol mirror these. Because all error codes are consistent
between drivers, it is possible to share error handling code between
all the drivers written in a particular language, making the job of
the client-side implementors easier.

The Remote WebDriver Server is simply a Java servlet that acts as a
multiplexer, routing any commands it receives to an appropriate
WebDriver instance. It's the sort of thing that a second year graduate
student could write. The Firefox driver also implements the remote
WebDriver protocol, and its architecture is far more interesting, so
let's follow a request through from the call in the language bindings
to that back-end until it returns to the user.

\pagebreak

Assuming that we're using Java, and that ``element'' is an instance of
WebElement, it all starts here:

\begin{verbatim}
element.getAttribute(``row'');
\end{verbatim}

\noindent Internally, the element has an opaque ``id'' that the server-side uses
to identify which element we're talking about. For the sake of this
discussion, we'll imagine it has the value ``some\_opaque\_id''. This
is encoded into a Java \code{Command} object with a \code{Map} holding
the (now named) parameters \code{id} for the element ID
and \code{name} for the name of the attribute being queried.

A quick look up in a table indicates that the correct URL is:

\begin{verbatim}
/session/:sessionId/element/:id/attribute/:name
\end{verbatim}

\noindent Any section of the URL that begins with a colon is assumed to be a
variable that requires substitution. We've been given the \code{id}
and \code{name} parameters already, and the \code{sessionId} is
another opaque handle that is used for routing when a server can
handle more than one session at a time (which the Firefox driver
cannot). This URL therefore typically expands to something like:

\begin{verbatim}
http://localhost:7055/hub/session/XXX/element/some_opaque_id/attribute/row
\end{verbatim}

As an aside, WebDriver's remote wire protocol was originally developed
at the same time as URL Templates were proposed as a draft RFC\@. Both
our scheme for specifying URLs and URL Templates allow variables to be
expanded (and therefore derived) within a URL\@. Sadly, although URL
Templates were proposed at the same time, we only became aware of them
relatively late in the day, and therefore they are not used to
describe the wire protocol.

Because the method we're executing is idempotent\footnote{I.e., always
returns the same result.}, the correct HTTP method to use is a
GET\@. We delegate down to a Java library that can handle HTTP (the
Apache HTTP Client) to call the server.

\aosafigure{../images/selenium/firefox-driver-overall-architecture.eps}{Overview of the Firefox Driver Architecture}{fig.sel.ffdr}

The Firefox driver is implemented as a Firefox extension, the basic
design of which is shown in~\aosafigref{fig.sel.ffdr}.
Somewhat unusually, it has an embedded HTTP server.  Although
originally we used one that we had built ourselves, writing HTTP
servers in XPCOM wasn't one of our core competencies, so when the
opportunity arose we replaced it with a basic HTTPD written by Mozilla
themselves.  Requests are received by the HTTPD and almost straight
away passed to a \code{dispatcher} object.

The dispatcher takes the request and iterates over a known list of
supported URLs, attempting to find one that matches the request. This
matching is done with knowledge of the variable interpolation that
went on in the client side. Once an exact match is found, including
the verb being used, a JSON object, representing the command to
execute, is constructed. In our case it looks like:

\begin{verbatim}
{
  'name': 'getElementAttribute',
  'sessionId': { 'value': 'XXX' },
  'parameters': {
    'id': 'some_opaque_key',
    'name': 'rows'
  }
}
\end{verbatim}

\noindent This is then passed as a JSON string to a custom XPCOM component we've
written called the CommandProcessor. Here's the code:

\begin{verbatim}
var jsonResponseString = JSON.stringify(json);
var callback = function(jsonResponseString) {
  var jsonResponse = JSON.parse(jsonResponseString);

  if (jsonResponse.status != ErrorCode.SUCCESS) {
    response.setStatus(Response.INTERNAL_ERROR);
  }

  response.setContentType('application/json');
  response.setBody(jsonResponseString);
  response.commit();
};

// Dispatch the command.
Components.classes['@googlecode.com/webdriver/command-processor;1'].
    getService(Components.interfaces.nsICommandProcessor).
    execute(jsonString, callback);
\end{verbatim}

\pagebreak

There's quite a lot of code here, but there are two key points. First,
we converted the object above to a JSON string.  Secondly, we pass a
callback to the execute method that causes the HTTP response to be
sent.

The execute method of the command processor looks up the ``name'' to
determine which function to call, which it then does. The first
parameter given to this implementing function is a \code{respond}''
object (so called because it was originally just the function used to
send the response back to the user), which encapsulates not only the
possible values that might be sent, but also has a method that allows
the response to be dispatched back to the user and mechanisms to find
out information about the DOM\@. The second parameter is the value of
the \code{parameters} object seen above (in this case, \code{id} and
\code{name}). The advantage of this scheme is that each function has a
uniform interface that mirrors the structure used on the client
side. This means that the mental models used for thinking about the
code on each side are similar.  Here's the underlying implementation
of \code{getAttribute}, which you've seen before in
\aosasecref{sec.selenium.layers}:

\begin{verbatim}
FirefoxDriver.prototype.getElementAttribute = function(respond, parameters) {
  var element = Utils.getElementAt(parameters.id,
                                  respond.session.getDocument());
  var attributeName = parameters.name;

  respond.value = webdriver.element.getAttribute(element, attributeName);
  respond.send();
};
\end{verbatim}

\noindent In order to make element references consistent, the first line simply
looks up the element referred to by the opaque ID in a cache. In the
Firefox driver, that opaque ID is a UUID and the ``cache'' is simply a
map.  The \code{getElementAt} method also checks to see if the
referred to element is both known and attached to the DOM\@. If either
check fails, the ID is removed from the cache (if necessary) and an
exception is thrown and returned to the user.

The second line from the end makes use of the browser automation atoms
discussed earlier, this time compiled as a monolithic script and
loaded as part of the extension.

In the final line, the \code{send} method is called.  This does a
simple check to ensure that we only send a response once before it
calls the callback given to the execute method. The response is sent
back to the user in the form of a JSON string, which is decanted into
an object that looks like (assuming that \code{getAttribute} returned
``7'', meaning the element wasn't found):

\begin{verbatim}
{
  'value': '7',
  'status': 0,
  'sessionId': 'XXX'
}
\end{verbatim}

The Java client then checks the value of the status field. If that
value is non-zero, it converts the numeric status code into an
exception of the correct type and throws that, using the ``value''
field to help set the message sent to the user. If the status is zero
the value of the ``value'' field is returned to to the user.

Most of this makes a certain amount of sense, but there was one piece
that an astute reader will raise questions about: why did the
dispatcher convert the object it had into a string before calling the
\code{execute} method?

The reason for this is that the Firefox Driver also supports running
tests written in pure Javascript. Normally, this would be an extremely
difficult thing to support: the tests are running in the context of
the browser's Javascript security sandbox, and so may not do a range
of things that are useful in tests, such as traveling between domains
or uploading files. The WebDriver Firefox extension, however, provides
an escape hatch from the sandbox. It announces its presence by adding
a \code{webdriver} property to the document element. The WebDriver
Javascript API uses this as an indicator that it can add JSON
serialized command objects as the value of a \code{command} property on
the document element, fire a custom \code{webdriverCommand} event and
then listen for a \code{webdriverResponse} event on the same element to
be notified that the \code{response} property has been set.

This suggests that browsing the web in a copy of Firefox with the
WebDriver extension installed is a seriously bad idea as it makes it
trivially easy for someone to remotely control the browser.

Behind the scenes, there is a DOM messenger, waiting for the
\code{webdriverCommand} this reads the serialized JSON object and calls
the \code{execute} method on the command processor. This time, the
callback is one that simply sets the \code{response} attribute on the
document element and then fires the expected \code{webdriverResponse}
event.

\end{aosasect1}

\begin{aosasect1}{The IE Driver}

Internet Explorer is an interesting browser. It's constructed of a
number of COM interfaces working in concert. This extends all the way
into the Javascript engine, where the familiar Javascript variables
actually refer to underlying COM instances. That Javascript
\code{window} is an \code{IHTMLWindow}.  \code{document} is an
instance of the COM interface \code{IHTMLDocument}. Microsoft have
done an excellent job in maintaining existing behavior as they
enhanced their browser. This means that if an application worked with
the COM classes exposed by IE6 it will still continue to work with
IE9.

The Internet Explorer driver has an architecture that's evolved over
time. One of the major forces upon its design has been a requirement
to avoid an installer. This is a slightly unusual requirement, so
perhaps needs some explanation. The first reason not to require an
installer is that it makes it harder for WebDriver to pass the ``5
minute test'', where a developer downloads a package and tries it out
for a brief period of time. More importantly, it is relatively common
for users of WebDriver to not be able to install software on their own
machines. It also means that no-one needs to remember to log on to the
continuous integration servers to run an installer when a project
wants to start testing with IE\@. Finally, running installers just isn't
in the culture of some languages. The common Java idiom is to simply
drop JAR files on to the CLASSPATH, and, in my experience, those
libraries that require installers tend not to be as well-liked or
used.

So, no installer. There are consequences to this choice.

The natural language to use for programming on Windows would be
something that ran on .Net, probably C\#. The IE driver integrates
tightly with IE by making use of the IE COM Automation interfaces that
ship with every version of Windows. In particular, we use COM
interfaces from the native MSHTML and ShDocVw DLLs, which form part of
IE\@. Prior to C\# 4, CLR/COM interoperability was achieved via the use of
separate Primary Interop Assemblies (PIAs) A PIA is essentially a
generated bridge between the managed world of the CLR and that of COM.

Sadly, using C\# 4 would mean using a very modern version of the .Net
runtime, and many companies avoid living on the leading edge,
preferring the stability and known issues of older releases. By using
C\# 4 we would automatically exclude a reasonable percentage of our
user-base. There are also other disadvantages to using a PIA\@. Consider
licensing restrictions.  After consultation with Microsoft, it became
clear that the Selenium project would not have the rights to
distribute the PIAs of either the MSHTML or ShDocVw libraries. Even if
those rights had been granted, each installation of Windows and IE has
a unique combination of these libraries, which means that we would
have needed to ship a vast number of these things. Building the PIAs
on the client machine on demand is also a non-starter, as they require
developer tools that may not exist on a normal user's machine.

So, although C\# would have been an attractive language to do the bulk
of the coding in, it wasn't an option. We needed to use something
native, at least for the communication with IE\@. The next natural
choice for this is C++, and this is the language that we chose in the
end. Using C++ has the advantage that we don't need to use PIAs, but
it does mean that we need to redistribute the Visual Studio C++
runtime DLL unless we statically link against them.  Since we'd need
to run an installer in order to make that DLL available, we statically
link our library for communicating with IE.

That's a fairly high cost to pay for a requirement not to use an
installer. However, going back to the theme of where complexity should
live, it is worth the investment as it makes our users' lives
considerably easier. It is a decision we re-evaluate on an ongoing
basis, as the benefit to the user is a trade-off with the fact that
the pool of people able to contribute to an advanced C++ Open Source
project seems significantly smaller than those able to contribute to
an equivalent C\# project.

The initial design of the IE driver is shown in
\aosafigref{fig.sel.sie}.

\aosafigure{../images/selenium/simple-ie-stack.eps}{Original IE Driver}{fig.sel.sie}

Starting from the bottom of that stack, you can see that we're using
IE's COM Automation interfaces.  In order to make these easier to deal
with on a conceptual level, we wrapped those raw interfaces with a set
of C++ classes that closely mirrored the main WebDriver API\@. In order
to get the Java classes communicating with the C++ we made use of JNI,
with the implementations of the JNI methods using the C++ abstractions
of the COM interfaces.

This approach worked reasonably well while Java was the only client
language, but it would have been a source of pain and complexity if
each language we supported needed us to alter the underlying
library. Thus, although JNI worked, it didn't provide the correct
level of abstraction.

What was the correct level of abstraction? Every language that we
wanted to support had a mechanism for calling down to straight C
code. In C\#, this takes the form of PInvoke. In Ruby there is FFI, and
Python has ctypes. In the Java world, there is an excellent library
called JNA (Java Native Architecture). We needed to expose our API
using this lowest common denominator. This was done by taking our
object model and flattening it, using a simple two or three letter
prefix to indicate the ``home interface'' of the method: ``wd'' for
``WebDriver'' and ``wde'' for WebDriver Element. Thus
\code{WebDriver.get} became \code{wdGet}, and
\code{WebElement.getText} became \code{wdeGetText}. Each method
returns an integer representing a status code, with ``out'' parameters
being used to allow functions to return more meaningful data. Thus we
ended up with method signatures such as:

\begin{verbatim}
int wdeGetAttribute(WebDriver*, WebElement*, const wchar_t*, StringWrapper**)
\end{verbatim}

\noindent To calling code, the \code{WebDriver}, \code{WebElement} and
\code{StringWrapper} are opaque types: we expressed the difference in
the API to make it clear what value should be used as that parameter,
though could just as easily have been ``void *''. You can also see
that we were using wide characters for text, since we wanted to deal
with internationalized text properly.

On the Java side, we exposed this library of functions via an
interface, which we then adapted to make it look like the normal object-oriented
interface presented by WebDriver. For example, the Java definition of
the \code{getAttribute} method looks like:

\begin{verbatim}
public String getAttribute(String name) {
  PointerByReference wrapper = new PointerByReference();
  int result = lib.wdeGetAttribute(
      parent.getDriverPointer(), element, new WString(name), wrapper);

  errors.verifyErrorCode(result, "get attribute of");

  return wrapper.getValue() == null ? null : new StringWrapper(lib, wrapper).toString();
}
\end{verbatim}

\noindent This lead to the design shown in \aosafigref{fig.sel.ient}.

\aosafigure{../images/selenium/ie-stack-no-threads.eps}{Modified IE Driver}{fig.sel.ient}

While all the tests were running on the local machine, this worked out
well, but once we started using the IE driver in the remote WebDriver
we started running into random lock ups. We traced this problem back
to a constraint on the IE COM Automation interfaces. They are designed
to be used in a ``Single Thread Apartment'' model. Essentially, this
boils down to a requirement that we call the interface from the same
thread every time. While running locally, this happens by
default. Java app servers, however, spin up multiple threads to handle
the expected load. The end result? We had no way of being sure that
the same thread would be used to access the IE driver in all cases.

One solution to this problem would have been to run the IE driver in a
single-threaded executor and serialize all access via Futures in the
app server, and for a while this was the design we chose. However, it
seemed unfair to push this complexity up to the calling code, and it's
all too easy to imagine instances where people accidentally make use
of the IE driver from multiple threads. We decided to sink the
complexity down into the driver itself. We did this by holding the IE
instance in a separate thread and using the \code{PostThreadMessage}
Win32 API to communicate across the thread boundary. Thus, at the time
of writing, the design of the IE driver looks like
\aosafigref{fig.sel.fullie}.

\aosafigure{../images/selenium/full-ie-stack.eps}{IE Driver as of Selenium 2.0 alpha 7}{fig.sel.fullie}

This isn't the sort of design that I would have chosen voluntarily,
but it has the advantage of working and surviving the horrors that our
users may chose to inflict upon it.

One drawback to this design is that it can be hard to determine
whether the IE instance has locked itself solid. This may happen if a
modal dialog opens while we're interacting with the DOM, or it may
happen if there's a catastrophic failure on the far side of the thread
boundary. We therefore have a timeout associated with every thread
message we post, and this is set to what we thought was a relatively
generous 2 minutes. From user feedback on the mailing lists, this
assumption, while generally true, isn't always correct, and later
versions of the IE driver may well make the timeout configurable.

Another drawback is that debugging the internals can be deeply
problematic, requiring a combination of speed (after all, you've got
two minutes to trace the code through as far as possible), the
judicious use of break points and an understanding of the expected
code path that will be followed across the thread boundary. Needless
to say, in an Open Source project with so many other interesting
problems to solve, there is little appetite for this sort of grungy
work. This significantly reduces the bus factor of the system, and as
a project maintainer, this worries me.

To address this, more and more of the IE driver is being moved to sit
upon the same Automation Atoms as the Firefox driver and Selenium
Core. We do this by compiling each of the atoms we plan to use and
preparing it as a C++ header file, exposing each function as a
constant. At runtime, we prepare the Javascript to execute from these
constants. This approach means that we can develop and test a
reasonable percentage of code for the IE driver without needing a C
compiler involved, allowing far more people to contribute to finding
and resolving bugs. In the end, the goal is to leave only the
interaction APIs in native code, and rely on the atoms as much as
possible.

Another approach we're exploring is to rewrite the IE driver to make
use of a lightweight HTTP server, allowing us to treat it as a remote
WebDriver. If this occurs, we can remove a lot of the complexity
introduced by the thread boundary, reducing the total amount of code
required and making the flow of control significantly easier to
follow.

\end{aosasect1}

\begin{aosasect1}{Selenium RC}

It's not always possible to bind tightly to a particular browser. In
those cases, WebDriver falls back to the original mechanism used by
Selenium. This means using Selenium Core, a pure Javascript framework,
which introduces a number of drawbacks as it executes firmly in the
context of the Javascript sandbox. From a user of WebDriver's APIs
this means that the list of supported browsers falls into tiers, with
some being tightly integrated with and offering exceptional control,
and others being driven via Javascript and offering the same level of
control as the original Selenium RC.

Conceptually, the design used is pretty simple, as you can see in
\aosafigref{fig.sel.arch}.

\aosafigureTop[250pt]{../images/selenium/selenium-rc.eps}{Outline of Selenium RC's Architecture}{fig.sel.arch}

As you can see, there are three moving pieces here: the client code,
the intermediate server and the Javascript code of Selenium Core
running in the browser. The client side is just an HTTP client that
serializes commands to the server-side piece. Unlike the remote
WebDriver, there is just a single end-point, and the HTTP verb used is
largely irrelevant. This is partly because the Selenium RC protocol is
derived from the table-based API offered by Selenium Core, and this
means that the entire API can be described using three URL query
parameters.

When the client starts a new session, the Selenium server looks up the
requested ``browser string'' to identify a matching browser
launcher. The launcher is responsible for configuring and starting an
instance of the requested browser. In the case of Firefox, this is as
simple as expanding a pre-built profile with a handful of extensions
pre-installed (one for handling a ``quit'' command, and another for
modeling ``document.readyState'' which wasn't present on older
Firefox releases that we still support). The key piece of
configuration that's done is that the server configures itself as a
proxy for the browser, meaning that at least some requests (those for
``/selenium-server'') are routed through it.  Selenium RC can operate
in one of three modes: controlling a frame in a single window
(``singlewindow'' mode), in a separate window controlling the AUT in a
second window (``multiwindow'' mode) or by injecting itself into the
page via a proxy (``proxyinjection'' mode). Depending on the mode of
operation, all requests may be proxied.

Once the browser is configured, it is started, with an initial URL
pointing to a page hosted on the Selenium
server---\code{RemoteRunner.html}. This page is responsible for
bootstrapping the process by loading all the required Javascript files
for Selenium Core. Once complete, the ``runSeleniumTest'' function is
called. This uses reflection of the \code{Selenium} object to
initialize the list of available commands that are available before
kicking off the main command processing loop.

The Javascript executing in the browser opens an XMLHttpRequest to a
URL on the waiting server (\code{/selenium-server/driver}), relying on
the fact that the server is proxying all requests to ensure that the
request actually goes somewhere valid. Rather than making a request,
the first thing that this does is send the response from the
previously executed command, or ``OK'' in the case where the browser
is just starting up. The server then keeps the request open until a
new command is received from the user's test via the client, which is
then sent as the response to the waiting Javascript.  This mechanism
was originally dubbed ``Response/Request'', but would now be more
likely to be called ``Comet with AJAX long polling''.

Why does RC work this way? The server needs to be configured as a
proxy so that it can intercept any requests that are made to it
without causing the calling Javascript to fall foul of the ``Single
Host Origin'' policy, which states that only resources from the same
server that the script was served from can be requested via
Javascript.  This is in place as a security measure, but from the
point of view of a browser automation framework developer, it's pretty
frustrating and requires a hack such as this.

The reason for making an XmlHttpRequest call to the server is
two-fold.  Firstly, and most importantly, until WebSockets, a part of
HTML5, become available in the majority of browsers there is no way to
start up a server process reliably within a browser.  That means that
the server had to live elsewhere. Secondly, an XMLHttpRequest calls
the response callback asynchronously, which means that while we're
waiting for the next command the normal execution of the browser is
unaffected. The other two ways to wait for the next command would have
been to poll the server on a regular basis to see if there was another
command to execute, which would have introduced latency to the users
tests, or to put the Javascript into a busy loop which would have
pushed CPU usage through the roof and would have prevented other
Javascript from executing in the browser (since there is only ever one
Javascript thread executing in the context of a single window).

Inside Selenium Core there are two major moving pieces. These are the
main \code{selenium} object, which acts as the host for all available
commands and mirrors the API offered to users. The second piece is the
\code{browserbot}. This is used by the Selenium object to abstract
away the differences present in each browser and to present an
idealized view of commonly used browser functionality. This means that
the functions in \code{selenium} are clearer and easier to maintain,
whilst the \code{browserbot} is tightly focused.

Increasingly, Core is being converted to make use of the Automation
Atoms. Both \code{selenium} and \code{browserbot} will probably need
to remain as there is an extensive amount of code that relies on using
the APIs it exposes, but it is expected that they will ultimately be
shell classes, delegating to the atoms as quickly as possible.

\end{aosasect1}

\begin{aosasect1}{Looking Back}

Building a browser automation framework is a lot like painting a room;
at first glance, it looks like something that should be pretty easy to
do.  All it takes is a few coats of paint, and the job's done. The
problem is, the closer you get, the more tasks and details emerge, and
the longer the task becomes. With a room, it's things like working
around light fittings, radiators and the skirting boards that start to
consume time.  For a browser automation framework, it's the quirks and
differing capabilities of browsers that make the situation more
complex. The extreme case of this was expressed by Daniel Wagner-Hall
as he sat next to me working on the Chrome driver; he banged his hands
on the desk and in frustration muttered, "It's all edge cases!" It
would be nice to be able to go back and tell myself that, and that the
project is going to take a lot longer than I expected.

I also can't help but wonder where the project would be if we'd
identified and acted upon the need for a layer like the automation
atoms sooner than we did. It would certainly have made some of the
challenges the project faced, internal and external, technically and
socially, easier to deal with. Core and RC were implemented in a
focused set of languages---essentially just Javascript and Java. Jason
Huggins used to refer to this as providing Selenium with a level of
``hackability'', which made it easy for people to get involved with
the project. It's only with the atoms that this level of hackability
has become widely available in WebDriver. Balanced against this, the
reason why the atoms can be so widely applied is because of the
Closure compiler, which we adopted almost as soon as it was released
as Open Source.

It's also interesting to reflect on the things that we got right. The
decision to write the framework from the viewpoint of the user is
something that I still feel is correct. Initially, this paid off as
early adopters highlighted areas for improvement, allowing the utility
of the tool to increase rapidly. Later, as WebDriver gets asked to do
more and harder things and the number of developers using it
increases, it means that new APIs are added with care and attention,
keeping the focus of the project tight. Given the scope of what we're
trying to do, this focus is vital.

Binding tightly to the browser is something that is both right and
wrong.  It's right, as it has allowed us to emulate the user with
extreme fidelity, and to control the browser extremely well. It's
wrong because this approach is extremely technically demanding,
particularly when finding the necessary hook point into the
browser. The constant evolution of the IE driver is a demonstration of
this in action, and, although it's not covered here, the same is true
of the Chrome driver, which has a long and storied history. At some
point, we'll need to find a way to deal with this complexity.

\end{aosasect1}

\begin{aosasect1}{Looking to the Future}

There will always be browsers that WebDriver can't integrate tightly
to, so there will always be a need for Selenium Core. Migrating this
from its current traditional design to a more modular design based on
the same Closure Library that the atoms are using is underway. We also
expect to embed the atoms more deeply within the existing WebDriver
implementations.

One of the initial goals of WebDriver was to act as a building block
for other APIs and tools. Of course, Selenium doesn't live in a
vacuum: there are plenty of other Open Source browser automation
tools. One of these is Watir (Web Application Testing In Ruby), and
work has begun, as a joint effort by the Selenium and Watir
developers, to place the Watir API over the WebDriver core. We're keen
to work with other projects too, as successfully driving all the
browsers out there is hard work. It would be nice to have a solid
kernel that others could build on. Our hope is that the kernel is
WebDriver.

A glimpse of this future is offered by Opera Software, who have
independently implemented the WebDriver API, using the WebDriver test
suites to verify the behavior of their code, and who will be
releasing their own OperaDriver. Members of the Selenium team are also
working with members of the Chromium team to add better hooks and
support for WebDriver to that browser, and by extension to Chrome
too. We have a friendly relationship with Mozilla, who have
contributed code for the FirefoxDriver, and with the developers of the
popular HtmlUnit Java browser emulator.

One view of the future sees this trend continue, with automation hooks
being exposed in a uniform way across many different browsers. The
advantages for people keen to write tests for web applications are
clear, and the advantages for browser manufacturers are also
obvious. For example, given the relative expense of manual testing,
many large projects rely heavily on automated testing. If it's not
possible, or even if it's ``only'' extremely taxing, to test with a
particular browser, then tests just aren't run for it, with knock-on
effects for how well complex applications work with that
browser. Whether those automation hooks are going to be based on
WebDriver is an open question, but we can hope!

The next few years are going to be very interesting. As we're an open
source project, you'd be welcome to join us for the journey at
\url{http://selenium.googlecode.com/}.

\end{aosasect1}

\end{aosachapter}
