\begin{aosachapter}{Sendmail}{s:sendmail}{Eric Allman}
%% Based on EN-Revision r229

%% Most people think of electronic mail as the program that they interact
%% with---their mail client, technically known as a Mail User Agent
%% (MUA).  But another important part of electronic mail is the software
%% that actually transfers the mail from the sender to the
%% recipient---the Mail Transfer Agent (MTA).  The first MTA on the
%% Internet, and still the most prevalent, was sendmail.
電子メールのプログラムと聞いて、たいていの人が思い浮かべるのはメールクライアントだろう。厳密には、Mail User Agent (MUA)と呼ばれているものだ。しかし、電子メールを扱うソフトウェアには、もうひとつ重要なものがある。それが、実際に送信者から受信者にメールを配送するソフトウェア---Mail Transfer Agent (MTA)である。インターネット上で最初に登場したMTAであり、現在でも幅広く使われているものがsendmailだ。

%% Sendmail was first created before the Internet officially existed. It
%% has been extraordinarily successful, having grown from 1981, when it
%% wasn't at all obvious that the Internet was going to be more than an
%% academic experiment with only a few hundred hosts, to today, with over
%% 800 million Internet hosts as of January
%% 2011\footnote{\url{http://ftp.isc.org/www/survey/reports/2011/01/}}.
%% Sendmail remains among the most used implementations of SMTP on the Internet.
Sendmailが最初に作られたのは、まだインターネットが公式には存在しなかったころである。そして、それから大成功を収めた。1981年に最初に作られたころ、インターネットはまだ学術的な実験段階で、接続されているホストはたかだか数百台に過ぎなかった。それが今や、2011年1月の時点でインターネットに接続されているホストは8億を超える\footnote{\url{http://ftp.isc.org/www/survey/reports/2011/01/}}。当時、今のこの状況を想像できた人はあまりいなかっただろう。Sendmailは、インターネット上のSMTPの実装として、今でも最も使われているものである。

%% \begin{aosasect1}{Once Upon a Time\ldots}
\begin{aosasect1}{むかしむかし\ldots}

%% The first versions of the program that would become known as sendmail
%% were written in 1980. It started as a quick hack to forward messages
%% between different networks.  The Internet was being developed but was
%% not functional at that time.  In fact, many different networks had been
%% proposed with no obvious consensus emerging.  The Arpanet was in use
%% in the United States and the Internet was being designed as an
%% upgrade, but Europe had thrown its weight behind the OSI (Open Systems
%% Interconnect) effort, and for a while it appeared that OSI might
%% triumph. Both of these used leased lines from the phone companies; in
%% the US that speed was 56 Kbps.
後にsendmailとして知られることになるプログラムの最初のバージョンが書かれたのは1980年のことだった。もともとは、メッセージを別のネットワークに転送するためのちょっとしたハックだった。インターネットは構築されていたが、当時はまだそれほど機能的ではなかった。実際のところ、当時はさまざまなネットワークがコンセンサスを得ないままに提案されていた。アメリカではArpanetが使われており、インターネットはその上位版として設計された。しかしヨーロッパではOSI (Open Systems Interconnect)の取り組みを支持しており、おそらくOSIのほうが勝ちを収めるだろうとみられていたこともあった。どちらも電話会社から借りた専用線を使っており、米国での速度は56 Kbpsだった。

%% Probably the most successful network of the time,
%% in terms of numbers of computers
%% and people connected, was the UUCP network, which was unusual in that
%% it had absolutely no central authority. It was, in some sense, the
%% original peer-to-peer network, which ran over dialup phone lines: 9600
%% bps was about the fastest available for some time. The fastest network
%% (at 3 Mbps) was based on the Ethernet from Xerox, which ran a protocol
%% called XNS (Xerox Network Systems)---but it didn't work outside of a
%% local installation.
おそらく、(接続するマシンやユーザーの数で考えると)当時最も成功していたネットワークはUUCPネットワークだろう。このネットワークの特徴は、中央管理型の権限がどこにもなかったことだ。ある意味では、ピアツーピアのネットワークの元祖と言えるだろう。このネットワークは電話回線でのダイアルアップで動いており、最大で9600 bpsの速度しか出なかった。その当時の最速のネットワーク(3 Mbps)はXeroxのEthernetをベースとしたもので、これはXNS (Xerox Network Systems)というプロトコルで動いていた---しかし、ローカルにインストールした環境以外では動作しなかった。

%% The environment of the time was rather different than what exists
%% today.  Computers were highly heterogeneous, to the extent that there
%% wasn't even complete agreement to use 8-bit bytes.
%% For example, other machines included the PDP-10 (36 bit words, 9 bit bytes),
%% the PDP-11 (16 bit words, 8 bit bytes),
%% the CDC 6000 series (60 bit words, 6 bit characters),
%% the IBM 360 (32 bit words, 8 bit bytes),
%% the XDS 940, the ICL 470, and the Sigma 7.
%% One of the up-and-coming platforms was Unix,
%% which at that time came from Bell Laboratories.
%% Most Unix-based
%% machines had 16-bit addresses spaces: at that time the PDP-11 was the
%% major Unix machine, with the Data General 8/32 and the VAX-11/780 just
%% appearing.  Threads didn't exist---in fact, the concept of dynamic
%% processes was still fairly new (Unix had them, but ``serious'' systems
%% such as IBM's OS/360 did not). File locking was not supported in the
%% Unix kernel (but tricks were possible using filesystem links).
当時の環境は、今とは異なっていた。コンピュータはそれぞれ異質なものであり、そもそも1バイトを8ビットにするかどうかさえ完全には合意されていなかった。たとえばPDP-10(1ワードが36ビット、1バイトが9ビット)、PDP-11(1ワードが16ビット、1バイトが8ビット)、CDC 6000シリーズ(1ワードが60ビット、1文字が6ビット)、IBM 360(1ワードが32ビット、1バイトが8ビット)、そしてXDS 940やICL 470、Sigma 7などがあった。そのころ赤丸急上昇中だったプラットフォームのひとつがUnixで、これはベル研究所が発表したものだった。Unixベースのマシンの大半は16ビットアドレス空間を保持していた。当時のUnixマシンの主流はPDP-11で、Data General 8/32やVAX-11/780が登場してきたころだった。スレッドはまだ存在しなかった---実際のところ、ダイナミックプロセスという概念自体がまだまだ出たてのものだったのだ(Unixにはスレッドがあったが、IBMのOS/360みたいな``本物の''システムにはまだ実装されていなかった)。ファイルのロックはまだUnixカーネルではサポートされていなかった(ファイルシステムのリンクを使うという裏技はあった)。

%% To the extent they existed at all, networks were generally low speed
%% (many based on 9600-baud TTY lines; the truly rich might have had
%% Ethernet available, but for local use only). The venerable socket
%% interface wasn't going to be invented for many years. Public key
%% encryption hadn't been invented either, so most network security as we
%% know it today wasn't feasible.
とりあえずあるにはあったものの、ネットワークは一般的に低速だった(その多くは9600ボーのTTY回線を使っていた。お金持ちの中にはEthernetを使っているところもあったが、それでもローカルネットワークだけの話だった)。あの由緒あるソケットインターフェイスが発明されるのは、それから何年か後のことである。公開鍵による暗号化の仕組みもまだ発明されていなかったので、ネットワークセキュリティに関して我々の知る仕組みのほとんどは実現不可能だった。

%% Network email already existed on Unix, but it was created using
%% hacks.  The primary user agent at the time was the \code{/bin/mail}
%% command (today sometimes referred to as \code{binmail} or
%% \code{v7mail}), but some sites had other user agents such as
%% \code{Mail} from Berkeley, which actually understood how to
%% treat messages as individual items rather than being a glorified
%% \code{cat} program.  Every user agent read (and usually wrote!)
%% \code{/usr/spool/mail} directly; there was no abstraction for how the
%% messages were actually stored.
ネットワーク上を流れる電子メールはUnixで既に実現されていたが、そのためにはちょっとしたハックを要した。当時のユーザーエージェントとして一番よく使われていたのが\code{/bin/mail}コマンド(現在では\code{binmail}や\code{v7mail}と呼ばれることもある)だが、それ以外のユーザーエージェントを使っているところもあった。たとえばバークレーの\code{Mail}は、メッセージを個別のアイテムとして扱う方法を知っていた。単に\code{cat}プログラムに任せるだけではなかった。どのユーザーエージェントも、\code{/usr/spool/mail}ディレクトリを直接読んだり(そして直接書き込んだり!)していた。実際に格納されるメッセージを抽象化することなどなかったのだ。

%% The logic to route a message to the network versus local e-mail was nothing
%% more than seeing if the address contained an exclamation point (UUCP)
%% or a colon (BerkNET). People with Arpanet access had to use a
%% completely separate mail program, which would not interoperate with
%% other networks, and which even stored local mail in a different place
%% and in a different format.
メッセージをネットワークに送るかローカルの電子メールに送るかの振り分けロジックは、単にアドレスを見て中に感嘆符(UUTPの場合)あるいはコロン(BerkNETの場合)が含まれているかどうかを見るだけのことでしかなかった。Arpanetにアクセスする人たちは、完全に個別のメールプログラムを使う必要があった。これは他のネットワークとの相互運用ができず、ローカルメールも別の場所に異なる書式で保存していた。

%% To make things even more interesting, there was virtually no
%% standardization on the format of the messages themselves.  There was
%% general agreement that there would be a block of header fields at the
%% top of the message, that each header field would be on a new line, and
%% that header field names and values would be separated by a
%% colon. Beyond that, there was very little standardization in either the
%% selection of header field names or the syntaxes of individual
%% fields. For example, some systems used \code{Subj:} instead of
%% \code{Subject:}, \code{Date:} fields were different syntaxes, and some
%% systems didn't understand full names in a \code{From:} field. On top
%% of all of this, what was documented was often ambiguous or not quite
%% what was actually in use. In particular, RFC~733 (which purported to
%% describe the format of Arpanet messages) was different from what was
%% actually used in subtle but sometimes important ways, and the method
%% of actually transmitting messages was not officially documented at all
%% (although several RFCs made reference to the mechanism, none defined
%% it). The result was that there was somewhat of a priesthood around
%% messaging systems.
さらにおもしろいのが、この時点ではまだ、メッセージ自体の書式についても事実上標準化されていなかったということだ。大まかに決まっていたのは、メッセージの先頭にヘッダフィールドを置き、各ヘッダフィールドの後には改行を置いて、フィールド名とその値はコロンで区切るということくらいだった。それ以外のこと、たとえばどんなヘッダフィールド名を使うかや個々のフィールドの構文などについてはほとんど標準化されていなかった。たとえば\code{Subject:}の代わりに\code{Subj:}を使うシステムもあったし\code{Date:}フィールドの構文もそれぞればらばらだったし、システムによっては\code{From:}フィールドのフルネームを解釈できないものもあった。さらに、ドキュメントもあいまいなものばかりだったし、ドキュメントが実際の内容と食い違っていることもあった。特にRFC~733(Arpanetメッセージについての説明)は実際に使われている内容と微妙に異なっていた(しかし重要な違いであることもあった)。そして、実際のメッセージ送信の仕組みは実際のところ公式には文書化されていなかった(その仕組みを扱うRFCはいくつかあったが、きちんとした定義はどこにもなかった)。メッセージングシステムまわりはこんなひどい状況だった。

%% In 1979, the INGRES Relational Database Management Project
%% (a.k.a. my day job)
%% got a DARPA grant, and with it a 9600bps Arpanet connection
%% to our PDP-11.
%% At the time it was the only Arpanet connection available in
%% the Computer Science Division,
%% so everyone wanted access to our machine so they could get to the Arpanet.
%% However, that machine was already maxed out,
%% and so we could only make two login ports available for everyone in the
%% department to share.
%% This caused substantial contention and frequent conflicts.
%% However, I noticed that what people wanted most of all was not
%% remote login or file transfer, but e-mail.
1979年、INGRES Relational Database Management Project(私の昼間の業務だった)はDARPAの資金援助を受け、9600bpsのArpanet接続を我々のPDP-11につなげることになった。その当時、コンピュータサービス部門でArpanetに接続しているマシンはそれだけだったので、誰もが我々のマシンを使ってArpanetを使いたがった。しかし、そのマシンはすでに最大限に使い切っており、使えるログインポートはあとふたつしか残っていなかった。そこで、そのふたつを部門内で共有した。そのため、頻繁に争いが発生した。しかし、私は気づいた。我々の大半はリモートログインやファイル転送を求めているのではなく、単に電子メールが使いたいだけだったのだ。

%% Into this, sendmail (initially called delivermail) emerged as an
%% attempt to unify the chaos into one place. Every MUA (mail user agent,
%% or mail client) would just call delivermail to deliver email rather than
%% figuring out how to do it on an ad hoc (and often incompatible) basis.
%% Delivermail/sendmail made no attempt to dictate how local mail should be stored or
%% delivered; it did absolutely nothing except shuffle mail between other
%% programs. (This changed when SMTP was added, as we'll see shortly.)
%% In some sense it was just glue to hold the various mail systems
%% together rather than being a mail system in its own right.
そんなところに、sendmail(当初はdelivermailという名前だった)が登場した。このカオスを何とか統一しようという試みだ。すべてのMUA(Mail User Agent、あるいはメールクライアント)は、単にdelivermailを呼び出すだけでメールを配送することができた。いちいち調査をしてアドホックな(そして、たいてい互換性のない)対応をする必要もない。Delivermail/sendmailは、ローカルメールの保存方法や配送方法に関しては一切関知しなかった。つまり、単に他のプログラムとの間でメールを入れ替える以外のことは何もしなかったのだ(これは、後にSMTPが追加されたことで変わった。詳しくは後ほど)。ある意味では、自分自身がメールシステムであるというよりは、さまざまなメールシステムの間をつなぐ糊に過ぎなかったのだ。

\pagebreak

%% During the development of sendmail the Arpanet was transformed into
%% the Internet. The changes were extensive, from the low level packets
%% on the wire up through application protocols, and did not happen
%% instantly.  Sendmail was literally developed concurrently with the
%% standards, and in some cases influenced them. It's also notable that
%% sendmail has survived and even thrived as ``the network'' (as we think
%% of it today) scaled from a few hundred hosts to hundreds of millions
%% of hosts.
Sendmailの開発を進めるうちに、Arpanetはインターネットに姿を変えた。その変更点は幅広く、ローレベルのパケットからその上にのるアプリケーションプロトコルまですべて変化した。そして、その変化は一斉に起こったわけではなかった。Sendmailの開発は標準規格の策定とまさに文字通り並行して進んでおり、Sendmailが標準規格に影響を及ぼすこともあった。もうひとつ特筆すべき点は、いま我々が思い浮かべる``ネットワーク''がまだ数百台規模だったころから何億台規模に成長するまでの間、sendmailがずっと生き残って成長を続けたという事実である。

%% \begin{aosabox}{Another Network}
\begin{aosabox}{もうひとつのネットワーク}

%% It's worth mentioning that another completely separate
%% mail standard was proposed at the
%% time called X.400, which was a part of ISO/OSI (International
%% Standards Organization/Open Systems Interconnect).  X.400 was a binary
%% protocol, with the message encoded using ASN.1 (Abstract Syntax
%% Notation 1), which is still in use in some Internet protocols today
%% such as LDAP\@. LDAP was in turn a simplification of X.500, which was
%% the directory service used by X.400.  Sendmail made no attempt
%% whatsoever to be directly compatible with X.400, although there were
%% some gateway services extant at the time. Although X.400 was initially
%% adopted by many of the commercial vendors at the time, Internet mail
%% and SMTP
%% ended up winning in the marketplace.
ちなみに、当時まったく別のメール標準規格も提案されていた。それはX.400と呼ばれ、ISO/OSI (International Standards Organization/Open Systems Interconnect)の一部だった。X.400はバイナリプロトコルで、メッセージはASN.1(Abstract Syntax Notation 1)で符号化する。この仕組みは、今でもLDAPなどの一部のインターネットプロトコルで使われている。LDAPはX.500を単純化したものであり、X.500はX.400が使っているディレクトリサービスだ。SendmailはX.400との直接の互換性に関しては一切考慮していない。しかし、両者の橋渡しをするゲートウェイが存在する。X.400は当初から多くの商用ベンダーに採用されていたが、最終的に市場を制したのはインターネットメールとSMTPだった。

\end{aosabox}

\end{aosasect1}

%% \begin{aosasect1}{Design Principles}
\begin{aosasect1}{設計の原則}

While developing sendmail, I adhered to several design principles. All
of these in some sense came down to one thing: do as little as
possible.  This is in sharp contrast to some of the other efforts of
the time that had much broader goals and required much larger
implementations.

%% \begin{aosasect2}{Accept that One Programmer Is Finite}
\begin{aosasect2}{プログラマには限界があることを受け入れる}

I wrote sendmail as a part-time, unpaid project. It was intended to be
a quick way of making Arpanet mail more accessible to people at
U.C. Berkeley. The key was to forward mail between existing networks,
all of which were implemented as standalone programs that were unaware
that more than one network even existed. Modifying more than a tiny
amount of the existing software was infeasible with only one part-time
programmer.  The design had to minimize the amount of existing code
that needed to be modified as well as the amount of new code that
needed to be written.  This constraint drove most of the rest of the
design principles. As it turned out, in most cases they would have
been the right thing to do even if there had been a larger team
available.

\end{aosasect2}

%% \begin{aosasect2}{Don't Redesign User Agents}
\begin{aosasect2}{ユーザーエージェントの再設計はしない}

A Mail User Agent (MUA) is what most end users think of as the ``mail
system''---it's the program that they use to read, write, and answer
mail. It is quite distinct from the Mail Transfer Agent (MTA), which
routes email from the sender to the receiver. At the time sendmail was
written, many implementations at least partly combined these two
functions, so they were often developed in tandem. Trying to work on
both at the same time would have been too much, so Sendmail completely
punted on the user interface problem: the only changes to MUAs were to
have them invoke sendmail instead of doing their own routing. In
particular, there were already several user agents, and people were
often quite emotional about how they interacted with mail.
Trying to work on both at the same time would have been too much.
This
separation of the MUA from the MTA is accepted wisdom now, but was far
from standard practice at the time.

\end{aosasect2}

%% \begin{aosasect2}{Don't Redesign the Local Mail Store}
\begin{aosasect2}{ローカルのメールストアの再設計はしない}

The local mail store (where messages would be saved until the
recipient came along to read them) was not formally standardized. Some sites
liked to store them in a centralized place, such as \code{/usr/mail},
\code{/var/mail}, or \code{/var/spool/mail}. Other sites liked to
store them in the recipient's home directory (e.g., as a file called
\code{.mail}). Most sites started each message with a line beginning
``From'' followed by a space character (an extraordinarily bad
decision, but that was the convention at the time), but sites that
were Arpanet-focused usually stored messages separated by a line
containing four control-A characters. Some sites attempted to lock the
mailbox to prevent collisions, but they used different locking
conventions (file locking primitives were not yet available). In
short, the only reasonable thing to do was treat local mail storage as
a black box.

On nearly all sites, the actual mechanism for doing local mailbox
storage was embodied in the \code{/bin/mail} program. This had a
(quite primitive) user interface, routing, and storage built into one
program. To incorporate sendmail, the routing portion was pulled out
and replaced with a call to sendmail.  A \code{-d} flag was added to
force final delivery, i.e., it prevented \code{/bin/mail} from calling
sendmail to do the routing. In later years the code used to deliver a
message to a physical mailbox was extracted into another program
called \code{mail.local}. The \code{/bin/mail} program exists today
only to include a lowest common denominator for scripts to send
mail.

\end{aosasect2}

%% \begin{aosasect2}{Make Sendmail Adapt to the World, Not the Other Way Around}
\begin{aosasect2}{Sendmailが世界に合わせるのであって、世界をSendmailに合わせるのではない}

Protocols such as UUCP and BerkNET were already implemented as
separate programs that had their own, sometimes quirky, command line
structure. In some cases they were being actively developed at the
same time as sendmail. It was clear that reimplementing them (for
example, to convert them to standard calling conventions) was going to
be painful. This led directly to the principle that sendmail should
adapt to the rest of the world rather than trying to make the rest of
the world adapt to sendmail.

\end{aosasect2}

%% \begin{aosasect2}{Change as Little as Possible}
\begin{aosasect2}{変更はできる限り少なく}

To the fullest extent possible, during the development of sendmail I
didn't touch anything I didn't absolutely have to touch. Besides just
not having enough time to do it, there was a culture at Berkeley at
the time that eschewed most formal code ownership in favor of a policy
of ``the last person who touched the code is the go-to person for that
program'' (or more simply, ``you touch it, you own it''). Although
that sounds chaotic by most modern-day standards, it worked quite well
in a world where no one at Berkeley was assigned full time to work on
Unix; individuals worked on parts of the system that they were
interested in and committed to and didn't touch the rest of the code
base except in dire circumstances.

\end{aosasect2}

%% \begin{aosasect2}{Think About Reliability Early}
\begin{aosasect2}{信頼性をまず考える}

The mail system prior to sendmail (including most of the transport
systems) wasn't terribly concerned about reliability. For example,
versions of Unix prior to 4.2BSD did not have native file locking,
although it could be simulated by creating a temporary file and then
linking it to a lock file (if the lock file already existed the link
call would fail). However, sometimes different programs writing the
same data file wouldn't agree on how the locking should be done (for
example, they might use a different lock file name or even make no attempt
to do locking at all), and so it wasn't
that uncommon to lose mail.  Sendmail took the approach that losing
mail wasn't an option (possibly a result of my background as a
database guy, where losing data is a mortal sin).

\end{aosasect2}

%% \begin{aosasect2}{What Was Left Out}
\begin{aosasect2}{その他}

There were many things that were not done in the early versions.
I did not try to re-architect the mail system
or build a completely general solution:
functionality could be added as the need arose.
Very early versions were not even intended to be completely configurable
without access to the source code and a compiler
(although this changed fairly early on).
In general, the modus operandi for sendmail
was to get something working quickly
and then enhance working code as needed
and as the problem was better understood.

\end{aosasect2}

\end{aosasect1}

%% \begin{aosasect1}{Development Phases}
\begin{aosasect1}{開発フェーズ}

Like most long-lived software, sendmail was developed in phases, each
with its own basic theme and feeling.

%% \begin{aosasect2}{Wave 1: delivermail}
\begin{aosasect2}{第一波: delivermail}

The first instantiation of sendmail was known as delivermail. It was
extremely simple, if not simplistic. Its sole job was to forward mail
from one program to another; in particular, it had no SMTP support,
and so never made any direct network connections. No queuing
was necessary because each network already had its own queue, so the
program was really just a crossbar switch. Since delivermail had no
direct network protocol support, there was no reason for it to run as
a daemon---it would be invoked to route each message as it was
submitted, pass it to the appropriate program that would implement the
next hop, and terminate.  Also, there was no attempt to rewrite
headers to match the network to which a message was being
delivered. This commonly resulted in messages being forwarded that
could not be replied to. The situation was so bad that an entire book
was written about addressing mail (called, fittingly,
\emph{!\%\@:: A Directory of Electronic Mail Addressing \& Networks}
\cite{bib:adams:email}).

All configuration in delivermail was compiled in and was based only on
special characters in each address.  The characters had
precedence. For example, a host configuration might search for an
``@'' sign and, if one was found, send the entire address to a
designated Arpanet relay host. Otherwise, it might search for a colon,
and send the message to BerkNET with the designated host and user if
it found one, then could check for an exclamation point (``!'')
signalling that the message should be forwarded to a designated UUCP
relay.  Otherwise it would attempt local delivery. This configuration
might result in the following:

\begin{table}[h!]\centering
\begin{tabular}{|ll|}
\hline
Input & Sent To {net, host, user} \\
\hline
foo@bar & \{Arpanet, bar, foo\} \\
foo:bar & \{Berknet, foo, bar\} \\
foo!bar!baz & \{Uucp, foo, bar!baz\} \\
foo!bar@baz & \{Arpanet, baz, foo!bar\} \\
\hline
\end{tabular}
\end{table}

\noindent
Note that address delimiters differed in their associativity,
resulting in ambiguities that could only be resolved using heuristics.
For example, the last example might reasonably be parsed as
\{Uucp, foo, bar@baz\} 
at another site.

The configuration was compiled in for several reasons: first, with a
16 bit address space and limited memory, parsing a runtime
configuration was too expensive. Second, the systems of the time had
been so highly customized that recompiling was a good idea, just to
make sure you had the local versions of the libraries (shared
libraries did not exist with Unix 6th Edition).

Delivermail was distributed with 4.0 and 4.1 BSD and was more
successful than expected; Berkeley was far from the only site with
hybrid network architectures.  It became clear that more work was
required.

\end{aosasect2}

%% \begin{aosasect2}{Wave 2: sendmail 3, 4, and 5}
\begin{aosasect2}{第二波: sendmail 3, 4, そして 5}

Versions 1 and 2 were distributed under the delivermail name. In March
1981 work began on version 3, which would be distributed under the
sendmail name. At this point the 16-bit PDP-11 was still in common use
but the 32-bit VAX-11 was becoming popular, so many of the original
constraints associated with small address spaces were starting to be
relaxed.

The initial goals of sendmail were to convert to runtime
configuration, allow message modification to provide compatibility
across networks for forwarded mail, and have a richer language on
which to make routing decisions. The technique used was essentially
textual rewriting of addresses (based on tokens rather than
character strings),
a mechanism used in some expert systems at the time. There was ad hoc
code to extract and save any comment strings (in parentheses) as well
as to re-insert them after the programmatic rewriting completed. It
was also important to be able to add or augment header fields (e.g.,
adding a \code{Date} header field or including the full name of the
sender in the \code{From} header if it was known).

SMTP development started in November 1981.
The Computer Science Research Group (CSRG) at U.C. Berkeley had gotten
the DARPA contract to produce a Unix-based platform to support DARPA
funded research, with the intent of making sharing between projects
easier.  The initial work on the TCP/IP stack was done by that time,
although the details of the socket interface were still changing.
Basic application protocols such as Telnet and FTP were done, but SMTP
had yet to be implemented. In fact, the SMTP protocol wasn't even
finalized at that point; there had been a huge debate about how mail
should be sent using a protocol to be creatively named Mail Transfer
Protocol (MTP). As the debate raged, MTP got more and more complex
until in frustration SMTP (Simple Mail Transfer Protocol) was drafted
more-or-less by fiat (but not officially published until August 1982).
Officially, I was working on the INGRES Relational Database Management
System, but since I knew more about the mail system than anyone else
around Berkeley at the time, I got talked into implementing SMTP.

My initial thought was to create a separate SMTP mailer that would
have its own queueing and daemon; that subsystem would attach to
sendmail to do the routing. However, several features of SMTP made
this problematic. For example, the \code{EXPN} and \code{VRFY} commands required
access to the parsing, aliasing, and local address verification
modules. Also, at the time I thought it was important that the \code{RCPT}
command return immediately if the address was unknown, rather than
accepting the message and then having to send a delivery failure
message later. This turns out to have been a prescient
decision. Ironically, later MTAs often got this wrong, exacerbating
the spam backscatter problem. These issues drove the decision to
include SMTP as part of sendmail itself.

Sendmail 3 was distributed with 4.1a and 4.1c BSD (beta versions),
sendmail 4 was distributed with 4.2 BSD, and sendmail 5 was
distributed with 4.3 BSD.

\end{aosasect2}

%% \begin{aosasect2}{Wave 3: The Chaos Years}
\begin{aosasect2}{第三波: カオスな日々}

After I left Berkeley and went to a startup company, my time available
to work on sendmail rapidly decreased. But the Internet was starting
to seriously explode and sendmail was being used in a variety of new
(and larger) environments. Most of the Unix system vendors (Sun, DEC,
and IBM in particular) created their own versions of sendmail, all of
which were mutually incompatible. There were also attempts to build
open source versions, notably IDA sendmail and KJS.

IDA sendmail came from Link\"{o}ping University.  IDA included extensions
to make it easier to install and manage in larger environments and a
completely new configuration system. One of the major new features was
the inclusion of dbm(3) database maps to support highly dynamic
sites. These were available using a new syntax in the configuration
file and were used for many functions including mapping of addresses
to and from external syntax (for example, sending out mail as
john\_doe@example.com instead of johnd@example.com) and routing.

King James Sendmail (KJS, produced by Paul Vixie) was an attempt to
unify all the various versions of sendmail that had sprung
up. Unfortunately, it never really got enough traction to have the
desired effect.  This era was also driven by a plethora of new
technologies that were reflected in the mail system. For example,
Sun's creation of diskless clusters added the YP (later NIS) directory
services and NFS, the Network File System.  In particular, YP had to be
visible to sendmail,
since aliases were stored in YP rather than in local files.

\end{aosasect2}

%% \begin{aosasect2}{Wave 4: sendmail 8}
\begin{aosasect2}{第四波: sendmail 8}

After several years, I returned to Berkeley as a staff member. My job
was to manage a group installing and supporting shared infrastructure
for research around the Computer Science department. For that to
succeed, the largely ad hoc environments of individual research groups
had to be unified in some rational way. Much like the early days of
the Internet, different research groups were running on radically
different platforms, some of which were quite old. In general, every
research group ran its own systems, and although some of them were
well managed, most of them suffered from ``deferred maintenance.''

In most cases email was similarly fractured. Each person's email
address was ``\path{person@host.berkeley.edu}'', where \code{host} was
the name of the workstation in their office or the shared server they
used (the campus didn't even have internal subdomains) with the
exception of a few special people who had \code{@berkeley.edu} addresses. The
goal was to switch to internal subdomains (so all individual hosts
would be in the \code{cs.berkeley.edu} subdomain) and have a unified
mail system (so each person would have an \code{@cs.berkeley.edu}
address). This goal was most easily realized by creating a new version
of sendmail that could be used throughout the department.

I began by studying many of the variants of sendmail that had become
popular.  My intent was not to start from a different code base but
rather to understand the functionality that others had found useful.
Many of those ideas found their way into sendmail 8, often with
modifications to merge related ideas or make them more generic. For
example, several versions of sendmail had the ability to access
external databases such as dbm(3) or NIS; sendmail 8 merged these into
one ``map'' mechanism that could handle multiple types of databases
(and even arbitrary non-database transformations).  Similarly, the
``generics'' database (internal to external name mapping)
from IDA sendmail was incorporated.

Sendmail 8 also included a new configuration package using the m4(1)
macro processor.
This was
intended to be more declarative than the sendmail 5 configuration
package, which had been largely procedural. That is, the sendmail 5
configuration package required the administrator to essentially lay
out the entire configuration file by hand, really only using the
``include'' facility from m4 as shorthand. The sendmail 8
configuration file allowed the administrator to just declare what
features, mailers, and so on were required, and m4 laid out the final
configuration file.

Much of \aosasecref{sec.sendmail.evo} discusses the enhancements in
sendmail 8.

\end{aosasect2}

%% \begin{aosasect2}{Wave 5: The Commercial Years}
\begin{aosasect2}{第五波: 商売の日々}

As the Internet grew and the number of sendmail sites expanded,
support for the ever larger user base became more problematic.  For a
while I was able to continue support by setting up a group of
volunteers (informally called the ``Sendmail Consortium'',
a.k.a.\ sendmail.org) who provided free support via e-mail and
newsgroup.  But by the late 1990s, the installed base had grown to such
an extent that it was nearly impossible to support it on a volunteer
basis.  Together with a more business-savvy friend I founded Sendmail,
Inc.\footnote{\url{http://www.sendmail.com}},
with the expectation of getting new resources to bear on the
code.

Although the commercial product was originally based largely on
configuration and management tools, many new features were added to
the open-source MTA to support the needs of the commercial world.
Notably, the company added support for TLS (connection encryption),
SMTP Authentication, site security enhancements such as Denial of
Service protection, and most importantly mail filtering plugins (the
Milter interface discussed below).

At of this writing the commercial product has expanded to include a large
suite of e-mail based applications, nearly all of which are constructed
on the extensions added to sendmail during the first few years of the
company.

\end{aosasect2}

%% \begin{aosasect2}{Whatever Happened to sendmail 6 and 7?}
\begin{aosasect2}{sendmail 6や7に何が起こったのか?}

Sendmail 6 was essentially the beta for sendmail 8. It was never
officially released, but was distributed fairly widely. Sendmail 7
never existed at all; sendmail jumped directly to version 8 because
all the other source files for the BSD distribution were bumped to
version 8 when 4.4 BSD was released in June 1993.

\end{aosasect2}

\end{aosasect1}

%% \begin{aosasect1}{Design Decisions}
\begin{aosasect1}{設計に関する決定}

Some design decisions were right. Some started out right and became
wrong as the world changed. Some were dubious and haven't become any
less so.

%% \begin{aosasect2}{The Syntax of the Configuration File}
\begin{aosasect2}{設定ファイルの構文}

The syntax of the configuration file was driven by a couple of issues.
First, the entire application had to fit into a 16-bit address space,
so the parser had to be small. Second, early configurations were quite
short (under one page), so while the syntax was obscure, the file was
still comprehensible.  However, as time passed, more operational
decisions moved out of the C code into the configuration file, and the
file started to grow. The configuration file acquired a reputation for
being arcane. One particular frustration for many people was the
choice of the tab character as an active syntax item. This was a
mistake that was copied from other systems of the time, notably
\code{make}. That particular problem became more acute as window
systems (and hence cut-and-paste, which usually did not preserve the
tabs) became available.

In retrospect, as the file got larger and 32-bit machines took over,
it would have made sense to reconsider the syntax. There was a time
when I thought about doing this but decided against it because I
didn't want to break the ``large'' installed base (which at that point
was probably a few hundred machines). In retrospect this was a
mistake; I had simply not appreciated how large the install base would
grow and how many hours it would save me had I changed the syntax
early.  Also, when the standards stabilized a fair amount of the
generality could have been pushed back into the C code base, thus
simplifying the configurations.

Of particular interest was how more functionality got moved into the
configuration file. I was developing sendmail at the same time as the
SMTP standard was evolving. By moving operational decisions into the
configuration file I was able to respond rapidly to design
changes---usually in under 24 hours. I believe that this improved the
SMTP standard, since it was possible to get operational experience
with a proposed design change quite quickly, but only at the cost of
making the configuration file difficult to understand.

\end{aosasect2}

%% \begin{aosasect2}{Rewriting Rules}
\begin{aosasect2}{ルールの書き換え}

One of the difficult decisions when writing sendmail was how to do the
necessary rewriting to allow forwarding between networks without
violating the standards of the receiving network. The transformations
required changing metacharacters (for example, BerkNET used colon as a
separator, which was not legal in SMTP addresses), rearranging address
components, adding or deleting components, etc. For example, the
following rewrites would be needed under certain circumstances:

\begin{table}[h!]\centering
\begin{tabular}{|ll|}
\hline
From & To \\
\hline
a:foo & a.foo@berkeley.edu \\
a!b!c & b!c@a.uucp \\
{\textless}@a.net,@b.org:user@c.com{\textgreater} & {\textless}@b.org:user@c.com{\textgreater} \\
\hline
\end{tabular}
\end{table}

Regular expressions were not a good choice because they didn't have
good support for word boundaries, quoting, etc. It quickly became
obvious that it would be nearly impossible to write regular
expressions that were accurate, much less intelligible. In particular,
regular expressions reserve a number of metacharacters, including
``.'', ``*'', ``+'', ``{[}'', and ``{]}'', all of which can appear in
e-mail addresses. These could have been escaped in configuration files,
but I deemed that to be complicated, confusing, and a bit
ugly. (This was tried by UPAS from Bell Laboratories, the mailer for
Unix Eighth Edition, but it never caught
on\footnote{\url{http://doc.cat-v.org/bell_labs/upas_mail_system/upas.pdf}}.)
Instead, a scanning phase was necessary to produce tokens that could
then be manipulated much like characters in regular expressions. A
single parameter describing ``operator characters'', which were
themselves both tokens and token separators, was sufficient. Blank
spaces separated tokens but were not tokens themselves. The rewriting
rules were just pattern match/replace pairs organized into what were
essentially subroutines.

Instead of a large number of metacharacters that had to be escaped to
lose their ``magic'' properties (as used in regular expressions), I
used a single ``escape'' character that combined with ordinary
characters to represent wildcard patterns (to match an arbitrary word,
for example).  The traditional Unix approach would be to use
backslash, but backslash was already used as a quote character in some
address syntaxes. As it turned out, ``\$'' was one of the few
characters that had not already been used as a punctuation character
in some email syntax.

One of the original bad decisions was, ironically, just a matter of how white
space was used. A space character was a separator, just as in most
scanned input, and so could have been used freely between tokens in
patterns. However, the original configuration files distributed did
not include spaces, resulting in patterns that were far harder to
understand than necessary. Consider the difference between the
following two (semantically identical) patterns:

\begin{verbatim}
$+ + $* @ $+ . $={mydomain}
$++$*@$+.$={mydomain}
\end{verbatim}

\end{aosasect2}

%% \begin{aosasect2}{Using Rewriting for Parsing}
\begin{aosasect2}{パース用の書き換えの使用}

Some have suggested that sendmail should have used conventional
grammar-based parsing techniques to parse addresses rather than
rewriting rules and leave the rewriting rules for address modification.
On the surface this would seem to make sense, given
that the standards define addresses using a grammar.  The main reason
for reusing rewriting rules is that in some cases it was necessary to
parse header field addresses (e.g., in order to extract the sender
envelope from a header when receiving mail from a network that didn't
have a formal envelope). Such addresses aren't easy to parse using
(say) an LALR(1) parser such as YACC and a traditional scanner because
of the amount of lookahead required. For example, parsing the address:
\code{allman@foo.bar.baz.com
  {\textless}eric@example.com{\textgreater}} requires lookahead by
either the scanner or the parser; you can't know that the initial
``allman@\ldots'' is not an address until you see the
``{\textless}''.
Since LALR(1) parsers only have one token of lookahead
this would have had to be done in the scanner,
which would have complicated it substantially.
Since the rewriting rules already had arbitrary backtracking
(i.e., they could look ahead arbitrarily far),
they were sufficient.

A secondary reason was that it was relatively easy to make the
patterns recognize and fix broken input. Finally, rewriting was more
than powerful enough to do the job, and reusing any code was wise.

One unusual point about the rewriting rules: when doing the pattern
matching, it is useful for both the input and the pattern to be
tokenized. Hence, the same scanner is used for both the input
addresses and the patterns themselves. This requires that the scanner
be called with different character type tables for differing input.

\end{aosasect2}

%% \begin{aosasect2}{Embedding SMTP and Queueing in sendmail}
\begin{aosasect2}{SMTPやキューのsendmailへの埋め込み}

An ``obvious'' way to implement outgoing (client) SMTP would have been
to build it as an external mailer, similarly to UUCP,  But this would
raise a number of other questions. For example, would queueing be done
in sendmail or in the SMTP client module? If it was done in sendmail
then either separate copies of messages would have to be sent to each
recipient (i.e., no ``piggybacking'', wherein a single connection can
be opened and then multiple \code{RCPT} commands can be sent) or a much
richer communication back-path would be necessary to convey the
necessary per-recipient status than was possible using simple Unix
exit codes. If queueing was done in the client module then there was a
potential for large amounts of replication; in particular, at the time
other networks such as XNS were still possible
contenders. Additionally, including the queue into sendmail itself
provided a more elegant way of dealing with certain kinds of failures,
notably transient problems such as resource exhaustion.

Incoming (server) SMTP involved a different set of decisions. At the
time, I felt it was important to implement the \code{VRFY} and
\code{EXPN} SMTP commands faithfully, which required access to the
alias mechanism.  This would once again require a much richer protocol
exchange between the server SMTP module and sendmail than was possible
using command lines and exit codes---in fact, a protocol akin to SMTP
itself.

I would be much more inclined today to leave queueing in the core
sendmail but move both sides of the SMTP implementation into other
processes. One reason is to gain security: once the server side has an
open instance of port 25 it no longer needs access to root
permissions. Modern extensions such as TLS and DKIM signing complicate
the client side (since the private keys should not be accessible to
unprivileged users), but strictly speaking root access is still not
necessary. Although the security issue is still an issue here, if the
client SMTP is running as a non-root user who can read the private
keys, that user by definition has special privileges, and hence should
not be communicating directly with other sites. All of these issues
can be finessed with a bit of work.

\end{aosasect2}

%% \begin{aosasect2}{The Implementation of the Queue}
\begin{aosasect2}{キューの実装}

Sendmail followed the conventions of the time for storing queue files.
In fact, the format used is extremely similar to the lpr subsystem of
the time.  Each job had two files, one with the control information
and one with the data. The control file was a flat text file with the
first character of each line representing the meaning of that line.

When sendmail wanted to process the queue it had to read all of the
control files, storing the relevant information in memory, and then
sort the list. That worked fine with a relatively small number of
messages in the queue, but started to break down at around 10,000
queued messages. Specifically, when the directory got large enough to
require indirect blocks in the filesystem, there was a serious
performance knee that could reduce performance by as much as an order
of magnitude. It was possible to ameliorate this problem by having
sendmail understand multiple queue directories, but that was at best a
hack.

An alternative implementation might be to store all the control files
in one database file. This wasn't done because when sendmail coding
began there was no generally available database package, and when
dbm(3) became available it had several flaws, including the inability
to reclaim space, a requirement that all keys that hashed together fit
on one (512 byte) page, and a lack of locking. Robust database
packages didn't appear for many years.

Another alternative implementation would have been to have a separate
daemon that would keep the state of the queue in memory, probably
writing a log to allow recovery. Given the relatively low email
traffic volumes of the time, the lack of memory on most machines,
the relatively high cost of background processes,
and the complexity of implementing such a process,
this didn't seem
like a good tradeoff at the time.

Another design decision was to store the message header in the queue
control file rather than the data file. The rationale was that most
headers needed considerable rewriting that varied from destination to
destination (and since messages could have more than one destination,
they would have to be customized multiple times), and the cost of
parsing the headers seemed high, so storing them in a pre-parsed
format seemed like a savings. In retrospect this was not a good decision,
as was storing the message body in Unix-standard format (with newline
endings) rather than in the format in which it was received (which
could use newlines, carriage-return/line-feed, bare carriage-return,
or line-feed/carriage-return). As the e-mail world evolved and
standards were adopted, the need for rewriting diminished, and even
seemingly innocuous rewriting has the risk of error.

\end{aosasect2}

%% \begin{aosasect2}{Accepting and Fixing Bogus Input}
\begin{aosasect2}{誤入力の受け入れと修正}

Since sendmail was created in a world of multiple protocols and
disturbingly few written standards, I decided to clean up malformed
messages wherever possible.
This matches the ``Robustness Principle'' (a.k.a. Postel's Law)
articulated in RFC~793\footnote{``Be conservative in what you do,
be liberal in what you accept from others''}.
Some of these changes were obvious and even required:
when sending a
UUCP message to the Arpanet, the UUCP addresses needed to be converted
to Arpanet addresses, if only to allow ``reply'' commands to work
correctly,
line terminations needed to be converted between the conventions used
by various platforms, and so on.
Some were less obvious: if a message was received that did
not include a \code{From:} header field required in the Internet
specifications, should you add a \code{From:} header field, pass the
message on without the \code{From:} header field, or reject the message?  At
the time, my prime consideration was interoperability, so sendmail
patched the message, e.g., by adding the \code{From:} header
field. However, this is claimed to have allowed other broken mail
systems to be perpetuated long past the time when they should have
been fixed or killed off.

I believe my decision was correct for the time, but is problematic
today. A high degree of interoperability was important to let mail
flow unimpeded. Had I rejected malformed messages, most messages at
the time would have been rejected. Had I passed them through unfixed,
recipients would have received messages that they couldn't reply to
and in some cases couldn't even determine who sent the message---that
or the message would have been rejected by another mailer.

Today the standards are written, and for the most part those standards
are accurate and complete. It is no longer the case that most messages
would be rejected, and yet there is still mail software out there that
send out mangled messages. This unnecessarily creates numerous
problems for other software on the Internet.

\end{aosasect2}

%% \begin{aosasect2}{Configuration and the Use of M4}
\begin{aosasect2}{M4の設定と使用}

For a period I was both making regular changes to the sendmail
configuration files and personally supporting many machines. Since a
large amount of the configuration file was the same between different
machines, the use of a tool to build the configuration files was
desirable. The m4 macro processor was included with Unix. It was
designed as a front end for programming languages (notably
ratfor). Most importantly, it had ``include'' capabilities, like
``\#include'' in the C language.
The original configuration files used little
more than this capability and some minor macro expansions.

IDA sendmail also used m4, but in a dramatically different way. In
retrospect I should have probably studied these prototypes in more
detail. They contained many clever ideas, in particular the way they
handled quoting.

Starting with sendmail 6, the m4 configuration files were completely
rewritten to be in a more declarative style and much smaller. This
used considerably more of the power of the m4 processor, which was
problematic when the introduction of GNU m4 changed some of
the semantics in subtle ways.

The original plan was that the m4
configurations would follow the 80/20 rule: they would be simple
(hence 20\% of the work), and would cover 80\% of the cases. This broke
down fairly quickly, for two reasons.
The minor reason was that it turned out to be relatively easy to
handle the vast majority of the cases, at least in the beginning. It
became much harder as sendmail and the world evolved, notably with the
inclusion of features such as TLS encryption and SMTP Authentication,
but those didn't come until quite a bit later.

The important reason was that it was becoming clear that the raw
configuration file was just too difficult for most people to
manage. In essence, the \code{.cf} (raw) format had become assembly
code---editable in principle, but in reality quite opaque. The
``source code'' was an m4 script stored in the \code{.mc} file.

Another important distinction is that the raw format configuration
file was really a programming language. It had procedural code
(rulesets), subroutine calls, parameter expansion, and loops (but no
gotos). The syntax was obscure, but in many ways resembled
the \code{sed} and \code{awk} commands, at
least conceptually. The m4 format was declarative: although it was
possible to drop into the low-level raw language, in practice these
details were hidden from the user.

It isn't clear that this decision was correct or incorrect.
I felt at the time (and still feel) that with complex systems
it can be useful to implement what amounts to a Domain Specific Language (DSL)
for building certain portions of that system.
However, exposing that DSL to end users as a configuration methodology
essentially converts all attempts to configure a system
into a programming problem.
Great power results from this, but at a non-trivial cost.

\end{aosasect2}

\end{aosasect1}

%% \begin{aosasect1}{Other Considerations}
\begin{aosasect1}{その他の検討事項}

Several other architectural and development points deserve to be
mentioned.

%% \begin{aosasect2}{A Word About Optimizing Internet Scale Systems}
\begin{aosasect2}{インターネットにスケールするシステムの最適化に関して一言}

In most network-based systems there is a tension between the client
and the server. A good strategy for the client may be the
wrong thing for the server and vice versa. For example, when possible
the server would like to minimize its processing costs by pushing as
much as possible back to the client, and of course the client feels
the same way but in the opposite direction. For example, a server
might want to keep a connection open while doing spam processing since
that lowers the cost of rejecting a message (which these days is the
common case), but the client wants to move on as quickly as
possible. Looking at the entire system, that is, the Internet as a
whole, the optimum solution may be to balance these two needs.

There have been cases of MTAs that have used strategies that
explicitly favor either the client or the server. They can do this
only because they have a relatively small installed base.  When your
system is used on a significant portion of the Internet you have to
design it in order to balance the load between both sides in an
attempt to optimize the Internet as a whole. This is complicated by
the fact that there will always be MTAs completely skewed in one
direction or the other---for example, mass mailing systems only care
about optimizing the outgoing side.

When designing a system that incorporates both sides of the
connection, it is important to avoid playing favorites. Note that this
is in stark contrast to the usual asymmetry of clients and
services---for example, web servers and web clients are generally not
developed by the same groups.

\end{aosasect2}

%% \begin{aosasect2}{Milter}
\begin{aosasect2}{Milter}

One of the most important additions to sendmail was the milter
(\emph{m}ail f\emph{ilter}) interface.  Milter allows for the use of
offboard plugins (i.e., they run in a separate process) for mail
processing. These were originally designed for anti-spam
processing. The milter protocol runs synchronously with the server
SMTP protocol. As each new SMTP command is received from the client,
sendmail calls the milters with the information from that command. The
milter has the opportunity to accept the command or send a rejection,
which rejects the phase of the protocol appropriate for the SMTP
command. Milters are modeled as callbacks, so as an SMTP command comes
in, the appropriate milter subroutine is called. Milters are threaded,
with a per-connection context pointer handed in to each routine to
allow passing state.

In theory milters could work as loadable modules in the sendmail
address space. We declined to do this for three reasons.  First, the
security issues were too significant: even if sendmail were running as
a unique non-root user id, that user would have access to all of the
state of other messages.  Similarly, it was inevitable that some milter
authors would try to access internal sendmail state.

Second, we wanted to create a firewall between sendmail and the
milters: if a milter crashed, we wanted it to be clear who was at
fault, and for mail to (potentially) continue to flow. Third, it was
much easier for a milter author to debug a standalone process than
sendmail as a whole.

It quickly became clear that the milter was useful for more than
anti-spam processing. In fact, the milter.org\footnote{\url{http://milter.org}}
web site lists milters
for anti-spam, anti-virus, archiving, content monitoring, logging,
traffic shaping, and many other categories, produced by commercial
companies and open source projects. The postfix
mailer\footnote{\url{http://postfix.org}}
has added
support for milters using the same interface. Milters have proven to
be one of sendmail's great successes.

\end{aosasect2}

%% \begin{aosasect2}{Release Schedules}
\begin{aosasect2}{リリーススケジュール}

There is a popular debate between ``release early and often'' and
``release stable systems'' schools of thought. Sendmail has used both
of these at various times. During times of considerable change I was
sometimes doing more than one release a day. My general philosophy was
to make a release after each change. This is similar to providing
public access to the source management system tree. I personally
prefer doing releases over providing public source trees, at least in
part because I use source management in what is now considered an
unapproved way: for large changes, I will check in non-functioning
snapshots while I am writing the code. If the tree is shared I will
use branches for these snapshots, but in any case they are available
for the world to see and can create considerable confusion. Also,
creating a release means putting a number on it, which makes it easier
to track the changes when going through a bug report.
Of course, this requires that releases be easy to generate,
which is not always true.

As sendmail became used in ever more critical production environments
this started to become problematic. It wasn't always easy for others
to tell the difference between changes that I wanted out there for
people to test versus changes that were really intended to be used in
the wild. Labeling releases as ``alpha'' or ``beta'' alleviates but
does not fix the problem. The result was that as sendmail matured it
moved toward less frequent but larger releases. This became especially
acute when sendmail got folded into a commercial company which had
customers who wanted both the latest and greatest but also only stable
versions, and wouldn't accept that the two are incompatible.

This tension between open source developer needs and commercial
product needs will never go away. There are many advantages to
releasing early and often, notably the potentially huge audience of
brave (and sometimes foolish) testers who stress the system in ways
that you could almost never expect to reproduce in a standard
development system. But as a project becomes successful it tends to
turn into a product (even if that product is open source and free),
and products have different needs than projects.

\end{aosasect2}

\end{aosasect1}

%% \begin{aosasect1}{Security}
\begin{aosasect1}{セキュリティ}

Sendmail has had a tumultuous life, security-wise. Some of this is
well deserved, but some not, as our concept of ``security'' changed
beneath us. The Internet started out with a user base of a few
thousand people, mostly in academic and research settings. It was, in
many ways, a kinder, gentler Internet than we know today. The network
was designed to encourage sharing, not to build firewalls (another
concept that did not exist in the early days). The net is now a
dangerous, hostile place, filled with spammers and crackers.
Increasingly it is being described as a war zone, and in war zones
there are civilian casualties.

It's hard to write network servers securely, especially when the
protocol is anything beyond the most simple. Nearly all programs have
had at least minor problems; even common TCP/IP implementations have
been successfully attacked. Higher-level implementation languages have
proved no panacea, and have even created vulnerabilities of their
own. The necessary watch phrase is ``distrust all input,'' no matter
where it comes from. Distrusting input includes secondary input, for
example, from DNS servers and milters. Like most early network
software, sendmail was far too trusting in its early versions.

But the biggest problem with sendmail was that early versions ran with
root permissions. Root permission is needed in order to open the SMTP
listening socket, to read individual users' forwarding information,
and to deliver to individual users' mailboxes and home
directories. However, on most systems today the concept of a mailbox
name has been divorced from the concept of a system user, which
effectively eliminates the need for root access except to open the
SMTP listening socket. Today sendmail has the ability to give up root
permissions before it processes a connection, eliminating this concern
for environments that can support it. It's worth noting that on those
systems that do not deliver directly to users' mailboxes, sendmail can
also run in a chrooted environment, allowing further permission
isolation.

Unfortunately, as sendmail gained a reputation for poor security, it
started to be blamed for problems that had nothing to do with
sendmail.  For example, one system administrator made his \code{/etc}
directory world writable and then blamed sendmail when someone
replaced the \code{/etc/passwd} file. It was incidents like this that
caused us to tighten security substantially, including explicitly
checking the ownerships and modes on files and directories that
sendmail accesses. These were so draconian that we were obliged to
include the \code{DontBlameSendmail} option to (selectively) turn off
these checks.

There are other aspects of security that are not related to protecting
the address space of the program itself. For example, the rise of spam
also caused a rise in address harvesting. The \code{VRFY} and \code{EXPN} commands
in SMTP were designed specifically to validate individual addresses
and expand the contents of mailing lists respectively. These have been
so badly abused by spammers that most sites now turn them off
entirely. This is unfortunate, at least with \code{VRFY}, as this command was
sometimes used by some anti-spam agents to validate the purported
sending address.

Similarly, anti-virus protection was once seen as a desktop problem,
but rose in importance to the point where any commercial-grade MTA had
to have anti-virus checking available.
Other security-related requirements in modern settings
include mandatory encryption of sensitive data,
data loss protection,
and enforcement of regulatory requirements,
for example, for HIPPA.

One of the principles that sendmail took to heart early on was
reliability---every message should either be delivered or reported
back to the sender. But the problem of joe-jobs (attackers forging
the return address on a message, viewed by many as a security issue)
has caused many sites to turn off the
creation of bounce messages. If a failure can be determined while
the SMTP connection is still open, the server can report the problem
by failing the command, but after the SMTP connection is closed an
incorrectly addressed message will silently disappear. To be fair,
most legitimate mail today is single hop, so problems will be
reported, but at least in principle the world has decided that
security wins over reliability.

\end{aosasect1}

%% \begin{aosasect1}{Evolution of Sendmail}
\begin{aosasect1}{Sendmailの進化}
\label{sec.sendmail.evo}

Software doesn't survive in a rapidly changing environment without
evolving to fit the changing environment. New hardware technologies
appear, which push changes in the operating system, which push changes
in libraries and frameworks, which push changes in applications. If an
application succeeds, it gets used in ever more problematic
environments.  Change is inevitable; to succeed you have to accept and
embrace change.  This section describes some of the more important
changes that have occurred as sendmail evolved.

%% \begin{aosasect2}{Configuration Became More Verbose}
\begin{aosasect2}{設定はより冗長に}

The original configuration of sendmail was quite terse. For example,
the names of options and macros were all single characters. There were
three reasons for this. First, it made parsing very simple (important in
a 16-bit environment). Second, there weren't very many options, so it
wasn't hard to come up with mnemonic names.
Third,
the single character convention was already established
with command-line flags.

Similarly, rewriting rulesets were originally numbered instead of
named. This was perhaps tolerable with a small number of rulesets, but
as their number grew it became important that they have more mnemonic
names.

As the environment in which sendmail operated became more complex, and
as the 16-bit environment faded away, the need for a richer
configuration language became evident. Fortunately, it was possible to
make these changes in a backward compatible way. These changes
dramatically improved the understandability of the configuration file.

\end{aosasect2}

%% \begin{aosasect2}{More Connections with Other Subsystems: Greater Integration}
\begin{aosasect2}{他のサブシステムとの接続: さらなる統合}

When sendmail was written the mail system was largely isolated from
the rest of the operating system.  There were a few services that
required integration, e.g., the \code{/etc/passwd} and
\code{/etc/hosts} files.  Service switches had not been invented,
directory services were nonexistent, and configuration was small and
hand-maintained.

That quickly changed. One of the first additions was DNS\@. Although the
system host lookup abstraction (\code{gethostbyname}) worked for
looking up IP addresses, email had to use other queries such as
MX\@. Later, IDA sendmail included an external database lookup
functionality using dbm(3) files.  Sendmail 8 updated that to a
general mapping service that allowed other database types, including
external databases and internal transformations that could not be done
using rewriting (e.g., dequoting an address).

Today, the email system relies on many external services that are, in
general, not designed specifically for the exclusive use of email.
This has moved sendmail toward more abstractions in the code. It has
also made maintaining the mail system more difficult as more ``moving
parts'' are added.

\end{aosasect2}

%% \begin{aosasect2}{Adaptation to a Hostile World}
\begin{aosasect2}{敵対勢力との順応}

Sendmail was developed in a world that seems completely foreign by
today's standards. The user population on the early network were
mostly researchers who were relatively benign, despite the sometimes
vicious academic politics.  Sendmail reflected the world in which it
was created, putting a lot of emphasis on getting the mail through as
reliably as possible, even in the face of user errors.

Today's world is much more hostile. The vast majority of email is
malicious. The goal of an MTA has transitioned from getting the mail
through to keeping the bad mail out. Filtering is probably the first
priority for any MTA today.  This required a number of changes in
sendmail.

For example, many rulesets have been added to allow checking of
parameters on incoming SMTP commands in order to catch problems as
early as possible. It is much cheaper to reject a message when reading
the envelope than after you have committed to reading the entire
message, and even more expensive after you have accepted the message
for delivery. In the early days filtering was generally done by
accepting the message, passing it to a filter program, and then
sending it to another instance of sendmail if the message passed (the
so-called ``sandwich'' configuration). This is just far too expensive
in today's world.

Similarly, sendmail has gone from being a quite vanilla consumer of
TCP/IP connections to being much more sophisticated, doing things like
``peeking'' at network input to see if the sender is transmitting
commands before the previous command has been acknowledged. This
breaks down some of the previous abstractions that were designed to
make sendmail adaptable to multiple network types.  Today, it would
involve considerable work to connect sendmail to an XNS or DECnet
network, for example, since the knowledge of TCP/IP has been built
into so much of the code.

Many configuration features were added to address the hostile world,
such as support for access tables, Realtime Blackhole Lists, address
harvesting mitigation, denial-of-service protection, and spam
filtering.  This has dramatically complicated the task of configuring
a mail system, but was absolutely necessary to adapt to today's world.

\end{aosasect2}

%% \begin{aosasect2}{Incorporation of New Technologies}
\begin{aosasect2}{新技術の採用}

Many new standards have come along over the years that required
significant changes to sendmail. For example, the addition of TLS
(encryption) required significant changes through much of the
code. SMTP pipelining required peering into the low-level TCP/IP
stream to avoid deadlocks. The addition of the submission port (587)
required the ability to listen to multiple incoming ports, including
having different behaviors depending on the arrival port.

Other pressures were forced by circumstances rather than standards.
For example, the addition of the milter interface was a direct
response to spam. Although milter was not a published standard, it was
a major new technology.

In all cases, these changes enhanced the mail system in some way, be
it increased security, better performance, or new
functionality. However, they all came with costs, in nearly all cases
complicating both the code base and the configuration file.

\end{aosasect2}

\end{aosasect1}

%% \begin{aosasect1}{What If I Did It Today?}
\begin{aosasect1}{もし今やりなおせるとしたら?}

Hindsight is 20/20. There are many things I would do differently
today.  Some were unforeseeable at the time (e.g., how spam would
change our perception of e-mail, what modern toolsets would look like,
etc.), and some were eminently predictable. Some were just that in the
process of writing sendmail I learned a lot about e-mail, about TCP/IP,
and about programming itself---everyone grows as they code.

But there are also many things I would do the same, some in
contradiction to the standard wisdom.

%% \begin{aosasect2}{Things I Would Do Differently}
\begin{aosasect2}{違うやりかたにしたいところ}

Perhaps my biggest mistake with sendmail was to not recognize early
enough how important it was going to be. I had several opportunities
to nudge the world in the correct direction but didn't take them; in
fact, in some cases I did damage, e.g., by not making sendmail
stricter about bad input when it became appropriate to do
so. Similarly, I recognized that the configuration file syntax needed
to be improved fairly early on, when there were perhaps a few hundred
sendmail instances deployed, but decided not to change things because
I didn't want to cause the installed user base undue pain. In
retrospect it would have been better to improve things early and
cause temporary pain in order to produce a better long-term result.

%% \begin{aosasect3}{Version 7 Mailbox Syntax}
\begin{aosasect3}{バージョン7のMailboxの構文}

One example of this was the way version 7 mailboxes separated
messages.  They used a line beginning ``From\textvisiblespace'' (where
``\textvisiblespace'' represents the ASCII space character, 0x20) to separate
messages. If a message came in containing the word ``From\textvisiblespace''
at the
beginning of the line, local mailbox software converted it to
``{\textgreater}From\textvisiblespace''.
One refinement on some but not all systems was
to require a preceding blank line, but this could not be relied
upon. To this day, ``{\textgreater}From'' appears in extremely
unexpected places that aren't obviously related to email (but clearly
were processed by email at one time or another). In retrospect I
probably could have converted the BSD mail system to use a new syntax.
I would have been roundly cursed at the time, but I would have saved
the world a heap of trouble.

\end{aosasect3}

%% \begin{aosasect3}{Syntax and Contents of Configuration File}
\begin{aosasect3}{設定ファイルの構文とその中身}

Perhaps my biggest mistake in the syntax of the configuration file was
the use of tab (HT, 0x09) in rewriting rules
to separate the pattern from the
replacement.  At the time I was emulating make, only to learn years
later that Stuart Feldman, the author of make, thought that was one of
his biggest mistakes. Besides being non-obvious when looking at the
configuration on a screen, the tab character doesn't survive
cut-and-paste in most window systems.

Although I believe that rewriting rules were the correct idea (see
below), I would change the general structure of the configuration
file.  For example, I did not anticipate the need for hierarchies in
the configuration (e.g., options that would be set differently for
different SMTP listener ports). At the time the configuration file was
designed there were no ``standard'' formats. Today, I would be inclined
to use an Apache-style configuration---it's clean, neat, and has
adequate expressive power---or perhaps even embed a language such as Lua.

When sendmail was developed the address spaces were small and the
protocols were still in flux. Putting as much as possible into the
configuration file seemed like a good idea. Today, that looks like a
mistake: we have plenty of address space (for an MTA) and the
standards are fairly static.
Furthermore, part of the ``configuration file'' is really code
that needs to be updated in new releases.
The \code{.mc} configuration file fixes that,
but having to rebuild your configuration every time you update the software
is a pain.
A simple solution to this would simply be
to have two configuration files that sendmail would read, one hidden
and installed with each new software release and the other exposed and
used for local configuration.

\end{aosasect3}

%% \begin{aosasect3}{Use of Tools}
\begin{aosasect3}{使うツール}

There are many new tools available today---for example, for
configuring and building the software. Tools can be good leverage if
you need them, but they can also be overkill, making it harder than
necessary to understand the system. For example, you should never use
a yacc(1) grammar when all you need is strtok(3). But reinventing the
wheel isn't a good idea either.  In particular, despite some
reservations I would almost certainly use autoconf today.

\end{aosasect3}

%% \begin{aosasect3}{Backward Compatibility}
\begin{aosasect3}{後方互換性}

With the benefit of hindsight, and knowing how ubiquitous sendmail
became, I would not worry so much about breaking existing
installations in the early days of development.  When existing
practice is seriously broken it should be fixed, not accommodated
for. That said, I would still not do strict checking of all message
formats; some problems can be easily and safely ignored or
patched. For example, I would probably still insert a \code{Message-Id:}
header field into messages that did not have one, but I would be more
inclined to reject messages without a \code{From:} header field rather than
try to create one from the information in the envelope.

\end{aosasect3}

%% \begin{aosasect3}{Internal Abstractions}
\begin{aosasect3}{内部的な抽象化}

There are certain internal abstractions that I would not attempt
again, and others that I would add. For example, I would not use
null-terminated strings, opting instead for a length/value pair,
despite the fact that this means that much of the Standard C Library
becomes difficult to use. The security implications of this alone make
it worthwhile. Conversely, I would not attempt to build exception
handling in C, but I would create a consistent status code system that
would be used throughout the code rather than having routines return
\code{null}, \code{false}, or negative numbers to represent errors.

I would certainly abstract the concept of mailbox names from Unix user
ids. At the time I wrote sendmail the model was that you only sent
messages to Unix users. Today, that is almost never the case; even on
systems that do use that model, there are system accounts that should
never receive e-mail.

\end{aosasect3}

\end{aosasect2}

%% \begin{aosasect2}{Things I Would Do The Same}
\begin{aosasect2}{同じようにしたいところ}

Of course, some things \emph{did} work well\ldots

%% \begin{aosasect3}{Syslog}
\begin{aosasect3}{Syslog}

One of the successful side projects from sendmail was syslog. At the
time sendmail was written, programs that needed to log had a specific
file that they would write. These were scattered around the
filesystem.  Syslog was difficult to write at the time (UDP didn't
exist yet, so I used something called mpx files), but well worth
it. However, I would make one specific change: I would pay more
attention to making the syntax
of logged messages
machine parseable---essentially, I
failed to predict the existence of log monitoring.

\end{aosasect3}

%% \begin{aosasect3}{Rewriting Rules}
\begin{aosasect3}{リライトルール}

Rewriting rules have been much maligned, but I would use them again
(although probably not for as many things as they are used for now).
Using the tab character was a clear mistake, but given the limitations
of ASCII and the syntax of e-mail addresses, some escape character is
probably required\footnote{Somehow I suspect that using Unicode for
configuration would not prove popular.}. In general, the concept of
using a pattern-replace paradigm worked well and was very flexible.

\end{aosasect3}

%% \begin{aosasect3}{Avoid Unnecessary Tools}
\begin{aosasect3}{不要なツールは使わない}

Despite my comment above that I would use more existing tools, I am
reluctant to use many of the run-time libraries available today. In my
opinion far too many of them are so bloated as to be dangerous.
Libraries should be chosen with care, balancing the merits of reuse
against the problems of using an overly powerful tool to solve a
simple problem. One particular tool I would avoid is XML, at least as
a configuration language. I believe that the syntax is too baroque for
much of what it is used for. XML has its place, but it is overused
today.

\end{aosasect3}

%% \begin{aosasect3}{Code in C}
\begin{aosasect3}{コードをCで書く}

Some people have suggested that a more natural implementation language
would be Java or C++. Despite the well-known problems with C, I would
still use it as my implementation language. In part this is personal:
I know C much better than I know Java or C++. But I'm also
disappointed by the cavalier attitude that most object-oriented
languages take toward memory allocation. Allocating memory has many
performance concerns that can be difficult to characterize.  Sendmail
uses object-oriented concepts internally where appropriate (for
example, the implementation of map classes), but in my opinion going
completely object-oriented is wasteful and overly restrictive.

\end{aosasect3}

\end{aosasect2}

\end{aosasect1}

%% \begin{aosasect1}{Conclusions}
\begin{aosasect1}{まとめ}

The sendmail MTA was born into a world of immense upheaval, a sort of
``wild west'' that existed when e-mail was ad hoc and the current mail
standards were not yet formulated.  In the intervening 31 years the
``e-mail problem'' has changed from just working reliably to working
with large messages and heavy load to protecting sites from spam and
viruses and finally today to being used as a platform for a plethora
of e-mail-based applications.  Sendmail has evolved into a work-horse
that is embraced by even the most risk-averse corporations, even as
e-mail has evolved from pure text person-to-person communications into
a multimedia-based mission-critical part of the infrastructure.

The reasons for this success are not always obvious.  Building a
program that survives and even thrives in a rapidly changing world
with only a handful of part-time developers can't be done using
conventional software development methodologies.  I hope I've provided
some insights into how sendmail succeeded.

\end{aosasect1}

\end{aosachapter}
