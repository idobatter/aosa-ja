\begin{aosachapter}{VTK}{s:vtk}{Berk Geveci and Will Schroeder}
%% Based on EN-Revision r272

%% The Visualization Toolkit (VTK) is a widely used software system for
%% data processing and visualization. It is used in scientific computing,
%% medical image analysis, computational geometry, rendering, image
%% processing and informatics. In this chapter we provide a brief
%% overview of VTK, including some of the basic design patterns that make
%% it a successful system.
Visualization Toolkit (VTK)は、データ処理や可視化用のソフトウェアシステムとして幅広く使われているものだ。
科学計算や医用画像解析、計算幾何学、レンダリング、画像処理や情報科学などの分野で使われている。
この章では、VTKの概要とその基本的な設計パターンに関して説明する。

%% To really understand a software system it is essential to not only
%% understand what problem it solves, but also the particular culture in
%% which it emerged. In the case of VTK, the software was ostensibly
%% developed as a 3D visualization system for scientific data. But the
%% cultural context in which it emerged adds a significant back story to
%% the endeavor, and helps explains why the software was designed and
%% deployed as it was.
何かのソフトウェアシステムについて完全に理解するには、
単にそのソフトウェアが解決しようとしている問題が何かを理解するだけでは不十分だ。
そのソフトウェアが生み出す文化も理解しないといけない。
VTKに関して言うと、上っ面だけを見れば、科学データ用の3D可視化システムに過ぎない。
しかし、それが生み出す文化を踏まえるとその背景には大きなストーリーがあり、
なぜ開発されたのかやなぜこの形で公開されたのかなどを理解する助けになるだろう。

%% At the time VTK was conceived and written, its initial authors (Will
%% Schroeder, Ken Martin, Bill Lorensen) were researchers at GE Corporate
%% R\&D\@. We were heavily invested in a precursor system known as LYMB
%% which was a Smalltalk-like environment implemented in the C
%% programming language. While this was a great system for its time, as
%% researchers we were consistently frustrated by two major barriers when
%% trying to promote our work: 1) IP issues and 2) non-standard,
%% proprietary software. IP issues were a problem because trying to
%% distribute the software outside of GE was nearly impossible once the
%% corporate lawyers became involved. Second, even if we were deploying
%% the software inside of GE, many of our customers balked at learning a
%% proprietary, non-standard system since the effort to master it did not
%% transition with an employee once she left the company, and it did not
%% have the widespread support of a standard tool set. Thus in the end
%% the primary motivation for VTK was to develop an open standard, or
%% \emph{collaboration platform} through which we could easily transition
%% technology to our customers. Thus choosing an open source license for
%% VTK was probably the most important design decision that we made.
VTKの元になる概念が生まれて実際に書き始めた当時、作者(Will Schroeder、Ken Martin、Bill Lorensen)
はみなGE Corporate R\&Dの研究者だった。
当時の私たちが使っていたのは、LYMBという先進的なシステムだった。
これはSmalltalk風の環境で、C言語で実装されていた。
当時としてはすばらしいシステムだったが、研究者としての私たちは二つの大きな障害に悩まされていた。
知的所有権の問題と、非標準のプロプライエタリなソフトウェアを使っているという問題だ。
知的所有権の問題のせいで、いったん顧問弁護士に口出しされると
ソフトウェアをGEの外部に公開することは事実上不可能だった。
さらに、ソフトウェアをGEの内部でだけ公開することにしたとしても、
顧客の多くはプロプライエタリで非標準なシステムを覚えようとはしたがらないだろう。
がんばってマスターしたところで、転職してしまえば何の役にも立たなくなる。
さらに、標準のツールセットの恩恵を受けることもできない。
そんな状況があったので、VTKを開発する最大の動機として「オープンな標準規格を使うこと」
つまり「\emph{コラボレーションプラットフォーム}として開発し、
その技術を顧客にも使ってもらいやすくすること」がうまれた。
VTKにオープンソースライセンスを適用したことは、
これまでの設計上の判断の中でもおそらく最も重要なものだった。

%% The final choice of a non-reciprocal, permissive license (i.e., BSD
%% not GPL) in hindsight was an exemplary decision made by the authors
%% because it ultimately enabled the service and consulting based
%% business that became Kitware. At the time we made the decision we were
%% mostly interested in reduced barriers to collaborating with academics,
%% research labs, and commercial entities. We have since discovered that
%% reciprocal licenses are avoided by many organizations because of the
%% potential havoc they can wreak. In fact we would argue that reciprocal
%% licenses do much to slow the acceptance of open source software, but
%% that is an argument for another time. The point here is: one of the
%% major design decisions to make relative to any software system is the
%% choice of copyright license. It's important to review the goals of the
%% project and then address IP issues appropriately.
最終的に、伝播性のないライセンスを選んだ(つまり、GPLではなくBSDを選んだ)
ことは、今思えばとてもよい判断だった。そのおかげで、サービスやコンサルのビジネスもできるようになり、
それが後のKitwareにつながったのだから。
当時の私たちが重視していたのは、学校や研究所そして企業などが共同作業をするための障壁を下げることだった。
多くの組織では、伝播性のあるライセンスが避けられていた。問題に巻き込まれる恐れがあったからだ。
実際私たちも、伝播性のあるライセンスのせいでオープンソースソフトウェアの普及がなかなか進まないと思っている。
まあ、それはまた別の話。
言いたいのは、ソフトウェアシステムを作るときにはライセンスの選択が重要のポイントの一つになるっていうことだ。
プロジェクトのゴールを見定めて、知的所有権の問題に適切に対応することが大切だ。

%% \begin{aosasect1}{What Is VTK?}
\begin{aosasect1}{VTKとは}

VTK was initially conceived as a scientific data visualization
system. Many people outside of the field naively consider
visualization a particular type of geometric rendering: examining
virtual objects and interacting with them. While this is indeed part
of visualization, in general data visualization includes the whole
process of transforming data into sensory input, typically images, but
also includes tactile, auditory, and other forms. The data forms not
only consist of geometric and topological constructs, including such
abstractions as meshes or complex spatial decompositions, but
attributes to the core structure such as scalars (e.g., temperature or
pressure), vectors (e.g., velocity), tensors (e.g., stress and strain)
plus rendering attributes such as surface normals and texture
coordinate.

Note that data representing spatial-temporal information is generally
considered part of scientific visualization. However there are more
abstract data forms such as marketing demographics, web pages,
documents and other information that can only be represented through
abstract (i.e., non-spatial temporal) relationships such as
unstructured documents, tables, graphs, and trees. These abstract data
are typically addressed by methods from information
visualization. With the help of the community, VTK is now capable of
both scientific and information visualization.

As a visualization system, the role of VTK is to take data in these
forms and ultimately transform them into forms comprehensible by the
human sensory apparatus. Thus one of the core requirements of VTK is
its ability to create data flow pipelines that are capable of
ingesting, processing, representing and ultimately rendering
data. Hence the toolkit is necessarily architected as a flexible
system and its design reflects this on many levels. For example, we
purposely designed VTK as a toolkit with many interchangeable
components that can be combined to process a wide variety of data.

\end{aosasect1}

%% \begin{aosasect1}{Architectural Features}
\begin{aosasect1}{アーキテクチャの機能}

Before getting too far into the specific architectural features of
VTK, there are high-level concepts that have significant impact on
developing and using the system. One of these is VTK's hybrid wrapper
facility. This facility automatically generates language bindings to
Python, Java, and Tcl from VTK's C++ implementation (additional
languages could be and have been added). Most high-powered developers
will work in C++. User and application developers may use C++ but
often the interpreted languages mentioned above are preferred. This
hybrid compiled/interpreted environment combines the best of both
worlds: high performance compute-intensive algorithms and flexibility
when prototyping or developing applications. In fact this approach to
multi-language computing has found favor with many in the scientific
computing community and they often use VTK as a template for
developing their own software.

In terms of software process, VTK has adopted CMake to control the
build; CDash/CTest for testing; and CPack for cross-platform
deployment. Indeed VTK can be compiled on almost any computer
including supercomputers which are often notoriously primitive
development environments. In addition, web pages, wiki, mailing lists
(user and developer), documentation generation facilities (i.e.,
Doxygen) and a bug tracker (Mantis) round out the development tools.

%% \begin{aosasect2}{Core Features}
\begin{aosasect2}{コア機能}

As VTK is an object-oriented system, the access of class and instance
data members is carefully controlled in VTK\@. In general, all data
members are either protected or private. Access to them is through
\code{Set} and \code{Get} methods, with special variations for Boolean
data, modal data, strings and vectors. Many of these methods are
actually created by inserting macros into the class header files. So
for example:

\begin{verbatim}
vtkSetMacro(Tolerance,double);
vtkGetMacro(Tolerance,double);
\end{verbatim}

\noindent become on expansion:

\begin{verbatim}
virtual void SetTolerance(double);
virtual double GetTolerance();
\end{verbatim}

There are many reasons for using these macros beyond simply code
clarity. In VTK there are important data members controlling
debugging, updating an object's modified time (MTime), and properly
managing reference counting. These macros correctly manipulate these
data and their use is highly recommended. For example, a particularly
pernicious bug in VTK occurs when the object's MTime is not managed
properly. In this case code may not execute when it should, or may
execute too often.

One of the strengths of VTK is its relatively simplistic means of
representing and managing data. Typically various data arrays of
particular types (e.g., \code{vtkFloatArray}) are used to represent
contiguous pieces of information. For example, a list of three XYZ
points would be represented with a \code{vtkFloatArray} of nine
entries (x,y,z, x,y,z, etc.) There is the notion of a tuple in these
arrays, so a 3D point is a 3-tuple, whereas a symmetric $3{\times}3$
tensor matrix is represented by a 6-tuple (where symmetry space
savings are possible). This design was adopted purposely because in
scientific computing it is common to interface with systems
manipulating arrays (e.g., Fortran) and it is much more efficient to
allocate and deallocate memory in large contiguous chunks. Further,
communication, serializing and performing IO is generally much more
efficient with contiguous data. These core data arrays (of various
types) represent much of the data in VTK and have a variety of
convenience methods for inserting and accessing information, including
methods for fast access, and methods that automatically allocate
memory as needed when adding more data. Data arrays are subclasses of
the \code{vtkDataArray} abstract class meaning that generic virtual
methods can be used to simplify coding. However, for higher
performance static, templated functions are used which switch based on
type, with subsequent, direct access into the contiguous data arrays.

In general C++ templates are not visible in the public class API;
although templates are used widely for performance reasons. This goes
for STL as well: we typically employ the
PIMPL\footnote{\url{http://en.wikipedia.org/wiki/Opaque\_pointer}.}
design pattern to hide the complexities of a template implementation
from the user or application developer. This has served us
particularly well when it comes to wrapping the code into interpreted
code as described previously. Avoiding the complexity of the templates
in the public API means that the VTK implementation, from the
application developer point of view, is mostly free of the
complexities of data type selection. Of course under the hood the code
execution is driven by the data type which is typically determined at
run time when the data is accessed.

Some users wonder why VTK uses reference counting for memory
management versus a more user-friendly approach such as garbage
collection. The basic answer is that VTK needs complete control over
when data is deleted, because the data sizes can be huge. For example,
a volume of byte data $1000{\times}1000{\times}1000$ in size is a
gigabyte in size. It is not a good idea to leave such data lying
around while the garbage collector decides whether or not it is time
to release it. In VTK most classes (subclasses of \code{vtkObject})
have the built-in capability for reference counting. Every object
contains a reference count that it initialized to one when the object
is instantiated.  Every time a use of the object is registered, the
reference count is increased by one. Similarly, when a use of the
object is unregistered (or equivalently the object is deleted) the
reference count is reduced by one. Eventually the object's reference
count is reduced to zero, at which point it self destructs.  A typical
example looks like the following:

\begin{verbatim}
vtkCamera *camera = vtkCamera::New();   //reference count is 1
camera->Register(this);                 //reference count is 2
camera->Unregister(this);               //reference count is 1
renderer->SetActiveCamera(camera);      //reference count is 2
renderer->Delete();                     //ref count is 1 when renderer is deleted
camera->Delete();                       //camera self destructs
\end{verbatim}

There is another important reason why reference counting is important
to VTK---it provides the ability to efficiently copy data. For
example, imagine a data object D1 that consists of a number of data
arrays: points, polygons, colors, scalars and texture coordinates. Now
imagine processing this data to generate a new data object D2 which is
the same as the first plus the addition of vector data (located on the
points). One wasteful approach is to completely (deep) copy D1 to
create D2, and then add the new vector data array to
D2. Alternatively, we create an empty D2 and then pass the arrays from
D1 to D2 (shallow copy), using reference counting to keep track of
data ownership, finally adding the new vector array to D2. The latter
approach avoids copying data which, as we have argued previously, is
essential to a good visualization system. As we will see later in this
chapter, the data processing pipeline performs this type of operation
routinely, i.e., copying data from the input of an algorithm to the
output, hence reference counting is essential to VTK.

Of course there are some notorious problems with reference
counting. Occasionally reference cycles can exist, with objects in the
cycle referring to each other in a mutually supportive
configuration. In this case, intelligent intervention is required, or
in the case of VTK, the special facility implemented in
\code{vtkGarbageCollector} is used to manage objects which are
involved in cycles. When such a class is identified (this is
anticipated during development), the class registers itself with the
garbage collector and overloads its own \code{Register} and
\code{UnRegister} methods. Then a subsequent object deletion (or
unregister) method performs a topological analysis on the local
reference counting network, searching for detached islands of mutually
referencing objects. These are then deleted by the garbage collector.

Most instantiation in VTK is performed through an object factory
implemented as a static class member. The typical syntax appears as
follows:

\begin{verbatim}
vtkLight *a = vtkLight::New();
\end{verbatim}

What is important to recognize here is what is actually instantiated
may not be a \code{vtkLight}, it could be a subclass of
\code{vtkLight} (e.g., \code{vtkOpenGLLight}). There are a variety of
motivations for the object factory, the most important being
application portability and device independence. For example, in the
above we are creating a light in a rendered scene. In a particular
application on a particular platform, \code{vtkLight::New} may result
in an OpenGL light, however on different platforms there is potential
for other rendering libraries or methods for creating a light in the
graphics system. Exactly what derived class to instantiate is a
function of run-time system information. In the early days of VTK
there were a myriad of options including gl, PHIGS, Starbase, XGL, and
OpenGL\@. While most of these have now vanished, new approaches have
appeared including DirectX and GPU-based approaches. Over time, an
application written with VTK has not had to change as developers have
derived new device specific subclasses to \code{vtkLight} and other
rendering classes to support evolving technology. Another important
use of the object factory is to enable the run-time replacement of
performance-enhanced variations. For example, a \code{vtkImageFFT} may
be replaced with a class that accesses special-purpose hardware or a
numerics library.

\end{aosasect2}

%% \begin{aosasect2}{Representing Data}
\begin{aosasect2}{データ表現}

One of the strengths of VTK is its ability to represent complex forms
of data. These data forms range from simple tables to complex
structures such as finite element meshes. All of these data forms are
subclasses of \code{vtkDataObject} as shown in
\aosafigref{fig.vtk.dataclass} (note this is a partial inheritance
diagram of the many data object classes).

%% \aosafigure{../images/vtk/dataclasses.eps}{Data Object Classes}{fig.vtk.dataclass}
\aosafigure{../images/vtk/dataclasses.eps}{データオブジェクトクラス}{fig.vtk.dataclass}

One of the most important characteristics of \code{vtkDataObject} is
that it can be processed in a visualization pipeline (next
subsection). Of the many classes shown, there are just a handful that
are typically used in most real world applications. \code{vtkDataSet}
and derived classes are used for scientific visualization
(\aosafigref{fig.vtk.dataset}). For example, \code{vtkPolyData} is
used to represent polygonal meshes; \code{vtkUnstructuredGrid} to
represent meshes, and \code{vtkImageData} represents 2D and 3D pixel
and voxel data.

%% \aosafigure[250pt]{../images/vtk/dataset.eps}{Data Set Classes}{fig.vtk.dataset}
\aosafigure[250pt]{../images/vtk/dataset.eps}{データセットクラス}{fig.vtk.dataset}

\end{aosasect2}

%% \begin{aosasect2}{Pipeline Architecture}
\begin{aosasect2}{パイプラインのアーキテクチャ}

VTK consists of several major subsystems. Probably the subsystem most
associated with visualization packages is the data flow/pipeline
architecture. In concept, the pipeline architecture consists of three
basic classes of objects: objects to represent data (the
\code{vtkDataObject}s discussed above), objects to process, transform,
filter or map data objects from one form into another
(\code{vtkAlgorithm}); and objects to execute a pipeline
(\code{vtkExecutive}) which controls a connected graph of interleaved
data and process objects (i.e., the
pipeline). \aosafigref{fig.vtk.pipeline} depicts a typical pipeline.

%% \aosafigure{../images/vtk/pipeline.eps}{Typical Pipeline}{fig.vtk.pipeline}
\aosafigure{../images/vtk/pipeline.eps}{典型的なパイプライン}{fig.vtk.pipeline}

While conceptually simple, actually implementing the pipeline
architecture is challenging. One reason is that the representation of
data can be complex. For example, some datasets consist of hierarchies
or grouping of data, so executing across the data requires non-trivial
iteration or recursion. To compound matters, parallel
processing (whether using shared-memory or scalable, distributed
approaches) require partitioning data into pieces, where pieces may be
required to overlap in order to consistently compute boundary
information such as derivatives.

The algorithm objects also introduce their own special
complexity. Some algorithms may take multiple inputs and/or produce
multiple outputs of different types. Some can operate locally on data
(e.g., compute the center of a cell) while others require global
information, for example to compute a histogram. In all cases, the
algorithms treat their inputs as immutable, algorithms only read their
input in order to produce their output. This is because data may be
available as input to multiple algorithms, and it is not a good idea
for one algorithm to trample on the input of another.

Finally the executive can be complicated depending on the particulars
of the execution strategy. In some cases we may wish to cache
intermediate results between filters. This minimizes the amount of
recomputation that must be performed if something in the pipeline
changes. On the other hand, visualization data sets can be huge, in
which case we may wish to release data when it is no longer needed for
computation. Finally, there are complex execution strategies, such as
multi-resolution processing of data, which require the pipeline to
operate in iterative fashion.

To demonstrate some of these concepts and further explain the pipeline
architecture, consider the following C++ example:

\begin{verbatim}
vtkPExodusIIReader *reader = vtkPExodusIIReader::New();
reader->SetFileName("exampleFile.exo");

vtkContourFilter *cont = vtkContourFilter::New();
cont->SetInputConnection(reader->GetOutputPort());
cont->SetNumberOfContours(1);
cont->SetValue(0, 200);

vtkQuadricDecimation *deci = vtkQuadricDecimation::New();
deci->SetInputConnection(cont->GetOutputPort());
deci->SetTargetReduction( 0.75 );

vtkXMLPolyDataWriter *writer = vtkXMLPolyDataWriter::New();
writer->SetInputConnection(deci->GetOuputPort());
writer->SetFileName("outputFile.vtp");
writer->Write();
\end{verbatim}

\noindent In this example, a reader object reads a large unstructured grid (or
mesh) data file. The next filter generates an isosurface from the
mesh. The \code{vtkQuadricDecimation} filter reduces the size of the
isosurface, which is a polygonal dataset, by decimating it (i.e.,
reducing the number of triangles representing the isocontour). Finally
after decimation the new, reduced data file is written back to
disk. The actual pipeline execution occurs when the \code{Write}
method is invoked by the writer (i.e., upon demand for the data).

As this example demonstrates, VTK's pipeline execution mechanism is
demand driven. When a sink such as a writer or a mapper (a data
rendering object) needs data, it asks its input. If the input filter
already has the appropriate data, it simply returns the execution
control to the sink. However, if the input does not have the
appropriate data, it needs to compute it. Consequently, it must first
ask its input for data. This process will continue upstream along the
pipeline until a filter or source that has "appropriate data" or the
beginning of the pipeline is reached, at which point the filters will
execute in correct order and the data will flow to the point in the
pipeline at which it was requested.

Here we should expand on what "appropriate data" means. By default,
after a VTK source or filter executes, its output is cached by the
pipeline in order to avoid unnecessary executions in the future. This
is done to minimize computation and/or I/O at the cost of memory, and
is configurable behavior. The pipeline caches not only the data
objects but also the metadata about the conditions under which these
data objects were generated. This metadata includes a time stamp
(i.e., ComputeTime) that captures when the data object was
computed. So in the simplest case, the "appropriate data" is one that
was computed after all of the pipeline objects upstream from it were
modified. It is easier to demonstrate this behavior by considering the
following examples. Let's add the following to the end of the previous
VTK program:

\begin{verbatim}
vtkXMLPolyDataWriter *writer2 = vtkXMLPolyDataWriter::New();
writer2->SetInputConnection(deci->GetOuputPort());
writer2->SetFileName("outputFile2.vtp");
writer2->Write();
\end{verbatim}

As explained previously, the first \code{writer-{\textgreater}Write}
call causes the execution of the entire pipeline. When
\code{writer2-{\textgreater}Write()} is called, the pipeline will
realize that the cached output of the decimation filter is up to date
when it compares the time stamp of the cache with the modification
time of the decimation filter, the contour filter and the
reader. Therefore, the data request does not have to propagate past
\code{writer2}. Now, let's consider the following change.

\begin{verbatim}
cont->SetValue(0, 400);

vtkXMLPolyDataWriter *writer2 = vtkXMLPolyDataWriter::New();
writer2->SetInputConnection(deci->GetOuputPort());
writer2->SetFileName("outputFile2.vtp");
writer2->Write();
\end{verbatim}

Now the pipeline executive will realize that the contour filter was
modified after the outputs of the contour and decimation filters were
last executed. Thus, the cache for these two filters are stale and
they have to be re-executed. However, since the reader was not
modified prior to the contour filter its cache is valid and hence the
reader does not have to re-execute.

The scenario described here is the simplest example of a demand-driven
pipeline. VTK's pipeline is much more sophisticated. When a filter or
a sink requires data, it can provide additional information to request
specific data subsets. For example, a filter can perform out-of-core
analysis by streaming pieces of data. Let's change our previous
example to demonstrate.

\begin{verbatim}
vtkXMLPolyDataWriter *writer = vtkXMLPolyDataWriter::New();
writer->SetInputConnection(deci->GetOuputPort());
writer->SetNumberOfPieces(2);

writer->SetWritePiece(0);
writer->SetFileName("outputFile0.vtp");
writer->Write();

writer->SetWritePiece(1);
writer->SetFileName("outputFile1.vtp");
writer->Write();
\end{verbatim}

Here the writer asks the upstream pipeline to load and process data in
two pieces each of which are streamed independently. You may have
noticed that the simple execution logic described previously will not
work here. By this logic when the \code{Write} function is called for
the second time, the pipeline should not re-execute because nothing
upstream changed. Thus to address this more complex case, the
executives have additional logic to handle piece requests such as
this. VTK's pipeline execution actually consists of multiple
passes. The computation of the data objects is actually the last
pass. The pass before then is a request pass. This is where sinks and
filters can tell upstream what they want from the forthcoming
computation. In the example above, the writer will notify its input
that it wants piece 0 of 2. This request will actually propagate all
the way to the reader. When the pipeline executes, the reader will
then know that it needs to read a subset of the data. Furthermore,
information about which piece the cached data corresponds to is stored
in the metadata for the object. The next time a filter asks for data
from its input, this metadata will be compared with the current
request. Thus in this example the pipeline will re-execute in order to
process a different piece request.

There are several more types of request that a filter can make. These
include requests for a particular time step, a particular structured
extent or the number of ghost layers (i.e., boundary layers for
computing neighborhood information). Furthermore, during the request
pass, each filter is allowed to modify requests from downstream. For
example, a filter that is not able to stream (e.g., the streamline
filter) can ignore the piece request and ask for the whole data.

\end{aosasect2}

%% \begin{aosasect2}{Rendering Subsystem}
\begin{aosasect2}{レンダリングサブシステム}

At first glance VTK has a simple object-oriented rendering model with
classes corresponding to the components that make up a 3D scene. For
example, \code{vtkActor}s are objects that are rendered by a
\code{vtkRenderer} in conjunction with a \code{vtkCamera}, with
possibly multiple \code{vtkRenderer}s existing in a
\code{vtkRenderWindow}. The scene is illuminated by one or more
\code{vtkLight}s. The position of each \code{vtkActor} is controlled
by a \code{vtkTransform}, and the appearance of an actor is specified
through a \code{vtkProperty}. Finally, the geometric representation of
an actor is defined by a \code{vtkMapper}. Mappers play an important
role in VTK, they serve to terminate the data processing pipeline, as
well as interface to the rendering system. Consider this example where
we decimate data and write the result to a file, and then visualize
and interact with the result by using a mapper:

\begin{verbatim}
vtkOBJReader *reader = vtkOBJReader::New();
reader->SetFileName("exampleFile.obj");

vtkTriangleFilter *tri = vtkTriangleFilter::New();
tri->SetInputConnection(reader->GetOutputPort());

vtkQuadricDecimation *deci = vtkQuadricDecimation::New();
deci->SetInputConnection(tri->GetOutputPort());
deci->SetTargetReduction( 0.75 );

vtkPolyDataMapper *mapper = vtkPolyDataMapper::New();
mapper->SetInputConnection(deci->GetOutputPort());

vtkActor *actor = vtkActor::New();
actor->SetMapper(mapper);

vtkRenderer *renderer = vtkRenderer::New();
renderer->AddActor(actor);

vtkRenderWindow *renWin = vtkRenderWindow::New();
renWin->AddRenderer(renderer);

vtkRenderWindowInteractor *interactor = vtkRenderWindowInteractor::New();
interactor->SetRenderWindow(renWin);

renWin->Render();
\end{verbatim}

Here a single actor, renderer and render window are created with the
addition of a mapper that connects the pipeline to the rendering
system. Also note the addition of a \code{vtkRenderWindowInteractor},
instances of which capture mouse and keyboard events and translate
them into camera manipulations or other actions. This translation
process is defined via a \code{vtkInteractorStyle} (more on this
below). By default many instances and data values are set behind the
scenes. For example, an identity transform is constructed, as well as a
single default (head) light and property.

Over time this object model has become more sophisticated. Much of the
complexity has come from developing derived classes that specialize on
an aspect of the rendering process. \code{vtkActor}s are now
specializations of \code{vtkProp} (like a prop found on stage), and
there are a whole slew of these props for rendering 2D overlay
graphics and text, specialized 3D objects, and even for supporting
advanced rendering techniques such as volume rendering or GPU
implementations (see \aosafigref{fig.vtk.painter}).

Similarly, as the data model supported by VTK has grown, so have the
various mappers that interface the data to the rendering
system. Another area of significant extension is the transformation
hierarchy. What was originally a simple linear $4{\times}4$
transformation matrix, has become a powerful hierarchy that supports
non-linear transformations including thin-plate spline
transformation. For example, the original \code{vtkPolyDataMapper} had
device-specific subclasses (e.g., \code{vtkOpenGLPolyDataMapper}). In
recent years it has been replaced with a sophisticated graphics
pipeline referred to as the ``painter'' pipeline illustrated in
\aosafigref{fig.vtk.painter}.

%% \aosafigure{../images/vtk/painter.eps}{Display Classes}{fig.vtk.painter}
\aosafigure{../images/vtk/painter.eps}{ディスプレイクラス}{fig.vtk.painter}

The painter design supports a variety of techniques for rendering data
that can be combined to provide special rendering effects. This
capability greatly surpasses the simple \code{vtkPolyDataMapper} that
was initially implemented in 1994.

Another important aspect of a visualization system is the selection
subsystem. In VTK there is a hierarchy of "pickers", roughly
categorized into objects that select \code{vtkProp}s based on
hardware-based methods versus software methods (e.g., ray-casting); as
well as objects that provide different levels of information after a
pick operations. For example, some pickers provide only a location in
XYZ world space without indicating which \code{vtkProp} they have
selected; others provide not only the selected \code{vtkProp} but a
particular point or cell that make up the mesh defining the prop
geometry.

\end{aosasect2}

%% \begin{aosasect2}{Events and Interaction}
\begin{aosasect2}{イベントとインタラクション}

Interacting with data is an essential part of visualization. In VTK
this occurs in a variety of ways. At its simplest level, users can
observe events and respond appropriately through commands (the
command/observer design pattern). All subclasses of \code{vtkObject}
maintain a list of observers which register themselves with the
object. During registration, the observers indicate which particular
event(s) they are interested in, with the addition of an associated
command that is invoked if and when the event occurs. To see how this
works, consider the following example in which a filter (here a
polygon decimation filter) has an observer which watches for the three
events \code{StartEvent}, \code{ProgressEvent}, and
\code{EndEvent}. These events are invoked when the filter begins to
execute, periodically during execution, and then on completion of
execution. In the following the \code{vtkCommand} class has an
\code{Execute} method that prints out the appropriate information
relative to the time it take to execute the algorithm:

\begin{verbatim}
class vtkProgressCommand : public vtkCommand
{
  public:
    static vtkProgressCommand *New() { return new vtkProgressCommand; }
    virtual void Execute(vtkObject *caller, unsigned long, void *callData)
    {
      double progress = *(static_cast<double*>(callData));
      std::cout << "Progress at " << progress<< std::endl;
    }
};

vtkCommand* pobserver = vtkProgressCommand::New();

vtkDecimatePro *deci = vtkDecimatePro::New();
deci->SetInputConnection( byu->GetOutputPort() );
deci->SetTargetReduction( 0.75 );
deci->AddObserver( vtkCommand::ProgressEvent, pobserver );
\end{verbatim}

While this is a primitive form of interaction, it is a foundational
element to many applications that use VTK\@. For example, the simple
code above can be easily converted to display and manage a GUI
progress bar. This Command/Observer subsystem is also central to the
3D widgets in VTK, which are sophisticated interaction objects for
querying, manipulating and editing data and are described below.

Referring to the example above, it is important to note that events in
VTK are predefined, but there is a back door for user-defined
events. The class \code{vtkCommand} defines the set of enumerated
events (e.g., \code{vtkCommand::ProgressEvent} in the above example)
as well as a user event. The \code{UserEvent}, which is simply an integral
value, is typically used as a starting offset value into a set of
application user-defined events. So for example
\code{vtkCommand::UserEvent+100} may refer to a specific event outside
the set of VTK defined events.

From the user's perspective, a VTK widget appears as an actor in a
scene except that the user can interact with it by manipulating
handles or other geometric features (the handle manipulation and
geometric feature manipulation is based on the picking functionality
described earlier.) The interaction with this widget is fairly
intuitive: a user grabs the spherical handles and moves them, or grabs
the line and moves it. Behind the scenes, however, events are emitted
(e.g., \code{InteractionEvent}) and a properly programmed application
can observe these events, and then take the appropriate action. For
example they often trigger on the \code{vtkCommand::InteractionEvent}
as follows:

\begin{verbatim}
vtkLW2Callback *myCallback = vtkLW2Callback::New();
  myCallback->PolyData = seeds;    // streamlines seed points, updated on interaction
  myCallback->Actor = streamline;  // streamline actor, made visible on interaction

vtkLineWidget2 *lineWidget = vtkLineWidget2::New();
  lineWidget->SetInteractor(iren);
  lineWidget->SetRepresentation(rep);
  lineWidget->AddObserver(vtkCommand::InteractionEvent,myCallback);
\end{verbatim}

VTK widgets are actually constructed using two objects: a subclass of
\code{vtkInterac\-torObserver} and a subclass of \code{vtkProp}. The
\code{vtkInteractorObserver} simply observes user interaction in the
render window (i.e., mouse and keyboard events) and processes
them. The subclasses of \code{vtkProp} (i.e., actors) are simply
manipulated by the \code{vtkInteractorObserver}. Typically such
manipulation consists of modifying the \code{vtkProp}'s geometry
including highlighting handles, changing cursor appearance, and/or
transforming data. Of course, the particulars of the widgets require
that subclasses are written to control the nuances of widget behavior,
and there are more than 50 different widgets currently in the system.

\end{aosasect2}

%% \begin{aosasect2}{Summary of Libraries}
\begin{aosasect2}{ライブラリのまとめ}

VTK is a large software toolkit. Currently the system consists of
approximately 1.5 million lines of code (including comments but not
including automatically generated wrapper software), and approximately
1000 C++ classes. To manage the complexity of the system and reduce
build and link times the system has been partitioned into dozens of
subdirectories. \aosatblref{tbl.vtk.dirs} lists these subdirectories,
with a brief summary describing what capabilities the library
provides.

\begin{table}\centering
  \begin{tabular}{ |ll| }
    \hline
    \code{Common} & core VTK classes \\
    \code{Filtering} & classes used to manage pipeline dataflow \\
    \code{Rendering} & rendering, picking, image viewing, and interaction \\
    \code{VolumeRendering} & volume rendering techniques \\
    \code{Graphics} & 3D geometry processing \\
    \code{GenericFiltering} &  non-linear 3D geometry processing \\
    \code{Imaging} & imaging pipeline \\
    \code{Hybrid} & classes requiring both graphics and imaging functionality \\
    \code{Widgets} & sophisticated interaction \\
    \code{IO} & VTK input and output \\
    \code{Infovis} & information visualization \\
    \code{Parallel} & parallel processing (controllers and communicators) \\
    \code{Wrapping} & support for Tcl, Python, and Java wrapping \\
    \code{Examples} & extensive, well-documented examples \\
    \hline
  \end{tabular}
  \caption{VTK Subdirectories}
  \label{tbl.vtk.dirs}
\end{table}

\end{aosasect2}

\end{aosasect1}

%% \begin{aosasect1}{Looking Back/Looking Forward}
\begin{aosasect1}{ふりかえり/今後の展望}

VTK has been an enormously successful system. While the first line of code was written in
1993, at the time of this writing VTK is still growing strong and if anything
the pace of development is increasing.\footnote{See the latest VTK code
analysis at \url{http://www.ohloh.net/p/vtk/analyses/latest}.} In this section
we talk about some lessons learned and future challenges.

%% \begin{aosasect2}{Managing Growth}
\begin{aosasect2}{プロジェクトの発展}

One of the most surprising aspects to the VTK adventure has been the
project's longevity. The pace of development is due to several major
reasons:

\begin{aosaitemize}

  \item New algorithms and capabilities continue to be added. For
  example, the informatics subsystem (Titan, primarily developed by
  Sandia National Labs and Kitware) is a recent significant
  addition. Additional charting and rendering classes are also being
  added, as well as capabilities for new scientific dataset
  types. Another important addition were the 3D interaction
  widgets. Finally, the on-going evolution of GPU-based rendering and
  data processing is driving new capabilities in VTK.

  \item The growing exposure and use of VTK is a self-perpetuating
  process that adds even more users and developers to the
  community. For example, ParaView is the most popular scientific
  visualization application built on VTK and is highly regarded in the
  high-performance computing community. 3D Slicer is a major
  biomedical computing platform that is largely built on VTK and
  received millions of dollars per year in funding.

  \item VTK's development process continues to evolve. In recent years
  the software process tools CMake, CDash, CTest, and CPack have been
  integrated into the VTK build environment. More recently, the VTK
  code repository has moved to Git and a more sophisticated work
  flow. These improvements ensure that VTK remains on the leading edge
  of software development in the scientific computing community.

\end{aosaitemize}

While growth is exciting, validates the creation of the software
system, and bodes well for the future of VTK, it can be extremely
difficult to manage well. As a result, the near term future of VTK
focuses more on managing the growth of the community as well as the
software. Several steps have been taken in this regard.

First, formalized management structures are being created. An
Architecture Review Board has been created to guide the development of
the community and technology, focusing on high-level, strategic
issues. The VTK community is also establishing a recognized team of
Topic Leads to guide the technical development of particular VTK
subsystems.

Next, there are plans to modularize the toolkit further, partially in
response to workflow capabilities introduced by git, but also to
recognize that users and developers typically want to work with small
subsystems of the toolkit, and do not want to build and link against
the entire package. Further, to support the growing community, it's
important that contributions of new functionality and subsystems are
supported, even if they are not necessarily part of the core of the
toolkit. By creating a loose, modularized collection of modules it is
possible to accommodate the large number of contributions on the
periphery while maintaining core stability.

\end{aosasect2}

%% \begin{aosasect2}{Technology Additions}
\begin{aosasect2}{技術的な補足}

Besides the software process, there are many technological innovations
in the development pipeline.

\begin{aosaitemize}

  \item Co-processing is a capability where the visualization engine is
  integrated into the simulation code, and periodically generates data
  extracts for visualization. This technology greatly reduces the need
  to output large amounts of complete solution data.

  \item The data processing pipeline in VTK is still too
  complex. Methods are under way to simplify and refactor this
  subsystem.

  \item The ability to directly interact with data is increasingly
  popular with users. While VTK has a large suite of widgets, many
  more interaction techniques are emerging including
  touch-screen-based and 3D methods. Interaction will continue its
  development at a rapid pace.

  \item Computational chemistry is increasing in importance to materials
  designers and engineers. The ability to visualize and interact with
  chemistry data is being added to VTK.

  \item The rendering system in VTK has been criticized for being too
  complex, making it difficult to derive new classes or support new
  rendering technology. In addition, VTK does not directly support the
  notion of a scene graph, again something that many users have
  requested.

  \item Finally new forms of data are constantly emerging. For example,
  in the medical field hierarchical volumetric datasets of varying
  resolution (e.g., confocal microscopy with local magnification).

\end{aosaitemize}

\end{aosasect2}

%% \begin{aosasect2}{Open Science}
\begin{aosasect2}{オープンサイエンス}

Finally Kitware and more generally the VTK community are committed to
Open Science. Pragmatically this is a way of saying we will promulgate
open data, open publication, and open source---the features necessary
to ensure that we are creating reproducible scientific systems. While
VTK has long been distributed as an open source and open data system,
the documentation process has been lacking. While there are decent
books~\cite{bib:vtk:userguide,bib:vtk:toolkit} there
have been a variety of ad hoc ways to collect technical publications
including new source code contributions. We are improving the
situation by developing new publishing mechanisms like the \emph{VTK
Journal}\footnote{\url{http://www.midasjournal.org/?journal=35}}
that enable of articles consisting of documentation, source code,
data, and valid test images. The journal also enables automated
reviews of the code (using VTK's quality software testing process) as
well as human reviews of the submission.

\end{aosasect2}

%% \begin{aosasect2}{Lessons Learned}
\begin{aosasect2}{教訓}

While VTK has been successful there are many things we didn't do
right:

\begin{aosadescription}

  \item{Design Modularity}: We did a good job choosing the
  modularity of our classes. For example, we didn't do something as
  silly as creating an object per pixel, rather we created the
  higher-level \code{vtkImageClass} that under the hood treats data arrays of
  pixel data. However in some cases we made our classes too high level
  and too complex, in many instances we've had to refactor them into
  smaller pieces, and are continuing this process. One prime example
  is the data processing pipeline.  Initially, the pipeline was
  implemented implicitly through interaction of the data and algorithm
  objects. We eventually realized that we had to create an explicit
  pipeline executive object to coordinate the interaction between data
  and algorithms, and to implement different data processing
  strategies.

  \item{Missed Key Concepts}: Once of our biggest regrets is not
  making widespread use of C++ iterators. In many cases the traversal
  of data in VTK is akin to the scientific programming language
  Fortran\@. The additional flexibility of iterators would have been a
  significant benefit to the system. For example, it is very
  advantageous to process a local region of data, or only data
  satisfying some iteration criterion.

  \item{Design Issues}: Of course there is a long list of design
  decisions that are not optimal. We have struggled with the data
  execution pipeline, having gone through multiple generations each
  time making the design better. The rendering system too is complex
  and hard to derive from. Another challenge resulted from the initial
  conception of VTK: we saw it as a read-only visualization system for
  viewing data. However, current customers often want it to be capable
  of editing data, which requires significantly different data
  structures.

\end{aosadescription}

One of the great things about an open source system like VTK is that
many of these mistakes can and will be rectified over time. We have an
active, capable development community that is improving the system
every day and we expect this to continue into the foreseeable future.

\end{aosasect2}

\end{aosasect1}

\end{aosachapter}

