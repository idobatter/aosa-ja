\begin{aosachapter}{LLVM}{s:llvm}{Chris Lattner}
%% Based on EN-Revision r272

This chapter discusses some of the design decisions that shaped
LLVM\footnote{\url{http://llvm.org}}, an umbrella project that hosts
and develops a set of close-knit low-level toolchain components (e.g.,
assemblers, compilers, debuggers, etc.), which are designed to be
compatible with existing tools typically used on Unix systems.  The
name ``LLVM'' was once an acronym, but is now just a brand for the
umbrella project.  While LLVM provides some unique capabilities, and
is known for some of its great tools (e.g., the Clang
compiler\footnote{\url{http://clang.llvm.org}}, a C/C++/Objective-C
compiler which provides a number of benefits over the GCC compiler),
the main thing that sets LLVM apart from other compilers is its
internal architecture.

From its beginning in December 2000, LLVM was designed as a set of
reusable libraries with well-defined
interfaces~\cite{bib:lattner:cgo}.  At the time, open source
programming language implementations were designed as special-purpose
tools which usually had monolithic executables. For example, it was
very difficult to reuse the parser from a static compiler (e.g., GCC)
for doing static analysis or refactoring.  While scripting languages
often provided a way to embed their runtime and interpreter into
larger applications, this runtime was a single monolithic lump of code
that was included or excluded.  There was no way to reuse pieces, and
very little sharing across language implementation projects.

Beyond the composition of the compiler itself, the communities
surrounding popular language implementations were usually strongly
polarized: an implementation usually provided \emph{either} a
traditional static compiler like GCC, Free Pascal, and FreeBASIC,
\emph{or} it provided a runtime compiler in the form of an
interpreter or Just-In-Time (JIT) compiler.  It was very uncommon to
see language implementation that supported both, and if they did,
there was usually very little sharing of code.

Over the last ten years, LLVM has substantially altered this
landscape.  LLVM is now used as a common infrastructure to implement a
broad variety of statically and runtime compiled languages (e.g., the
family of languages supported by GCC, Java, .NET, Python, Ruby,
Scheme, Haskell, D, as well as countless lesser known languages).  It
has also replaced a broad variety of special purpose compilers, such
as the runtime specialization engine in Apple's OpenGL stack and the
image processing library in Adobe's After Effects product.  Finally
LLVM has also been used to create a broad variety of new products,
perhaps the best known of which is the OpenCL GPU programming language
and runtime.

%% \begin{aosasect1}{A Quick Introduction to Classical Compiler Design}
\begin{aosasect1}{古典的なコンパイラの設計}

The most popular design for a traditional static compiler (like most C
compilers) is the three phase design whose major components are the
front end, the optimizer and the back end
(\aosafigref{fig.llvm.com}). The front end parses source code, checking
it for errors, and builds a language-specific Abstract Syntax Tree
(AST) to represent the input code.  The AST is optionally converted to
a new representation for optimization, and the optimizer and back end
are run on the code.

%% \aosafigure{../images/llvm/SimpleCompiler.eps}{Three Major Components of a Three-Phase Compiler}{fig.llvm.com}
\aosafigure{../images/llvm/SimpleCompiler.eps}{3フェーズコンパイラの主要なコンポーネント}{fig.llvm.com}

The optimizer is responsible for doing a broad variety of
transformations to try to improve the code's running time, such as
eliminating redundant computations, and is usually more or less
independent of language and target.  The back end (also known as the
code generator) then maps the code onto the target instruction set.
In addition to making \emph{correct} code, it is responsible for
generating \emph{good} code that takes advantage of unusual features
of the supported architecture.  Common parts of a compiler back end
include instruction selection, register allocation, and instruction
scheduling.

This model applies equally well to interpreters and JIT compilers.
The Java Virtual Machine (JVM) is also an implementation of this
model, which uses Java bytecode as the interface between the front end
and optimizer.

%% \begin{aosasect2}{Implications of this Design}
\begin{aosasect2}{この設計の実装}

The most important win of this classical design comes when a compiler
decides to support multiple source languages or target
architectures. If the compiler uses a common code representation in
its optimizer, then a front end can be written for any language that
can compile to it, and a back end can be written for any target that
can compile from it, as shown in \aosafigref{fig.llvm.rtc}.

%% \aosafigure{../images/llvm/RetargetableCompiler.eps}{Retargetablity}{fig.llvm.rtc}
\aosafigure{../images/llvm/RetargetableCompiler.eps}{再利用性}{fig.llvm.rtc}

With this design, porting the compiler to support a new source
language (e.g.,  Algol or BASIC) requires implementing a new front end,
but the existing optimizer and back end can be reused.  If these parts
weren't separated, implementing a new source language would require
starting over from scratch, so supporting \code{N} targets and
\code{M} source languages would need N*M compilers.

Another advantage of the three-phase design (which follows directly
from retargetability) is that the compiler serves a broader set of
programmers than it would if it only supported one source language and
one target. For an open source project, this means that there is a
larger community of potential contributors to draw from, which
naturally leads to more enhancements and improvements to the compiler.
This is the reason why open source compilers that serve many
communities (like GCC) tend to generate better optimized machine code
than narrower compilers like FreePASCAL\@.  This isn't the case for
proprietary compilers, whose quality is directly related to the
project's budget.  For example, the Intel ICC Compiler is widely known
for the quality of code it generates, even though it serves a narrow
audience.

A final major win of the three-phase design is that the skills
required to implement a front end are different than those required for
the optimizer and back end.  Separating these makes it easier for a
``front-end person'' to enhance and maintain their part of the
compiler.  While this is a social issue, not a technical one, it
matters a lot in practice, particularly for open source projects
that want to reduce the barrier to contributing as much as possible.

\end{aosasect2}

\end{aosasect1}

%% \begin{aosasect1}{Existing Language Implementations}
\begin{aosasect1}{既存の言語の実装}

While the benefits of a three-phase design are compelling and
well-documented in compiler textbooks, in practice it is almost never
fully realized. Looking across open source language implementations
(back when LLVM was started), you'd find that the implementations of
Perl, Python, Ruby and Java share no code.  Further, projects like the
Glasgow Haskell Compiler (GHC) and FreeBASIC are retargetable to
multiple different CPUs, but their implementations are very specific
to the one source language they support.  There is also a broad
variety of special purpose compiler technology deployed to implement
JIT compilers for image processing, regular expressions, graphics card
drivers, and other subdomains that require CPU intensive work.

That said, there are three major success stories for this model, the
first of which are the Java and .NET virtual machines.  These systems
provide a JIT compiler, runtime support, and a very well defined
bytecode format.  This means that any language that can compile to the
bytecode format (and there are dozens of
them\footnote{\url{http://en.wikipedia.org/wiki/List_of_JVM_languages}})
can take advantage of the effort put into the optimizer and JIT as
well as the runtime.  The tradeoff is that these implementations
provide little flexibility in the choice of runtime: they both
effectively force JIT compilation, garbage collection, and the use of
a very particular object model.  This leads to suboptimal performance
when compiling languages that don't match this model closely, such as
C (e.g., with the LLJVM project).

A second success story is perhaps the most unfortunate, but
also most popular way to reuse compiler technology: translate the input
source to C code (or some other language) and send it through existing
C compilers.  This allows reuse of the optimizer and code generator,
gives good flexibility, control over the runtime, and is really easy
for front-end implementers to understand, implement, and maintain.
Unfortunately, doing this prevents efficient implementation of
exception handling, provides a poor debugging experience, slows down
compilation, and can be problematic for languages that require guaranteed
tail calls (or other features not supported by C).

A final successful implementation of this model is GCC\footnote{A
backronym that now stands for ``GNU Compiler Collection''.}.  GCC
supports many front ends and back ends, and has an active and broad
community of contributors.  GCC has a long history of being a C
compiler that supports multiple targets with hacky support for a few
other languages bolted onto it.  As the years go by, the GCC community
is slowly evolving a cleaner design.  As of GCC~4.4, it has a new
representation for the optimizer (known as ``GIMPLE Tuples'') which is
closer to being separate from the front-end representation than
before.  Also, its Fortran and Ada front ends use a clean AST.

While very successful, these three approaches have strong limitations
to what they can be used for, because they are designed as monolithic
applications.  As one example, it is not realistically possible to
embed GCC into other applications, to use GCC as a runtime/JIT
compiler, or extract and reuse pieces of GCC without pulling in most
of the compiler.  People who have wanted to use GCC's C++ front end for
documentation generation, code indexing, refactoring, and static
analysis tools have had to use GCC as a monolithic application that
emits interesting information as XML, or write plugins to inject
foreign code into the GCC process.

There are multiple reasons why pieces of GCC cannot be reused as
libraries, including rampant use of global variables, weakly enforced
invariants, poorly-designed data structures, sprawling code base, and
the use of macros that prevent the codebase from being compiled to
support more than one front-end/target pair at a time.  The hardest
problems to fix, though, are the inherent architectural problems that
stem from its early design and age.  Specifically, GCC suffers from
layering problems and leaky abstractions: the back end walks front-end
ASTs to generate debug info, the front ends generate back-end data
structures, and the entire compiler depends on global data structures
set up by the command line interface.

\end{aosasect1}

%% \begin{aosasect1}{LLVM's Code Representation: LLVM IR}
\begin{aosasect1}{LLVMのコード表現: LLVM IR}

With the historical background and context out of the way, let's dive
into LLVM: The most important aspect of its design is the LLVM
Intermediate Representation (IR), which is the form it uses to
represent code in the compiler.  LLVM IR is designed to host mid-level
analyses and transformations that you find in the optimizer section of
a compiler.  It was designed with many specific goals in mind,
including supporting lightweight runtime optimizations,
cross-function/interprocedural optimizations, whole program analysis,
and aggressive restructuring transformations, etc.  The most important
aspect of it, though, is that it is itself defined as a first class
language with well-defined semantics. To make this concrete, here is a
simple example of a \code{.ll} file:

\begin{verbatim}
define i32 @add1(i32 %a, i32 %b) {
entry:
  %tmp1 = add i32 %a, %b
  ret i32 %tmp1
}

define i32 @add2(i32 %a, i32 %b) {
entry:
  %tmp1 = icmp eq i32 %a, 0
  br i1 %tmp1, label %done, label %recurse

recurse:
  %tmp2 = sub i32 %a, 1
  %tmp3 = add i32 %b, 1
  %tmp4 = call i32 @add2(i32 %tmp2, i32 %tmp3)
  ret i32 %tmp4

done:
  ret i32 %b
}
\end{verbatim}

\noindent This LLVM IR corresponds to this C code, which provides two different
ways to add integers:

\begin{verbatim}
unsigned add1(unsigned a, unsigned b) {
  return a+b;
}

// Perhaps not the most efficient way to add two numbers.
unsigned add2(unsigned a, unsigned b) {
  if (a == 0) return b;
  return add2(a-1, b+1);
}
\end{verbatim}

\noindent As you can see from this example, LLVM IR is a low-level RISC-like
virtual instruction set.  Like a real RISC instruction set, it
supports linear sequences of simple instructions like add, subtract,
compare, and branch.  These instructions are in three address form,
which means that they take some number of inputs and produce a result
in a different register.\footnote{This is in contrast to a two-address
instruction set, like X86, which destructively updates an input
register, or one-address machines which take one explicit operand
and operate on an accumulator or the top of the stack on a stack
machine.} LLVM IR supports labels and generally looks like a weird
form of assembly language.

Unlike most RISC instruction sets, LLVM is strongly typed with a
simple type system (e.g., \code{i32} is a 32-bit integer, \code{i32**}
is a pointer to pointer to 32-bit integer) and some details of the
machine are abstracted away.  For example, the calling convention is
abstracted through \code{call} and \code{ret} instructions and
explicit arguments.  Another significant difference from machine code
is that the LLVM IR doesn't use a fixed set of named registers, it
uses an infinite set of temporaries named with a \% character.

Beyond being implemented as a language, LLVM IR is actually defined in
three isomorphic forms: the textual format above, an in-memory data
structure inspected and modified by optimizations themselves, and an
efficient and dense on-disk binary ``bitcode'' format.  The LLVM
Project also provides tools to convert the on-disk format from text to
binary: \code{llvm-as} assembles the textual \code{.ll} file into a
\code{.bc} file containing the bitcode goop and \code{llvm-dis} turns a
\code{.bc} file into a \code{.ll} file.

The intermediate representation of a compiler is interesting because
it can be a ``perfect world'' for the compiler optimizer: unlike the
front end and back end of the compiler, the optimizer isn't constrained
by either a specific source language or a specific target machine.  On
the other hand, it has to serve both well: it has to be designed to be
easy for a front end to generate and be expressive enough to allow
important optimizations to be performed for real targets.

%% \begin{aosasect2}{Writing an LLVM IR Optimization}
\begin{aosasect2}{LLVM IRの最適化}

To give some intuition for how optimizations work, it is useful to
walk through some examples.  There are lots of different kinds of
compiler optimizations, so it is hard to provide a recipe for how to
solve an arbitrary problem.  That said, most optimizations follow a
simple three-part structure:

\begin{aosaitemize}

  \item Look for a pattern to be transformed.

  \item Verify that the transformation is safe/correct for the matched
  instance.

  \item Do the transformation, updating the code.

\end{aosaitemize}

The most trivial optimization is pattern matching on arithmetic
identities, such as: for any integer \code{X}, \code{X-X} is 0,
\code{X-0} is \code{X}, \code{(X*2)-X} is \code{X}.  The first
question is what these look like in LLVM IR\@.  Some examples are:

\begin{verbatim}
...
%example1 = sub i32 %a, %a
...
%example2 = sub i32 %b, 0
...
%tmp = mul i32 %c, 2
%example3 = sub i32 %tmp, %c
...
\end{verbatim}

For these sorts of ``peephole'' transformations, LLVM provides an
instruction simplification interface that is used as utilities by
various other higher level transformations.  These particular
transformations are in the \code{SimplifySubInst} function and look
like this:

\begin{verbatim}
// X - 0 -> X
if (match(Op1, m_Zero()))
  return Op0;

// X - X -> 0
if (Op0 == Op1)
  return Constant::getNullValue(Op0->getType());

// (X*2) - X -> X
if (match(Op0, m_Mul(m_Specific(Op1), m_ConstantInt<2>())))
  return Op1;

...

return 0;  // Nothing matched, return null to indicate no transformation.
\end{verbatim}

\noindent In this code, Op0 and Op1 are bound to the left and right operands of
an integer subtract instruction (importantly, these identities don't
necessarily hold for IEEE floating point!).  LLVM is implemented in
C++, which isn't well known for its pattern matching capabilities
(compared to functional languages like Objective Caml), but it does
offer a very general template system that allows us to implement
something similar.  The \code{match} function and the \code{m\_}
functions allow us to perform declarative pattern matching operations
on LLVM IR code.  For example, the \code{m\_Specific} predicate only
matches if the left hand side of the multiplication is the same as Op1.

Together, these three cases are all pattern matched and the function
returns the replacement if it can, or a null pointer if no replacement
is possible.  The caller of this function (\code{SimplifyInstruction})
is a dispatcher that does a switch on the instruction opcode,
dispatching to the per-opcode helper functions.  It is called from
various optimizations. A simple driver looks like this:

\begin{verbatim}
for (BasicBlock::iterator I = BB->begin(), E = BB->end(); I != E; ++I)
  if (Value *V = SimplifyInstruction(I))
    I->replaceAllUsesWith(V);
\end{verbatim}

\noindent This code simply loops over each instruction in a block, checking to
see if any of them simplify.  If so (because
\code{SimplifyInstruction} returns non-null), it uses the
\code{replaceAllUsesWith} method to update anything in the code using
the simplifiable operation with the simpler form.

\end{aosasect2}

\end{aosasect1}

%% \begin{aosasect1}{LLVM's Implementation of Three-Phase Design}
\begin{aosasect1}{LLVMでの3フェーズ設計の実装}

In an LLVM-based compiler, a front end is responsible for parsing,
validating and diagnosing errors in the input code, then translating
the parsed code into LLVM IR (usually, but not always, by building an
AST and then converting the AST to LLVM IR).  This IR is optionally
fed through a series of analysis and optimization passes which improve
the code, then is sent into a code generator to produce native machine
code, as shown in \aosafigref{fig.llvm.lcom}. This is a very
straightforward implementation of the three-phase design, but this
simple description glosses over some of the power and flexibility
that the LLVM architecture derives from LLVM IR.

%% \aosafigure{../images/llvm/LLVMCompiler1.eps}{LLVM's Implementation of the Three-Phase Design}{fig.llvm.lcom}
\aosafigure{../images/llvm/LLVMCompiler1.eps}{LLVMでの3フェーズ設計の実装}{fig.llvm.lcom}

%% \begin{aosasect2}{LLVM IR is a Complete Code Representation}
\begin{aosasect2}{LLVM IRは完全なコード表現である}

In particular, LLVM IR is both well specified and the \emph{only}
interface to the optimizer.  This property means that all you need to
know to write a front end for LLVM is what LLVM IR is, how it works,
and the invariants it expects.  Since LLVM IR has a first-class
textual form, it is both possible and reasonable to build a front end
that outputs LLVM IR as text, then uses Unix pipes to send it through
the optimizer sequence and code generator of your choice.

It might be surprising, but this is actually a pretty novel property
to LLVM and one of the major reasons for its success in a broad range
of different applications.  Even the widely successful and relatively
well-architected GCC compiler does not have this property: its GIMPLE
mid-level representation is not a self-contained representation.  As a
simple example, when the GCC code generator goes to emit DWARF debug
information, it reaches back and walks the source level ``tree'' form.
GIMPLE itself uses a ``tuple'' representation for the operations in
the code, but (at least as of GCC 4.5) still represents operands as
references back to the source level tree form.

The implications of this are that front-end authors need to know and
produce GCC's tree data structures as well as GIMPLE to write a GCC
front end.  The GCC back end has similar problems, so they also need to
know bits and pieces of how the RTL back end works as well.  Finally,
GCC doesn't have a way to dump out ``everything representing my
code'', or a way to read and write GIMPLE (and the related data
structures that form the representation of the code) in text form.
The result is that it is relatively hard to experiment with GCC, and
therefore it has relatively few front ends.

\end{aosasect2}

%% \begin{aosasect2}{LLVM is a Collection of Libraries}
\begin{aosasect2}{LLVMはライブラリ群である}

After the design of LLVM IR, the next most important
aspect of LLVM is that it is designed as a set of libraries, rather than as a
monolithic command line compiler like GCC or an opaque virtual machine
like the JVM or .NET virtual machines.  LLVM is an
infrastructure, a collection of useful compiler technology that can be
brought to bear on specific problems (like building a C compiler, or
an optimizer in a special effects pipeline).  While one of its most
powerful features, it is also one of its least understood design
points.

Let's look at the design of the optimizer as an example: it reads LLVM
IR in, chews on it a bit, then emits LLVM IR which hopefully will
execute faster.  In LLVM (as in many other compilers) the optimizer is
organized as a pipeline of distinct optimization passes each of which
is run on the input and has a chance to do something.  Common examples
of passes are the inliner (which substitutes the body of a function
into call sites), expression reassociation, loop invariant code
motion, etc.  Depending on the optimization level, different passes
are run: for example at -O0 (no optimization) the Clang compiler runs
no passes, at -O3 it runs a series of 67 passes in its optimizer (as
of LLVM 2.8).

Each LLVM pass is written as a C++ class that derives (indirectly)
from the \code{Pass} class.  Most passes are written in a single
\code{.cpp} file, and their subclass of the \code{Pass} class is
defined in an anonymous namespace (which makes it completely private
to the defining file).  In order for the pass to be useful, code
outside the file has to be able to get it, so a single function (to
create the pass) is exported from the file.  Here is a slightly
simplified example of a pass to make things concrete.\footnote{ For
all the details, please see \emph{Writing an LLVM Pass manual} at
\url{http://llvm.org/docs/WritingAnLLVMPass.html}.  }

\begin{verbatim}
namespace {
  class Hello : public FunctionPass {
  public:
    // Print out the names of functions in the LLVM IR being optimized.
    virtual bool runOnFunction(Function &F) {
      cerr << "Hello: " << F.getName() << "\n";
      return false;
    }
  };
}

FunctionPass *createHelloPass() { return new Hello(); }
\end{verbatim}

As mentioned, the LLVM optimizer provides dozens of different passes,
each of which are written in a similar style.  These passes are
compiled into one or more \code{.o} files, which are then built into a
series of archive libraries (\code{.a} files on Unix systems). These
libraries provide all sorts of analysis and transformation
capabilities, and the passes are as loosely coupled as possible: they
are expected to stand on their own, or explicitly declare their
dependencies among other passes if they depend on some other analysis
to do their job.  When given a series of passes to run, the LLVM
PassManager uses the explicit dependency information to satisfy these
dependencies and optimize the execution of passes.

Libraries and abstract capabilities are great, but they don't actually
solve problems.  The interesting bit comes when someone wants to build
a new tool that can benefit from compiler technology, perhaps a JIT
compiler for an image processing language.  The implementer of this
JIT compiler has a set of constraints in mind: for example, perhaps
the image processing language is highly sensitive to compile-time
latency and has some idiomatic language properties that are important
to optimize away for performance reasons.

The library-based design of the LLVM optimizer allows our implementer
to pick and choose both the order in which passes execute, and which
ones make sense for the image processing domain: if everything is
defined as a single big function, it doesn't make sense to waste time
on inlining.  If there are few pointers, alias analysis and memory
optimization aren't worth bothering about.  However, despite our best
efforts, LLVM doesn't magically solve all optimization problems!
Since the pass subsystem is modularized and the PassManager itself
doesn't know anything about the internals of the passes, the
implementer is free to implement their own language-specific passes to
cover for deficiencies in the LLVM optimizer or to explicit
language-specific optimization opportunities.
\aosafigref{fig.llvn.pass} shows a simple example for our hypothetical
XYZ image processing system:

%% \aosafigure[300pt]{../images/llvm/PassLinkage.eps}{Hypothetical XYZ System using LLVM}{fig.llvn.pass}
\aosafigure[300pt]{../images/llvm/PassLinkage.eps}{LLVMを使った想像上のXYZシステム}{fig.llvn.pass}

Once the set of optimizations is chosen (and similar decisions are
made for the code generator) the image processing compiler is built
into an executable or dynamic library.  Since the only reference to
the LLVM optimization passes is the simple \code{create} function
defined in each \code{.o} file, and since the optimizers live in
\code{.a} archive libraries, only the optimization passes \emph{that
are actually used} are linked into the end application, not the
entire LLVM optimizer.  In our example above, since there is a
reference to PassA and PassB, they will get linked in.  Since PassB
uses PassD to do some analysis, PassD gets linked in.  However, since
PassC (and dozens of other optimizations) aren't used, its code isn't
linked into the image processing application.

This is where the power of the library-based design of LLVM comes into
play. This straightforward design approach allows LLVM to provide a
vast amount of capability, some of which may only be useful to
specific audiences, without punishing clients of the libraries that
just want to do simple things. In contrast, traditional compiler
optimizers are built as a tightly interconnected mass of code, which
is much more difficult to subset, reason about, and come up to speed
on. With LLVM you can understand individual optimizers without knowing
how the whole system fits together.

This library-based design is also the reason why so many people
misunderstand what LLVM is all about: the LLVM libraries have many
capabilities, but they don't actually \emph{do} anything by themselves.
It is up to the designer of the client of the libraries (e.g., the
Clang C compiler) to decide how to put the pieces to best use.  This
careful layering, factoring, and focus on subset-ability is also why
the LLVM optimizer can be used for such a broad range of different
applications in different contexts.  Also, just because LLVM provides
JIT compilation capabilities, it doesn't mean that every client uses
it.

\end{aosasect2}

\end{aosasect1}

%% \begin{aosasect1}{Design of the Retargetable LLVM Code Generator}
\begin{aosasect1}{再利用性を考慮したLLVMコードジェネレータの設計}

The LLVM code generator is responsible for transforming LLVM IR into
target specific machine code.  On the one hand, it is the code
generator's job to produce the best possible machine code for any
given target.  Ideally, each code generator should be completely
custom code for the target, but on the other hand, the code generators
for each target need to solve very similar problems.  For example,
each target needs to assign values to registers, and though each
target has different register files, the algorithms used should be
shared wherever possible.

Similar to the approach in the optimizer, LLVM's code generator splits
the code generation problem into individual passes---instruction
selection, register allocation, scheduling, code layout optimization,
and assembly emission---and provides many builtin passes that are run
by default.  The target author is then given the opportunity to choose
among the default passes, override the defaults and implement
completely custom target-specific passes as required.  For example,
the x86 back end uses a register-pressure-reducing scheduler since it
has very few registers, but the PowerPC back end uses a latency
optimizing scheduler since it has many of them.  The x86 back end uses
a custom pass to handle the x87 floating point stack, and the ARM
back end uses a custom pass to place constant pool islands inside
functions where needed.  This flexibility allows target authors to
produce great code without having to write an entire code generator
from scratch for their target.

%% \begin{aosasect2}{LLVM Target Description Files}
\begin{aosasect2}{LLVMターゲット記述ファイル}

The ``mix and match'' approach allows target authors to choose what
makes sense for their architecture and permits a large amount of code
reuse across different targets.  This brings up another challenge:
each shared component needs to be able to reason about target specific
properties in a generic way. For example, a shared register allocator
needs to know the register file of each target and the constraints
that exist between instructions and their register operands.  LLVM's
solution to this is for each target to provide a target description in
a declarative domain-specific language (a set of \code{.td} files)
processed by the tblgen tool.  The (simplified) build process for the
x86 target is shown in \aosafigref{fig.llvm.x86}.

%% \aosafigure{../images/llvm/X86Target.eps}{Simplified x86 Target Definition}{fig.llvm.x86}
\aosafigure{../images/llvm/X86Target.eps}{単純化したx86ターゲット定義}{fig.llvm.x86}

The different subsystems supported by the \code{.td} files allow
target authors to build up the different pieces of their target. For
example, the x86 back end defines a register class that holds all of
its 32-bit registers named ``GR32'' (in the \code{.td} files, target
specific definitions are all caps) like this:

\begin{verbatim}
def GR32 : RegisterClass<[i32], 32,
  [EAX, ECX, EDX, ESI, EDI, EBX, EBP, ESP,
   R8D, R9D, R10D, R11D, R14D, R15D, R12D, R13D]> { ... }
\end{verbatim}

\noindent This definition says that registers in this class can hold 32-bit
integer values (``i32''), prefer to be 32-bit aligned, have the
specified 16 registers (which are defined elsewhere in the \code{.td}
files) and have some more information to specify preferred allocation
order and other things.  Given this definition, specific instructions
can refer to this, using it as an operand.  For example, the
``complement a 32-bit register'' instruction is defined as:

\begin{verbatim}
let Constraints = "$src = $dst" in
def NOT32r : I<0xF7, MRM2r,
               (outs GR32:$dst), (ins GR32:$src),
               "not{l}\t$dst",
               [(set GR32:$dst, (not GR32:$src))]>;
\end{verbatim}

\noindent This definition says that NOT32r is an instruction (it uses the
\code{I} tblgen class), specifies encoding information (\code{0xF7,
  MRM2r}), specifies that it defines an ``output'' 32-bit register
\code{\$dst} and has a 32-bit register ``input'' named \code{\$src}
(the \code{GR32} register class defined above defines which registers
are valid for the operand), specifies the assembly syntax for the
instruction (using the \code{\{\}} syntax to handle both AT\&T and
Intel syntax), specifies the effect of the instruction and provides
the pattern that it should match on the last line.  The ``let''
constraint on the first line tells the register allocator that the
input and output register must be allocated to the same physical
register.

This definition is a very dense description of the instruction, and
the common LLVM code can do a lot with information derived from it (by
the \code{tblgen} tool).  This one definition is enough for
instruction selection to form this instruction by pattern matching on
the input IR code for the compiler.  It also tells the register allocator how
to process it, is enough to encode and decode the instruction to
machine code bytes, and is enough to parse and print the instruction
in a textual form.  These capabilities allow the x86 target to support
generating a stand-alone x86 assembler (which is a drop-in replacement
for the ``gas'' GNU assembler) and disassemblers from the target
description as well as handle encoding the instruction for the JIT.

In addition to providing useful functionality, having multiple pieces
of information generated from the same ``truth'' is good for other
reasons.  This approach makes it almost infeasible for the assembler
and disassembler to disagree with each other in either assembly syntax
or in the binary encoding.  It also makes the target description
easily testable: instruction encodings can be unit tested without
having to involve the entire code generator.

While we aim to get as much target information as possible into the
\code{.td} files in a nice declarative form, we still don't have
everything. Instead, we require target authors to write some C++ code
for various support routines and to implement any target specific
passes they might need (like \code{X86FloatingPoint.cpp}, which
handles the x87 floating point stack).  As LLVM continues to grow new
targets, it becomes more and more important to increase the amount of
the target that can be expressed in the \code{.td} file, and we
continue to increase the expressiveness of the \code{.td} files to
handle this.  A great benefit is that it gets easier and easier write
targets in LLVM as time goes on.

\end{aosasect2}

\end{aosasect1}

%% \begin{aosasect1}{Interesting Capabilities Provided by a Modular Design}
\begin{aosasect1}{モジュラー設計がもたらす興味深い可能性}

Besides being a generally elegant design, modularity provides clients
of the LLVM libraries with several interesting capabilities. These
capabilities stem from the fact that LLVM provides functionality, but
lets the client decide most of the \emph{policies} on how to use it.

%% \begin{aosasect2}{Choosing When and Where Each Phase Runs}
\begin{aosasect2}{各フェーズがいつどこで走るかを選ぶ}

As mentioned earlier, LLVM IR can be efficiently (de)serialized
to/from a binary format known as LLVM bitcode.  Since LLVM IR is
self-contained, and serialization is a lossless process, we can do
part of compilation, save our progress to disk, then continue work at
some point in the future.  This feature provides a number of
interesting capabilities including support for link-time and
install-time optimization, both of which delay code generation from
``compile time''.

Link-Time Optimization (LTO) addresses the problem where the compiler
traditionally only sees one translation unit (e.g., a \code{.c} file
with all its headers) at a time and therefore cannot do optimizations
(like inlining) across file boundaries.  LLVM compilers like Clang
support this with the \code{-flto} or \code{-O4} command line option.
This option instructs the compiler to emit LLVM bitcode to the
\code{.o}file instead of writing out a native object file, and delays
code generation to link time, shown in \aosafigref{fig.llvm.lto}.

%% \aosafigure{../images/llvm/LTO.eps}{Link-Time Optimization}{fig.llvm.lto}
\aosafigure{../images/llvm/LTO.eps}{リンク時の最適化}{fig.llvm.lto}

Details differ depending on which operating system you're on, but the
important bit is that the linker detects that it has LLVM bitcode in
the \code{.o} files instead of native object files.  When it sees
this, it reads all the bitcode files into memory, links them together,
then runs the LLVM optimizer over the aggregate.  Since the optimizer
can now see across a much larger portion of the code, it can inline,
propagate constants, do more aggressive dead code elimination, and
more across file boundaries.  While many modern compilers support LTO,
most of them (e.g., GCC, Open64, the Intel compiler, etc.) do so by
having an expensive and slow serialization process.  In LLVM, LTO
falls out naturally from the design of the system, and works across
different source languages (unlike many other compilers) because the
IR is truly source language neutral.

Install-time optimization is the idea of delaying code generation even
later than link time, all the way to install time, as shown in
\aosafigref{fig.llvm.ito}.  Install time is a very interesting time
(in cases when software is shipped in a box, downloaded, uploaded to a
mobile device, etc.), because this is when you find out the specifics
of the device you're targeting.  In the x86 family for example, there
are broad variety of chips and characteristics.  By delaying
instruction choice, scheduling, and other aspects of code generation,
you can pick the best answers for the specific hardware an application
ends up running on.

%% \aosafigure{../images/llvm/InstallTime.eps}{Install-Time Optimization}{fig.llvm.ito}
\aosafigure{../images/llvm/InstallTime.eps}{インストール時の最適化}{fig.llvm.ito}

\end{aosasect2}

\vspace{-.5cm}
%% \begin{aosasect2}{Unit Testing the Optimizer}
\begin{aosasect2}{オプティマイザのユニットテスト}

Compilers are very complicated, and quality is important, therefore
testing is critical.  For example, after fixing a bug that caused a
crash in an optimizer, a regression test should be added to make sure
it doesn't happen again.  The traditional approach to testing this is
to write a \code{.c} file (for example) that is run through the
compiler, and to have a test harness that verifies that the compiler
doesn't crash.  This is the approach used by the GCC test suite, for
example.

The problem with this approach is that the compiler consists of many
different subsystems and even many different passes in the optimizer,
all of which have the opportunity to change what the input code looks
like by the time it gets to the previously buggy code in question.  If
something changes in the front end or an earlier optimizer, a test
case can easily fail to test what it is supposed to be testing.

By using the textual form of LLVM IR with the modular optimizer, the
LLVM test suite has highly focused regression tests that can load LLVM
IR from disk, run it through exactly one optimization pass, and verify
the expected behavior.  Beyond crashing, a more complicated behavioral
test wants to verify that an optimization is actually performed.  Here
is a simple test case that checks to see that the constant propagation
pass is working with add instructions:

\begin{verbatim}
; RUN: opt < %s -constprop -S | FileCheck %s
define i32 @test() {
  %A = add i32 4, 5
  ret i32 %A
  ; CHECK: @test()
  ; CHECK: ret i32 9
}
\end{verbatim}

\noindent The \code{RUN} line specifies the command to execute: in this case,
the \code{opt} and \code{FileCheck} command line tools.  The
\code{opt} program is a simple wrapper around the LLVM pass manager,
which links in all the standard passes (and can dynamically load
plugins containing other passes) and exposes them through to the
command line.  The \code{FileCheck} tool verifies that its standard
input matches a series of \code{CHECK} directives.  In this case, this
simple test is verifying that the \code{constprop} pass is folding the
\code{add} of 4 and 5 into 9.

While this might seem like a really trivial example, this is very
difficult to test by writing .c files: front ends often do constant
folding as they parse, so it is very difficult and fragile to write
code that makes its way downstream to a constant folding optimization
pass.  Because we can load LLVM IR as text and send it through the
specific optimization pass we're interested in, then dump out the
result as another text file, it is really straightforward to test
exactly what we want, both for regression and feature tests.

\end{aosasect2}

%% \begin{aosasect2}{Automatic Test Case Reduction with BugPoint}
\begin{aosasect2}{BugPointによる、自動的なテストケースの削減}

When a bug is found in a compiler or other client of the LLVM
libraries, the first step to fixing it is to get a test case that
reproduces the problem.  Once you have a test case, it is best to
minimize it to the smallest example that reproduces the problem, and
also narrow it down to the part of LLVM where the problem happens,
such as the optimization pass at fault.  While you eventually learn
how to do this, the process is tedious, manual, and particularly
painful for cases where the compiler generates incorrect code but does
not crash.

The LLVM BugPoint
tool\footnote{\url{http://llvm.org/docs/Bugpoint.html}} uses the IR
serialization and modular design of LLVM to automate this process.
For example, given an input \code{.ll} or \code{.bc} file along with a
list of optimization passes that causes an optimizer crash, BugPoint
reduces the input to a small test case and determines which optimizer
is at fault.  It then outputs the reduced test case and the \code{opt}
command used to reproduce the failure.  It finds this by using
techniques similar to ``delta debugging'' to reduce the input and the
optimizer pass list.  Because it knows the structure of LLVM IR,
BugPoint does not waste time generating invalid IR to input to the
optimizer, unlike the standard ``delta'' command line tool.

In the more complex case of a miscompilation, you can specify the
input, code generator information, the command line to pass to the
executable, and a reference output. BugPoint will first determine if
the problem is due to an optimizer or a code generator, and will then
repeatedly partition the test case into two pieces: one that is sent
into the ``known good'' component and one that is sent into the
``known buggy'' component.  By iteratively moving more and more code
out of the partition that is sent into the known buggy code generator,
it reduces the test case.

BugPoint is a very simple tool and has saved countless hours of test
case reduction throughout the life of LLVM\@.  No other open source
compiler has a similarly powerful tool, because it relies on a
well-defined intermediate representation.  That said, BugPoint isn't
perfect, and would benefit from a rewrite.  It dates back to 2002, and
is typically only improved when someone has a really tricky bug to
track down that the existing tool doesn't handle well. It has grown
over time, accreting new features (such as JIT debugging) without a
consistent design or owner.

\end{aosasect2}

\end{aosasect1}

%% \begin{aosasect1}{Retrospective and Future Directions}
\begin{aosasect1}{ふりかえりと今後の方針}

LLVM's modularity wasn't originally designed to directly achieve any
of the goals described here. It was a self-defense mechanism: it was
obvious that we wouldn't get everything right on the first try.  The
modular pass pipeline, for example, exists to make it easier to
isolate passes so that they can be discarded after being replaced by
better implementations\footnote{I often say that none of the
subsystems in LLVM are really good until they have been rewritten at
least once.}.

Another major aspect of LLVM remaining nimble (and a controversial
topic with clients of the libraries) is our willingness to reconsider
previous decisions and make widespread changes to APIs without
worrying about backwards compatibility.  Invasive changes to LLVM IR
itself, for example, require updating all of the optimization passes
and cause substantial churn to the C++ APIs.  We've done this on
several occasions, and though it causes pain for clients, it is the
right thing to do to maintain rapid forward progress.  To make life
easier for external clients (and to support bindings for other
languages), we provide C wrappers for many popular APIs (which are
intended to be extremely stable) and new versions of LLVM aim to
continue reading old \code{.ll} and \code{.bc} files.

Looking forward, we would like to continue making LLVM more modular
and easier to subset.  For example, the code generator is still too
monolithic: it isn't currently possible to subset LLVM based on
features.  For example, if you'd like to use the JIT, but have no need
for inline assembly, exception handling, or debug information
generation, it should be possible to build the code generator without
linking in support for these features.  We are also continuously
improving the quality of code generated by the optimizer and code
generator, adding IR features to better support new language and
target constructs, and adding better support for performing high-level
language-specific optimizations in LLVM.

The LLVM project continues to grow and improve in numerous ways.  It
is really exciting to see the number of different ways that LLVM is
being used in other projects and how it keeps turning up in surprising
new contexts that its designers never even thought about.  The new
LLDB debugger is a great example of this: it uses the
C/C++/Objective-C parsers from Clang to parse expressions, uses the
LLVM JIT to translate these into target code, uses the LLVM
disassemblers, and uses LLVM targets to handle calling conventions
among other things.  Being able to reuse this existing code allows
people developing debuggers to focus on writing the debugger logic, instead of
reimplementing yet another (marginally correct) C++ parser.

Despite its success so far, there is still a lot left to be done, as
well as the ever-present risk that LLVM will become less nimble and
more calcified as it ages.  While there is no magic answer to this
problem, I hope that the continued exposure to new problem domains, a
willingness to reevaluate previous decisions, and to
redesign and throw away code will help.  After all, the goal isn't to
be perfect, it is to keep getting better over time.

\end{aosasect1}

\end{aosachapter}
