\begin{aosachapter}{Python Packaging}{s:packaging}{Tarek Ziad\'{e}}
%% Based on EN-Revision r375

%% \begin{aosasect1}{Introduction}
\begin{aosasect1}{はじめに}

There are two schools of thought when it comes to installing
applications.  The first, common to Windows and Mac OS X, is that
applications should be self-contained, and their installation should
not depend on anything else.  This philosophy simplifies the
management of applications: each application is its own standalone
``appliance'', and installing and removing them should not disturb the
rest of the OS\@.  If the application needs an uncommon library, that
library is included in the application's distribution.

The second school, which is the norm for Linux-based systems, treats
software as a collection of small self-contained units called
\emph{packages}. Libraries are bundled into packages, any given
library package might depend on other packages.
Installing an application might involve finding and installing
particular versions of dozens of other libraries.  These dependencies
are usually fetched from a central repository that contains thousands
of packages.  This philosophy is why Linux distributions use complex
package management systems like \code{dpkg} and \code{RPM} to track
dependencies and prevent installation of two applications that use
incompatible versions of the same library.

There are pros and cons to each approach.  Having a highly modular
system where every piece can be updated or replaced makes management
easier, because each library is present in a single place, and all
applications that use it benefit when it is updated.  For instance, a
security fix in a particular library will reach all applications that
use it at once, whereas if an application ships with its own library,
that security fix will be more complex to deploy, especially if
different applications use different versions of the library.

But that modularity is seen as a drawback by some developers, because
they're not in control of their applications and dependencies.  It is
easier for them to provide a standalone software appliance to be sure
that the application environment is stable and not subject to
``dependency hell'' during system upgrades.

Self-contained applications also make the developer's life easier when
she needs to support several operating systems.  Some projects go so
far as to release portable applications that remove \emph{any}
interaction with the hosting system by working in a self-contained
directory, even for log files.

Python's packaging system was intended to make the second
philosophy---multiple dependencies for each install---as developer-,
admin-, packager-, and user-friendly as possible. Unfortunately it had
(and has) a variety of flaws which caused or allowed all kinds of
problems: unintuitive version schemes, mishandled data files,
difficulty re-packaging, and more.  Three years ago I and a group of
other Pythoneers decided to reinvent it to address these problems.  We
call ourselves the Fellowship of the Packaging, and this chapter
describes the problems we have been trying to fix, and what our
solution looks like.

%% \begin{aosabox}{Terminology}
\begin{aosabox}{用語}

In Python a \emph{package} is a directory containing Python
files. Python files are called \emph{modules}.  That definition makes
the usage of the word ``package'' a bit vague since it is also used by
many systems to refer to a \emph{release} of a project.

Python developers themselves are sometimes vague about this.  One way
to remove this ambiguity is to use the term ``Python packages'' when
we talk about a directory containing Python modules.  The term
``release'' is used to define one version of a project, and the term
``distribution'' defines a source or a binary distribution of a
release as something like a tarball or zip file.

\end{aosabox}

\end{aosasect1}

%% \begin{aosasect1}{The Burden of the Python Developer}
\begin{aosasect1}{Python開発者の苦悩}

Most Python programmers want their programs to be usable in any
environment.  They also usually want to use a mix of standard Python
libraries and system-dependent libraries.  But unless you package your
application separately for every existing packaging system, you are
doomed to provide Python-specific releases---a Python-specific release
is a release aimed to be installed within a Python installation no
matter what the underlying Operating System is---and hope that:

\begin{aosaitemize}

  \item packagers for every target system will be able to repackage
  your work,

  \item the dependencies you have will themselves be repackaged in
  every target system, and

  \item system dependencies will be clearly described.

\end{aosaitemize}

Sometimes, this is simply impossible.  For example, Plone (a
full-fledged Python-powered CMS) uses hundreds of small pure Python
libraries that are not always available as packages in every packaging
system out there.  This means that Plone \emph{must} ship everything
that it needs in a portable application. To do this, it uses
\code{zc.buildout}, which collects all its dependencies and creates a
portable application that will run on any system within a single
directory. It is effectively a binary release, since any piece of C
code will be compiled in place.

This is a big win for developers: they just have to describe their
dependencies using the Python standards described below
and use \code{zc.buildout} to
release their application.  But as discussed earlier, this type of
release sets up a fortress within the system, which most Linux sysadmins
will hate.  Windows admins won't mind, but those managing
CentOS or Debian will, because those systems base their management on
the assumption that every file in the system is registered,
classified, and known to admin tools.

Those admins will want to repackage your application according to
their own standards.  The question we need to answer is, ``Can Python
have a packaging system that can be automatically translated into
other packaging systems?''  If so, one application or library can be
installed on any system without requiring extra packaging work.  Here,
``automatically'' doesn't necessarily mean that the work should be
fully done by a script: \code{RPM} or \code{dpkg} packagers will tell
you that's impossible---they always need to add some specifics in the
projects they repackage.  They'll also tell you that they
often have a hard time re-packaging a piece of code because its
developers were not aware of a few basic packaging rules.

Here's one example of what you can do to annoy packagers using the
existing Python packaging system: release a library called
``MathUtils'' with the version name ``Fumanchu''.  The brilliant
mathematician who wrote the library have found it amusing to use
his cats' names for his project versions.  But how can a packager know
that ``Fumanchu'' is his second cat's name, and that the first one was
called ``Phil'', so that the ``Fumanchu'' version comes after the
``Phil'' one?

This may sound extreme, but it can happen with today's tools and
standards.  The worst thing is that tools like \code{easy\_install} or
\code{pip} use their own non-standard registry to keep track of
installed files, and will sort the ``Fumanchu'' and ``Phil'' versions
alphanumerically.

Another problem is how to handle data files.  For example, what if
your application uses an SQLite database?  If you put it inside your
package directory, your application might fail because the system
forbids you to write in that part of the tree.  Doing this will
also compromise the assumptions Linux systems make about where
application data is for backups (\code{/var}).

In the real world, system administrators need to be able to place your
files where they want without breaking your application, and you need
to tell them what those files are.  So let's rephrase the question: is
it possible to have a packaging system in Python that can provide all
the information needed to repackage an application with any
third-party packaging system out there without having to read the
code, and make everyone happy?

\end{aosasect1}

%% \begin{aosasect1}{The Current Architecture of Packaging}
\begin{aosasect1}{現在のパッケージングのアーキテクチャ}

The \code{Distutils} package that comes with the Python standard
library is riddled with the problems described above.  Since it's the
standard, people either live with it and its flaws, or use more
advanced tools like \code{Setuptools}, which add features on the top of it,
or \code{Distribute}, a fork of \code{Setuptools}. There's also \code{Pip},
a more advanced installer, that relies on \code{Setuptools}.

However, these newer tools are all based on \code{Distutils} and inherit its
problems.  Attempts were made to fix \code{Distutils} in place, but
the code is so deeply used by other tools that any change to it, even
its internals, is a potential regression in the whole Python packaging
ecosystem.

We therefore decided to freeze \code{Distutils} and start the
development of \code{Distutils2} from the same code base, without
worrying too much about backward compatibility.  To understand what
changed and why, let's have a closer look at \code{Distutils}.

%% \begin{aosasect2}{Distutils Basics and Design Flaws}
\begin{aosasect2}{Distutilsの基本とその設計ミス}
\label{sec.packaging.flaws}

\code{Distutils} contains commands, each of which is a class with a
\code{run} method that can be called with some options.  \code{Distutils}
also provides a \code{Distribution} class that contains global values
every command can look at.

To use \code{Distutils}, a developer adds a single Python module to a
project, conventionally called \code{setup.py}. This module contains a
call to \code{Distutils}' main entry point: the \code{setup} function.  This
function can take many options, which are held by a
\code{Distribution} instance and used by commands.  Here's an example
that defines a few standard options like the name and version of the
project, and a list of modules it contains:

\pagebreak

\begin{verbatim}
from distutils.core import setup

setup(name='MyProject', version='1.0', py_modules=['mycode.py'])
\end{verbatim}

\noindent This module can then be used to run \code{Distutils} commands like
\code{sdist}, which creates a source distribution in an archive and
places it in a \code{dist} directory:

\begin{verbatim}
$ python setup.py sdist
\end{verbatim}

\noindent Using the same script, you can install the project using the
\code{install} command:

\begin{verbatim}
$ python setup.py install
\end{verbatim}

\code{Distutils} provides other commands such as:

\begin{aosaitemize}

  \item \code{upload} to upload a distribution into an online repository.

  \item \code{register} to register the metadata of a project in an online
  repository without necessary uploading a distribution,

  \item \code{bdist} to creates a binary distribution, and

  \item \code{bdist\_msi} to create a \code{.msi} file for Windows.

\end{aosaitemize}

\noindent
It will also let you get information about the project via other
command line options.

So installing a project or getting information about it is always done
by invoking \code{Distutils} through this file. For example, to find
out the name of the project:

\begin{verbatim}
$ python setup.py --name
MyProject
\end{verbatim}

\noindent \code{setup.py} is therefore how everyone interacts with the project,
whether to build, package, publish, or install it. The developer
describes the content of his project through options passed to a
function, and uses that file for all his packaging tasks. The file is
also used by installers to install the project on a target system.

\aosafigure[300pt]{../images/packaging/setup-py.eps}{Setup}{fig.packaging.setup}

Having a single Python module used for packaging, releasing,
\emph{and} installing a project is one of \code{Distutils}' main
flaws.  For example, if you want to get the \code{name} from the \code{lxml}
project, \code{setup.py} will do a lot of things besides returning a
simple string as expected:

\begin{verbatim}
$ python setup.py --name
Building lxml version 2.2.
NOTE: Trying to build without Cython, pre-generated 'src/lxml/lxml.etree.c'
needs to be available.
Using build configuration of libxslt 1.1.26
Building against libxml2/libxslt in the following directory: /usr/lib/lxml
\end{verbatim}

\noindent It might even fail to work on some projects, since
developers make the assumption that \code{setup.py} is used only to
install, and that other \code{Distutils} features are only used by
them during development.  The multiple roles of the \code{setup.py}
script can easily cause confusion.

\end{aosasect2}

%% \begin{aosasect2}{Metadata and PyPI}
\begin{aosasect2}{MetadataとPyPI}

When \code{Distutils} builds a distribution, it creates a
\code{Metadata} file that follows the standard described in
PEP~314\footnote{The Python Enhancement Proposals, or PEPs, that we
refer to are summarized at the end of this chapter}.  It contains a
static version of all the usual metadata, like the name of the project
or the version of the release.  The main metadata fields are:

\begin{aosaitemize}

  \item \code{Name}: The name of the project.

  \item \code{Version}: The version of the release.

  \item \code{Summary}: A one-line description.

  \item \code{Description}: A detailed description.

  \item \code{Home-Page}: The URL of the project.

  \item \code{Author}: The author name.

  \item \code{Classifiers}: Classifiers for the project. Python provides a list
  of classifiers for the license, the maturity of the release (beta,
  alpha, final), etc.

  \item \code{Requires}, \code{Provides}, and \code{Obsoletes}:
  Used to define dependencies with modules.

\end{aosaitemize}

\noindent
These fields are for the most part easy to map to equivalents in other
packaging systems.

The Python Package Index (PyPI)\footnote{Formerly known as the
CheeseShop.}, a central repository of packages like CPAN, is able to
register projects and publish releases via \code{Distutils}'
\code{register} and \code{upload} commands.  \code{register} builds
the \code{Metadata} file and sends it to PyPI, allowing people and
tools---like installers---to browse them via web pages or via web
services.

\aosafigure[400pt]{../images/packaging/pypi.eps}{The PyPI Repository}{fig.packaging.pypi}

You can browse projects by \code{Classifiers}, and get the author name
and project URL\@.  Meanwhile, \code{Requires} can be used to define
dependencies on Python modules.  The \code{requires} option can be
used to add a \code{Requires} metadata element to the project:

\begin{verbatim}
from distutils.core import setup

setup(name='foo', version='1.0', requires=['ldap'])
\end{verbatim}

Defining a dependency on the \code{ldap} module is purely declarative:
no tools or installers ensure that such a module exists.  This would be
satisfactory if Python defined requirements at the module level
through a \code{require} keyword like Perl does.  Then it would
just be a matter of the installers browsing the dependencies at PyPI
and installing them; that's basically what CPAN does.  But that's not
possible in Python since a module named \code{ldap} can exist in any
Python project. Since \code{Distutils} allows people to release
projects that can contain several packages and modules, this metadata
field is not useful at all.

Another flaw of \code{Metadata} files is that they are created by a
Python script, so they are specific to the platform they are executed
in.  For example, a project that provides features specific to Windows
could define its \code{setup.py} as:

\begin{verbatim}
from distutils.core import setup

setup(name='foo', version='1.0', requires=['win32com'])
\end{verbatim}

\noindent But this assumes that the project only works under Windows, even if it
provides portable features.  One way to solve this is to make the
\code{requires} option specific to Windows:

\begin{verbatim}
from distutils.core import setup
import sys
\end{verbatim}
\begin{verbatim}
if sys.platform == 'win32':
    setup(name='foo', version='1.0', requires=['win32com'])
else:
    setup(name='foo', version='1.0')
\end{verbatim}

\noindent This actually makes the issue worse. Remember, the script is used to
build source archives that are then released to the world via PyPI.
This means that the static \code{Metadata} file sent to PyPI is
dependent on the platform that was used to compile it.  In other
words, there is no way to indicate statically in the metadata field
that it is platform-specific.

\end{aosasect2}

%% \begin{aosasect2}{Architecture of PyPI}
\begin{aosasect2}{PyPIのアーキテクチャ}

\aosafigure[250pt]{../images/packaging/pypi-workflow.eps}{PyPI Workflow}{fig.packaging.workflow}

As indicated earlier, PyPI is a central index of Python projects where
people can browse existing projects by category or register their own
work.  Source or binary distributions can be uploaded and added to an
existing project, and then downloaded for installation or study.  PyPI
also offers web services that can be used by tools like installers.

%% \begin{aosasect3}{Registering Projects and Uploading Distributions}
\begin{aosasect3}{プロジェクトの登録と配布物のアップロード}

Registering a project to PyPI is done with the \code{Distutils}
\code{register} command.  It builds a POST request containing the
metadata of the project, whatever its version is.  The request
requires an Authorization header, as PyPI uses Basic Authentication to
make sure every registered project is associated with a user that has
first registered with PyPI\@.  Credentials are kept in the local
\code{Distutils} configuration or typed in the prompt every time a
\code{register} command is invoked.  An example of its use is:

\begin{verbatim}
$ python setup.py register
running register
Registering MPTools to http://pypi.python.org/pypi
Server response (200): OK
\end{verbatim}

\noindent Each registered project gets a web page with an HTML version of the
metadata, and packagers can upload distributions to PyPI using
\code{upload}:

\begin{verbatim}
$ python setup.py sdist upload
running sdist
...
running upload
Submitting dist/mopytools-0.1.tar.gz to http://pypi.python.org/pypi
Server response (200): OK
\end{verbatim}

It's also possible to point users to another location via the
\code{Download-URL} metadata field rather than uploading files directly to PyPI.

\end{aosasect3}

%% \begin{aosasect3}{Querying PyPI}
\begin{aosasect3}{PyPIへの問い合わせ}

Besides the HTML pages PyPI publishes for web users, it provides two
services that tools can use to browse the content: the Simple Index
protocol and the XML-RPC APIs.

The Simple Index protocol starts at
\url{http://pypi.python.org/simple/}, a plain HTML page that contains
relative links to every registered project:

\begin{verbatim}
<html><head><title>Simple Index</title></head><body>
...
<a href='MontyLingua/'>MontyLingua</a><br/>
<a href='mootiro_web/'>mootiro_web</a><br/>
<a href='Mopidy/'>Mopidy</a><br/>
<a href='mopowg/'>mopowg</a><br/>
<a href='MOPPY/'>MOPPY</a><br/>
<a href='MPTools/'>MPTools</a><br/>
<a href='morbid/'>morbid</a><br/>
<a href='Morelia/'>Morelia</a><br/>
<a href='morse/'>morse</a><br/>
...
</body></html>
\end{verbatim}

\noindent For example, the MPTools project has a \code{MPTools/} link, which
means that the project exists in the index.  The site it points at
contains a list of all the links related to the project:

\begin{aosaitemize}

  \item links for every distribution stored at PyPI

  \item links for every Home URL defined in the \code{Metadata}, for
  each version of the project registered

  \item links for every Download-URL defined in the \code{Metadata}, for
  each version as well.

\end{aosaitemize}

\noindent
The page for MPTools contains:

\begin{verbatim}
<html><head><title>Links for MPTools</title></head>
<body><h1>Links for MPTools</h1>
<a href="../../packages/source/M/MPTools/MPTools-0.1.tar.gz">MPTools-0.1.tar.gz</a><br/>
<a href="http://bitbucket.org/tarek/mopytools" rel="homepage">0.1 home_page</a><br/>
</body></html>
\end{verbatim}

\noindent
Tools like installers that want to find distributions of a project can
look for it in the index page, or simply check if
\url{http://pypi.python.org/simple/PROJECT_NAME/} exists.

This protocol has two main limitations.  First, PyPI is a single
server right now, and while people usually have local copies of its
content, we have experienced several downtimes in the past two years
that have paralyzed developers that are constantly working with
installers that browse PyPI to get all the dependencies a project
requires when it is built. For instance, building a Plone application
will generate several hundreds queries at PyPI to get all the required
bits, so PyPI may act as a single point of failure.

Second, when the distributions are not stored at PyPI and a
Download-URL link is provided in the Simple Index page, installers
have to follow that link and hope that the location will be up and
will really contain the release. These indirections weakens any Simple
Index-based process.

The Simple Index protocol's goal is to give to installers a list of
links they can use to install a project. The project metadata is not
published there; instead, there are XML-RPC methods to get extra
information about registered projects:

\begin{verbatim}
>>> import xmlrpclib
>>> import pprint
>>> client = xmlrpclib.ServerProxy('http://pypi.python.org/pypi')
>>> client.package_releases('MPTools')
['0.1']
>>> pprint.pprint(client.release_urls('MPTools', '0.1'))
[{'comment_text': '',
'downloads': 28,
'filename': 'MPTools-0.1.tar.gz',
'has_sig': False,
'md5_digest': '6b06752d62c4bffe1fb65cd5c9b7111a',
'packagetype': 'sdist',
'python_version': 'source',
'size': 3684,
'upload_time': <DateTime '20110204T09:37:12' at f4da28>,
'url': 'http://pypi.python.org/packages/source/M/MPTools/MPTools-0.1.tar.gz'}]
>>> pprint.pprint(client.release_data('MPTools', '0.1'))
{'author': 'Tarek Ziade',
'author_email': 'tarek@mozilla.com',
'classifiers': [],
'description': 'UNKNOWN',
'download_url': 'UNKNOWN',
'home_page': 'http://bitbucket.org/tarek/mopytools',
'keywords': None,
'license': 'UNKNOWN',
'maintainer': None,
'maintainer_email': None,
'name': 'MPTools',
'package_url': 'http://pypi.python.org/pypi/MPTools',
'platform': 'UNKNOWN',
'release_url': 'http://pypi.python.org/pypi/MPTools/0.1',
'requires_python': None,
'stable_version': None,
'summary': 'Set of tools to build Mozilla Services apps',
'version': '0.1'}
\end{verbatim}

\noindent The issue with this approach is that some of the data that the XML-RPC
APIs are publishing could have been stored as static files and
published in the Simple Index page to simplify the work of client
tools.  That would also avoid the extra work PyPI has to do to handle
those queries.  It's fine to have non-static data like the number of
downloads per distribution published in a specialized web service, but
it does not make sense to have to use two different services to get
all static data about a project.

\end{aosasect3}

\end{aosasect2}

%% \begin{aosasect2}{Architecture of a Python Installation}
\begin{aosasect2}{Python Installationのアーキテクチャ}

If you install a Python project using \code{python setup.py install},
\code{Distutils}---which is included in the standard library---will
copy the files onto your system.

\begin{aosaitemize}

  \item \emph{Python packages} and modules will land in the Python
  directory that is loaded when the interpreter starts: under the
  latest Ubuntu they will wind up in
  \code{/usr/local/lib/python2.6/dist\-packages/} and under Fedora in
  \code{/usr/local/lib/python2.6/sites-packages/}.

  \item \emph{Data files} defined in a project can land anywhere
  on the system.

  \item The \emph{executable script} will land in a \code{bin} directory
  on the system. Depending on the platform, this could be
  \code{/usr/local/bin} or in a bin directory specific to the Python
  installation.

\end{aosaitemize}

Ever since Python~2.5, the metadata file is copied alongside the modules
and packages as \code{project\-version.egg-info}. For example, the
\code{virtualenv} project could have a
\code{virtualenv-1.4.9.egg\-info} file.  These metadata files can be
considered a database of installed projects, since it's possible to
iterate over them and build a list of projects with their versions.
However, the \code{Distutils} installer does not record the list of
files it installs on the system.  In other words, there is no way to
remove all files that were copied in the system.  This is a shame
since the \code{install} command has a \code{--record} option that can
be used to record all installed files in a text file. However, this
option is not used by default and \code{Distutils}' documentation
barely mentions it.

\end{aosasect2}

%% \begin{aosasect2}{Setuptools, Pip and the Like}
\begin{aosasect2}{Setuptools、PipそしてLike}

As mentioned in the introduction, some projects tried to fix some of
the problems with \code{Distutils}, with varying degrees of success.

%% \begin{aosasect3}{The Dependencies Issue}
\begin{aosasect3}{依存関係の問題}

PyPI allowed developers to publish Python projects that could include
several modules organized into Python packages. But at the same time,
projects could define module-level dependencies via \code{Require}.
Both ideas are reasonable, but their combination is not.

The right thing to do was to have project-level dependencies, which is
exactly what \code{Setuptools} added as a feature on the top of
\code{Distutils}.  It also provided a script called
\code{easy\_install} to automatically fetch and install dependencies
by looking for them on PyPI\@.  In practice, module-level dependency was
never really used, and people jumped on \code{Setuptools}' extensions.
But since these features were added in options specific to
\code{Setuptools}, and ignored by \code{Distutils} or PyPI,
\code{Setuptools} effectively created its own standard and became a
hack on a top of a bad design.

\code{easy\_install} therefore needs to download the archive of the
project and run its \code{setup.py} script again to get the metadata
it needs, and it has to do this again for every dependency. The
dependency graph is built bit by bit after each download.

Even if the new metadata was accepted by PyPI and browsable online,
\code{easy\_install} would still need to download all archives
because, as said earlier, metadata published at PyPI is specific to
the platform that was used to upload it, which can differ from the
target platform.  But this ability to install a project and its
dependencies was good enough in 90\% of the cases and was a great
feature to have. So \code{Setuptools} became widely used, although it
still suffers from other problems:

\begin{aosaitemize}

  \item If a dependency install fails, there is no rollback and the
  system can end up in a broken state.

  \item The dependency graph is built on the fly during installation, so
  if a dependency conflict is encountered the system can end up in a
  broken state as well.

\end{aosaitemize}

\end{aosasect3}

%% \begin{aosasect3}{The Uninstall Issue}
\begin{aosasect3}{アンインストールの問題}

\code{Setuptools} did not provide an uninstaller, even though its
custom metadata could have contained a file listing the installed
files.  \code{Pip}, on the other hand, extended \code{Setuptools}'
metadata to record installed files, and is therefore able to
uninstall.  But that's yet another custom set of metadata, which means
that a single Python installation may contain up to four different
flavours of metadata for each installed project:

\begin{aosaitemize}

  \item \code{Distutils}' \code{egg-info}, which is a single metadata
  file.

  \item \code{Setuptools}' \code{egg-info}, which is a directory
  containing the metadata and extra \code{Setuptools} specific
  options.

  \item \code{Pip}'s \code{egg-info}, which is an extended version of
  the previous.

  \item Whatever the hosting packaging system creates.

\end{aosaitemize}

\end{aosasect3}

\end{aosasect2}

%% \begin{aosasect2}{What About Data Files?}
\begin{aosasect2}{データファイルとは?}

In \code{Distutils}, data files can be installed anywhere on the
system.  If you define some package data files in \code{setup.py}
script like this:

\begin{verbatim}
setup(...,
  packages=['mypkg'],
  package_dir={'mypkg': 'src/mypkg'},
  package_data={'mypkg': ['data/*.dat']},
  )
\end{verbatim}

\noindent then all files with the \code{.dat} extension in the \code{mypkg}
project will be included in the distribution and eventually installed
along with the Python modules in the Python installation.

For data files that need to be installed outside the Python distribution, 
there's another option that stores files in the archive but puts them in defined
locations:

\begin{verbatim}
setup(...,
    data_files=[('bitmaps', ['bm/b1.gif', 'bm/b2.gif']),
                ('config', ['cfg/data.cfg']),
                ('/etc/init.d', ['init-script'])]
    )
\end{verbatim}

\noindent
This is terrible news for OS packagers for several reasons:

\begin{aosaitemize}

  \item Data files are not part of the metadata, so packagers need to read
  \code{setup.py} and sometimes dive into the project's code.

  \item The developer should not be the one deciding where data files
  should land on a target system.

  \item There are no categories for these data files: images, \code{man}
  pages, and everything else are all treated the same way.

\end{aosaitemize}

A packager who needs to repackage a project with such a file has no
choice but to patch the \code{setup.py} file so that it works as
expected for her platform.  To do that, she must review the code
and change every line that uses those files, since the developer made
an assumption about their location.  \code{Setuptools} and \code{Pip}
did not improve this.

\end{aosasect2}

\end{aosasect1}

%% \begin{aosasect1}{Improved Standards}
\begin{aosasect1}{標準規約の改良}

So we ended up with with a mixed up and confused packaging
environment, where everything is driven by a single Python module,
with incomplete metadata and no way to describe everything a project
contains.  Here's what we're doing to make things better.

%% \begin{aosasect2}{Metadata}
\begin{aosasect2}{メタデータ}

The first step is to fix our \code{Metadata} standard.  PEP~345
defines a new version that includes:

\begin{aosaitemize}

  \item a saner way to define versions

  \item project-level dependencies

  \item a static way to define platform-specific values

\end{aosaitemize}

%% \begin{aosasect3}{Version}
\begin{aosasect3}{バージョン}

One goal of the metadata standard is to make sure that all tools that
operate on Python projects are able to classify them the same
way. For versions, it means that every tool should be able to know
that ``1.1'' comes after ``1.0''. But if project have custom
versioning schemes, this becomes much harder.

The only way to ensure consistent versioning is to publish a standard
that projects will have to follow.  The scheme we chose is a
classical sequence-based scheme.  As defined in PEP~386, its format
is:

\begin{verbatim}
N.N[.N]+[{a|b|c|rc}N[.N]+][.postN][.devN]
\end{verbatim}

\noindent where:

\begin{aosaitemize}

  \item \emph{N} is an integer. You can use as many Ns as you want and
  separate them by dots, as long as there are at least two
  (MAJOR.MINOR).

  \item \emph{a}, \emph{b}, \emph{c} and \emph{rc} are \emph{alpha},
  \emph{beta} and \emph{release candidate} markers. They are followed
  by an integer. Release candidates have two markers because we wanted
  the scheme to be compatible with Python, which uses \emph{rc}. But
  we find \emph{c} simpler.

  \item \emph{dev} followed by a number is a dev marker.

  \item \emph{post} followed by a number is a post-release marker.

\end{aosaitemize}

\noindent
Depending on the project release process, dev or post markers can be used for
all intermediate versions between two final releases. Most process use
dev markers.

\pagebreak

Following this scheme, PEP~386 defines a strict ordering:

\begin{aosaitemize}

  \item alpha {\textless} beta {\textless} rc {\textless} final

  \item dev {\textless} non-dev {\textless} post, where non-dev can be a alpha, beta,
  rc or final

\end{aosaitemize}

\noindent
Here's a full ordering example:

\begin{verbatim}
1.0a1 < 1.0a2.dev456 < 1.0a2 < 1.0a2.1.dev456
  < 1.0a2.1 < 1.0b1.dev456 < 1.0b2 < 1.0b2.post345
    < 1.0c1.dev456 < 1.0c1 < 1.0.dev456 < 1.0
      < 1.0.post456.dev34 < 1.0.post456
\end{verbatim}

\noindent
The goal of this scheme is to make it easy for other packaging systems
to translate Python projects' versions into their own schemes.  PyPI
now rejects any projects that upload PEP~345 metadata with version numbers
that don't follow PEP~386.

\end{aosasect3}

%% \begin{aosasect3}{Dependencies}
\begin{aosasect3}{依存関係}

PEP~345 defines three new fields that replace PEP~314 \code{Requires},
\code{Provides}, and \code{Obsoletes}. Those fields are
\code{Requires-Dist}, \code{Provides-Dist}, and \code{Obsoletes-Dist},
and can be used multiple times in the metadata.

For \code{Requires-Dist}, each entry contains a string naming some
other \code{Distutils} project required by this distribution.  The
format of a requirement string is identical to that of a
\code{Distutils} project name (e.g., as found in the \code{Name} field)
optionally followed by a version declaration within parentheses.
These \code{Distutils} project names should correspond to names as
found at PyPI, and version declarations must follow the rules
described in PEP~386. Some example are:

\begin{verbatim}
Requires-Dist: pkginfo
Requires-Dist: PasteDeploy
Requires-Dist: zope.interface (>3.5.0)
\end{verbatim}

\noindent \code{Provides-Dist} is used to define extra names contained in the
project.  It's useful when a project wants to merge with another
project. For example the ZODB project can include the
\code{transaction} project and state:

\begin{verbatim}
Provides-Dist: transaction
\end{verbatim}

\noindent \code{Obsoletes-Dist} is useful to mark another project as an obsolete
version:

\begin{verbatim}
Obsoletes-Dist: OldName
\end{verbatim}

\end{aosasect3}

%% \begin{aosasect3}{Environment Markers}
\begin{aosasect3}{環境マーカー}

An environment marker is a marker that can be added at the end of a
field after a semicolon to add a condition about the execution
environment.  Some examples are:

\begin{verbatim}
Requires-Dist: pywin32 (>1.0); sys.platform == 'win32'
Obsoletes-Dist: pywin31; sys.platform == 'win32'
Requires-Dist: foo (1,!=1.3); platform.machine == 'i386'
Requires-Dist: bar; python_version == '2.4' or python_version == '2.5'
Requires-External: libxslt; 'linux' in sys.platform
\end{verbatim}

The micro-language for environment markers is deliberately kept simple
enough for non-Python programmers to understand: it compares strings
with the \code{==} and \code{in} operators (and their opposites), and
allows the usual Boolean combinations.  The fields in PEP~345 that can
use this marker are:

\begin{aosaitemize}
  \item \code{Requires-Python}
  \item \code{Requires-External}
  \item \code{Requires-Dist}
  \item \code{Provides-Dist}
  \item \code{Obsoletes-Dist}
  \item \code{Classifier}
\end{aosaitemize}

\end{aosasect3}

\end{aosasect2}

%% \begin{aosasect2}{What's Installed?}
\begin{aosasect2}{何がインストールされるのか?}

Having a single installation format shared among all Python tools is
mandatory for interoperability. If we want Installer A to detect that
Installer B has previously installed project Foo, they both need to
share and update the same database of installed projects.

Of course, users should ideally use a single installer in their
system, but they may want to switch to a newer installer that has
specific features. For instance, Mac OS X ships \code{Setuptools}, so
users automatically have the \code{easy\_install} script. If they
want to switch to a newer tool, they will need it to be backward
compatible with the previous one.

Another problem when using a Python installer on a platform that has a
packaging system like RPM is that there is no way to inform the system
that a project is being installed. What's worse, even if the Python
installer could somehow ping the central packaging system, we would
need to have a mapping between the Python metadata and the system
metadata. The name of the project, for instance, may be different for
each. That can occur for several reasons. The most common one is 
a conflict name: another project outside the Python land already uses
the same name for the RPM. Another cause is that the name used include
a \code{python} prefix that breaks the convention of the platform.
For example, if you name your project \code{foo-python}, there are high
chances that the Fedora RPM will be called \code{python-foo}.

One way to avoid this problem is to leave the global Python
installation alone, managed by the central packaging system, and work
in an isolated environment.  Tools like \code{Virtualenv} allows this.

In any case, we do need to have a single installation format in Python
because interoperability is also a concern for other packaging systems
when they install themselves Python projects.  Once a third-party 
packaging system has registered a newly installed project in its 
own database on the system, it needs to generate the right metadata 
for the Python installaton itself, so projects appear to be installed 
to Python installers or any APIs that query the Python installation.

The metadata mapping issue can be addressed in that case: since an RPM
knows which Python projects it wraps, it can generate the proper
Python-level metadata. For instance, it knows that
\code{python26-webob} is called \code{WebOb} in the PyPI ecosystem.

Back to our standard: PEP~376 defines a standard for installed
packages whose format is quite similar to those used by
\code{Setuptools} and \code{Pip}.  This structure is a directory with
a \code{dist-info} extension that contains:

\begin{aosaitemize}

  \item \code{METADATA}: the metadata, as described in PEP~345, PEP~314
  and PEP~241.

  \item \code{RECORD}: the list of installed files in a csv-like format.

  \item \code{INSTALLER}: the name of the tool used to install the
  project.

  \item \code{REQUESTED}: the presence of this file indicates that the
  project installation was explicitly requested (i.e., not installed
  as a dependency).

\end{aosaitemize}

\noindent
Once all tools out there understand this format, we'll be able to
manage projects in Python without depending on a particular installer
and its features.  Also, since PEP~376 defines the metadata as a
directory, it will be easy to add new files to extend it.  As a matter
of fact, a new metadata file called \code{RESOURCES}, described in the
next section, might be added in a near future without modifying
PEP~376. Eventually, if this new file turns out to be useful for all
tools, it will be added to the PEP.

\end{aosasect2}

%% \begin{aosasect2}{Architecture of Data Files}
\begin{aosasect2}{データファイルのアーキテクチャ}

As described earlier, we need to let the packager decide where to put
data files during installation without breaking the developer's
code. At the same time, the developer must be able to work with data
files without having to worry about their location.  Our solution is
the usual one: indirection.

%% \begin{aosasect3}{Using Data Files}
\begin{aosasect3}{データファイルの利用}

Suppose your \code{MPTools} application needs to work with a
configuration file.  The developer will put that file in a Python
package and use \code{\_\_file\_\_} to reach it:

\begin{verbatim}
import os

here = os.path.dirname(__file__)
cfg = open(os.path.join(here, 'config', 'mopy.cfg'))
\end{verbatim}

\noindent This implies that configuration files are installed like code, and
that the developer \emph{must} place it alongside her code: in this
example, in a subdirectory called \code{config}.

The new architecture of data files we have designed
uses the project tree as the root
of all files, and allows access to any file in the tree, whether it is
located in a Python package or a simple directory.  This allowed
developers to create a dedicated directory for data files and access
them using \code{pkgutil.open}:

\begin{verbatim}
import os
import pkgutil

# Open the file located in config/mopy.cfg in the MPTools project
cfg = pkgutil.open('MPTools', 'config/mopy.cfg')
\end{verbatim}

\noindent \code{pkgutil.open} looks for the project metadata and see if it
contains a \code{RESOURCES} file. This is a simple map of files to
locations that the system may contain:

\begin{verbatim}
config/mopy.cfg {confdir}/{distribution.name}
\end{verbatim}

\noindent Here the \code{{confdir}} variable points to the
system's configuration directory, and \code{{distribution.name}}
contains the name of the Python project as found in the metadata.

\aosafigure[300pt]{../images/packaging/find-file.eps}{Finding a File}{fig.packaging.findfile}

As long as this \code{RESOURCES} metadata file is created at
installation time, the API will find the location of \code{mopy.cfg}
for the developer.  And since \code{config/mopy.cfg} is the path
relative to the project tree, it means that we can also offer a
development mode where the metadata for the project are generated
in-place and added in the lookup paths for \code{pkgutil}.

\end{aosasect3}

%% \begin{aosasect3}{Declaring Data Files}
\begin{aosasect3}{データファイルの宣言}

In practice, a project can define where data files should land by
defining a mapper in their \code{setup.cfg} file.  A mapper is a
list of \code{(glob-style pattern, target)} tuples. Each pattern
points to one of several files in the project tree, while the target
is an installation path that may contain variables in brackets.  For
example, \code{MPTools}'s \code{setup.cfg} could look like this:

\begin{verbatim}
[files]
resources =
        config/mopy.cfg {confdir}/{application.name}/
        images/*.jpg    {datadir}/{application.name}/
\end{verbatim}

\noindent
The \code{sysconfig} module will provide and document a specific list
of variables that can be used, and default values for each platform.
For example \code{{confdir}} is \code{/etc} on Linux.  Installers can
therefore use this mapper in conjunction with \code{sysconfig} at
installation time to know where the files should be placed.
Eventually, they will generate the \code{RESOURCES} file mentioned
earlier in the installed metadata so \code{pkgutil} can find back the
files.

\aosafigure[300pt]{../images/packaging/installer.eps}{Installer}{fig.packaging.installer}

\end{aosasect3}

\end{aosasect2}

%% \begin{aosasect2}{PyPI Improvements}
\begin{aosasect2}{PyPIの改良}

I said earlier that PyPI was effectively a single point of
failure. PEP~380 addresses this problem by defining a mirroring
protocol so that users can fall back to alternative servers when PyPI
is down. The goal is to allow members of the community to run mirrors
around the world.

\aosafigure[250pt]{../images/packaging/mirroring.eps}{Mirroring}{fig.packaging.mirroring}

The mirror list is provided as a list of host names of the form
\code{X.pypi.python.org}, where \code{X} is in the sequence
\code{a,b,c,...,aa,ab,...}.  \code{a.pypi.python.org} is the master
server and mirrors start with b. A CNAME record
\code{last.pypi.python.org} points to the last host name so clients
that are using PyPI can get the list of the mirrors by looking at the
CNAME.

For example, this call tells use that the last mirror is
\code{h.pypi.python.org}, meaning that PyPI currently has 6 mirrors (b
through h):

\begin{verbatim}
>>> import socket
>>> socket.gethostbyname_ex('last.pypi.python.org')[0]
'h.pypi.python.org'
\end{verbatim}

\noindent Potentially, this protocol allows clients to redirect requests to the
nearest mirror by localizing the mirrors by their IPs, and also fall
back to the next mirror if a mirror or the master server is down.  The
mirroring protocol itself is more complex than a simple rsync because
we wanted to keep downloads statistics accurate and provide minimal
security.

%% \begin{aosasect3}{Synchronization}
\begin{aosasect3}{同期処理}

Mirrors must reduce the amount of data transferred between the central
server and the mirror. To achieve that, they \emph{must} use the
\code{changelog} PyPI XML-RPC call, and only refetch the packages that
have been changed since the last time.  For each package P, they
\emph{must} copy documents \code{/simple/P/} and \code{/serversig/P}.

If a package is deleted on the central server, they \emph{must} delete
the package and all associated files. To detect modification of
package files, they may cache the file's ETag, and may request
skipping it using the \code{If-None-Match} header.  Once the
synchronization is over, the mirror changes its \code{/last-modified}
to the current date.

\end{aosasect3}

%% \begin{aosasect3}{Statistics Propagation}
\begin{aosasect3}{統計情報の伝搬}

When you download a release from any of the mirrors, the protocol
ensures that the download hit is transmitted to the master PyPI
server, then to other mirrors.  Doing this ensures that people or
tools browsing PyPI to find out how many times a release was downloaded
will get a value summed across all mirrors.

Statistics are grouped into daily and weekly CSV files in the
\code{stats} directory at the central PyPI itself.  Each mirror needs
to provide a \code{local-stats} directory that contains its own
statistics. Each file provides the number of downloads for each
archive, grouped by use agents.  The central server visits mirrors
daily to collect those statistics, and merge them back into the global
\code{stats} directory, so each mirror must keep \code{/local-stats}
up-to-date at least once a day.

\end{aosasect3}

%% \begin{aosasect3}{Mirror Authenticity}
\begin{aosasect3}{信頼性のミラー}

With any distributed mirroring system, clients may want to verify that
the mirrored copies are authentic.  Some of the possible threats
include:

\begin{aosaitemize}

  \item the central index may be compromised

  \item the mirrors might be tampered with

  \item a man-in-the-middle attack between the central index and the end
  user, or between a mirror and the end user

\end{aosaitemize}

\noindent
To detect the first attack, package authors need to sign their
packages using PGP keys, so that users can verify that the package
comes from the author they trust.  The mirroring protocol itself only
addresses the second threat, though some attempt is made to detect
man-in-the-middle attacks.

The central index provides a DSA key at the URL \code{/serverkey}, in
the PEM format as generated by \code{openssl dsa -pubout}\footnote{I.e.,
RFC 3280 SubjectPublicKeyInfo,
with the algorithm 1.3.14.3.2.12.}. This URL must not be mirrored,
and clients must fetch the official \code{serverkey} from PyPI
directly, or use the copy that came with the PyPI client
software. Mirrors should still download the key so that they can
detect a key rollover.

For each package, a mirrored signature is provided at
\code{/serversig/package}.  This is the DSA signature of the parallel
URL \code{/simple/package}, in DER form, using SHA-1 with
DSA\footnote{I.e., as a RFC 3279 Dsa-Sig-Value, created by algorithm
  1.2.840.10040.4.3.}.

Clients using a mirror need to perform the following steps to verify a
package:

\begin{enumerate}

  \item Download the \code{/simple} page, and compute its SHA-1 hash.

  \item Compute the DSA signature of that hash.

  \item Download the corresponding \code{/serversig}, and compare it
  byte for byte with the value computed in step 2.

  \item Compute and verify (against the \code{/simple} page) the MD5
  hashes of all files they download from the mirror.

\end{enumerate}

Verification is not needed when downloading from central index, and
clients should not do it to reduce the computation overhead.

About once a year, the key will be replaced with a new one. Mirrors
will have to re-fetch all \code{/serversig} pages. Clients using
mirrors need to find a trusted copy of the new server key. One way to
obtain one is to download it from
\url{https://pypi.python.org/serverkey}.  To detect man-in-the-middle
attacks, clients need to verify the SSL server certificate, which will
be signed by the CACert authority.

\end{aosasect3}

\end{aosasect2}

\end{aosasect1}

%% \begin{aosasect1}{Implementation Details}
\begin{aosasect1}{実装の詳細}

The implementation of most of the improvements described in the previous
section are taking place in \code{Distutils2}.  The \code{setup.py}
file is not used anymore, and a project is completely described in
\code{setup.cfg}, a static \code{.ini}-like file.  By doing this, we
make it easier for packagers to change the behavior of a project
installation without having to deal with Python code.  Here's an
example of such a file:

\begin{verbatim}
[metadata]
name = MPTools
version = 0.1
author = Tarek Ziade
author-email = tarek@mozilla.com
summary = Set of tools to build Mozilla Services apps
description-file = README
home-page = http://bitbucket.org/tarek/pypi2rpm
project-url: Repository, http://hg.mozilla.org/services/server-devtools
classifier = Development Status :: 3 - Alpha
    License :: OSI Approved :: Mozilla Public License 1.1 (MPL 1.1)
\end{verbatim}

\begin{verbatim}
[files]
packages =
        mopytools
        mopytools.tests

extra_files =
        setup.py
        README
        build.py
        _build.py

resources =
    etc/mopytools.cfg {confdir}/mopytools
\end{verbatim}

\noindent \code{Distutils2} use this configuration file to:

\begin{aosaitemize}

  \item generate \code{META-1.2} metadata files that can be used for various
  actions, like registering at PyPI.

  \item run any package management command, like \code{sdist}.

  \item install a \code{Distutils2}-based project.

\end{aosaitemize}

\noindent
\code{Distutils2} also implements \code{VERSION} via its \code{version}
module.

The \code{INSTALL-DB} implementation will find its way to the standard library
in Python 3.3 and will be in the \code{pkgutil} module. In the
interim, a version of this module exists in \code{Distutils2} for
immediate use.  The provided APIs will let us browse an installation
and know exactly what's installed.

These APIs are the basis for some neat \code{Distutils2} features:

\begin{aosaitemize}
  \item installer/uninstaller
  \item dependency graph view of installed projects
\end{aosaitemize}

\end{aosasect1}

%% \begin{aosasect1}{Lessons learned}
\begin{aosasect1}{教訓}

%% \begin{aosasect2}{It's All About PEPs}
\begin{aosasect2}{PEPこそがすべて}

Changing an architecture as wide and complex as Python packaging 
needs to be carefully done by changing standards through a
PEP process. And changing or adding a new PEP takes in my
experience around a year.

One mistake the community made along the way was to deliver tools 
that solved some issues by extending the Metadata and the way Python 
applications were installed without trying to change the impacted 
PEPs.

In other words, depending on the tool you used, the standard library
\code{Distutils} or \code{Setuptools}, applications were installed
differently. The problems were solved for one part of the community 
that used these new tools, but added more problems for the rest of 
the world.
OS Packagers for instance, had to face several Python standards:
the official documented standard and the de-facto standard imposed by 
\code{Setuptools}.

But in the meantime, \code{Setuptols} had the opportunity to experiment
in a realistic scale (the whole community) some innovations in a
very fast pace, and the feedback was invaluable. We were able to 
write down new PEPs with more confidence in what worked and what did not,
and maybe it would have been impossible to do so differently.
So it's all about detecting when some third-party tools are contributing
innovations that are solving problems and that should ignite a PEP
change.

\end{aosasect2}

%% \begin{aosasect2}{A Package that Enters the Standard Library Has One Foot in the Grave}
\begin{aosasect2}{標準ライブラリ入りは死への第一歩}

I am paraphrasing Guido van Rossum in the section title, but that's
one aspect of the batteries-included philosophy of Python that 
impacts a lot our efforts.

\code{Distutils} is part of the standard library and \code{Distutils2} 
will soon be. A package that's in the standard library is very hard to 
make evolve. There are of course deprecation processes, where you can kill 
or change an API after 2 minor versions of Python. But once an API is 
published, it's going to stay there for years.

So any change you make in a package in the standard library that is 
not a bug fix, is a potential disturbance for the eco-system. So when 
you're doing important changes, you have to create a new package.

I've learned it the hard way with \code{Distutils} since I had to eventually
revert all the changes I had done in it for more that a year and create
\code{Distutils2}. In the future, if our standards change again in a drastic 
way, there are high chances that we will start a standalone \code{Distutils3}
project first, unless the standard library is released on its own at some point.

\end{aosasect2}

%% \begin{aosasect2}{Backward Compatibility}
\begin{aosasect2}{後方互換性}

Changing the way packaging works in Python is a very long process: the
Python ecosystem contains so many projects based on older packaging
tools that there is and will be a lot of resistance to change.
(Reaching consensus on some of the topics discussed in this chapter
took several years, rather than the few months I originally
expected.)  As with Python 3, it will take years before all projects
switch to the new standard.

That's why everything we are doing has to be backward-compatible with 
all previous tools, installations and standards, which makes the 
implementation of \code{Distutils2} a wicked problem.

For example, if a project that uses the new standards depends on 
another project that don't use them yet, we can't stop the installation
process by telling the end-user that the dependency is in an unknown 
format!

For example, the \code{INSTALL-DB} implementation contains compatibility code 
to browse projects installed by the original \code{Distutils},
\code{Pip}, \code{Distribute}, or \code{Setuptools}.
\code{Distutils2} is also able to install projects created by the
original \code{Distutils} by converting their metadata on the fly.

\end{aosasect2}


\end{aosasect1}

%% \begin{aosasect1}{References and Contributions}
\begin{aosasect1}{参考文献と謝辞}

Some sections in this paper were directly taken from the various PEP
documents we wrote for packaging. You can find the original documents
at \url{http://python.org}:

\begin{aosaitemize}
  \item PEP 241: Metadata for Python Software Packages 1.0: \url{http://python.org/peps/pep-0214.html}
  \item PEP 314: Metadata for Python Software Packages 1.1: \url{http://python.org/peps/pep-0314.html}
  \item PEP 345: Metadata for Python Software Packages 1.2: \url{http://python.org/peps/pep-0345.html}
  \item PEP 376: Database of Installed Python Distributions: \url{http://python.org/peps/pep-0376.html}
  \item PEP 381: Mirroring infrastructure for PyPI: \url{http://python.org/peps/pep-0381.html}
  \item PEP 386: Changing the version comparison module in Distutils: \url{http://python.org/peps/pep-0386.html}
\end{aosaitemize}

I would like to thank all the people that are working on packaging;
you will find their name in every PEP I've mentioned. I would also
like to give a special thank to all members of The Fellowship of the
Packaging.  Also, thanks to Alexis Metaireau, Toshio Kuratomi, Holger
Krekel and Stefane Fermigier for their feedback on this chapter.

The projects that were discussed in this chapter are:

\begin{aosaitemize}
  \item \code{Distutils}: \url{http://docs.python.org/distutils}
  \item \code{Distutils2}: \url{http://packages.python.org/Distutils2}
  \item \code{Distribute}: \url{http://packages.python.org/distribute}
  \item \code{Setuptools}: \url{http://pypi.python.org/pypi/setuptools}
  \item \code{Pip}: \url{http://pypi.python.org/pypi/pip}
  \item \code{Virtualenv}: \url{http://pypi.python.org/pypi/virtualenv}
\end{aosaitemize}

\end{aosasect1}

\end{aosachapter}
